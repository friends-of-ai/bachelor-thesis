\documentclass[10pt]{article}

% add links to document
\PassOptionsToPackage{hyphens}{url}\usepackage[hidelinks]{hyperref}

% add some packages
\usepackage{xcolor}
\usepackage[english]{babel}
\usepackage{nameref}
\usepackage{footnote}
\usepackage{refcount}
\usepackage{makecell}
\usepackage[a4paper, total={6in, 10in}]{geometry}
\usepackage{titlesec}
\usepackage{graphicx,import}
\usepackage{setspace}
\usepackage{url}
\usepackage{amsmath,amsfonts,amssymb}
%\usepackage{indentfirst}
\usepackage{float}
\usepackage{pgf}
\usepackage{pdfpages}
\usepackage{svg}
\usepackage{pstricks}
\usepackage{color,soul}
\usepackage{pst-plot}
\usepackage{import}
\usepackage{xifthen}
\usepackage{pdfpages}
\usepackage{transparent}
\usepackage[utf8]{inputenc}
\usepackage{csquotes}
\usepackage[style=verbose-ibid,backend=bibtex]{biblatex}
\usepackage{multirow}
\usepackage{caption}
\usepackage{colortbl}\usepackage{float}
\usepackage{placeins}
\usepackage{dblfloatfix}
\usepackage{tikz}
\usetikzlibrary{positioning}

\usetikzlibrary{math} %needed tikz library


\newcommand*\circled[1]{\tikz[baseline=(char.base)]{
   \node[shape=circle,draw,inner sep=1pt] (char) {#1};}}

\restylefloat{table}

% add more distance between footnotes and the main text
\addtolength{\skip\footins}{2pc plus 5pt}

% italic blockquotes
\renewcommand{\mkbegdispquote}[2]{\itshape}
\renewcommand{\mkbegdispquote}[2]{\openautoquote}
\renewcommand{\mkenddispquote}[2]{\closeautoquote#1#2}

% literature reference
\bibliography{references}

% document settings
\author{Björn Hempel <bjoern@hempel.li>}

% add nice formated tables
\makesavenoteenv{tabular}
\makesavenoteenv{table}

% add automatic numbering
\setcounter{secnumdepth}{5}
\setcounter{tocdepth}{4}
\setlength\bibitemsep{1.5\itemsep}

% some formating settings
\setlength{\parindent}{20pt}
\setlength{\parskip}{5pt}
\titleformat{\paragraph}
{\normalfont\normalsize\bfseries}{\theparagraph}{1em}{}
\titlespacing*{\paragraph}
{0pt}{3.25ex plus 1ex minus .2ex}{1.5ex plus .2ex}

% Footmarks always at the end of the page
\usepackage[bottom]{footmisc}

% allow some special characters
\DeclareUnicodeCharacter{2212}{-}

% pgf figures with images (add function \inputpgf)
\newcommand\inputpgf[2]{{
\let\pgfimageWithoutPath\pgfimage
\renewcommand{\pgfimage}[2][]{\pgfimageWithoutPath[##1]{#1/##2}}
\input{#1/#2}
}}

% import function of tex files
\newcommand{\includetex}[1]{
    \def\svgwidth{\columnwidth}
    \import{images/other/}{#1.pdf_tex}
}

% add language switcher functions \de and \en
\newif\ifen
\newif\ifde
\newcommand{\en}[1]{\ifen#1\fi}
\newcommand{\de}[1]{\ifde#1\fi}

% language switcher
%\detrue
\entrue



% -------------------- %
% Start document %
% -------------------- %
\begin{document}



	% -------------------- %
	% Flyleaf %
	% -------------------- %
	\thispagestyle{empty}

\includegraphics[width=0.4\textwidth]{images/tu_chemnitz}\\[2\baselineskip]

\begin{center}
\Large{TU Chemnitz}
\end{center}


\begin{center}
\Large{Faculty of Natural Sciences}\\
\Large{Institute of Physics}\\[4\baselineskip]
\end{center}

\begin{center}
\textbf{\LARGE{Bachelor Thesis}}\\[2\baselineskip]
\textbf{{In the course of degree in computational science (B.Sc.)}}
\end{center}

\begin{center}
\textbf{{To obtain the academic degree Bachelor of Science}}
\end{center}

\vspace*{\fill}
\begin{flushleft}
\begin{tabular}{llll}
\textbf{Topic:} & & Investigation of strategies for image classification \\
& & on small training data sets & \\
& & \\
\textbf{Author:} & & Björn Hempel \textless{bjoern@hempel.li}\textgreater & \\
& & Matriculation number 025038 & \\
& & \\
\textbf{Version from:} & & \today &\\
& & \\
\textbf{First assessor:} & & Prof. Dr. Angela Thränhardt &\\
\textbf{Second assessor:} & & Dr. David Urbansky &\\
\end{tabular}
\end{flushleft}




	\pagebreak



	% -------------------- %
	% Empty page %
	% -------------------- %
    \newpage\null\thispagestyle{empty}\newpage



	% -------------------- %
	% Abstract %
	% -------------------- %
	\section*{Abstract}
		\de{Neuronale Netze sind außerordentlich gut darin Muster in Daten zu finden. Hierzu müssen diese Netze vorher mit bekannten Datensätzen trainiert und entsprechend angepasst werden. Datensätze sind meist sehr teuer in der Beschaffung und sollten deshalb mit Bedacht und guter Qualität eingesetzt werden. Das Training des Netzes findet unter von vielen verschiedenen Parametern und Verfahrenstechniken statt. Dabei ist darauf zu achten das bestmögliche Modell mit seinen bestmöglichen Parametern zu verwenden. In dieser Arbeit werden gängige moderne Methoden der Bildklassifikation vorgestellt und miteinander verglichen. Das Hauptziel der Arbeit ist es, optimale Parameter und Techniken für die Klassifikation zu finden, die es auch ermöglichen, mit wenig Trainingsdaten ein optimales Modell zu erstellen.}
		\en{Neural networks are extraordinarily good at finding patterns in data. For this purpose, these networks must be trained with known data sets and adapted accordingly. Data sets are usually very expensive to obtain and should therefore be used with care and good quality. The training of the network takes place under many different parameters and process techniques. Care must be taken to use the best possible model with its best possible parameters. In this thesis, common modern methods of image classification will be presented and compared with each other. The main goal of the work is to find optimal parameters and techniques for the classification, which also allow to create an optimal model with little training data.}
	\pagebreak



	% -------------------- %
	% Table of contents %
	% -------------------- %
	\tableofcontents
	\pagebreak


	% -------------------- %
	% Introduction %
	% -------------------- %
	\section{Introduction}
		\de{In dieser Arbeit werden verschiedene Techniken der Bildklassifizierung verglichen. Variable Parameter beim Training werden einen entscheidenden Einfluss auf die Genauigkeit des Modells haben und werden hier im Detail verglichen. Nicht immer ist nur die Genauigkeit ein ausschlaggebender Faktor. Auch die benötigte Rechenzeit, welche notwendig ist das Modell zu bestimmen, soll nicht außer Acht gelassen werden und mit in die Auswertung einbezogen werden. Ich gehe davon aus, dass eine kleine Lernrate verbunden mit vielen Lern-Epochen und entsprechend mehr benötigter Rechenzeit bessere Ergebnisse erzielen werden, als wenige Lernepochen verbunden mit einer hohe Lernrate (langsame Anpassung vs. schnelle Anpassung). Auch gehe ich davon aus, dass eine hohe Qualität und eine größere Menge an Daten das Ergebnis entschieden positiv beeinflussen werden. Neue und komplexere Convolutional Neuronale Netzwerke stufe ich erfolgreicher in der Modellgenauigkeit ein als schön etwas ältere und kleinere Modelle.}
		\en{In this thesis different techniques of image classification are compared. Variable parameters during training will have a decisive influence on the accuracy of the model and are compared here in detail. Not always only the accuracy is a decisive factor. Also the required computing time, which is necessary to determine the model, should not be disregarded and should be included in the evaluation. I assume that a small learning rate combined with many learning epochs and correspondingly more computing time required will achieve better results than a few learning epochs combined with a high learning rate (slow adaptation vs. fast adaptation). I also assume that a high quality and a larger amount of data will have a decidedly positive influence on the result. New and more complex convolutional neural networks are more successful in model accuracy than older and smaller models.}


	% -------------------- %
	% Background %
	% -------------------- %
	\section{Background}

		% -------------------- %
		% Image Classification %
		% -------------------- %
		\subsection{Image Classification}
			\de{Klassifizierungen sind ein Prozess der Identifizierung, zu welcher Klasse ein unbeobachtetes Objekt gehört. Hierbei können eine Reihe von vordefinierten Klassen vorgegeben und anhand deren Eigenschaften versucht werden unbekannte und bisher unbeobachtete Objekt einzuordnen. Bei der Bildklassifizierung wird analog vorgegangen. Bei den zuvor genannten Objekten handelt es sich nun schlichtweg um Bilder.}
			\en{Classifications are a process of identifying to which class an unobserved object belongs. A number of predefined classes can be specified and, based on their properties, an attempt can be made to classify unknown and previously unobserved objects. The procedure for image classification is similar. The previously mentioned objects are now simply images.}

			\begin{figure}[H]
				\centering
				\includegraphics[width=0.4\textwidth]{images/cat_dog}
				\caption[Is it a dog or a cat?]{Is it a dog or a cat?\footnotemark}
				\label{fig:cat_or_dog}
			\end{figure}
			\footnotetext{Source: \url{https://towardsdatascience.com/image-classifier-cats-vs-dogs-with-convolutional-neural-networks-cnns-and-google-colabs-4e9af21ae7a8}}

			\de{Lange Zeit galt die automatische Erkennung von Objekten, Personen und Szenen in Bildern durch Computer als unmöglich. Die Komplexität schien zu groß, als dass man sie einem Algorithmus programmatisch beibringen könnte. Bis noch vor einigen Jahrzehnten hat man so versucht Bildklassifikation durch manuell entwickelte Algorithmen zu erreichen. Die automatisierte Klassifikation anhand von vorgegebenen und vorklassifizierten Bildern und dem automatisierten Erstellen von Modellen war ein neuer Schritt in eine neue Vorgehensweise. Die dabei entwickelten neuronale Netze spielten eine gewaltige Rolle und änderten dramatisch die Art der Herangehensweise! Mittlerweile ist die Bilderkennung ein weit verbreitetes Anwendungsgebiet des maschinellen Lernens. Häufig werden für Bilder sogenannte "Convolutional Neural Networks\footnote{Convolutional neural network, Wikipedia contributors, January 31, 2020, \url{https://en.wikipedia.org/wiki/Convolutional_neural_network}}" or "ConvNets" are often used for images.}
			\en{For a long time, the automatic recognition of objects, people and scenes in images by computers was considered impossible. The complexity seemed too great to be programmatically taught to an algorithm. Until a few decades ago, attempts were made to achieve image classification by manually developed algorithms. Automated classification based on given and pre-classified images and the automated creation of models was a new step into a new approach. The neural networks developed in this process played a huge role and dramatically changed the way of approach! In the meantime, image recognition has become a widespread application area of machine learning. So-called "Convolutional Neural Networks\footnote{Convolutional neural network, Wikipedia contributors, January 31, 2020, \url{https://en.wikipedia.org/wiki/Convolutional_neural_network}}" or "ConvNets" are often used for images.}

			\de{Der Bildklassifizierungsalgorithmus nimmt ein Bild als Eingabe und klassifiziert es in eine der Ausgabekategorien. Deep Learning hat den Bereich der Bildklassifizierung revolutioniert und großartige Ergebnisse erzielt. Verschiedene Deep Learning Netzwerke, wie ResNet, DenseNet, Inception, etc. wurden als hochpräzise Netzwerke für die Bildklassifikation entwickelt. Gleichzeitig wurden Bilddatensätze angelegt, um getaggte Bilddaten zu erfassen. Diese werden jetzt vorrangig dazu verwendet um bestehende Netzwerke zu trainieren und alljährliche Challenges zu veranstalten, welche sich mit den bisher bekannten und schon entwickelten Modellgenauigkeiten messen. ImageNet ist ein solch großer Datensatz mit mehr als 11 Millionen Bildern und über 11.000 Kategorien. Wenn ein Netzwerk einmal mit ImageNet-Daten trainiert wurde, kann es durch einfache Neuanpassung oder Optimierung mit anderen Datensätzen verallgemeinert werden. Bei diesem Transfer-Lernansatz wird ein Netzwerk mit Gewichten initialisiert, welche aus einem zuvor trainierten Netzwerk stammen. Dieses zuvor initialisierte Netzwerk wird nun für eine neue Bildklassifikationsaufgabe lediglich entsprechend angepasst.}
			\en{The image classification algorithm takes an image as input and classifies it into one of the output categories. Deep Learning has revolutionized the field of image classification and has achieved great results. Various Deep Learning networks, such as ResNet, DenseNet, Inception, etc. have been developed as high-precision networks for image classification. At the same time, image data sets were created to capture tagged image data. These are now primarily used to train existing networks and to organize annual challenges that compete with the model accuracies already known and developed. ImageNet is such a large data set with more than 11 million images and over 11,000 categories. Once a network has been trained with ImageNet data, it can be generalized with other data sets by simple re-compilation or optimization. In this transfer learning approach, a network is initialized with weights that come from a previously trained network. This previously initialized network is now simply adapted for a new image classification task.}

			\de{Die hier zugrunde liegende Arbeit beschäftigt sich hauptsächlich mit überwachtem Lernen, bei dem ein mathematisches Modell aufgrund bestehender bekannter Datensätze trainiert wird. Das Ziel des trainierten Modells ist es dabei auch für unbekannte Bilder bestmögliche Vorhersagen zu treffen. Diese bekannten Datensätze werden meist händisch erstellt (Ontologe), automatisiert aufgrund bekannter Tatsachen bestimmt oder auch in einem halbautomatischen Prozess ermittelt.}
			\en{The underlying work here is mainly concerned with supervised learning, in which a mathematical model is trained based on existing known data sets. The goal of the trained model is to make best possible predictions even for unknown images. These known data sets are usually created manually (ontologist), automatically determined based on known facts or determined in a semi-automatic process.}
			
		% -------------------- %
		% Deductive approach %
		% -------------------- %
		\subsubsection{Deductive approach}
		\label{sec:section_deductive_approach}
			\de{Seit den späten 1960er Jahren hat man versucht, Bilder mit selbstgeschriebenen Algorithmen zu klassifizieren. Dieser Teil der Computer Vision beschäftigt sich mit Techniken wie Bildentstehung, Bildbearbeitung und Bildsegmentierung. Im Bereich der Bildverarbeitung reihen sich bekannte Verfahren wie Kantenerkennungen, Merkmalsdetektoren, Randverknüpfungen, Kontrastverbesserungen, etc.\footnote{Szeliski, R.: Computer Vision: Algorithms and Applications,  Springer Science Business Media, 10 (2010)\label{springer_10}} Allen Techniken gemein ist die Verwendung des deduktiven Ansatzes. Beim deduktiven Ansatz erstellt man Regeln (Merkmalsdetektoren), welche das gewünschte Ergebnis vorhersagen sollen. Diese Regeln werden vorgegeben und beschrieben und erlauben damit später eine Klassifizierung von unbekannten Objekten. Da das Modell und sein Algorithmus hinreichend bekannt ist, wird dieses Verfahren White-Box-Verfahren genannt.}
			\en{Since the late 1960s, attempts have been made to classify images with self-written algorithms. This part of Computer Vision deals with techniques such as image creation, image processing and image segmentation. In the field of image processing, well-known techniques such as edge detection, feature detectors, edge linking, contrast enhancement, etc. are used\footnote{Szeliski, R.: Computer Vision: Algorithms and Applications,  Springer Science Business Media, 10 (2010)\label{springer_10}}. Common to all techniques is the use of the deductive approach. With the deductive approach, one creates rules (feature detectors) which are supposed to predict the desired result. These rules are given and described and thus allow later classification of unknown objects. Since the model and its algorithm are sufficiently well known, this procedure is called white-box procedure.}

			\begin{figure}[H]
				\begin{center}
					\scalebox{1.0}{\includetex{deductive_approach}}
				\end{center}
				\caption{Deductive approach}
				\label{fig:overview_deductive_approach}
			\end{figure}

			\de{}

		% -------------------- %
		% Inductive approach %
		% -------------------- %
		\subsubsection{Inductive approach}
		\label{sec:section_inductive_approach}
			\de{Der induktive Ansatz hingegen verfolgt einen anderen Ansatz Bilder zu klassifizieren. Das Ziel ist nicht die Vorgabe einer Regel, sondern das Vorgehen aus schon bekannten einzelnen Objekten eine Regel (Modell) automatisiert zu erlernen. Ein Modell meist eine komplexe Funktion und eine mathematische Abbildung eines Raumes (VC Dimension\footnote{Vapnik–Chervonenkis dimension, Wikipedia contributors, January 31, 2020, \url{https://en.wikipedia.org/wiki/Vapnik\%E2\%80\%93Chervonenkis_dimension}}), in welcher einzelne Objekte mit ihren Eigenschaften abgebildet und getrennt werden können. Das Modell wird Stück für Stück den bekannten Objekten so angepasst, dass der Eingabewert dem Ausgabewert entspricht bzw. weitgehend entspricht (Backpropagation). Das Ziel ist es mit diesem Modell eine Funktion zu erstellen, welche in der Lage ist auch unbekannte Objekte bestmöglichst zu klassifizieren. Da der Raum dieses Modell meist fern der Vorstellungskraft und der Erklärungsmöglichkeit liegt, wird dieses Verfahren auch Black-Box-Verfahren genannt. Der hier beschriebene Vorgang wird meist bei jeder Art von \hyperref[sec:section_supervised_classification]{überwachtem Lernen} angewendet und ist ein Teil des \hyperref[sec:section_machine_learning]{maschinellem Lernens}.}
			\en{The inductive approach, on the other hand, takes a different approach to classifying images. The goal is not to specify a rule, but to learn a rule (model) automatically from already known individual objects. A model is usually a complex function and a mathematical representation of a space (VC dimension\footnote{Vapnik–Chervonenkis dimension, Wikipedia contributors, January 31, 2020, \url{https://en.wikipedia.org/wiki/Vapnik\%E2\%80\%93Chervonenkis_dimension}}), in which individual objects with their properties can be mapped and separated. The model is adapted piece by piece to the known objects in such a way that the input value corresponds to the output value or corresponds to a large extent (backpropagation). The goal is to create a function with this model, which is able to classify unknown objects in the best possible way. Because the space of this model is mostly far away from the imagination and the possibility of explanation, this procedure is also called black box procedure. The procedure described here is mostly used for any kind of \hyperref[sec:section_supervised_classification]{supervised learning} and is a part of \hyperref[sec:section_machine_learning]{machine learning}.}

			\begin{figure}[H]
				\begin{center}
					\scalebox{1.0}{\includetex{inductive_approach}}
				\end{center}
				\caption{Inductive approach}
				\label{fig:overview_inductive_approach}
			\end{figure}

		% -------------------- %
		% Balanced training data set %
		% -------------------- %
		\subsubsection{Balanced training data set}
		\label{sec:section_balanced_training_data_set}
			\de{Neuronale Netze haben in den letzten Jahren enorme Fortschritte im Bereich der Mustererkennung gemacht. Dabei ist ein entscheidender Faktor, dass die Daten zum Lernen eine hohe Qualität und eine einfache Verarbeitung für das Netzwerk aufweisen müssen. Falsch klassifizierte oder irrelevante Daten könnten dazu führen, dass das Netzwerk was Falsches lernt. Das gilt auch für eine nicht vorhandene bzw. nicht geeignete Vorverarbeitung.}
			\en{Neural networks have made enormous progress in the field of pattern recognition in recent years. A decisive factor is that the data for learning must be of high quality and easy for the network to process. Wrongly classified or irrelevant data could cause the network to learn something wrong. This also applies to non-existent or unsuitable pre-processing\autocite{osinga2019data}.}

			\de{Mit dem Beginn eines Klassifizierungsprojektes steht die Frage was genau man klassifizieren möchte und wie umfangreich die Klassifzierung ausfallen soll. Angenommen man möchte verschiedene Klassen von Essen identifizieren, so könnten das Klassen wie Pizza, Burger, Donuts und Lasagne sein (etc.). Zu diesen Klassen benötigt man nun eine große Anzahl an Bildern. Diese Daten sollten im Idealfall die Realität möglichst gut widerspiegeln. Eine große Variation ist von Vorteil (ausbalancierter Datenset): verschiedene Blickwinkel, Größe, Position, Farbhelligkeiten, Variationen, Anzahl, etc. Bilder von z.B. nur einer Farbhelligkeit oder nur einem Blickwinkel sollten vermieden werden. Sind die Daten nicht ausbalanciert, so müssen diese entsprechend korrigiert werden: z.B. durch Hinzufügen weiterer Daten, Bildverarbeitung oder durch Entfernen von Daten, welche für eine Unausgewogenheit sorgen. Weiterhin sollten die ausgewählten Klassen untereinander klar optisch trennbar sein. Sind sich zwei Klassen optisch sehr ähnlich und selbst durch einen Menschen nicht wirklich unterscheidbar, sollte darüber nachgedacht werden diese zusammenzufassen (z.B. "burger" und "veggie burger"):}
			\en{With the beginning of a classification project the question is what exactly you want to classify and how extensive the classification should be. Assuming you want to identify different classes of food, this could be classes like pizza, burgers, donuts and lasagna (etc.). For these classes you now need a large number of images. Ideally, this data should reflect reality as well as possible. A large variation is advantageous (balanced data set): different viewing angles, size, position, colour brightness, variations, number, etc. Images of e.g. only one colour brightness or only one viewing angle should be avoided. If the data are not balanced, they must be corrected accordingly: e.g. by adding further data, image processing or by removing data that causes an imbalance. Furthermore, the selected classes should be clearly optically separable from each other. If two classes are visually very similar and not really distinguishable even by a human, consideration should be given to combining them (e.g. "burger" and "veggie burger"):}

			\begin{figure}[H]
				\centering
				\includegraphics[width=0.19\textwidth]{images/data/burger/burger28.jpg}
				\includegraphics[width=0.19\textwidth]{images/data/burger/burger77.jpg}
				\includegraphics[width=0.19\textwidth]{images/data/burger/burger89.jpg}
				\includegraphics[width=0.19\textwidth]{images/data/burger/burger162.jpg}
				\includegraphics[width=0.19\textwidth]{images/data/burger/burger449.jpeg}
				\caption[Example pictures of a burger class]{Example pictures of a burger class}
				\label{fig:class_burger}
			\end{figure}

			\begin{figure}[H]
				\centering
				\includegraphics[width=0.19\textwidth]{images/data/donut/donut116.jpg}
				\includegraphics[width=0.19\textwidth]{images/data/donut/donut155.jpg}
				\includegraphics[width=0.19\textwidth]{images/data/donut/donut176.jpg}
				\includegraphics[width=0.19\textwidth]{images/data/donut/donut205.jpg}
				\includegraphics[width=0.19\textwidth]{images/data/donut/donut440.jpg}
				\caption[Example pictures of a donut class]{Example pictures of a donut class}
				\label{fig:class_donut}
			\end{figure}

			\begin{figure}[H]
				\centering
				\includegraphics[width=0.19\textwidth]{images/data/pizza/pizza73.png}
				\includegraphics[width=0.19\textwidth]{images/data/pizza/pizza76.jpg}
				\includegraphics[width=0.19\textwidth]{images/data/pizza/pizza92.jpg}
				\includegraphics[width=0.19\textwidth]{images/data/pizza/pizza108.jpg}
				\includegraphics[width=0.19\textwidth]{images/data/pizza/pizza239.jpg}
				\caption[Example pictures of a pizza class]{Example pictures of a pizza class}
				\label{fig:class_pizza}
			\end{figure}

			\de{An Daten zu gelangen ist oftmals nicht so einfach. Jede Datenquelle hat ihre eigenen Besonderheiten. Eine Möglichkeit an Daten zu gelangen wäre ein automatische Crawling von Bilddatenbanken, Suchmaschinen oder Rezensionen, in welchen Bilder vorkommen. Ein gewisses Maß an Kreativität ist von Vorteil:}
			\en{Accessing data is often not that easy. Every data source has its own special features. One way to access data would be an automatic crawling of image databases, search engines or reviews in which images appear. A certain amount of creativity is advantageous:}

			\begin{itemize}
				\item Google
				\item Bing
				\item Flickr
				\item TripAdvisor
				\item etc.
			\end{itemize}
			
			\de{Die wahrscheinlich teuerste Variante an Daten zu gelangen ist die händische Suche und Klassifizierung durch z.B. einen Ontologen. Dieser beurteilt und sucht verschiedenen Bilder und ordnet diese händisch in die entsprechenden Klassen ein. Auch eine kombinierte Variante ist möglich und wahrscheinlich zu bevorzugen: Automatisches Crawling und händisches aussortieren falscher, ungünstiger oder irrelevanter Bilder.}
			\en{Probably the most expensive way to obtain data is to search and classify them manually, e.g. by an ontologist. The ontologist evaluates and searches for different images and manually classifies them in the appropriate classes. A combined variant is also possible and probably preferable: automatic crawling and manual sorting out of incorrect, unfavorable or irrelevant images.}

		% -------------------- %
		% Training, test and evaluation data set %
		% -------------------- %
		\subsubsection{Training, test and evaluation data set}
		\label{sec:section_training_test_and_evaluation}
			\de{Vor dem Beginn mit dem Training von ausbalancierten Bildern müssen diese in einen Trainings-, einen Test- und eventuell in einen Validierungsdatensatz aufgeteilt werden. Dies ist notwendig, da neuronale Netze zu einem Teil nicht verallgemeinern, sondern auswendig lernen werden (overfitting\footnote{``Overfitting'', Wikipedia contributors, January 31, 2020, \url{https://en.wikipedia.org/wiki/Overfitting}}). Die Idee ist es mit einem Trainingsdatensatz zu trainieren, während mit dem Validierungsdatensatz die Allgemeingültigkeit des Netzes und seiner Parameter überwacht wird. Anhand der Ergebnisse werden zur Laufzeit Anpassungen vorgenommen. Da die Anpassung der Parameter anhand der Testdaten vorgenommen wird, gibt es noch einen unabhängigen Testdatensatz, welcher eine erneute Überprüfung des Modells auf bisher unbeteiligte Daten vornimmt. Dieser stellt sicher, dass nicht versehentlich Hyperparameter nur speziell für den Validierungsdatensatz optimiert werden\autocite{osinga2019data}. Die Verwendung des Testdatensatz ist optional und simuliert das Modell unter realen Bedingungen. Ist die Anzahl der Daten begrenzt und kann dieser Datensatz z.B. auch dem Trainingsdatensatz hinzugefügt werden. In dieser Arbeit wird auf den Testdatensatz verzichtet und sämtliche Auswertungen beziehen sich auf den Validierungsdatensatz.}
			\en{Before starting the training of balanced images, they must be divided into a training, a test and possibly a validation data set. This is necessary because neural networks will not generalize to some extent, but will learn by heart (overfitting\footnote{``Overfitting'', Wikipedia contributors, January 31, 2020, \url{https://en.wikipedia.org/wiki/Overfitting}}). The idea is to train with a training data set, while the validation data set is used to monitor the general validity of the network and its parameters. Based on the results, adjustments are made at runtime. Since the adjustment of the parameters is carried out using the test data, there is also an independent test data set, which carries out a renewed check of the model for previously uninvolved data. This ensures that hyperparameters are not inadvertently optimized for the validation data set only\autocite{osinga2019data}. The use of the test data set is optional and simulates the model under real conditions. If the number of data is limited, this data record can also be added to the training data record, for example. In this thesis, the test data set is not used and all evaluations refer to the validation data set.}

			\de{Eine optimale Aufteilung des Trainings- und Validierungsdatensatzes ist abhängig von dem vorliegenden Klassifizierungsproblem und die Anzahl der Daten, welche zur Verfügung stehen. In dieser Arbeit wird ein Verhältnis aus 80 Prozent Trainingsdaten und 20 Prozent Validierungsdaten verwendet, sofern nicht anders angegeben.}
			\en{An optimal division of the training and validation data set depends on the existing classification problem and the amount of data available. In this paper a ratio of 80 percent training data and 20 percent validation data is used, unless otherwise stated.}

			\de{\hl{Hier muss die Frage noch geklärt werden warum 80 Prozent Trainingsdaten und 20 Prozent Validierungsdaten verwendet werden. Gibt es ein Paper bzw. eine Studie dazu? Oder ist noch ein Test erforderlich?}}
			\en{\hl{The question of why 80 percent training data and 20 percent validation data are used remains to be clarified. Is there a paper or a study on this? Or is another test required?}}			

		% -------------------- %
		% Methods of machine learning %
		% -------------------- %
		\subsubsection{Methods of machine learning}
		\label{sec:section_methods_of_machine_learning}
			\de{Je nach Art und Vorgehensweise der Überwachung des Trainings lassen sich verschiedene Machine-Learning-System einordnen. Dabei wird unterschieden, welche Art von Daten uns vorliegen oder diese selbst bestimmt werden müssen.}
			\en{Different machine learning systems can be classified according to the type and procedure of monitoring the training. A distinction is made between the type of data we have or the data we need to determine ourselves.}

			% -------------------- %
			% Supervised learning %
			% -------------------- %
			\paragraph{Supervised learning}
			\label{sec:section_supervised_learning}
				\de{Das überwachte Lernen bezieht sich auf ein maschinelles Lernen mit bekannten Trainingsdatensätzen (siehe auch Kapitel ``induktiver Ansatz''\hyperref[sec:section_inductive_approach]). Der Lernprozess wiederrum bezieht sich auf die Fähigkeit einer künstlichen Intelligenz, Regelmäßigkeiten und Muster zu reproduzieren. Die Ergebnisse sind durch Naturgesetze oder Expertenwissen bekannt und werden für die Lehre des Systems verwendet, in dem ein Trainingsset erstellt wird, welcher die gewünschten Lösungen enthält. Man nennt dies auch gelabelte Daten. Der Lernalgorithmus versucht nun epochenweise eine Hypothese zu finden, die eine möglichst genaue Vorhersagen auf unbekannten Daten ermöglicht. Eine Hypothese ist in diesem Fall ein Bild, das jedem Eingabewert (das Bild selbst) den angenommenen Ausgabewert (die vorhergesagte Klasse) zuordnet. Diese Arbeit macht ausgiebigen Gebrauch von überwachtem Lernen.}
				\en{Supervised learning refers to machine learning with known training data sets (see also chapter ``inductive approach''\hyperref[sec:section_inductive_approach]). The learning process in turn refers to the ability of an artificial intelligence to reproduce regularities and patterns. The results are known by laws of nature or expert knowledge and are used to teach the system by creating a training set containing the desired solutions. This is also called labelled data. The learning algorithm now tries to find a hypothesis epoch by epoch, which allows the most accurate predictions on unknown data. A hypothesis in this case is an image that assigns the assumed output value (the predicted class) to each input value (the image itself). This work makes extensive use of supervised learning.}
	
				\de{Beim Überwachten Lernen wird einer Klassifizierungsfunktion (meist ein künstliches neuronales Netz) ein Eingangsvektor zugeführt. Der Eingangsvektor erzeugt mit Hilfe der Klassifizierungsfunktion einen Ausgabevektor, die dieses neuronale Netz in seinem aktuellen Zustand produziert\footnote{Das neuronale Netzwerk besteht aus vielen (meist Millionen) Parametern, welche während des Lernprozesses angepasst werden können, um den Fehler zu minimieren.}. Dieser Wert wird mit dem Wert verglichen, den es eigentlich ausgeben soll. Der Vergleich des Soll- und Istzustandes gibt Auskunft wie und in welcher Form Änderungen am Netzwerk vorgenommen werden müssen, um dem Istzustand weiter anzugleichen und den Fehler zu minimieren. Für künstliche neuronale Netzwerke ohne versteckter Schicht (einlagiges Perzeptron\footnote{``Perceptron'', Wikipedia contributors, February 2, 2020, \url{https://en.wikipedia.org/wiki/Perceptron}}) kann die Delta-Regel\footnote{``Least mean squares'' filter also known as ``delta rule'', Wikipedia contributors, February 2, 2020, \url{https://en.wikipedia.org/wiki/Least_mean_squares_filter}} für die Korrektur vorgenommen werden. Bei Netzwerken mit einer oder mehreren versteckten Schichten verwendet man Backpropagation\footnote{``Backpropagation'', Wikipedia contributors, February 2, 2020, \url{https://en.wikipedia.org/wiki/Backpropagation}} um den Fehler zu minimieren. Backpropagation ist eine Verallgemeinerung der Delta-Regel.}
				\en{In supervised learning, an input vector is fed to a classification function (usually an artificial neural network). The input vector generates an output vector using the classification function, which produces this neural network in its current state\footnote{The neural network consists of many (usually millions) parameters, which can be adjusted during the learning process to minimize the error.}. This value is compared with the value that it should actually output. The comparison of the nominal and actual state provides information on how and in what form changes must be made to the network in order to further approximate the actual state and minimize the error. For artificial neural networks without a hidden layer (single-layer perceptron\footnote{``Perceptron'', Wikipedia contributors, February 2, 2020, \url{https://en.wikipedia.org/wiki/Perceptron}}), the delta rule\footnote{``Least mean squares'' filter also known as ``delta rule'', Wikipedia contributors, February 2, 2020, \url{https://en.wikipedia.org/wiki/Least_mean_squares_filter}} for correction can be applied. For networks with one or more hidden layers backpropagation\footnote{``Backpropagation'', Wikipedia contributors, February 2, 2020, \url{https://en.wikipedia.org/wiki/Backpropagation}} is used to minimize the error. Backpropagation is a generalization of the delta rule.}
				
				\de{Das neuronale Netzwerke ist nur ein Algorithmus aus der Kategorie der überwachten Lernalgorithmen. Der Vollständigkeit hier noch eine Liste von weiteren Verfahren:}
				\en{The neural network is only one algorithm from the category of supervised learning algorithms. For completeness here is a list of further algorithms:}
				
				\begin{itemize}
					\item k-nearest neighbors\footnote{``k-nearest neighbors algorithm'', Wikipedia contributors, February 2, 2020, \url{https://en.wikipedia.org/wiki/K-nearest_neighbors_algorithm}}
					\item Linear regression\footnote{``Linear regression'', Wikipedia contributors, February 2, 2020, \url{https://en.wikipedia.org/wiki/Linear_regression}}
					\item Logistic regression\footnote{``Logistic regression'', Wikipedia contributors, February 2, 2020, \url{https://en.wikipedia.org/wiki/Logistic_regression}}
					\item Support-vector machine\footnote{``Support-vector machine'', Wikipedia contributors, February 2, 2020, \url{https://en.wikipedia.org/wiki/Support-vector_machine}}
					\item Random forest\footnote{``Random forest'', Wikipedia contributors, February 2, 2020, \url{https://en.wikipedia.org/wiki/Random_forest}}
					\item etc.
				\end{itemize}
		
			% -------------------- %
			% Unsupervised learning %
			% -------------------- %
			\paragraph{Unsupervised learning}
			\label{sec:section_unsupervised_learning}
				\de{Beim unüberwachtem Lernen versucht man auch ohne gelabelte Daten an eine Kenntnis von Mustern zu erlangen. Angenommen man hätte mehrere Bilder von Burgern, Pizza und Donuts, welche sich unsortiert in einem Datenset befinden. Das unüberwachte Lernen versucht nun Ähnlichkeiten zu finden, um diese Bilder zu gruppieren. Im besten Fall erhält man am Ende drei unbenannte Gruppen \(A\), \(B\) und \(C\). Analysten werden sich diese Gruppen im Nachgang genauer anschauen und ein Fazit daraus ziehen, sofern dies möglich ist: Gruppe \(A\) sind Burger, Gruppe \(B\) sind Pizzen, etc.}
				\en{In unsupervised learning, one tries to gain knowledge of patterns even without labelled data. Suppose you have several pictures of burgers, pizza and donuts, which are unsorted in a data set. Unsupervised learning now tries to find similarities in order to cluster these images. In the best case you get three unnamed groups \(A\), \(B\) and \(C\) at the end. Analysts will take a closer look at these groups afterwards and draw a conclusion if possible: Group \(A\) is burgers, group \(B\) is pizzas, etc. }
				
				\de{\noindent Folgende unüberwachte Lernalgorithmen können zur Gruppierung verwendet werden:}
				\en{\noindent The following unsupervised learning algorithms can be used for clustering:}
				
				\begin{itemize}
					\item k-means\footnote{``k-means clustering'', Wikipedia contributors, February 2, 2020, \url{https://en.wikipedia.org/wiki/K-means_clustering}}
					\item Hierarchical clustering\footnote{``Hierarchical clustering'', Wikipedia contributors, February 2, 2020, \url{https://en.wikipedia.org/wiki/Hierarchical_clustering}}
					\item Expectation–maximization \footnote{``Expectation–maximization algorithm'', Wikipedia contributors, February 2, 2020, \url{Expectation–maximization algorithm}}
					\item etc.
				\end{itemize}
			
				\de{Eine Technik, die hierarchische Clusteranalyse, wird später verwendet um die Einführung von Hierarchien zu erleichtern. Für die allgemeine Analyse, den Finden von optimalen Parametern für das Lernen von Modellen, wird diese Art des Lernens in dieser Arbeit nicht verwendet.}
				\en{One technique, hierarchical clustering, is used later to facilitate the introduction of hierarchies. For the general analysis, the finding of optimal parameters for learning models, this kind of learning is not used in this thesis.}

			% -------------------- %
			% Reinforcement learning %
			% -------------------- %
			\paragraph{Reinforcement learning}
			\label{sec:section_reinforcement_learning}
				\de{Reinforcement learning\footnote{``Reinforcement learning'', Wikipedia contributors, January 31, 2020, \url{https://en.wikipedia.org/wiki/Reinforcement_learning}} ist eine Art des maschinellen Lernens, bei der ein Agent selbstständig die bestmögliche Strategie für die Erreichung eines Zieles erlernt. Für die Erreichung des Ziels sind Aktionen notwendig, welche zu bestimmten Zeitpunkten Belohnungen hervorbringen. Diese Belohnungen können auch negative sein (Bestrafung). Anhand dieser Belohnungen gilt es in Summe den bestmöglichen Belohnungswert zu erzielen. Bei der Klassifizierung von Bildern ist diese Art des Lernens nicht relevant, weshalb hier auch nicht weiter darauf eingegangen wird.}
				\en{Reinforcement learning\footnote{``Reinforcement learning'', Wikipedia contributors, January 31, 2020, \url{https://en.wikipedia.org/wiki/Reinforcement_learning}} is a type of machine learning in which an agent independently learns the best possible strategy for achieving a goal. To achieve the goal, actions are necessary which produce rewards at certain points in time. These rewards can also be negative (punishment). Based on these rewards, the aim is to achieve the best possible reward value. This type of learning is not relevant for the classification of images, which is why it will not be discussed further here.}



		% -------------------- %
		% Classification Metrics %
		% -------------------- %
		\subsubsection{Classification Metrics and confusion matrix}
			\de{Die Wahl der richtigen Metrik ist bei der Bewertung von Modellen des maschinellen Lernens von entscheidender Bedeutung. Metriken werden zur Überwachung und Messung der Leistung eines Modells während des Trainings und des Tests verwendet. Im nachfolgenden werden einige wichtige Metriken erklärt.}
			\en{Choosing the right metric is crucial in evaluating machine learning models. Metrics are used to monitor and measure the performance of a model during training and testing. Some important metrics are explained below.}

			% -------------------- %
			% Confusion Matrix %
			% -------------------- %
			\paragraph{Confusion Matrix}
			 	\de{Die Confusion Matrix ist eine spezielle quadratische Matrix auf dem Gebiet des maschinellen Lernens, das die Visualisierung der Leistung eines Vorhersage-Modells ermöglicht. Jede Zeile der Matrix repräsentiert die tatsächliche Klasse, während jede Spalte die Anzahl oder eine Zahlenangaben in Prozent der vorhergesagten Klasse angibt (oder umgekehrt)\footnote{``Confusion matrix'', Wikipedia contributors, February 5, 2020, \url{https://en.wikipedia.org/wiki/Confusion_matrix}}:}
			 	\en{The Confusion Matrix is a special quadratic matrix in the field of machine learning that allows the visualization of the performance of a predictive model. Each row of the matrix represents the actual class, while each column indicates the number or a numerical value as a percentage of the predicted class (or vice versa)\footnote{``Confusion matrix'', Wikipedia contributors, February 5, 2020, \url{https://en.wikipedia.org/wiki/Confusion_matrix}}:}

				\begin{table}[htb]
					\centering
					{\def\arraystretch{2}\tabcolsep=5pt
						\begin{tabular}{cc|c|c|c|c|}
							\cline{3-6}
							& & \multicolumn{4}{c|}{\textbf{predicted}} \\ \cline{3-6} 
							& & \boldmath\(class_1\) & \boldmath\(class_2\) & \textbf{\ldots} & \boldmath\(class_n\) \\ \hline
							\multicolumn{1}{|l|}{\multirow{4}{*}{\rotatebox{90}{\textbf{actual}}}} & \boldmath\(class_1\) & \(TP\)                  & \multicolumn{3}{c|}{\(FN\)} \\ \cline{2-6} 
							\multicolumn{1}{|c|}{} & \boldmath\(class_2\) & \multirow{3}{*}{\(FP\)} & \multicolumn{3}{c|}{\multirow{3}{*}{\(TN\)}} \\ \cline{2-2}
							\multicolumn{1}{|c|}{} & \textbf{\ldots} & & \multicolumn{3}{c|}{}                   \\ \cline{2-2}
							\multicolumn{1}{|c|}{} & \boldmath\(class_n\) & & \multicolumn{3}{c|}{}                   \\ \hline
						\end{tabular}
					}
					\captionof{table}{Confusion matrix}\label{tbl:table_confusion_matrix}
				\end{table}

			 	\de{Zum Schluß besitzt die Confusion Matrix folgenden Aufbau, wobei die Anzahl von Elementen in der Klasse \(C_{i,P}\) vorhergesagt wurde, obwohl (\(\boldsymbol{\cong}\)) es hätte Klasse \(C_{j,A}\) sein müssen:}
				\en{Finally, the Confusion Matrix has the following structure, where the number of elements in the class \(C_{i,P}\) was predicted although (\(\boldsymbol{\cong}\)) it should have been class \(C_{j,A}\):}

				\begin{equation}
					\boldsymbol{M}_{confusion} = \begin{bmatrix}
						\#C_{1,P}\boldsymbol{\cong} C_{1,A} & \dots & \#(C_{n,P}\boldsymbol{\cong} C_{1,A}) \\
						\vdots & \ddots & \vdots \\
						\#(C_{1,P}\boldsymbol{\cong} C_{n,A}) & \dots & \#(C_{n,P}\boldsymbol{\cong} C_{n,A})
					\end{bmatrix} = (a_{nn})
				\end{equation}

			% -------------------- %
			% Accuracy %
			% -------------------- %
			\paragraph{Accuracy}
				\de{Die \textbf{Top-1-Genauigkeit} ist wahrscheinlich die wichtigste Genauigkeit. Sie sagt sagt aus, zu wieviel Prozent die jeweils beste Aussage des Modells auf die Daten des Validierungssets mit der erwarteten Klasse übereinstimmt.}
				\en{\textbf{Top-1 accuracy} is probably the most important accuracy. It tells you the percentage of the model's best prediction of the data in the validation set that matches the expected class.}

				\begin{equation}
					Accuracy = {{TP + TN} \over {TP + TN + FP + FN}} = {{\sum_{i,j=1}^{n} a_{ij}} \over {\sum_{i=1}^{n} \sum_{j=1}^{n} a_{ij}}} = {{Correct_{all}} \over {CorrectPossible_{all}}}
				\end{equation}

				\de{Die \textbf{Top-5-Genauigkeit} ist eine weitere Genauigkeitsangabe. Jedoch wird hier nicht nur der beste Treffer einbezogen, sondern auch die nächsten weiteren vier. Sobald die richtige Klasse innerhalb der ersten fünf vorhergesagten Klassen gefunden werden kann, so ist auch diese Vorhersage wahr:}
				\en{The \textbf{Top 5 Accuracy} is another accuracy specification. However, not only the best hit is included here, but also the next four. As soon as the correct class can be found within the first five predicted classes, this prediction is also true:}
				
				\begin{equation}
					Accuracy_{top-5} = {{CorrectWithinTheBest5Classes_{all}} \over {CorrectPossible_{all}}}
				\end{equation}

				\de{Die Genauigkeit des gesamten Modells ist eine gute Aussagekraft über die Leistungsfähigkeit des Modells. Ein Problem tritt jedoch in Extremfällen auf, bei denen nicht mehr zuverlässig Annahmen gemacht werden können. Zum Beispiel wenn man es mit einem unbalancierten Datensatz zu tun hat\autocite{geron2017supervisedlearningConfusionMatrix}. Beispiel: Nehmen wir an, wir hätten ein Modell, dass immer die Klasse \(class_1\) vorhersagt. Die Klasse \(class_1\) besteht aus 9990 Elementen und von den anderen Klassen \(class_2\) bis \(class_n\) haben wir genau 10. Dann sieht die Confusion Matrix wie folgt aus:}
				\en{The accuracy of the entire model is a good indication of the performance of the model. However, a problem occurs in extreme cases where assumptions can no longer be made reliably. For example, if you are working with an unbalanced dataset\autocite{geron2017supervisedlearningConfusionMatrix}. Example: Suppose we have a model that always predicts the class \(class_1\). The class \(class_1\) consists of 9990 elements and of the other classes \(class_2\) to \(class_n\) we have exactly 10. Then the Confusion Matrix looks like this:}

				%\clearpage
				\begin{table}[htb]
					\centering
					{\def\arraystretch{2}\tabcolsep=5pt
						\begin{tabular}{cc|c|c|c|c|}
							\cline{3-6}
							& & \multicolumn{4}{c|}{\textbf{predicted}} \\ \cline{3-6} 
							& & \boldmath\(class_1\) & \boldmath\(class_2\) & \textbf{\ldots} & \boldmath\(class_n\) \\ \hline
							\multicolumn{1}{|l|}{\multirow{4}{*}{\rotatebox{90}{\textbf{actual}}}} & \boldmath\(class_1\) & \(TP = 9990\)                  & \multicolumn{3}{c|}{\(FN = 0\)} \\ \cline{2-6} 
							\multicolumn{1}{|c|}{} & \boldmath\(class_2\) & \multirow{3}{*}{\(FP = 10\)} & \multicolumn{3}{c|}{\multirow{3}{*}{\(TN = 0\)}} \\ \cline{2-2}
							\multicolumn{1}{|c|}{} & \textbf{\ldots} & & \multicolumn{3}{c|}{}                   \\ \cline{2-2}
							\multicolumn{1}{|c|}{} & \boldmath\(class_n\) & & \multicolumn{3}{c|}{}                   \\ \hline
						\end{tabular}
					}
					\captionof{table}{Confusion matrix}\label{tbl:table_confusion_matrix}
				\end{table}

				\de{Die Modell-Genauigkeit ist ein ist in diesem Fall 99,9\%, obwohl es ein schlechtes Modell ist:}
				\en{The model accuracy in this case is 99.9\%, although it is a bad model:}

				\begin{equation}
					Accuracy = 99,9\%
				\end{equation}

				\de{Deswegen gibt es weitere Performance Metriken wie Precision, Recall und F-Measure.}
				\en{Therefore there are additional performance metrics like Precision, Recall and F-Measure.}

			% -------------------- %
			% Precision %
			% -------------------- %
			\paragraph{Precision}
				\de{Precision\footnote{``Precision and recall'', Wikipedia contributors, February 7, 2020, \url{https://en.wikipedia.org/wiki/Precision_and_recall}} sagt aus, wie zuverlässig die Aussage einer Vorhersage einer Klasse ist:}
				\en{Precision\footnote{``Precision and recall'', Wikipedia contributors, February 7, 2020, \url{https://en.wikipedia.org/wiki/Precision_and_recall}} expresses how reliable the statement of a prediction of a class is:}

				\begin{equation}
					Precision = {{Correct} \over {Actual}} = {{TP} \over {TP + FP}}
				\end{equation}

				\de{Oder genauer für die Klasse c:}
				\en{Or more precisely for class c:}

				\begin{equation}
					Precision_{@c} = {{a_{cc}} \over {\sum_{i=1}^{n} a_{ic}}}
				\end{equation}

			% -------------------- %
			% Recall %
			% -------------------- %
			\paragraph{Recall}
				\de{Recall\footnote{``Precision and recall'', Wikipedia contributors, February 7, 2020, \url{https://en.wikipedia.org/wiki/Precision_and_recall}} ist die Genauigkeit einer Klasse. Das bedeutet wie gut konnte die Klasse vorhergesehen werden:}
				\en{Recall\footnote{``Precision and recall'', Wikipedia contributors, February 7, 2020, \url{https://en.wikipedia.org/wiki/Precision_and_recall}} is the accuracy of a class. This means how well the class could be predicted:}

				\begin{equation}
					Recall = {{Correct} \over {CorrectPossible}} = {{TP} \over {TP + FN}}
				\end{equation}

				\de{Oder genauer für die Klasse c:}
				\en{Or more precisely for class c:}

				\begin{equation}
					Recall_{@c} = {{a_{cc}} \over {\sum_{i=1}^{n} a_{ci}}}
				\end{equation}

			% -------------------- %
			% F-Measure %
			% -------------------- %
			\paragraph{F-Measure}
				\de{F-Measure\footnote{``F1 score'', Wikipedia contributors, February 7, 2020, \url{https://en.wikipedia.org/wiki/F1_score}} kombiniert Präzision und Rückruf, wobei der Parameter \(\beta\) die Gewichtung darstellt:}
				\en{F-Measure\footnote{``F1 score'', Wikipedia contributors, February 7, 2020, \url{https://en.wikipedia.org/wiki/F1_score}} combines precision and recall, with the parameter \(\beta\) representing the weighting:}
				\begin{equation}
					F_\beta =
					(1 + \beta^2) \cdot {{Precision\cdot Recall}\over{\beta^2 \cdot Precision + Recall}} = 
					{{(1 + \beta^2) \cdot TP}\over{(1 + \beta^2) \cdot TP + \beta^2 \cdot FN + FP}}
				\end{equation}

				\de{Je höher der \(\beta\), desto mehr Wert wird auf Precision statt auf Recall gelegt. Das ist wichtig, wenn man mehr Wert auf die Qualität der Verhersage legt, als auf die Erkennungsgenauigkeit. Z.B. bei einer Vorhersage von Krankheiten: Zuordnung von Klasse Krank bei gesunde Menschen ist hier genauso fatal wie auch die Zuordnung von Klasse Gesund bei kranken Menschen (Obwohl Fall zwei fataler wäre als Fall eins). Mit einem Wert von \(\beta = 0,5\) erhalten wir eine Gleichverteilung beider Werte und wird F1 score genannt:}
				\en{The higher \(\beta\) the more importance is placed on precision instead of recall. This is important if you put more importance on the quality of the prediction than on the accuracy of prediction. For example, when predicting diseases: assigning class ``ill'' to healthy people is just as fatal as assigning class ``healthy'' to sick people (although case two would be even more fatal than case one). With a beta value of \(\beta = 0,5\) we get an equal distribution of both values and is called F1 score:}

				\begin{equation}
					F_1 =
					{{2 \cdot Precision \cdot Recall}\over{Precision + Recall}} = 
					{{2 \cdot TP}\over{2 \cdot TP + FN + FP}}
				\end{equation}

			% -------------------- %
			% loss function %
			% -------------------- %
			\paragraph{Loss function}
				\de{Da es sich hierbei um ein Klassifzierungsproblem und nicht um ein Regressionsproblem handelt, greift man bei der Vorhersage auf die Softmax-Regression zurück. Dabei wird nach jeder Vorhersage ein Vektor \(\lambda\) der Größe \(n\) zurückgegeben, wobei \(n\) der Anzahl der zu unterscheidenden Klassen entspricht\autocite{geron2017supervisedlearningDecisionBoundaries}. Jeder einzelne \(\hat{p}\) Wert entspricht dabei der Wahrscheinlichkeit, dass es die Klasse \(class_n\) ist:}
				\en{Since this is a classification problem and not a regression problem, the prediction is based on softmax regression. After each prediction, a vector \(\lambda\) of the size \(n\) is returned, where \(n\) corresponds to the number of classes to be distinguished\autocite{geron2017supervisedlearningDecisionBoundaries}. Each individual \(\hat{p}\) value corresponds to the probability that it is class \(class_n\):}

				\begin{equation}
					\lambda = 
					\left(
						\begin{array}{c}
						\hat{p}_{1}\\
						\hat{p}_{2}\\
						\vdots\\
						\hat{p}_{n}\\
						\end{array}
					\right)
					\quad\Biggl\lvert \quad \sum_{i=1}^n \hat{p_i} = 1
				\end{equation}

				\de{Der erwartete Wert der Parameterfunktion und somit der aktuellen Klasse wird als One Hot Vektor zurückgegeben. Der Wert 1 entspricht der erwarteten Klasse. Alle anderen Klassen geben 0 zurück. Dies nennt man auch One Hot Encoding:}
				\en{The expected value of the parameter function and thus of the current class is returned as a one hot vector. The value 1 corresponds to the expected class. All other classes return 0. This is also called one hot encoding:}

				\begin{equation}
					g(\vartheta_2) = 
					\left(
						\begin{array}{c}
						0\\
						1\\
						\vdots\\
						0\\
						\end{array}
					\right)
				\end{equation}

				\de{Die Verlustfunktion\footnote{``Verlustfunktion (Statistik)'', Wikipedia contributors, February 7, 2020, \url{https://de.wikipedia.org/wiki/Verlustfunktion_(Statistik)}} ordnet jeder Vorhersage einen Schaden zu, der durch den Vergleich mit dem wahren Wert bzw. Parameter entsteht. Dazu wird der Abstand der vorhergesagten Klasse zur wahren Klasse berechnet (sind sie gleich ist der Abstand 0). Bessert sich mit Anpassung des Modells (Lernvorgang) die Vorhersage aller vorhergesagten Klassen zu ihren wahren Klassen, so verkleinert sich auch der Wert der Verlustfunktion. Eine typische Verlustfunktion ist z.B. für einen r-dimensionalen Raum:}
				\en{The loss function\footnote{``Verlustfunktion (Statistik)'', Wikipedia contributors, February 7, 2020, \url{https://de.wikipedia.org/wiki/Verlustfunktion_(Statistik)}} assigns a loss to each prediction, which results from the comparison with the true value or parameter. For this purpose, the distance between the predicted class and the true class is calculated (if they are equal, the distance is 0).  If the adaptation of the model (learning process) improves the prediction of all predicted classes to their true classes, the value of the loss function is also reduced.  A typical loss function is e.g. for an r-dimensional space:}

				\begin{equation}
					L_r(\vartheta, \lambda) := \lVert \lambda - g(\vartheta) \rVert^r
				\end{equation}

				\de{\(\lambda\) stellt hierbei den geschätzte Wert und \(g(\vartheta)\) die Parameterfunktion dar, welche den realen Wert für \(\vartheta\) zurückliefert. Der mittlere Verlust auf den gesamten Datensatz mit \(n\) Elementen beträgt somit:}
				\en{\(\lambda\) represents the estimated value and \(g(\vartheta)\) the parameter function which returns the actual value for \(\vartheta\). The average loss on the entire data set with \(n\) elements is thus:}

				\begin{equation}
					\hat{L} = {{1}\over{k}}\sum_{i=1}^{k} L_r(\vartheta_i, \lambda_i)
				\end{equation}

		\subsection{Machine Learning}
		\label{sec:section_machine_learning}
			\de{Maschinelles Lernen ist ein Oberbegriff für die künstliche Generierung von Wissen aus Erfahrung. Es verfolgt den Ansatz des induktiven Lernens (siehe auch Kapitel ``induktiver Ansatz''\hyperref[sec:section_inductive_approach]).}
			\en{Machine learning is a generic term for the artificial generation of knowledge from experience. It follows the approach of inductive learning (see also chapter ``inductive approach''\hyperref[sec:section_inductive_approach]).}

			\subsubsection{Artificial neural network}

				\de{Künstliche neuronale Netzwerke stellen Funktionen bereit, welche in der Lage sind hochkomplexe Daten im mehrdimensionalem Raum zu trennen. Bei großen und hochgradig komplexen Aufgaben, wie beispielsweise der Klassifizierung von Milliarden von Bildern, Sprach- und Texterkennungen, schneiden neuronale Netze meist besser ab, als andere Machine Learning Verfahren. Der erhebliche Zuwachs an Rechenkapizität seit den 1990ern ermöglicht das Trainieren großer neuronaler Netzwerke innerhalb eines sinnvollen Zeitraumes. Künstliche neuronale Netzwerke sind die Kernkomponente des Deep Learnings.}
				\en{Artificial neural networks provide functions that are able to separate highly complex data in multidimensional space. For large and highly complex tasks, such as the classification of billions of images, speech and text recognition, neural networks usually perform better than other machine learning methods. The significant increase in computational capacity since the 1990s allows the training of large neural networks within a reasonable period of time. Artificial neural networks are the core component of deep learning.}

				\de{Neuronale Netzwerke verarbeiten einen Eingabevektor und wandeln ihn in einen neuen Ausgabevektor um. Sie sind Netze aus vielen hintereinander und parallel geschalteten künstlichen Neuronen. Ein künstliches Neuron wiederrum wandelt einen Vektor in ein Skalar um, indem es die Eingänge \(\bar{x}\) mit den veränderlichen Parametern \(\bar{\omega}\) skaliert, aufsummiert und mit einem Bias \(b\) korrigiert (der Bias ist ebenso eine veränderliche Variable). Die Aktivierungsfunktion stellt sicher, dass aus dem Polynom ersten Grades (lineares Regressionsmodell) eine nichtlineare Funktion wird\footnote{``Aktivierungsfunktionen, ihre Arten und Verwendungsmöglichkeiten'', https://www.ai-united.de/, February 8, 2020, \url{https://www.ai-united.de/aktivierungsfunktionen-ihre-arten-und-verwendungsmoeglichkeiten/}}:}
				\en{Neural networks process an input vector \(\bar{x}\) and convert it into a new output vector \(\hat{\bar{x}}\). They are networks of many artificial neurons connected in series and parallel. An artificial neuron in turn converts a vector into a scalar by scaling and summing the inputs \(\bar{x}\) with the changeable parameters \(\bar{\omega}\) and correcting them with a bias \(b\) (the bias is also a changeable variable). The activation function ensures that the first degree polynomial (linear regression model) becomes a nonlinear function\footnote{``Activation functions, their types and uses'', https://www.ai-united.de/, February 8, 2020, \url{https://www.ai-united.de/aktivierungsfunktionen-ihre-arten-und-verwendungsmoeglichkeiten/}}:}

				\begin{figure}[H]
					\[
						\begin{array}{c}
							x_1\\
							x_2\\
							\color{white}\vdots\\
							x_n\\
							1\\
						\end{array}
						\begin{array}{c}
							\rightarrow\\
							\rightarrow\\
							\vdots\\
							\rightarrow\\
							\rightarrow\\
						\end{array}
						\begin{array}{c}
							\omega_1\\
							\omega_2\\
							\color{white}\vdots\\
							\omega_n\\
							b\\
						\end{array}
						\begin{array}{c}
							\diagdown\\
							\\
							\\
							\\
							\diagup\\
						\end{array}
						\circled{${\sum\atop\ }\over{\ \atop step(z)}$}
						\longrightarrow h
						\quad\Biggl\lvert \quad h = step(z) = step(\bar{\omega}^\intercal \cdot \bar{x} + b)
					\]
					\caption{The construction of an artificial neuron.}
				\end{figure}

				\de{Das künstliche neuronale Netz baut sich aus vielen hintereinandergeschalteten Layern zusammen, welche wiederrum parallel geschaltete Neuronen enthalten:}
				\en{The artificial neural network is composed of many layers connected in series, which again contain neurons connected in parallel:}

				\begin{figure}[H]
					\centering

					\tikzset{%
						every neuron/.style={
							circle,
							draw
					  	},
					  	every inputneuron/.style={
					  		circle,
					  		draw,
					  		minimum size=0.25cm
					  	},
					  	neuron 1/.style={
							circle,
							draw,
							minimum size=0.75cm,
							execute at begin node=\color{black}${\sum\atop\ }\over{\ \atop step(z)}$
					  	},
					  	neuron 2/.style={
							circle,
							draw,
							minimum size=0.75cm,
							execute at begin node=\color{black}${\sum\atop\ }\over{\ \atop step(z)}$
					  	},
					  	neuron missing/.style={
					  		draw=none,
					  		scale=2,
					  		text height=0.333cm,
					  		execute at begin node=\color{black}$\vdots$
					  	},
					  	inputneuron missing/.style={
					  		draw=none, 
					  		scale=2,
					  		text height=0.333cm,
					  		execute at begin node=\color{black}$\vdots$
						},
					}
					
					\begin{tikzpicture}[x=1.5cm, y=1.5cm, >=stealth]
						% print input neurons
						\foreach \m/\l [count=\y] in {1,2,3,missing,4}
							\node [every inputneuron/.try, inputneuron \m/.try] (input-\m) at (0,2.5-\y) {};
						
						% print hidden neurons
						\foreach \m [count=\y] in {1,missing,2}
							\node [every neuron/.try, neuron \m/.try ] (hidden-\m) at (2,2-\y*1.25) {};
						
						% print output neurons
						\foreach \m [count=\y] in {1,missing,2}
							\node [every neuron/.try, neuron \m/.try ] (output-\m) at (4,1.5-\y) {};
						
						% x_n
						\foreach \l [count=\i] in {1,2,3,n}
							\draw [<-] (input-\i) -- ++(-1,0)
								node [above, midway] {$x_\l$};
						
						% h_n
						\foreach \l [count=\i] in {1,n}
							\node [above] at (hidden-\i.north) {$h_\l$};
						
						% ^x_n
						\foreach \l [count=\i] in {1,n}
						  \draw [->] (output-\i) -- ++(1,0)
						    node [above, midway] {$\hat{x}_\l$};
						
						% lines between input and hidden layer
						\foreach \i in {1,...,4}
						  \foreach \j in {1,...,2}
						    \draw [->] (input-\i) -- (hidden-\j);
						
						% lines between hidden and output layer
						\foreach \i in {1,...,2}
						  \foreach \j in {1,...,2}
						    \draw [->] (hidden-\i) -- (output-\j);
						
						% print labels
						\foreach \l [count=\x from 0] in {Input, Hidden, Ouput}
						  \node [align=center, above] at (\x*2,2) {\l \\ layer};
					\end{tikzpicture}
					\caption{The construction of a simple neural network.}
				\end{figure}
	
				\de{Ein neuronales Netzwerk ist in der Lage komplexe Eingaben zu klassifizieren. Doch wie genau kann man sich das vorstellen? Betrachten wir eine einfache Klassifikationsfunktion, wobei \(\bar{x}\) die Koordinaten der einzelnen Klassenpunkte darstellen und \(\bar{w}\) und \(b\) lernbare Parameter sind:}
				\en{A neural network is able to classify complex inputs. But how exactly can this be imagined? Let us consider a simple classification function, where \(\bar{x}\) represent the coordinates of the individual class points and \(\bar{w}\) and \(b\) are learnable parameters:}
	
				\begin{equation}
					f(\bar{x}, \bar{\omega}, b) = sgn(\bar{\omega}^\intercal \cdot \bar{x} + b)
				\end{equation}

				\de{Mit dieser linearen Funktion lässt sich das nachfolgende Problem leicht klassifizieren:}
				\en{With this linear function the following problem can be easily classified:}

				\begin{figure}[H]
					\begin{center}
						\scalebox{0.5}{%% Creator: Matplotlib, PGF backend
%%
%% To include the figure in your LaTeX document, write
%%   \input{<filename>.pgf}
%%
%% Make sure the required packages are loaded in your preamble
%%   \usepackage{pgf}
%%
%% Figures using additional raster images can only be included by \input if
%% they are in the same directory as the main LaTeX file. For loading figures
%% from other directories you can use the `import` package
%%   \usepackage{import}
%% and then include the figures with
%%   \import{<path to file>}{<filename>.pgf}
%%
%% Matplotlib used the following preamble
%%   \usepackage{fontspec}
%%   \setmainfont{DejaVuSerif.ttf}[Path=C:/Users/bjoern/Anaconda3/envs/keras-gpu/lib/site-packages/matplotlib/mpl-data/fonts/ttf/]
%%   \setsansfont{DejaVuSans.ttf}[Path=C:/Users/bjoern/Anaconda3/envs/keras-gpu/lib/site-packages/matplotlib/mpl-data/fonts/ttf/]
%%   \setmonofont{DejaVuSansMono.ttf}[Path=C:/Users/bjoern/Anaconda3/envs/keras-gpu/lib/site-packages/matplotlib/mpl-data/fonts/ttf/]
%%
\begingroup%
\makeatletter%
\begin{pgfpicture}%
\pgfpathrectangle{\pgfpointorigin}{\pgfqpoint{6.400000in}{4.800000in}}%
\pgfusepath{use as bounding box, clip}%
\begin{pgfscope}%
\pgfsetbuttcap%
\pgfsetmiterjoin%
\definecolor{currentfill}{rgb}{1.000000,1.000000,1.000000}%
\pgfsetfillcolor{currentfill}%
\pgfsetlinewidth{0.000000pt}%
\definecolor{currentstroke}{rgb}{1.000000,1.000000,1.000000}%
\pgfsetstrokecolor{currentstroke}%
\pgfsetdash{}{0pt}%
\pgfpathmoveto{\pgfqpoint{0.000000in}{0.000000in}}%
\pgfpathlineto{\pgfqpoint{6.400000in}{0.000000in}}%
\pgfpathlineto{\pgfqpoint{6.400000in}{4.800000in}}%
\pgfpathlineto{\pgfqpoint{0.000000in}{4.800000in}}%
\pgfpathclose%
\pgfusepath{fill}%
\end{pgfscope}%
\begin{pgfscope}%
\pgfsetbuttcap%
\pgfsetmiterjoin%
\definecolor{currentfill}{rgb}{1.000000,1.000000,1.000000}%
\pgfsetfillcolor{currentfill}%
\pgfsetlinewidth{0.000000pt}%
\definecolor{currentstroke}{rgb}{0.000000,0.000000,0.000000}%
\pgfsetstrokecolor{currentstroke}%
\pgfsetstrokeopacity{0.000000}%
\pgfsetdash{}{0pt}%
\pgfpathmoveto{\pgfqpoint{0.800000in}{0.528000in}}%
\pgfpathlineto{\pgfqpoint{5.760000in}{0.528000in}}%
\pgfpathlineto{\pgfqpoint{5.760000in}{4.224000in}}%
\pgfpathlineto{\pgfqpoint{0.800000in}{4.224000in}}%
\pgfpathclose%
\pgfusepath{fill}%
\end{pgfscope}%
\begin{pgfscope}%
\pgfsetbuttcap%
\pgfsetroundjoin%
\definecolor{currentfill}{rgb}{0.000000,0.000000,0.000000}%
\pgfsetfillcolor{currentfill}%
\pgfsetlinewidth{0.803000pt}%
\definecolor{currentstroke}{rgb}{0.000000,0.000000,0.000000}%
\pgfsetstrokecolor{currentstroke}%
\pgfsetdash{}{0pt}%
\pgfsys@defobject{currentmarker}{\pgfqpoint{0.000000in}{0.000000in}}{\pgfqpoint{0.000000in}{0.048611in}}{%
\pgfpathmoveto{\pgfqpoint{0.000000in}{0.000000in}}%
\pgfpathlineto{\pgfqpoint{0.000000in}{0.048611in}}%
\pgfusepath{stroke,fill}%
}%
\begin{pgfscope}%
\pgfsys@transformshift{0.800000in}{0.528000in}%
\pgfsys@useobject{currentmarker}{}%
\end{pgfscope}%
\end{pgfscope}%
\begin{pgfscope}%
\pgfsetbuttcap%
\pgfsetroundjoin%
\definecolor{currentfill}{rgb}{0.000000,0.000000,0.000000}%
\pgfsetfillcolor{currentfill}%
\pgfsetlinewidth{0.803000pt}%
\definecolor{currentstroke}{rgb}{0.000000,0.000000,0.000000}%
\pgfsetstrokecolor{currentstroke}%
\pgfsetdash{}{0pt}%
\pgfsys@defobject{currentmarker}{\pgfqpoint{0.000000in}{-0.048611in}}{\pgfqpoint{0.000000in}{0.000000in}}{%
\pgfpathmoveto{\pgfqpoint{0.000000in}{0.000000in}}%
\pgfpathlineto{\pgfqpoint{0.000000in}{-0.048611in}}%
\pgfusepath{stroke,fill}%
}%
\begin{pgfscope}%
\pgfsys@transformshift{0.800000in}{4.224000in}%
\pgfsys@useobject{currentmarker}{}%
\end{pgfscope}%
\end{pgfscope}%
\begin{pgfscope}%
\definecolor{textcolor}{rgb}{0.000000,0.000000,0.000000}%
\pgfsetstrokecolor{textcolor}%
\pgfsetfillcolor{textcolor}%
\pgftext[x=0.800000in,y=0.479389in,,top]{\color{textcolor}\sffamily\fontsize{12.000000}{14.400000}\selectfont −1.0}%
\end{pgfscope}%
\begin{pgfscope}%
\pgfsetbuttcap%
\pgfsetroundjoin%
\definecolor{currentfill}{rgb}{0.000000,0.000000,0.000000}%
\pgfsetfillcolor{currentfill}%
\pgfsetlinewidth{0.803000pt}%
\definecolor{currentstroke}{rgb}{0.000000,0.000000,0.000000}%
\pgfsetstrokecolor{currentstroke}%
\pgfsetdash{}{0pt}%
\pgfsys@defobject{currentmarker}{\pgfqpoint{0.000000in}{0.000000in}}{\pgfqpoint{0.000000in}{0.048611in}}{%
\pgfpathmoveto{\pgfqpoint{0.000000in}{0.000000in}}%
\pgfpathlineto{\pgfqpoint{0.000000in}{0.048611in}}%
\pgfusepath{stroke,fill}%
}%
\begin{pgfscope}%
\pgfsys@transformshift{2.040000in}{0.528000in}%
\pgfsys@useobject{currentmarker}{}%
\end{pgfscope}%
\end{pgfscope}%
\begin{pgfscope}%
\pgfsetbuttcap%
\pgfsetroundjoin%
\definecolor{currentfill}{rgb}{0.000000,0.000000,0.000000}%
\pgfsetfillcolor{currentfill}%
\pgfsetlinewidth{0.803000pt}%
\definecolor{currentstroke}{rgb}{0.000000,0.000000,0.000000}%
\pgfsetstrokecolor{currentstroke}%
\pgfsetdash{}{0pt}%
\pgfsys@defobject{currentmarker}{\pgfqpoint{0.000000in}{-0.048611in}}{\pgfqpoint{0.000000in}{0.000000in}}{%
\pgfpathmoveto{\pgfqpoint{0.000000in}{0.000000in}}%
\pgfpathlineto{\pgfqpoint{0.000000in}{-0.048611in}}%
\pgfusepath{stroke,fill}%
}%
\begin{pgfscope}%
\pgfsys@transformshift{2.040000in}{4.224000in}%
\pgfsys@useobject{currentmarker}{}%
\end{pgfscope}%
\end{pgfscope}%
\begin{pgfscope}%
\definecolor{textcolor}{rgb}{0.000000,0.000000,0.000000}%
\pgfsetstrokecolor{textcolor}%
\pgfsetfillcolor{textcolor}%
\pgftext[x=2.040000in,y=0.479389in,,top]{\color{textcolor}\sffamily\fontsize{12.000000}{14.400000}\selectfont −0.5}%
\end{pgfscope}%
\begin{pgfscope}%
\pgfsetbuttcap%
\pgfsetroundjoin%
\definecolor{currentfill}{rgb}{0.000000,0.000000,0.000000}%
\pgfsetfillcolor{currentfill}%
\pgfsetlinewidth{0.803000pt}%
\definecolor{currentstroke}{rgb}{0.000000,0.000000,0.000000}%
\pgfsetstrokecolor{currentstroke}%
\pgfsetdash{}{0pt}%
\pgfsys@defobject{currentmarker}{\pgfqpoint{0.000000in}{0.000000in}}{\pgfqpoint{0.000000in}{0.048611in}}{%
\pgfpathmoveto{\pgfqpoint{0.000000in}{0.000000in}}%
\pgfpathlineto{\pgfqpoint{0.000000in}{0.048611in}}%
\pgfusepath{stroke,fill}%
}%
\begin{pgfscope}%
\pgfsys@transformshift{3.280000in}{0.528000in}%
\pgfsys@useobject{currentmarker}{}%
\end{pgfscope}%
\end{pgfscope}%
\begin{pgfscope}%
\pgfsetbuttcap%
\pgfsetroundjoin%
\definecolor{currentfill}{rgb}{0.000000,0.000000,0.000000}%
\pgfsetfillcolor{currentfill}%
\pgfsetlinewidth{0.803000pt}%
\definecolor{currentstroke}{rgb}{0.000000,0.000000,0.000000}%
\pgfsetstrokecolor{currentstroke}%
\pgfsetdash{}{0pt}%
\pgfsys@defobject{currentmarker}{\pgfqpoint{0.000000in}{-0.048611in}}{\pgfqpoint{0.000000in}{0.000000in}}{%
\pgfpathmoveto{\pgfqpoint{0.000000in}{0.000000in}}%
\pgfpathlineto{\pgfqpoint{0.000000in}{-0.048611in}}%
\pgfusepath{stroke,fill}%
}%
\begin{pgfscope}%
\pgfsys@transformshift{3.280000in}{4.224000in}%
\pgfsys@useobject{currentmarker}{}%
\end{pgfscope}%
\end{pgfscope}%
\begin{pgfscope}%
\definecolor{textcolor}{rgb}{0.000000,0.000000,0.000000}%
\pgfsetstrokecolor{textcolor}%
\pgfsetfillcolor{textcolor}%
\pgftext[x=3.280000in,y=0.479389in,,top]{\color{textcolor}\sffamily\fontsize{12.000000}{14.400000}\selectfont 0.0}%
\end{pgfscope}%
\begin{pgfscope}%
\pgfsetbuttcap%
\pgfsetroundjoin%
\definecolor{currentfill}{rgb}{0.000000,0.000000,0.000000}%
\pgfsetfillcolor{currentfill}%
\pgfsetlinewidth{0.803000pt}%
\definecolor{currentstroke}{rgb}{0.000000,0.000000,0.000000}%
\pgfsetstrokecolor{currentstroke}%
\pgfsetdash{}{0pt}%
\pgfsys@defobject{currentmarker}{\pgfqpoint{0.000000in}{0.000000in}}{\pgfqpoint{0.000000in}{0.048611in}}{%
\pgfpathmoveto{\pgfqpoint{0.000000in}{0.000000in}}%
\pgfpathlineto{\pgfqpoint{0.000000in}{0.048611in}}%
\pgfusepath{stroke,fill}%
}%
\begin{pgfscope}%
\pgfsys@transformshift{4.520000in}{0.528000in}%
\pgfsys@useobject{currentmarker}{}%
\end{pgfscope}%
\end{pgfscope}%
\begin{pgfscope}%
\pgfsetbuttcap%
\pgfsetroundjoin%
\definecolor{currentfill}{rgb}{0.000000,0.000000,0.000000}%
\pgfsetfillcolor{currentfill}%
\pgfsetlinewidth{0.803000pt}%
\definecolor{currentstroke}{rgb}{0.000000,0.000000,0.000000}%
\pgfsetstrokecolor{currentstroke}%
\pgfsetdash{}{0pt}%
\pgfsys@defobject{currentmarker}{\pgfqpoint{0.000000in}{-0.048611in}}{\pgfqpoint{0.000000in}{0.000000in}}{%
\pgfpathmoveto{\pgfqpoint{0.000000in}{0.000000in}}%
\pgfpathlineto{\pgfqpoint{0.000000in}{-0.048611in}}%
\pgfusepath{stroke,fill}%
}%
\begin{pgfscope}%
\pgfsys@transformshift{4.520000in}{4.224000in}%
\pgfsys@useobject{currentmarker}{}%
\end{pgfscope}%
\end{pgfscope}%
\begin{pgfscope}%
\definecolor{textcolor}{rgb}{0.000000,0.000000,0.000000}%
\pgfsetstrokecolor{textcolor}%
\pgfsetfillcolor{textcolor}%
\pgftext[x=4.520000in,y=0.479389in,,top]{\color{textcolor}\sffamily\fontsize{12.000000}{14.400000}\selectfont 0.5}%
\end{pgfscope}%
\begin{pgfscope}%
\pgfsetbuttcap%
\pgfsetroundjoin%
\definecolor{currentfill}{rgb}{0.000000,0.000000,0.000000}%
\pgfsetfillcolor{currentfill}%
\pgfsetlinewidth{0.803000pt}%
\definecolor{currentstroke}{rgb}{0.000000,0.000000,0.000000}%
\pgfsetstrokecolor{currentstroke}%
\pgfsetdash{}{0pt}%
\pgfsys@defobject{currentmarker}{\pgfqpoint{0.000000in}{0.000000in}}{\pgfqpoint{0.048611in}{0.000000in}}{%
\pgfpathmoveto{\pgfqpoint{0.000000in}{0.000000in}}%
\pgfpathlineto{\pgfqpoint{0.048611in}{0.000000in}}%
\pgfusepath{stroke,fill}%
}%
\begin{pgfscope}%
\pgfsys@transformshift{0.800000in}{0.528000in}%
\pgfsys@useobject{currentmarker}{}%
\end{pgfscope}%
\end{pgfscope}%
\begin{pgfscope}%
\pgfsetbuttcap%
\pgfsetroundjoin%
\definecolor{currentfill}{rgb}{0.000000,0.000000,0.000000}%
\pgfsetfillcolor{currentfill}%
\pgfsetlinewidth{0.803000pt}%
\definecolor{currentstroke}{rgb}{0.000000,0.000000,0.000000}%
\pgfsetstrokecolor{currentstroke}%
\pgfsetdash{}{0pt}%
\pgfsys@defobject{currentmarker}{\pgfqpoint{-0.048611in}{0.000000in}}{\pgfqpoint{0.000000in}{0.000000in}}{%
\pgfpathmoveto{\pgfqpoint{0.000000in}{0.000000in}}%
\pgfpathlineto{\pgfqpoint{-0.048611in}{0.000000in}}%
\pgfusepath{stroke,fill}%
}%
\begin{pgfscope}%
\pgfsys@transformshift{5.760000in}{0.528000in}%
\pgfsys@useobject{currentmarker}{}%
\end{pgfscope}%
\end{pgfscope}%
\begin{pgfscope}%
\definecolor{textcolor}{rgb}{0.000000,0.000000,0.000000}%
\pgfsetstrokecolor{textcolor}%
\pgfsetfillcolor{textcolor}%
\pgftext[x=0.346685in,y=0.464686in,left,base]{\color{textcolor}\sffamily\fontsize{12.000000}{14.400000}\selectfont −1.0}%
\end{pgfscope}%
\begin{pgfscope}%
\pgfsetbuttcap%
\pgfsetroundjoin%
\definecolor{currentfill}{rgb}{0.000000,0.000000,0.000000}%
\pgfsetfillcolor{currentfill}%
\pgfsetlinewidth{0.803000pt}%
\definecolor{currentstroke}{rgb}{0.000000,0.000000,0.000000}%
\pgfsetstrokecolor{currentstroke}%
\pgfsetdash{}{0pt}%
\pgfsys@defobject{currentmarker}{\pgfqpoint{0.000000in}{0.000000in}}{\pgfqpoint{0.048611in}{0.000000in}}{%
\pgfpathmoveto{\pgfqpoint{0.000000in}{0.000000in}}%
\pgfpathlineto{\pgfqpoint{0.048611in}{0.000000in}}%
\pgfusepath{stroke,fill}%
}%
\begin{pgfscope}%
\pgfsys@transformshift{0.800000in}{1.452000in}%
\pgfsys@useobject{currentmarker}{}%
\end{pgfscope}%
\end{pgfscope}%
\begin{pgfscope}%
\pgfsetbuttcap%
\pgfsetroundjoin%
\definecolor{currentfill}{rgb}{0.000000,0.000000,0.000000}%
\pgfsetfillcolor{currentfill}%
\pgfsetlinewidth{0.803000pt}%
\definecolor{currentstroke}{rgb}{0.000000,0.000000,0.000000}%
\pgfsetstrokecolor{currentstroke}%
\pgfsetdash{}{0pt}%
\pgfsys@defobject{currentmarker}{\pgfqpoint{-0.048611in}{0.000000in}}{\pgfqpoint{0.000000in}{0.000000in}}{%
\pgfpathmoveto{\pgfqpoint{0.000000in}{0.000000in}}%
\pgfpathlineto{\pgfqpoint{-0.048611in}{0.000000in}}%
\pgfusepath{stroke,fill}%
}%
\begin{pgfscope}%
\pgfsys@transformshift{5.760000in}{1.452000in}%
\pgfsys@useobject{currentmarker}{}%
\end{pgfscope}%
\end{pgfscope}%
\begin{pgfscope}%
\definecolor{textcolor}{rgb}{0.000000,0.000000,0.000000}%
\pgfsetstrokecolor{textcolor}%
\pgfsetfillcolor{textcolor}%
\pgftext[x=0.346685in,y=1.388686in,left,base]{\color{textcolor}\sffamily\fontsize{12.000000}{14.400000}\selectfont −0.5}%
\end{pgfscope}%
\begin{pgfscope}%
\pgfsetbuttcap%
\pgfsetroundjoin%
\definecolor{currentfill}{rgb}{0.000000,0.000000,0.000000}%
\pgfsetfillcolor{currentfill}%
\pgfsetlinewidth{0.803000pt}%
\definecolor{currentstroke}{rgb}{0.000000,0.000000,0.000000}%
\pgfsetstrokecolor{currentstroke}%
\pgfsetdash{}{0pt}%
\pgfsys@defobject{currentmarker}{\pgfqpoint{0.000000in}{0.000000in}}{\pgfqpoint{0.048611in}{0.000000in}}{%
\pgfpathmoveto{\pgfqpoint{0.000000in}{0.000000in}}%
\pgfpathlineto{\pgfqpoint{0.048611in}{0.000000in}}%
\pgfusepath{stroke,fill}%
}%
\begin{pgfscope}%
\pgfsys@transformshift{0.800000in}{2.376000in}%
\pgfsys@useobject{currentmarker}{}%
\end{pgfscope}%
\end{pgfscope}%
\begin{pgfscope}%
\pgfsetbuttcap%
\pgfsetroundjoin%
\definecolor{currentfill}{rgb}{0.000000,0.000000,0.000000}%
\pgfsetfillcolor{currentfill}%
\pgfsetlinewidth{0.803000pt}%
\definecolor{currentstroke}{rgb}{0.000000,0.000000,0.000000}%
\pgfsetstrokecolor{currentstroke}%
\pgfsetdash{}{0pt}%
\pgfsys@defobject{currentmarker}{\pgfqpoint{-0.048611in}{0.000000in}}{\pgfqpoint{0.000000in}{0.000000in}}{%
\pgfpathmoveto{\pgfqpoint{0.000000in}{0.000000in}}%
\pgfpathlineto{\pgfqpoint{-0.048611in}{0.000000in}}%
\pgfusepath{stroke,fill}%
}%
\begin{pgfscope}%
\pgfsys@transformshift{5.760000in}{2.376000in}%
\pgfsys@useobject{currentmarker}{}%
\end{pgfscope}%
\end{pgfscope}%
\begin{pgfscope}%
\definecolor{textcolor}{rgb}{0.000000,0.000000,0.000000}%
\pgfsetstrokecolor{textcolor}%
\pgfsetfillcolor{textcolor}%
\pgftext[x=0.486334in,y=2.312686in,left,base]{\color{textcolor}\sffamily\fontsize{12.000000}{14.400000}\selectfont 0.0}%
\end{pgfscope}%
\begin{pgfscope}%
\pgfsetbuttcap%
\pgfsetroundjoin%
\definecolor{currentfill}{rgb}{0.000000,0.000000,0.000000}%
\pgfsetfillcolor{currentfill}%
\pgfsetlinewidth{0.803000pt}%
\definecolor{currentstroke}{rgb}{0.000000,0.000000,0.000000}%
\pgfsetstrokecolor{currentstroke}%
\pgfsetdash{}{0pt}%
\pgfsys@defobject{currentmarker}{\pgfqpoint{0.000000in}{0.000000in}}{\pgfqpoint{0.048611in}{0.000000in}}{%
\pgfpathmoveto{\pgfqpoint{0.000000in}{0.000000in}}%
\pgfpathlineto{\pgfqpoint{0.048611in}{0.000000in}}%
\pgfusepath{stroke,fill}%
}%
\begin{pgfscope}%
\pgfsys@transformshift{0.800000in}{3.300000in}%
\pgfsys@useobject{currentmarker}{}%
\end{pgfscope}%
\end{pgfscope}%
\begin{pgfscope}%
\pgfsetbuttcap%
\pgfsetroundjoin%
\definecolor{currentfill}{rgb}{0.000000,0.000000,0.000000}%
\pgfsetfillcolor{currentfill}%
\pgfsetlinewidth{0.803000pt}%
\definecolor{currentstroke}{rgb}{0.000000,0.000000,0.000000}%
\pgfsetstrokecolor{currentstroke}%
\pgfsetdash{}{0pt}%
\pgfsys@defobject{currentmarker}{\pgfqpoint{-0.048611in}{0.000000in}}{\pgfqpoint{0.000000in}{0.000000in}}{%
\pgfpathmoveto{\pgfqpoint{0.000000in}{0.000000in}}%
\pgfpathlineto{\pgfqpoint{-0.048611in}{0.000000in}}%
\pgfusepath{stroke,fill}%
}%
\begin{pgfscope}%
\pgfsys@transformshift{5.760000in}{3.300000in}%
\pgfsys@useobject{currentmarker}{}%
\end{pgfscope}%
\end{pgfscope}%
\begin{pgfscope}%
\definecolor{textcolor}{rgb}{0.000000,0.000000,0.000000}%
\pgfsetstrokecolor{textcolor}%
\pgfsetfillcolor{textcolor}%
\pgftext[x=0.486334in,y=3.236686in,left,base]{\color{textcolor}\sffamily\fontsize{12.000000}{14.400000}\selectfont 0.5}%
\end{pgfscope}%
\begin{pgfscope}%
\pgfpathrectangle{\pgfqpoint{0.800000in}{0.528000in}}{\pgfqpoint{4.960000in}{3.696000in}}%
\pgfusepath{clip}%
\pgfsetbuttcap%
\pgfsetroundjoin%
\pgfsetlinewidth{1.505625pt}%
\definecolor{currentstroke}{rgb}{0.000000,0.000000,0.878431}%
\pgfsetstrokecolor{currentstroke}%
\pgfsetdash{{1.500000pt}{2.475000pt}}{0.000000pt}%
\pgfpathmoveto{\pgfqpoint{2.332027in}{3.361475in}}%
\pgfusepath{stroke}%
\end{pgfscope}%
\begin{pgfscope}%
\pgfpathrectangle{\pgfqpoint{0.800000in}{0.528000in}}{\pgfqpoint{4.960000in}{3.696000in}}%
\pgfusepath{clip}%
\pgfsetbuttcap%
\pgfsetroundjoin%
\definecolor{currentfill}{rgb}{0.000000,0.000000,0.878431}%
\pgfsetfillcolor{currentfill}%
\pgfsetlinewidth{0.501875pt}%
\definecolor{currentstroke}{rgb}{0.000000,0.000000,0.690196}%
\pgfsetstrokecolor{currentstroke}%
\pgfsetdash{}{0pt}%
\pgfsys@defobject{currentmarker}{\pgfqpoint{-0.027778in}{-0.027778in}}{\pgfqpoint{0.027778in}{0.027778in}}{%
\pgfpathmoveto{\pgfqpoint{0.000000in}{-0.027778in}}%
\pgfpathcurveto{\pgfqpoint{0.007367in}{-0.027778in}}{\pgfqpoint{0.014433in}{-0.024851in}}{\pgfqpoint{0.019642in}{-0.019642in}}%
\pgfpathcurveto{\pgfqpoint{0.024851in}{-0.014433in}}{\pgfqpoint{0.027778in}{-0.007367in}}{\pgfqpoint{0.027778in}{0.000000in}}%
\pgfpathcurveto{\pgfqpoint{0.027778in}{0.007367in}}{\pgfqpoint{0.024851in}{0.014433in}}{\pgfqpoint{0.019642in}{0.019642in}}%
\pgfpathcurveto{\pgfqpoint{0.014433in}{0.024851in}}{\pgfqpoint{0.007367in}{0.027778in}}{\pgfqpoint{0.000000in}{0.027778in}}%
\pgfpathcurveto{\pgfqpoint{-0.007367in}{0.027778in}}{\pgfqpoint{-0.014433in}{0.024851in}}{\pgfqpoint{-0.019642in}{0.019642in}}%
\pgfpathcurveto{\pgfqpoint{-0.024851in}{0.014433in}}{\pgfqpoint{-0.027778in}{0.007367in}}{\pgfqpoint{-0.027778in}{0.000000in}}%
\pgfpathcurveto{\pgfqpoint{-0.027778in}{-0.007367in}}{\pgfqpoint{-0.024851in}{-0.014433in}}{\pgfqpoint{-0.019642in}{-0.019642in}}%
\pgfpathcurveto{\pgfqpoint{-0.014433in}{-0.024851in}}{\pgfqpoint{-0.007367in}{-0.027778in}}{\pgfqpoint{0.000000in}{-0.027778in}}%
\pgfpathclose%
\pgfusepath{stroke,fill}%
}%
\begin{pgfscope}%
\pgfsys@transformshift{2.332027in}{3.361475in}%
\pgfsys@useobject{currentmarker}{}%
\end{pgfscope}%
\end{pgfscope}%
\begin{pgfscope}%
\pgfpathrectangle{\pgfqpoint{0.800000in}{0.528000in}}{\pgfqpoint{4.960000in}{3.696000in}}%
\pgfusepath{clip}%
\pgfsetbuttcap%
\pgfsetroundjoin%
\pgfsetlinewidth{1.505625pt}%
\definecolor{currentstroke}{rgb}{0.000000,0.000000,0.878431}%
\pgfsetstrokecolor{currentstroke}%
\pgfsetdash{{1.500000pt}{2.475000pt}}{0.000000pt}%
\pgfpathmoveto{\pgfqpoint{1.707304in}{3.458535in}}%
\pgfusepath{stroke}%
\end{pgfscope}%
\begin{pgfscope}%
\pgfpathrectangle{\pgfqpoint{0.800000in}{0.528000in}}{\pgfqpoint{4.960000in}{3.696000in}}%
\pgfusepath{clip}%
\pgfsetbuttcap%
\pgfsetroundjoin%
\definecolor{currentfill}{rgb}{0.000000,0.000000,0.878431}%
\pgfsetfillcolor{currentfill}%
\pgfsetlinewidth{0.501875pt}%
\definecolor{currentstroke}{rgb}{0.000000,0.000000,0.690196}%
\pgfsetstrokecolor{currentstroke}%
\pgfsetdash{}{0pt}%
\pgfsys@defobject{currentmarker}{\pgfqpoint{-0.027778in}{-0.027778in}}{\pgfqpoint{0.027778in}{0.027778in}}{%
\pgfpathmoveto{\pgfqpoint{0.000000in}{-0.027778in}}%
\pgfpathcurveto{\pgfqpoint{0.007367in}{-0.027778in}}{\pgfqpoint{0.014433in}{-0.024851in}}{\pgfqpoint{0.019642in}{-0.019642in}}%
\pgfpathcurveto{\pgfqpoint{0.024851in}{-0.014433in}}{\pgfqpoint{0.027778in}{-0.007367in}}{\pgfqpoint{0.027778in}{0.000000in}}%
\pgfpathcurveto{\pgfqpoint{0.027778in}{0.007367in}}{\pgfqpoint{0.024851in}{0.014433in}}{\pgfqpoint{0.019642in}{0.019642in}}%
\pgfpathcurveto{\pgfqpoint{0.014433in}{0.024851in}}{\pgfqpoint{0.007367in}{0.027778in}}{\pgfqpoint{0.000000in}{0.027778in}}%
\pgfpathcurveto{\pgfqpoint{-0.007367in}{0.027778in}}{\pgfqpoint{-0.014433in}{0.024851in}}{\pgfqpoint{-0.019642in}{0.019642in}}%
\pgfpathcurveto{\pgfqpoint{-0.024851in}{0.014433in}}{\pgfqpoint{-0.027778in}{0.007367in}}{\pgfqpoint{-0.027778in}{0.000000in}}%
\pgfpathcurveto{\pgfqpoint{-0.027778in}{-0.007367in}}{\pgfqpoint{-0.024851in}{-0.014433in}}{\pgfqpoint{-0.019642in}{-0.019642in}}%
\pgfpathcurveto{\pgfqpoint{-0.014433in}{-0.024851in}}{\pgfqpoint{-0.007367in}{-0.027778in}}{\pgfqpoint{0.000000in}{-0.027778in}}%
\pgfpathclose%
\pgfusepath{stroke,fill}%
}%
\begin{pgfscope}%
\pgfsys@transformshift{1.707304in}{3.458535in}%
\pgfsys@useobject{currentmarker}{}%
\end{pgfscope}%
\end{pgfscope}%
\begin{pgfscope}%
\pgfpathrectangle{\pgfqpoint{0.800000in}{0.528000in}}{\pgfqpoint{4.960000in}{3.696000in}}%
\pgfusepath{clip}%
\pgfsetbuttcap%
\pgfsetroundjoin%
\pgfsetlinewidth{1.505625pt}%
\definecolor{currentstroke}{rgb}{0.000000,0.000000,0.878431}%
\pgfsetstrokecolor{currentstroke}%
\pgfsetdash{{1.500000pt}{2.475000pt}}{0.000000pt}%
\pgfpathmoveto{\pgfqpoint{1.210904in}{3.899443in}}%
\pgfusepath{stroke}%
\end{pgfscope}%
\begin{pgfscope}%
\pgfpathrectangle{\pgfqpoint{0.800000in}{0.528000in}}{\pgfqpoint{4.960000in}{3.696000in}}%
\pgfusepath{clip}%
\pgfsetbuttcap%
\pgfsetroundjoin%
\definecolor{currentfill}{rgb}{0.000000,0.000000,0.878431}%
\pgfsetfillcolor{currentfill}%
\pgfsetlinewidth{0.501875pt}%
\definecolor{currentstroke}{rgb}{0.000000,0.000000,0.690196}%
\pgfsetstrokecolor{currentstroke}%
\pgfsetdash{}{0pt}%
\pgfsys@defobject{currentmarker}{\pgfqpoint{-0.027778in}{-0.027778in}}{\pgfqpoint{0.027778in}{0.027778in}}{%
\pgfpathmoveto{\pgfqpoint{0.000000in}{-0.027778in}}%
\pgfpathcurveto{\pgfqpoint{0.007367in}{-0.027778in}}{\pgfqpoint{0.014433in}{-0.024851in}}{\pgfqpoint{0.019642in}{-0.019642in}}%
\pgfpathcurveto{\pgfqpoint{0.024851in}{-0.014433in}}{\pgfqpoint{0.027778in}{-0.007367in}}{\pgfqpoint{0.027778in}{0.000000in}}%
\pgfpathcurveto{\pgfqpoint{0.027778in}{0.007367in}}{\pgfqpoint{0.024851in}{0.014433in}}{\pgfqpoint{0.019642in}{0.019642in}}%
\pgfpathcurveto{\pgfqpoint{0.014433in}{0.024851in}}{\pgfqpoint{0.007367in}{0.027778in}}{\pgfqpoint{0.000000in}{0.027778in}}%
\pgfpathcurveto{\pgfqpoint{-0.007367in}{0.027778in}}{\pgfqpoint{-0.014433in}{0.024851in}}{\pgfqpoint{-0.019642in}{0.019642in}}%
\pgfpathcurveto{\pgfqpoint{-0.024851in}{0.014433in}}{\pgfqpoint{-0.027778in}{0.007367in}}{\pgfqpoint{-0.027778in}{0.000000in}}%
\pgfpathcurveto{\pgfqpoint{-0.027778in}{-0.007367in}}{\pgfqpoint{-0.024851in}{-0.014433in}}{\pgfqpoint{-0.019642in}{-0.019642in}}%
\pgfpathcurveto{\pgfqpoint{-0.014433in}{-0.024851in}}{\pgfqpoint{-0.007367in}{-0.027778in}}{\pgfqpoint{0.000000in}{-0.027778in}}%
\pgfpathclose%
\pgfusepath{stroke,fill}%
}%
\begin{pgfscope}%
\pgfsys@transformshift{1.210904in}{3.899443in}%
\pgfsys@useobject{currentmarker}{}%
\end{pgfscope}%
\end{pgfscope}%
\begin{pgfscope}%
\pgfpathrectangle{\pgfqpoint{0.800000in}{0.528000in}}{\pgfqpoint{4.960000in}{3.696000in}}%
\pgfusepath{clip}%
\pgfsetbuttcap%
\pgfsetroundjoin%
\pgfsetlinewidth{1.505625pt}%
\definecolor{currentstroke}{rgb}{0.000000,0.000000,0.878431}%
\pgfsetstrokecolor{currentstroke}%
\pgfsetdash{{1.500000pt}{2.475000pt}}{0.000000pt}%
\pgfpathmoveto{\pgfqpoint{1.751588in}{3.835205in}}%
\pgfusepath{stroke}%
\end{pgfscope}%
\begin{pgfscope}%
\pgfpathrectangle{\pgfqpoint{0.800000in}{0.528000in}}{\pgfqpoint{4.960000in}{3.696000in}}%
\pgfusepath{clip}%
\pgfsetbuttcap%
\pgfsetroundjoin%
\definecolor{currentfill}{rgb}{0.000000,0.000000,0.878431}%
\pgfsetfillcolor{currentfill}%
\pgfsetlinewidth{0.501875pt}%
\definecolor{currentstroke}{rgb}{0.000000,0.000000,0.690196}%
\pgfsetstrokecolor{currentstroke}%
\pgfsetdash{}{0pt}%
\pgfsys@defobject{currentmarker}{\pgfqpoint{-0.027778in}{-0.027778in}}{\pgfqpoint{0.027778in}{0.027778in}}{%
\pgfpathmoveto{\pgfqpoint{0.000000in}{-0.027778in}}%
\pgfpathcurveto{\pgfqpoint{0.007367in}{-0.027778in}}{\pgfqpoint{0.014433in}{-0.024851in}}{\pgfqpoint{0.019642in}{-0.019642in}}%
\pgfpathcurveto{\pgfqpoint{0.024851in}{-0.014433in}}{\pgfqpoint{0.027778in}{-0.007367in}}{\pgfqpoint{0.027778in}{0.000000in}}%
\pgfpathcurveto{\pgfqpoint{0.027778in}{0.007367in}}{\pgfqpoint{0.024851in}{0.014433in}}{\pgfqpoint{0.019642in}{0.019642in}}%
\pgfpathcurveto{\pgfqpoint{0.014433in}{0.024851in}}{\pgfqpoint{0.007367in}{0.027778in}}{\pgfqpoint{0.000000in}{0.027778in}}%
\pgfpathcurveto{\pgfqpoint{-0.007367in}{0.027778in}}{\pgfqpoint{-0.014433in}{0.024851in}}{\pgfqpoint{-0.019642in}{0.019642in}}%
\pgfpathcurveto{\pgfqpoint{-0.024851in}{0.014433in}}{\pgfqpoint{-0.027778in}{0.007367in}}{\pgfqpoint{-0.027778in}{0.000000in}}%
\pgfpathcurveto{\pgfqpoint{-0.027778in}{-0.007367in}}{\pgfqpoint{-0.024851in}{-0.014433in}}{\pgfqpoint{-0.019642in}{-0.019642in}}%
\pgfpathcurveto{\pgfqpoint{-0.014433in}{-0.024851in}}{\pgfqpoint{-0.007367in}{-0.027778in}}{\pgfqpoint{0.000000in}{-0.027778in}}%
\pgfpathclose%
\pgfusepath{stroke,fill}%
}%
\begin{pgfscope}%
\pgfsys@transformshift{1.751588in}{3.835205in}%
\pgfsys@useobject{currentmarker}{}%
\end{pgfscope}%
\end{pgfscope}%
\begin{pgfscope}%
\pgfpathrectangle{\pgfqpoint{0.800000in}{0.528000in}}{\pgfqpoint{4.960000in}{3.696000in}}%
\pgfusepath{clip}%
\pgfsetbuttcap%
\pgfsetroundjoin%
\pgfsetlinewidth{1.505625pt}%
\definecolor{currentstroke}{rgb}{0.000000,0.000000,0.878431}%
\pgfsetstrokecolor{currentstroke}%
\pgfsetdash{{1.500000pt}{2.475000pt}}{0.000000pt}%
\pgfpathmoveto{\pgfqpoint{3.085882in}{2.944507in}}%
\pgfusepath{stroke}%
\end{pgfscope}%
\begin{pgfscope}%
\pgfpathrectangle{\pgfqpoint{0.800000in}{0.528000in}}{\pgfqpoint{4.960000in}{3.696000in}}%
\pgfusepath{clip}%
\pgfsetbuttcap%
\pgfsetroundjoin%
\definecolor{currentfill}{rgb}{0.000000,0.000000,0.878431}%
\pgfsetfillcolor{currentfill}%
\pgfsetlinewidth{0.501875pt}%
\definecolor{currentstroke}{rgb}{0.000000,0.000000,0.690196}%
\pgfsetstrokecolor{currentstroke}%
\pgfsetdash{}{0pt}%
\pgfsys@defobject{currentmarker}{\pgfqpoint{-0.027778in}{-0.027778in}}{\pgfqpoint{0.027778in}{0.027778in}}{%
\pgfpathmoveto{\pgfqpoint{0.000000in}{-0.027778in}}%
\pgfpathcurveto{\pgfqpoint{0.007367in}{-0.027778in}}{\pgfqpoint{0.014433in}{-0.024851in}}{\pgfqpoint{0.019642in}{-0.019642in}}%
\pgfpathcurveto{\pgfqpoint{0.024851in}{-0.014433in}}{\pgfqpoint{0.027778in}{-0.007367in}}{\pgfqpoint{0.027778in}{0.000000in}}%
\pgfpathcurveto{\pgfqpoint{0.027778in}{0.007367in}}{\pgfqpoint{0.024851in}{0.014433in}}{\pgfqpoint{0.019642in}{0.019642in}}%
\pgfpathcurveto{\pgfqpoint{0.014433in}{0.024851in}}{\pgfqpoint{0.007367in}{0.027778in}}{\pgfqpoint{0.000000in}{0.027778in}}%
\pgfpathcurveto{\pgfqpoint{-0.007367in}{0.027778in}}{\pgfqpoint{-0.014433in}{0.024851in}}{\pgfqpoint{-0.019642in}{0.019642in}}%
\pgfpathcurveto{\pgfqpoint{-0.024851in}{0.014433in}}{\pgfqpoint{-0.027778in}{0.007367in}}{\pgfqpoint{-0.027778in}{0.000000in}}%
\pgfpathcurveto{\pgfqpoint{-0.027778in}{-0.007367in}}{\pgfqpoint{-0.024851in}{-0.014433in}}{\pgfqpoint{-0.019642in}{-0.019642in}}%
\pgfpathcurveto{\pgfqpoint{-0.014433in}{-0.024851in}}{\pgfqpoint{-0.007367in}{-0.027778in}}{\pgfqpoint{0.000000in}{-0.027778in}}%
\pgfpathclose%
\pgfusepath{stroke,fill}%
}%
\begin{pgfscope}%
\pgfsys@transformshift{3.085882in}{2.944507in}%
\pgfsys@useobject{currentmarker}{}%
\end{pgfscope}%
\end{pgfscope}%
\begin{pgfscope}%
\pgfpathrectangle{\pgfqpoint{0.800000in}{0.528000in}}{\pgfqpoint{4.960000in}{3.696000in}}%
\pgfusepath{clip}%
\pgfsetbuttcap%
\pgfsetroundjoin%
\pgfsetlinewidth{1.505625pt}%
\definecolor{currentstroke}{rgb}{0.000000,0.000000,0.878431}%
\pgfsetstrokecolor{currentstroke}%
\pgfsetdash{{1.500000pt}{2.475000pt}}{0.000000pt}%
\pgfpathmoveto{\pgfqpoint{3.260773in}{2.754740in}}%
\pgfusepath{stroke}%
\end{pgfscope}%
\begin{pgfscope}%
\pgfpathrectangle{\pgfqpoint{0.800000in}{0.528000in}}{\pgfqpoint{4.960000in}{3.696000in}}%
\pgfusepath{clip}%
\pgfsetbuttcap%
\pgfsetroundjoin%
\definecolor{currentfill}{rgb}{0.000000,0.000000,0.878431}%
\pgfsetfillcolor{currentfill}%
\pgfsetlinewidth{0.501875pt}%
\definecolor{currentstroke}{rgb}{0.000000,0.000000,0.690196}%
\pgfsetstrokecolor{currentstroke}%
\pgfsetdash{}{0pt}%
\pgfsys@defobject{currentmarker}{\pgfqpoint{-0.027778in}{-0.027778in}}{\pgfqpoint{0.027778in}{0.027778in}}{%
\pgfpathmoveto{\pgfqpoint{0.000000in}{-0.027778in}}%
\pgfpathcurveto{\pgfqpoint{0.007367in}{-0.027778in}}{\pgfqpoint{0.014433in}{-0.024851in}}{\pgfqpoint{0.019642in}{-0.019642in}}%
\pgfpathcurveto{\pgfqpoint{0.024851in}{-0.014433in}}{\pgfqpoint{0.027778in}{-0.007367in}}{\pgfqpoint{0.027778in}{0.000000in}}%
\pgfpathcurveto{\pgfqpoint{0.027778in}{0.007367in}}{\pgfqpoint{0.024851in}{0.014433in}}{\pgfqpoint{0.019642in}{0.019642in}}%
\pgfpathcurveto{\pgfqpoint{0.014433in}{0.024851in}}{\pgfqpoint{0.007367in}{0.027778in}}{\pgfqpoint{0.000000in}{0.027778in}}%
\pgfpathcurveto{\pgfqpoint{-0.007367in}{0.027778in}}{\pgfqpoint{-0.014433in}{0.024851in}}{\pgfqpoint{-0.019642in}{0.019642in}}%
\pgfpathcurveto{\pgfqpoint{-0.024851in}{0.014433in}}{\pgfqpoint{-0.027778in}{0.007367in}}{\pgfqpoint{-0.027778in}{0.000000in}}%
\pgfpathcurveto{\pgfqpoint{-0.027778in}{-0.007367in}}{\pgfqpoint{-0.024851in}{-0.014433in}}{\pgfqpoint{-0.019642in}{-0.019642in}}%
\pgfpathcurveto{\pgfqpoint{-0.014433in}{-0.024851in}}{\pgfqpoint{-0.007367in}{-0.027778in}}{\pgfqpoint{0.000000in}{-0.027778in}}%
\pgfpathclose%
\pgfusepath{stroke,fill}%
}%
\begin{pgfscope}%
\pgfsys@transformshift{3.260773in}{2.754740in}%
\pgfsys@useobject{currentmarker}{}%
\end{pgfscope}%
\end{pgfscope}%
\begin{pgfscope}%
\pgfpathrectangle{\pgfqpoint{0.800000in}{0.528000in}}{\pgfqpoint{4.960000in}{3.696000in}}%
\pgfusepath{clip}%
\pgfsetbuttcap%
\pgfsetroundjoin%
\pgfsetlinewidth{1.505625pt}%
\definecolor{currentstroke}{rgb}{0.000000,0.000000,0.878431}%
\pgfsetstrokecolor{currentstroke}%
\pgfsetdash{{1.500000pt}{2.475000pt}}{0.000000pt}%
\pgfpathmoveto{\pgfqpoint{2.427569in}{4.061940in}}%
\pgfusepath{stroke}%
\end{pgfscope}%
\begin{pgfscope}%
\pgfpathrectangle{\pgfqpoint{0.800000in}{0.528000in}}{\pgfqpoint{4.960000in}{3.696000in}}%
\pgfusepath{clip}%
\pgfsetbuttcap%
\pgfsetroundjoin%
\definecolor{currentfill}{rgb}{0.000000,0.000000,0.878431}%
\pgfsetfillcolor{currentfill}%
\pgfsetlinewidth{0.501875pt}%
\definecolor{currentstroke}{rgb}{0.000000,0.000000,0.690196}%
\pgfsetstrokecolor{currentstroke}%
\pgfsetdash{}{0pt}%
\pgfsys@defobject{currentmarker}{\pgfqpoint{-0.027778in}{-0.027778in}}{\pgfqpoint{0.027778in}{0.027778in}}{%
\pgfpathmoveto{\pgfqpoint{0.000000in}{-0.027778in}}%
\pgfpathcurveto{\pgfqpoint{0.007367in}{-0.027778in}}{\pgfqpoint{0.014433in}{-0.024851in}}{\pgfqpoint{0.019642in}{-0.019642in}}%
\pgfpathcurveto{\pgfqpoint{0.024851in}{-0.014433in}}{\pgfqpoint{0.027778in}{-0.007367in}}{\pgfqpoint{0.027778in}{0.000000in}}%
\pgfpathcurveto{\pgfqpoint{0.027778in}{0.007367in}}{\pgfqpoint{0.024851in}{0.014433in}}{\pgfqpoint{0.019642in}{0.019642in}}%
\pgfpathcurveto{\pgfqpoint{0.014433in}{0.024851in}}{\pgfqpoint{0.007367in}{0.027778in}}{\pgfqpoint{0.000000in}{0.027778in}}%
\pgfpathcurveto{\pgfqpoint{-0.007367in}{0.027778in}}{\pgfqpoint{-0.014433in}{0.024851in}}{\pgfqpoint{-0.019642in}{0.019642in}}%
\pgfpathcurveto{\pgfqpoint{-0.024851in}{0.014433in}}{\pgfqpoint{-0.027778in}{0.007367in}}{\pgfqpoint{-0.027778in}{0.000000in}}%
\pgfpathcurveto{\pgfqpoint{-0.027778in}{-0.007367in}}{\pgfqpoint{-0.024851in}{-0.014433in}}{\pgfqpoint{-0.019642in}{-0.019642in}}%
\pgfpathcurveto{\pgfqpoint{-0.014433in}{-0.024851in}}{\pgfqpoint{-0.007367in}{-0.027778in}}{\pgfqpoint{0.000000in}{-0.027778in}}%
\pgfpathclose%
\pgfusepath{stroke,fill}%
}%
\begin{pgfscope}%
\pgfsys@transformshift{2.427569in}{4.061940in}%
\pgfsys@useobject{currentmarker}{}%
\end{pgfscope}%
\end{pgfscope}%
\begin{pgfscope}%
\pgfpathrectangle{\pgfqpoint{0.800000in}{0.528000in}}{\pgfqpoint{4.960000in}{3.696000in}}%
\pgfusepath{clip}%
\pgfsetbuttcap%
\pgfsetroundjoin%
\pgfsetlinewidth{1.505625pt}%
\definecolor{currentstroke}{rgb}{0.000000,0.000000,0.878431}%
\pgfsetstrokecolor{currentstroke}%
\pgfsetdash{{1.500000pt}{2.475000pt}}{0.000000pt}%
\pgfpathmoveto{\pgfqpoint{1.071035in}{3.892169in}}%
\pgfusepath{stroke}%
\end{pgfscope}%
\begin{pgfscope}%
\pgfpathrectangle{\pgfqpoint{0.800000in}{0.528000in}}{\pgfqpoint{4.960000in}{3.696000in}}%
\pgfusepath{clip}%
\pgfsetbuttcap%
\pgfsetroundjoin%
\definecolor{currentfill}{rgb}{0.000000,0.000000,0.878431}%
\pgfsetfillcolor{currentfill}%
\pgfsetlinewidth{0.501875pt}%
\definecolor{currentstroke}{rgb}{0.000000,0.000000,0.690196}%
\pgfsetstrokecolor{currentstroke}%
\pgfsetdash{}{0pt}%
\pgfsys@defobject{currentmarker}{\pgfqpoint{-0.027778in}{-0.027778in}}{\pgfqpoint{0.027778in}{0.027778in}}{%
\pgfpathmoveto{\pgfqpoint{0.000000in}{-0.027778in}}%
\pgfpathcurveto{\pgfqpoint{0.007367in}{-0.027778in}}{\pgfqpoint{0.014433in}{-0.024851in}}{\pgfqpoint{0.019642in}{-0.019642in}}%
\pgfpathcurveto{\pgfqpoint{0.024851in}{-0.014433in}}{\pgfqpoint{0.027778in}{-0.007367in}}{\pgfqpoint{0.027778in}{0.000000in}}%
\pgfpathcurveto{\pgfqpoint{0.027778in}{0.007367in}}{\pgfqpoint{0.024851in}{0.014433in}}{\pgfqpoint{0.019642in}{0.019642in}}%
\pgfpathcurveto{\pgfqpoint{0.014433in}{0.024851in}}{\pgfqpoint{0.007367in}{0.027778in}}{\pgfqpoint{0.000000in}{0.027778in}}%
\pgfpathcurveto{\pgfqpoint{-0.007367in}{0.027778in}}{\pgfqpoint{-0.014433in}{0.024851in}}{\pgfqpoint{-0.019642in}{0.019642in}}%
\pgfpathcurveto{\pgfqpoint{-0.024851in}{0.014433in}}{\pgfqpoint{-0.027778in}{0.007367in}}{\pgfqpoint{-0.027778in}{0.000000in}}%
\pgfpathcurveto{\pgfqpoint{-0.027778in}{-0.007367in}}{\pgfqpoint{-0.024851in}{-0.014433in}}{\pgfqpoint{-0.019642in}{-0.019642in}}%
\pgfpathcurveto{\pgfqpoint{-0.014433in}{-0.024851in}}{\pgfqpoint{-0.007367in}{-0.027778in}}{\pgfqpoint{0.000000in}{-0.027778in}}%
\pgfpathclose%
\pgfusepath{stroke,fill}%
}%
\begin{pgfscope}%
\pgfsys@transformshift{1.071035in}{3.892169in}%
\pgfsys@useobject{currentmarker}{}%
\end{pgfscope}%
\end{pgfscope}%
\begin{pgfscope}%
\pgfpathrectangle{\pgfqpoint{0.800000in}{0.528000in}}{\pgfqpoint{4.960000in}{3.696000in}}%
\pgfusepath{clip}%
\pgfsetbuttcap%
\pgfsetroundjoin%
\pgfsetlinewidth{1.505625pt}%
\definecolor{currentstroke}{rgb}{0.000000,0.000000,0.878431}%
\pgfsetstrokecolor{currentstroke}%
\pgfsetdash{{1.500000pt}{2.475000pt}}{0.000000pt}%
\pgfpathmoveto{\pgfqpoint{1.788873in}{2.496754in}}%
\pgfusepath{stroke}%
\end{pgfscope}%
\begin{pgfscope}%
\pgfpathrectangle{\pgfqpoint{0.800000in}{0.528000in}}{\pgfqpoint{4.960000in}{3.696000in}}%
\pgfusepath{clip}%
\pgfsetbuttcap%
\pgfsetroundjoin%
\definecolor{currentfill}{rgb}{0.000000,0.000000,0.878431}%
\pgfsetfillcolor{currentfill}%
\pgfsetlinewidth{0.501875pt}%
\definecolor{currentstroke}{rgb}{0.000000,0.000000,0.690196}%
\pgfsetstrokecolor{currentstroke}%
\pgfsetdash{}{0pt}%
\pgfsys@defobject{currentmarker}{\pgfqpoint{-0.027778in}{-0.027778in}}{\pgfqpoint{0.027778in}{0.027778in}}{%
\pgfpathmoveto{\pgfqpoint{0.000000in}{-0.027778in}}%
\pgfpathcurveto{\pgfqpoint{0.007367in}{-0.027778in}}{\pgfqpoint{0.014433in}{-0.024851in}}{\pgfqpoint{0.019642in}{-0.019642in}}%
\pgfpathcurveto{\pgfqpoint{0.024851in}{-0.014433in}}{\pgfqpoint{0.027778in}{-0.007367in}}{\pgfqpoint{0.027778in}{0.000000in}}%
\pgfpathcurveto{\pgfqpoint{0.027778in}{0.007367in}}{\pgfqpoint{0.024851in}{0.014433in}}{\pgfqpoint{0.019642in}{0.019642in}}%
\pgfpathcurveto{\pgfqpoint{0.014433in}{0.024851in}}{\pgfqpoint{0.007367in}{0.027778in}}{\pgfqpoint{0.000000in}{0.027778in}}%
\pgfpathcurveto{\pgfqpoint{-0.007367in}{0.027778in}}{\pgfqpoint{-0.014433in}{0.024851in}}{\pgfqpoint{-0.019642in}{0.019642in}}%
\pgfpathcurveto{\pgfqpoint{-0.024851in}{0.014433in}}{\pgfqpoint{-0.027778in}{0.007367in}}{\pgfqpoint{-0.027778in}{0.000000in}}%
\pgfpathcurveto{\pgfqpoint{-0.027778in}{-0.007367in}}{\pgfqpoint{-0.024851in}{-0.014433in}}{\pgfqpoint{-0.019642in}{-0.019642in}}%
\pgfpathcurveto{\pgfqpoint{-0.014433in}{-0.024851in}}{\pgfqpoint{-0.007367in}{-0.027778in}}{\pgfqpoint{0.000000in}{-0.027778in}}%
\pgfpathclose%
\pgfusepath{stroke,fill}%
}%
\begin{pgfscope}%
\pgfsys@transformshift{1.788873in}{2.496754in}%
\pgfsys@useobject{currentmarker}{}%
\end{pgfscope}%
\end{pgfscope}%
\begin{pgfscope}%
\pgfpathrectangle{\pgfqpoint{0.800000in}{0.528000in}}{\pgfqpoint{4.960000in}{3.696000in}}%
\pgfusepath{clip}%
\pgfsetbuttcap%
\pgfsetroundjoin%
\pgfsetlinewidth{1.505625pt}%
\definecolor{currentstroke}{rgb}{0.000000,0.000000,0.878431}%
\pgfsetstrokecolor{currentstroke}%
\pgfsetdash{{1.500000pt}{2.475000pt}}{0.000000pt}%
\pgfpathmoveto{\pgfqpoint{2.534037in}{3.019292in}}%
\pgfusepath{stroke}%
\end{pgfscope}%
\begin{pgfscope}%
\pgfpathrectangle{\pgfqpoint{0.800000in}{0.528000in}}{\pgfqpoint{4.960000in}{3.696000in}}%
\pgfusepath{clip}%
\pgfsetbuttcap%
\pgfsetroundjoin%
\definecolor{currentfill}{rgb}{0.000000,0.000000,0.878431}%
\pgfsetfillcolor{currentfill}%
\pgfsetlinewidth{0.501875pt}%
\definecolor{currentstroke}{rgb}{0.000000,0.000000,0.690196}%
\pgfsetstrokecolor{currentstroke}%
\pgfsetdash{}{0pt}%
\pgfsys@defobject{currentmarker}{\pgfqpoint{-0.027778in}{-0.027778in}}{\pgfqpoint{0.027778in}{0.027778in}}{%
\pgfpathmoveto{\pgfqpoint{0.000000in}{-0.027778in}}%
\pgfpathcurveto{\pgfqpoint{0.007367in}{-0.027778in}}{\pgfqpoint{0.014433in}{-0.024851in}}{\pgfqpoint{0.019642in}{-0.019642in}}%
\pgfpathcurveto{\pgfqpoint{0.024851in}{-0.014433in}}{\pgfqpoint{0.027778in}{-0.007367in}}{\pgfqpoint{0.027778in}{0.000000in}}%
\pgfpathcurveto{\pgfqpoint{0.027778in}{0.007367in}}{\pgfqpoint{0.024851in}{0.014433in}}{\pgfqpoint{0.019642in}{0.019642in}}%
\pgfpathcurveto{\pgfqpoint{0.014433in}{0.024851in}}{\pgfqpoint{0.007367in}{0.027778in}}{\pgfqpoint{0.000000in}{0.027778in}}%
\pgfpathcurveto{\pgfqpoint{-0.007367in}{0.027778in}}{\pgfqpoint{-0.014433in}{0.024851in}}{\pgfqpoint{-0.019642in}{0.019642in}}%
\pgfpathcurveto{\pgfqpoint{-0.024851in}{0.014433in}}{\pgfqpoint{-0.027778in}{0.007367in}}{\pgfqpoint{-0.027778in}{0.000000in}}%
\pgfpathcurveto{\pgfqpoint{-0.027778in}{-0.007367in}}{\pgfqpoint{-0.024851in}{-0.014433in}}{\pgfqpoint{-0.019642in}{-0.019642in}}%
\pgfpathcurveto{\pgfqpoint{-0.014433in}{-0.024851in}}{\pgfqpoint{-0.007367in}{-0.027778in}}{\pgfqpoint{0.000000in}{-0.027778in}}%
\pgfpathclose%
\pgfusepath{stroke,fill}%
}%
\begin{pgfscope}%
\pgfsys@transformshift{2.534037in}{3.019292in}%
\pgfsys@useobject{currentmarker}{}%
\end{pgfscope}%
\end{pgfscope}%
\begin{pgfscope}%
\pgfpathrectangle{\pgfqpoint{0.800000in}{0.528000in}}{\pgfqpoint{4.960000in}{3.696000in}}%
\pgfusepath{clip}%
\pgfsetbuttcap%
\pgfsetroundjoin%
\pgfsetlinewidth{1.505625pt}%
\definecolor{currentstroke}{rgb}{0.000000,0.000000,0.878431}%
\pgfsetstrokecolor{currentstroke}%
\pgfsetdash{{1.500000pt}{2.475000pt}}{0.000000pt}%
\pgfpathmoveto{\pgfqpoint{1.800897in}{3.857380in}}%
\pgfusepath{stroke}%
\end{pgfscope}%
\begin{pgfscope}%
\pgfpathrectangle{\pgfqpoint{0.800000in}{0.528000in}}{\pgfqpoint{4.960000in}{3.696000in}}%
\pgfusepath{clip}%
\pgfsetbuttcap%
\pgfsetroundjoin%
\definecolor{currentfill}{rgb}{0.000000,0.000000,0.878431}%
\pgfsetfillcolor{currentfill}%
\pgfsetlinewidth{0.501875pt}%
\definecolor{currentstroke}{rgb}{0.000000,0.000000,0.690196}%
\pgfsetstrokecolor{currentstroke}%
\pgfsetdash{}{0pt}%
\pgfsys@defobject{currentmarker}{\pgfqpoint{-0.027778in}{-0.027778in}}{\pgfqpoint{0.027778in}{0.027778in}}{%
\pgfpathmoveto{\pgfqpoint{0.000000in}{-0.027778in}}%
\pgfpathcurveto{\pgfqpoint{0.007367in}{-0.027778in}}{\pgfqpoint{0.014433in}{-0.024851in}}{\pgfqpoint{0.019642in}{-0.019642in}}%
\pgfpathcurveto{\pgfqpoint{0.024851in}{-0.014433in}}{\pgfqpoint{0.027778in}{-0.007367in}}{\pgfqpoint{0.027778in}{0.000000in}}%
\pgfpathcurveto{\pgfqpoint{0.027778in}{0.007367in}}{\pgfqpoint{0.024851in}{0.014433in}}{\pgfqpoint{0.019642in}{0.019642in}}%
\pgfpathcurveto{\pgfqpoint{0.014433in}{0.024851in}}{\pgfqpoint{0.007367in}{0.027778in}}{\pgfqpoint{0.000000in}{0.027778in}}%
\pgfpathcurveto{\pgfqpoint{-0.007367in}{0.027778in}}{\pgfqpoint{-0.014433in}{0.024851in}}{\pgfqpoint{-0.019642in}{0.019642in}}%
\pgfpathcurveto{\pgfqpoint{-0.024851in}{0.014433in}}{\pgfqpoint{-0.027778in}{0.007367in}}{\pgfqpoint{-0.027778in}{0.000000in}}%
\pgfpathcurveto{\pgfqpoint{-0.027778in}{-0.007367in}}{\pgfqpoint{-0.024851in}{-0.014433in}}{\pgfqpoint{-0.019642in}{-0.019642in}}%
\pgfpathcurveto{\pgfqpoint{-0.014433in}{-0.024851in}}{\pgfqpoint{-0.007367in}{-0.027778in}}{\pgfqpoint{0.000000in}{-0.027778in}}%
\pgfpathclose%
\pgfusepath{stroke,fill}%
}%
\begin{pgfscope}%
\pgfsys@transformshift{1.800897in}{3.857380in}%
\pgfsys@useobject{currentmarker}{}%
\end{pgfscope}%
\end{pgfscope}%
\begin{pgfscope}%
\pgfpathrectangle{\pgfqpoint{0.800000in}{0.528000in}}{\pgfqpoint{4.960000in}{3.696000in}}%
\pgfusepath{clip}%
\pgfsetbuttcap%
\pgfsetroundjoin%
\pgfsetlinewidth{1.505625pt}%
\definecolor{currentstroke}{rgb}{0.000000,0.000000,0.878431}%
\pgfsetstrokecolor{currentstroke}%
\pgfsetdash{{1.500000pt}{2.475000pt}}{0.000000pt}%
\pgfpathmoveto{\pgfqpoint{2.784162in}{3.602608in}}%
\pgfusepath{stroke}%
\end{pgfscope}%
\begin{pgfscope}%
\pgfpathrectangle{\pgfqpoint{0.800000in}{0.528000in}}{\pgfqpoint{4.960000in}{3.696000in}}%
\pgfusepath{clip}%
\pgfsetbuttcap%
\pgfsetroundjoin%
\definecolor{currentfill}{rgb}{0.000000,0.000000,0.878431}%
\pgfsetfillcolor{currentfill}%
\pgfsetlinewidth{0.501875pt}%
\definecolor{currentstroke}{rgb}{0.000000,0.000000,0.690196}%
\pgfsetstrokecolor{currentstroke}%
\pgfsetdash{}{0pt}%
\pgfsys@defobject{currentmarker}{\pgfqpoint{-0.027778in}{-0.027778in}}{\pgfqpoint{0.027778in}{0.027778in}}{%
\pgfpathmoveto{\pgfqpoint{0.000000in}{-0.027778in}}%
\pgfpathcurveto{\pgfqpoint{0.007367in}{-0.027778in}}{\pgfqpoint{0.014433in}{-0.024851in}}{\pgfqpoint{0.019642in}{-0.019642in}}%
\pgfpathcurveto{\pgfqpoint{0.024851in}{-0.014433in}}{\pgfqpoint{0.027778in}{-0.007367in}}{\pgfqpoint{0.027778in}{0.000000in}}%
\pgfpathcurveto{\pgfqpoint{0.027778in}{0.007367in}}{\pgfqpoint{0.024851in}{0.014433in}}{\pgfqpoint{0.019642in}{0.019642in}}%
\pgfpathcurveto{\pgfqpoint{0.014433in}{0.024851in}}{\pgfqpoint{0.007367in}{0.027778in}}{\pgfqpoint{0.000000in}{0.027778in}}%
\pgfpathcurveto{\pgfqpoint{-0.007367in}{0.027778in}}{\pgfqpoint{-0.014433in}{0.024851in}}{\pgfqpoint{-0.019642in}{0.019642in}}%
\pgfpathcurveto{\pgfqpoint{-0.024851in}{0.014433in}}{\pgfqpoint{-0.027778in}{0.007367in}}{\pgfqpoint{-0.027778in}{0.000000in}}%
\pgfpathcurveto{\pgfqpoint{-0.027778in}{-0.007367in}}{\pgfqpoint{-0.024851in}{-0.014433in}}{\pgfqpoint{-0.019642in}{-0.019642in}}%
\pgfpathcurveto{\pgfqpoint{-0.014433in}{-0.024851in}}{\pgfqpoint{-0.007367in}{-0.027778in}}{\pgfqpoint{0.000000in}{-0.027778in}}%
\pgfpathclose%
\pgfusepath{stroke,fill}%
}%
\begin{pgfscope}%
\pgfsys@transformshift{2.784162in}{3.602608in}%
\pgfsys@useobject{currentmarker}{}%
\end{pgfscope}%
\end{pgfscope}%
\begin{pgfscope}%
\pgfpathrectangle{\pgfqpoint{0.800000in}{0.528000in}}{\pgfqpoint{4.960000in}{3.696000in}}%
\pgfusepath{clip}%
\pgfsetbuttcap%
\pgfsetroundjoin%
\pgfsetlinewidth{1.505625pt}%
\definecolor{currentstroke}{rgb}{0.000000,0.000000,0.878431}%
\pgfsetstrokecolor{currentstroke}%
\pgfsetdash{{1.500000pt}{2.475000pt}}{0.000000pt}%
\pgfpathmoveto{\pgfqpoint{2.864616in}{3.474556in}}%
\pgfusepath{stroke}%
\end{pgfscope}%
\begin{pgfscope}%
\pgfpathrectangle{\pgfqpoint{0.800000in}{0.528000in}}{\pgfqpoint{4.960000in}{3.696000in}}%
\pgfusepath{clip}%
\pgfsetbuttcap%
\pgfsetroundjoin%
\definecolor{currentfill}{rgb}{0.000000,0.000000,0.878431}%
\pgfsetfillcolor{currentfill}%
\pgfsetlinewidth{0.501875pt}%
\definecolor{currentstroke}{rgb}{0.000000,0.000000,0.690196}%
\pgfsetstrokecolor{currentstroke}%
\pgfsetdash{}{0pt}%
\pgfsys@defobject{currentmarker}{\pgfqpoint{-0.027778in}{-0.027778in}}{\pgfqpoint{0.027778in}{0.027778in}}{%
\pgfpathmoveto{\pgfqpoint{0.000000in}{-0.027778in}}%
\pgfpathcurveto{\pgfqpoint{0.007367in}{-0.027778in}}{\pgfqpoint{0.014433in}{-0.024851in}}{\pgfqpoint{0.019642in}{-0.019642in}}%
\pgfpathcurveto{\pgfqpoint{0.024851in}{-0.014433in}}{\pgfqpoint{0.027778in}{-0.007367in}}{\pgfqpoint{0.027778in}{0.000000in}}%
\pgfpathcurveto{\pgfqpoint{0.027778in}{0.007367in}}{\pgfqpoint{0.024851in}{0.014433in}}{\pgfqpoint{0.019642in}{0.019642in}}%
\pgfpathcurveto{\pgfqpoint{0.014433in}{0.024851in}}{\pgfqpoint{0.007367in}{0.027778in}}{\pgfqpoint{0.000000in}{0.027778in}}%
\pgfpathcurveto{\pgfqpoint{-0.007367in}{0.027778in}}{\pgfqpoint{-0.014433in}{0.024851in}}{\pgfqpoint{-0.019642in}{0.019642in}}%
\pgfpathcurveto{\pgfqpoint{-0.024851in}{0.014433in}}{\pgfqpoint{-0.027778in}{0.007367in}}{\pgfqpoint{-0.027778in}{0.000000in}}%
\pgfpathcurveto{\pgfqpoint{-0.027778in}{-0.007367in}}{\pgfqpoint{-0.024851in}{-0.014433in}}{\pgfqpoint{-0.019642in}{-0.019642in}}%
\pgfpathcurveto{\pgfqpoint{-0.014433in}{-0.024851in}}{\pgfqpoint{-0.007367in}{-0.027778in}}{\pgfqpoint{0.000000in}{-0.027778in}}%
\pgfpathclose%
\pgfusepath{stroke,fill}%
}%
\begin{pgfscope}%
\pgfsys@transformshift{2.864616in}{3.474556in}%
\pgfsys@useobject{currentmarker}{}%
\end{pgfscope}%
\end{pgfscope}%
\begin{pgfscope}%
\pgfpathrectangle{\pgfqpoint{0.800000in}{0.528000in}}{\pgfqpoint{4.960000in}{3.696000in}}%
\pgfusepath{clip}%
\pgfsetbuttcap%
\pgfsetroundjoin%
\pgfsetlinewidth{1.505625pt}%
\definecolor{currentstroke}{rgb}{0.000000,0.000000,0.878431}%
\pgfsetstrokecolor{currentstroke}%
\pgfsetdash{{1.500000pt}{2.475000pt}}{0.000000pt}%
\pgfpathmoveto{\pgfqpoint{1.943116in}{4.145577in}}%
\pgfusepath{stroke}%
\end{pgfscope}%
\begin{pgfscope}%
\pgfpathrectangle{\pgfqpoint{0.800000in}{0.528000in}}{\pgfqpoint{4.960000in}{3.696000in}}%
\pgfusepath{clip}%
\pgfsetbuttcap%
\pgfsetroundjoin%
\definecolor{currentfill}{rgb}{0.000000,0.000000,0.878431}%
\pgfsetfillcolor{currentfill}%
\pgfsetlinewidth{0.501875pt}%
\definecolor{currentstroke}{rgb}{0.000000,0.000000,0.690196}%
\pgfsetstrokecolor{currentstroke}%
\pgfsetdash{}{0pt}%
\pgfsys@defobject{currentmarker}{\pgfqpoint{-0.027778in}{-0.027778in}}{\pgfqpoint{0.027778in}{0.027778in}}{%
\pgfpathmoveto{\pgfqpoint{0.000000in}{-0.027778in}}%
\pgfpathcurveto{\pgfqpoint{0.007367in}{-0.027778in}}{\pgfqpoint{0.014433in}{-0.024851in}}{\pgfqpoint{0.019642in}{-0.019642in}}%
\pgfpathcurveto{\pgfqpoint{0.024851in}{-0.014433in}}{\pgfqpoint{0.027778in}{-0.007367in}}{\pgfqpoint{0.027778in}{0.000000in}}%
\pgfpathcurveto{\pgfqpoint{0.027778in}{0.007367in}}{\pgfqpoint{0.024851in}{0.014433in}}{\pgfqpoint{0.019642in}{0.019642in}}%
\pgfpathcurveto{\pgfqpoint{0.014433in}{0.024851in}}{\pgfqpoint{0.007367in}{0.027778in}}{\pgfqpoint{0.000000in}{0.027778in}}%
\pgfpathcurveto{\pgfqpoint{-0.007367in}{0.027778in}}{\pgfqpoint{-0.014433in}{0.024851in}}{\pgfqpoint{-0.019642in}{0.019642in}}%
\pgfpathcurveto{\pgfqpoint{-0.024851in}{0.014433in}}{\pgfqpoint{-0.027778in}{0.007367in}}{\pgfqpoint{-0.027778in}{0.000000in}}%
\pgfpathcurveto{\pgfqpoint{-0.027778in}{-0.007367in}}{\pgfqpoint{-0.024851in}{-0.014433in}}{\pgfqpoint{-0.019642in}{-0.019642in}}%
\pgfpathcurveto{\pgfqpoint{-0.014433in}{-0.024851in}}{\pgfqpoint{-0.007367in}{-0.027778in}}{\pgfqpoint{0.000000in}{-0.027778in}}%
\pgfpathclose%
\pgfusepath{stroke,fill}%
}%
\begin{pgfscope}%
\pgfsys@transformshift{1.943116in}{4.145577in}%
\pgfsys@useobject{currentmarker}{}%
\end{pgfscope}%
\end{pgfscope}%
\begin{pgfscope}%
\pgfpathrectangle{\pgfqpoint{0.800000in}{0.528000in}}{\pgfqpoint{4.960000in}{3.696000in}}%
\pgfusepath{clip}%
\pgfsetbuttcap%
\pgfsetroundjoin%
\pgfsetlinewidth{1.505625pt}%
\definecolor{currentstroke}{rgb}{0.000000,0.000000,0.878431}%
\pgfsetstrokecolor{currentstroke}%
\pgfsetdash{{1.500000pt}{2.475000pt}}{0.000000pt}%
\pgfpathmoveto{\pgfqpoint{2.773386in}{2.955769in}}%
\pgfusepath{stroke}%
\end{pgfscope}%
\begin{pgfscope}%
\pgfpathrectangle{\pgfqpoint{0.800000in}{0.528000in}}{\pgfqpoint{4.960000in}{3.696000in}}%
\pgfusepath{clip}%
\pgfsetbuttcap%
\pgfsetroundjoin%
\definecolor{currentfill}{rgb}{0.000000,0.000000,0.878431}%
\pgfsetfillcolor{currentfill}%
\pgfsetlinewidth{0.501875pt}%
\definecolor{currentstroke}{rgb}{0.000000,0.000000,0.690196}%
\pgfsetstrokecolor{currentstroke}%
\pgfsetdash{}{0pt}%
\pgfsys@defobject{currentmarker}{\pgfqpoint{-0.027778in}{-0.027778in}}{\pgfqpoint{0.027778in}{0.027778in}}{%
\pgfpathmoveto{\pgfqpoint{0.000000in}{-0.027778in}}%
\pgfpathcurveto{\pgfqpoint{0.007367in}{-0.027778in}}{\pgfqpoint{0.014433in}{-0.024851in}}{\pgfqpoint{0.019642in}{-0.019642in}}%
\pgfpathcurveto{\pgfqpoint{0.024851in}{-0.014433in}}{\pgfqpoint{0.027778in}{-0.007367in}}{\pgfqpoint{0.027778in}{0.000000in}}%
\pgfpathcurveto{\pgfqpoint{0.027778in}{0.007367in}}{\pgfqpoint{0.024851in}{0.014433in}}{\pgfqpoint{0.019642in}{0.019642in}}%
\pgfpathcurveto{\pgfqpoint{0.014433in}{0.024851in}}{\pgfqpoint{0.007367in}{0.027778in}}{\pgfqpoint{0.000000in}{0.027778in}}%
\pgfpathcurveto{\pgfqpoint{-0.007367in}{0.027778in}}{\pgfqpoint{-0.014433in}{0.024851in}}{\pgfqpoint{-0.019642in}{0.019642in}}%
\pgfpathcurveto{\pgfqpoint{-0.024851in}{0.014433in}}{\pgfqpoint{-0.027778in}{0.007367in}}{\pgfqpoint{-0.027778in}{0.000000in}}%
\pgfpathcurveto{\pgfqpoint{-0.027778in}{-0.007367in}}{\pgfqpoint{-0.024851in}{-0.014433in}}{\pgfqpoint{-0.019642in}{-0.019642in}}%
\pgfpathcurveto{\pgfqpoint{-0.014433in}{-0.024851in}}{\pgfqpoint{-0.007367in}{-0.027778in}}{\pgfqpoint{0.000000in}{-0.027778in}}%
\pgfpathclose%
\pgfusepath{stroke,fill}%
}%
\begin{pgfscope}%
\pgfsys@transformshift{2.773386in}{2.955769in}%
\pgfsys@useobject{currentmarker}{}%
\end{pgfscope}%
\end{pgfscope}%
\begin{pgfscope}%
\pgfpathrectangle{\pgfqpoint{0.800000in}{0.528000in}}{\pgfqpoint{4.960000in}{3.696000in}}%
\pgfusepath{clip}%
\pgfsetbuttcap%
\pgfsetroundjoin%
\pgfsetlinewidth{1.505625pt}%
\definecolor{currentstroke}{rgb}{0.000000,0.000000,0.878431}%
\pgfsetstrokecolor{currentstroke}%
\pgfsetdash{{1.500000pt}{2.475000pt}}{0.000000pt}%
\pgfpathmoveto{\pgfqpoint{2.512543in}{4.067105in}}%
\pgfusepath{stroke}%
\end{pgfscope}%
\begin{pgfscope}%
\pgfpathrectangle{\pgfqpoint{0.800000in}{0.528000in}}{\pgfqpoint{4.960000in}{3.696000in}}%
\pgfusepath{clip}%
\pgfsetbuttcap%
\pgfsetroundjoin%
\definecolor{currentfill}{rgb}{0.000000,0.000000,0.878431}%
\pgfsetfillcolor{currentfill}%
\pgfsetlinewidth{0.501875pt}%
\definecolor{currentstroke}{rgb}{0.000000,0.000000,0.690196}%
\pgfsetstrokecolor{currentstroke}%
\pgfsetdash{}{0pt}%
\pgfsys@defobject{currentmarker}{\pgfqpoint{-0.027778in}{-0.027778in}}{\pgfqpoint{0.027778in}{0.027778in}}{%
\pgfpathmoveto{\pgfqpoint{0.000000in}{-0.027778in}}%
\pgfpathcurveto{\pgfqpoint{0.007367in}{-0.027778in}}{\pgfqpoint{0.014433in}{-0.024851in}}{\pgfqpoint{0.019642in}{-0.019642in}}%
\pgfpathcurveto{\pgfqpoint{0.024851in}{-0.014433in}}{\pgfqpoint{0.027778in}{-0.007367in}}{\pgfqpoint{0.027778in}{0.000000in}}%
\pgfpathcurveto{\pgfqpoint{0.027778in}{0.007367in}}{\pgfqpoint{0.024851in}{0.014433in}}{\pgfqpoint{0.019642in}{0.019642in}}%
\pgfpathcurveto{\pgfqpoint{0.014433in}{0.024851in}}{\pgfqpoint{0.007367in}{0.027778in}}{\pgfqpoint{0.000000in}{0.027778in}}%
\pgfpathcurveto{\pgfqpoint{-0.007367in}{0.027778in}}{\pgfqpoint{-0.014433in}{0.024851in}}{\pgfqpoint{-0.019642in}{0.019642in}}%
\pgfpathcurveto{\pgfqpoint{-0.024851in}{0.014433in}}{\pgfqpoint{-0.027778in}{0.007367in}}{\pgfqpoint{-0.027778in}{0.000000in}}%
\pgfpathcurveto{\pgfqpoint{-0.027778in}{-0.007367in}}{\pgfqpoint{-0.024851in}{-0.014433in}}{\pgfqpoint{-0.019642in}{-0.019642in}}%
\pgfpathcurveto{\pgfqpoint{-0.014433in}{-0.024851in}}{\pgfqpoint{-0.007367in}{-0.027778in}}{\pgfqpoint{0.000000in}{-0.027778in}}%
\pgfpathclose%
\pgfusepath{stroke,fill}%
}%
\begin{pgfscope}%
\pgfsys@transformshift{2.512543in}{4.067105in}%
\pgfsys@useobject{currentmarker}{}%
\end{pgfscope}%
\end{pgfscope}%
\begin{pgfscope}%
\pgfpathrectangle{\pgfqpoint{0.800000in}{0.528000in}}{\pgfqpoint{4.960000in}{3.696000in}}%
\pgfusepath{clip}%
\pgfsetbuttcap%
\pgfsetroundjoin%
\pgfsetlinewidth{1.505625pt}%
\definecolor{currentstroke}{rgb}{0.000000,0.000000,0.878431}%
\pgfsetstrokecolor{currentstroke}%
\pgfsetdash{{1.500000pt}{2.475000pt}}{0.000000pt}%
\pgfpathmoveto{\pgfqpoint{1.933751in}{2.865454in}}%
\pgfusepath{stroke}%
\end{pgfscope}%
\begin{pgfscope}%
\pgfpathrectangle{\pgfqpoint{0.800000in}{0.528000in}}{\pgfqpoint{4.960000in}{3.696000in}}%
\pgfusepath{clip}%
\pgfsetbuttcap%
\pgfsetroundjoin%
\definecolor{currentfill}{rgb}{0.000000,0.000000,0.878431}%
\pgfsetfillcolor{currentfill}%
\pgfsetlinewidth{0.501875pt}%
\definecolor{currentstroke}{rgb}{0.000000,0.000000,0.690196}%
\pgfsetstrokecolor{currentstroke}%
\pgfsetdash{}{0pt}%
\pgfsys@defobject{currentmarker}{\pgfqpoint{-0.027778in}{-0.027778in}}{\pgfqpoint{0.027778in}{0.027778in}}{%
\pgfpathmoveto{\pgfqpoint{0.000000in}{-0.027778in}}%
\pgfpathcurveto{\pgfqpoint{0.007367in}{-0.027778in}}{\pgfqpoint{0.014433in}{-0.024851in}}{\pgfqpoint{0.019642in}{-0.019642in}}%
\pgfpathcurveto{\pgfqpoint{0.024851in}{-0.014433in}}{\pgfqpoint{0.027778in}{-0.007367in}}{\pgfqpoint{0.027778in}{0.000000in}}%
\pgfpathcurveto{\pgfqpoint{0.027778in}{0.007367in}}{\pgfqpoint{0.024851in}{0.014433in}}{\pgfqpoint{0.019642in}{0.019642in}}%
\pgfpathcurveto{\pgfqpoint{0.014433in}{0.024851in}}{\pgfqpoint{0.007367in}{0.027778in}}{\pgfqpoint{0.000000in}{0.027778in}}%
\pgfpathcurveto{\pgfqpoint{-0.007367in}{0.027778in}}{\pgfqpoint{-0.014433in}{0.024851in}}{\pgfqpoint{-0.019642in}{0.019642in}}%
\pgfpathcurveto{\pgfqpoint{-0.024851in}{0.014433in}}{\pgfqpoint{-0.027778in}{0.007367in}}{\pgfqpoint{-0.027778in}{0.000000in}}%
\pgfpathcurveto{\pgfqpoint{-0.027778in}{-0.007367in}}{\pgfqpoint{-0.024851in}{-0.014433in}}{\pgfqpoint{-0.019642in}{-0.019642in}}%
\pgfpathcurveto{\pgfqpoint{-0.014433in}{-0.024851in}}{\pgfqpoint{-0.007367in}{-0.027778in}}{\pgfqpoint{0.000000in}{-0.027778in}}%
\pgfpathclose%
\pgfusepath{stroke,fill}%
}%
\begin{pgfscope}%
\pgfsys@transformshift{1.933751in}{2.865454in}%
\pgfsys@useobject{currentmarker}{}%
\end{pgfscope}%
\end{pgfscope}%
\begin{pgfscope}%
\pgfpathrectangle{\pgfqpoint{0.800000in}{0.528000in}}{\pgfqpoint{4.960000in}{3.696000in}}%
\pgfusepath{clip}%
\pgfsetbuttcap%
\pgfsetroundjoin%
\pgfsetlinewidth{1.505625pt}%
\definecolor{currentstroke}{rgb}{0.000000,0.000000,0.878431}%
\pgfsetstrokecolor{currentstroke}%
\pgfsetdash{{1.500000pt}{2.475000pt}}{0.000000pt}%
\pgfpathmoveto{\pgfqpoint{1.298434in}{2.500371in}}%
\pgfusepath{stroke}%
\end{pgfscope}%
\begin{pgfscope}%
\pgfpathrectangle{\pgfqpoint{0.800000in}{0.528000in}}{\pgfqpoint{4.960000in}{3.696000in}}%
\pgfusepath{clip}%
\pgfsetbuttcap%
\pgfsetroundjoin%
\definecolor{currentfill}{rgb}{0.000000,0.000000,0.878431}%
\pgfsetfillcolor{currentfill}%
\pgfsetlinewidth{0.501875pt}%
\definecolor{currentstroke}{rgb}{0.000000,0.000000,0.690196}%
\pgfsetstrokecolor{currentstroke}%
\pgfsetdash{}{0pt}%
\pgfsys@defobject{currentmarker}{\pgfqpoint{-0.027778in}{-0.027778in}}{\pgfqpoint{0.027778in}{0.027778in}}{%
\pgfpathmoveto{\pgfqpoint{0.000000in}{-0.027778in}}%
\pgfpathcurveto{\pgfqpoint{0.007367in}{-0.027778in}}{\pgfqpoint{0.014433in}{-0.024851in}}{\pgfqpoint{0.019642in}{-0.019642in}}%
\pgfpathcurveto{\pgfqpoint{0.024851in}{-0.014433in}}{\pgfqpoint{0.027778in}{-0.007367in}}{\pgfqpoint{0.027778in}{0.000000in}}%
\pgfpathcurveto{\pgfqpoint{0.027778in}{0.007367in}}{\pgfqpoint{0.024851in}{0.014433in}}{\pgfqpoint{0.019642in}{0.019642in}}%
\pgfpathcurveto{\pgfqpoint{0.014433in}{0.024851in}}{\pgfqpoint{0.007367in}{0.027778in}}{\pgfqpoint{0.000000in}{0.027778in}}%
\pgfpathcurveto{\pgfqpoint{-0.007367in}{0.027778in}}{\pgfqpoint{-0.014433in}{0.024851in}}{\pgfqpoint{-0.019642in}{0.019642in}}%
\pgfpathcurveto{\pgfqpoint{-0.024851in}{0.014433in}}{\pgfqpoint{-0.027778in}{0.007367in}}{\pgfqpoint{-0.027778in}{0.000000in}}%
\pgfpathcurveto{\pgfqpoint{-0.027778in}{-0.007367in}}{\pgfqpoint{-0.024851in}{-0.014433in}}{\pgfqpoint{-0.019642in}{-0.019642in}}%
\pgfpathcurveto{\pgfqpoint{-0.014433in}{-0.024851in}}{\pgfqpoint{-0.007367in}{-0.027778in}}{\pgfqpoint{0.000000in}{-0.027778in}}%
\pgfpathclose%
\pgfusepath{stroke,fill}%
}%
\begin{pgfscope}%
\pgfsys@transformshift{1.298434in}{2.500371in}%
\pgfsys@useobject{currentmarker}{}%
\end{pgfscope}%
\end{pgfscope}%
\begin{pgfscope}%
\pgfpathrectangle{\pgfqpoint{0.800000in}{0.528000in}}{\pgfqpoint{4.960000in}{3.696000in}}%
\pgfusepath{clip}%
\pgfsetbuttcap%
\pgfsetroundjoin%
\pgfsetlinewidth{1.505625pt}%
\definecolor{currentstroke}{rgb}{0.000000,0.000000,0.878431}%
\pgfsetstrokecolor{currentstroke}%
\pgfsetdash{{1.500000pt}{2.475000pt}}{0.000000pt}%
\pgfpathmoveto{\pgfqpoint{2.103100in}{3.519308in}}%
\pgfusepath{stroke}%
\end{pgfscope}%
\begin{pgfscope}%
\pgfpathrectangle{\pgfqpoint{0.800000in}{0.528000in}}{\pgfqpoint{4.960000in}{3.696000in}}%
\pgfusepath{clip}%
\pgfsetbuttcap%
\pgfsetroundjoin%
\definecolor{currentfill}{rgb}{0.000000,0.000000,0.878431}%
\pgfsetfillcolor{currentfill}%
\pgfsetlinewidth{0.501875pt}%
\definecolor{currentstroke}{rgb}{0.000000,0.000000,0.690196}%
\pgfsetstrokecolor{currentstroke}%
\pgfsetdash{}{0pt}%
\pgfsys@defobject{currentmarker}{\pgfqpoint{-0.027778in}{-0.027778in}}{\pgfqpoint{0.027778in}{0.027778in}}{%
\pgfpathmoveto{\pgfqpoint{0.000000in}{-0.027778in}}%
\pgfpathcurveto{\pgfqpoint{0.007367in}{-0.027778in}}{\pgfqpoint{0.014433in}{-0.024851in}}{\pgfqpoint{0.019642in}{-0.019642in}}%
\pgfpathcurveto{\pgfqpoint{0.024851in}{-0.014433in}}{\pgfqpoint{0.027778in}{-0.007367in}}{\pgfqpoint{0.027778in}{0.000000in}}%
\pgfpathcurveto{\pgfqpoint{0.027778in}{0.007367in}}{\pgfqpoint{0.024851in}{0.014433in}}{\pgfqpoint{0.019642in}{0.019642in}}%
\pgfpathcurveto{\pgfqpoint{0.014433in}{0.024851in}}{\pgfqpoint{0.007367in}{0.027778in}}{\pgfqpoint{0.000000in}{0.027778in}}%
\pgfpathcurveto{\pgfqpoint{-0.007367in}{0.027778in}}{\pgfqpoint{-0.014433in}{0.024851in}}{\pgfqpoint{-0.019642in}{0.019642in}}%
\pgfpathcurveto{\pgfqpoint{-0.024851in}{0.014433in}}{\pgfqpoint{-0.027778in}{0.007367in}}{\pgfqpoint{-0.027778in}{0.000000in}}%
\pgfpathcurveto{\pgfqpoint{-0.027778in}{-0.007367in}}{\pgfqpoint{-0.024851in}{-0.014433in}}{\pgfqpoint{-0.019642in}{-0.019642in}}%
\pgfpathcurveto{\pgfqpoint{-0.014433in}{-0.024851in}}{\pgfqpoint{-0.007367in}{-0.027778in}}{\pgfqpoint{0.000000in}{-0.027778in}}%
\pgfpathclose%
\pgfusepath{stroke,fill}%
}%
\begin{pgfscope}%
\pgfsys@transformshift{2.103100in}{3.519308in}%
\pgfsys@useobject{currentmarker}{}%
\end{pgfscope}%
\end{pgfscope}%
\begin{pgfscope}%
\pgfpathrectangle{\pgfqpoint{0.800000in}{0.528000in}}{\pgfqpoint{4.960000in}{3.696000in}}%
\pgfusepath{clip}%
\pgfsetbuttcap%
\pgfsetroundjoin%
\pgfsetlinewidth{1.505625pt}%
\definecolor{currentstroke}{rgb}{0.000000,0.000000,0.878431}%
\pgfsetstrokecolor{currentstroke}%
\pgfsetdash{{1.500000pt}{2.475000pt}}{0.000000pt}%
\pgfpathmoveto{\pgfqpoint{2.231698in}{2.457091in}}%
\pgfusepath{stroke}%
\end{pgfscope}%
\begin{pgfscope}%
\pgfpathrectangle{\pgfqpoint{0.800000in}{0.528000in}}{\pgfqpoint{4.960000in}{3.696000in}}%
\pgfusepath{clip}%
\pgfsetbuttcap%
\pgfsetroundjoin%
\definecolor{currentfill}{rgb}{0.000000,0.000000,0.878431}%
\pgfsetfillcolor{currentfill}%
\pgfsetlinewidth{0.501875pt}%
\definecolor{currentstroke}{rgb}{0.000000,0.000000,0.690196}%
\pgfsetstrokecolor{currentstroke}%
\pgfsetdash{}{0pt}%
\pgfsys@defobject{currentmarker}{\pgfqpoint{-0.027778in}{-0.027778in}}{\pgfqpoint{0.027778in}{0.027778in}}{%
\pgfpathmoveto{\pgfqpoint{0.000000in}{-0.027778in}}%
\pgfpathcurveto{\pgfqpoint{0.007367in}{-0.027778in}}{\pgfqpoint{0.014433in}{-0.024851in}}{\pgfqpoint{0.019642in}{-0.019642in}}%
\pgfpathcurveto{\pgfqpoint{0.024851in}{-0.014433in}}{\pgfqpoint{0.027778in}{-0.007367in}}{\pgfqpoint{0.027778in}{0.000000in}}%
\pgfpathcurveto{\pgfqpoint{0.027778in}{0.007367in}}{\pgfqpoint{0.024851in}{0.014433in}}{\pgfqpoint{0.019642in}{0.019642in}}%
\pgfpathcurveto{\pgfqpoint{0.014433in}{0.024851in}}{\pgfqpoint{0.007367in}{0.027778in}}{\pgfqpoint{0.000000in}{0.027778in}}%
\pgfpathcurveto{\pgfqpoint{-0.007367in}{0.027778in}}{\pgfqpoint{-0.014433in}{0.024851in}}{\pgfqpoint{-0.019642in}{0.019642in}}%
\pgfpathcurveto{\pgfqpoint{-0.024851in}{0.014433in}}{\pgfqpoint{-0.027778in}{0.007367in}}{\pgfqpoint{-0.027778in}{0.000000in}}%
\pgfpathcurveto{\pgfqpoint{-0.027778in}{-0.007367in}}{\pgfqpoint{-0.024851in}{-0.014433in}}{\pgfqpoint{-0.019642in}{-0.019642in}}%
\pgfpathcurveto{\pgfqpoint{-0.014433in}{-0.024851in}}{\pgfqpoint{-0.007367in}{-0.027778in}}{\pgfqpoint{0.000000in}{-0.027778in}}%
\pgfpathclose%
\pgfusepath{stroke,fill}%
}%
\begin{pgfscope}%
\pgfsys@transformshift{2.231698in}{2.457091in}%
\pgfsys@useobject{currentmarker}{}%
\end{pgfscope}%
\end{pgfscope}%
\begin{pgfscope}%
\pgfpathrectangle{\pgfqpoint{0.800000in}{0.528000in}}{\pgfqpoint{4.960000in}{3.696000in}}%
\pgfusepath{clip}%
\pgfsetbuttcap%
\pgfsetroundjoin%
\pgfsetlinewidth{1.505625pt}%
\definecolor{currentstroke}{rgb}{0.878431,0.000000,0.000000}%
\pgfsetstrokecolor{currentstroke}%
\pgfsetdash{{1.500000pt}{2.475000pt}}{0.000000pt}%
\pgfpathmoveto{\pgfqpoint{4.290526in}{1.734601in}}%
\pgfusepath{stroke}%
\end{pgfscope}%
\begin{pgfscope}%
\pgfpathrectangle{\pgfqpoint{0.800000in}{0.528000in}}{\pgfqpoint{4.960000in}{3.696000in}}%
\pgfusepath{clip}%
\pgfsetbuttcap%
\pgfsetroundjoin%
\definecolor{currentfill}{rgb}{0.878431,0.000000,0.000000}%
\pgfsetfillcolor{currentfill}%
\pgfsetlinewidth{0.501875pt}%
\definecolor{currentstroke}{rgb}{0.690196,0.000000,0.000000}%
\pgfsetstrokecolor{currentstroke}%
\pgfsetdash{}{0pt}%
\pgfsys@defobject{currentmarker}{\pgfqpoint{-0.027778in}{-0.027778in}}{\pgfqpoint{0.027778in}{0.027778in}}{%
\pgfpathmoveto{\pgfqpoint{0.000000in}{-0.027778in}}%
\pgfpathcurveto{\pgfqpoint{0.007367in}{-0.027778in}}{\pgfqpoint{0.014433in}{-0.024851in}}{\pgfqpoint{0.019642in}{-0.019642in}}%
\pgfpathcurveto{\pgfqpoint{0.024851in}{-0.014433in}}{\pgfqpoint{0.027778in}{-0.007367in}}{\pgfqpoint{0.027778in}{0.000000in}}%
\pgfpathcurveto{\pgfqpoint{0.027778in}{0.007367in}}{\pgfqpoint{0.024851in}{0.014433in}}{\pgfqpoint{0.019642in}{0.019642in}}%
\pgfpathcurveto{\pgfqpoint{0.014433in}{0.024851in}}{\pgfqpoint{0.007367in}{0.027778in}}{\pgfqpoint{0.000000in}{0.027778in}}%
\pgfpathcurveto{\pgfqpoint{-0.007367in}{0.027778in}}{\pgfqpoint{-0.014433in}{0.024851in}}{\pgfqpoint{-0.019642in}{0.019642in}}%
\pgfpathcurveto{\pgfqpoint{-0.024851in}{0.014433in}}{\pgfqpoint{-0.027778in}{0.007367in}}{\pgfqpoint{-0.027778in}{0.000000in}}%
\pgfpathcurveto{\pgfqpoint{-0.027778in}{-0.007367in}}{\pgfqpoint{-0.024851in}{-0.014433in}}{\pgfqpoint{-0.019642in}{-0.019642in}}%
\pgfpathcurveto{\pgfqpoint{-0.014433in}{-0.024851in}}{\pgfqpoint{-0.007367in}{-0.027778in}}{\pgfqpoint{0.000000in}{-0.027778in}}%
\pgfpathclose%
\pgfusepath{stroke,fill}%
}%
\begin{pgfscope}%
\pgfsys@transformshift{4.290526in}{1.734601in}%
\pgfsys@useobject{currentmarker}{}%
\end{pgfscope}%
\end{pgfscope}%
\begin{pgfscope}%
\pgfpathrectangle{\pgfqpoint{0.800000in}{0.528000in}}{\pgfqpoint{4.960000in}{3.696000in}}%
\pgfusepath{clip}%
\pgfsetbuttcap%
\pgfsetroundjoin%
\pgfsetlinewidth{1.505625pt}%
\definecolor{currentstroke}{rgb}{0.878431,0.000000,0.000000}%
\pgfsetstrokecolor{currentstroke}%
\pgfsetdash{{1.500000pt}{2.475000pt}}{0.000000pt}%
\pgfpathmoveto{\pgfqpoint{4.816969in}{1.776507in}}%
\pgfusepath{stroke}%
\end{pgfscope}%
\begin{pgfscope}%
\pgfpathrectangle{\pgfqpoint{0.800000in}{0.528000in}}{\pgfqpoint{4.960000in}{3.696000in}}%
\pgfusepath{clip}%
\pgfsetbuttcap%
\pgfsetroundjoin%
\definecolor{currentfill}{rgb}{0.878431,0.000000,0.000000}%
\pgfsetfillcolor{currentfill}%
\pgfsetlinewidth{0.501875pt}%
\definecolor{currentstroke}{rgb}{0.690196,0.000000,0.000000}%
\pgfsetstrokecolor{currentstroke}%
\pgfsetdash{}{0pt}%
\pgfsys@defobject{currentmarker}{\pgfqpoint{-0.027778in}{-0.027778in}}{\pgfqpoint{0.027778in}{0.027778in}}{%
\pgfpathmoveto{\pgfqpoint{0.000000in}{-0.027778in}}%
\pgfpathcurveto{\pgfqpoint{0.007367in}{-0.027778in}}{\pgfqpoint{0.014433in}{-0.024851in}}{\pgfqpoint{0.019642in}{-0.019642in}}%
\pgfpathcurveto{\pgfqpoint{0.024851in}{-0.014433in}}{\pgfqpoint{0.027778in}{-0.007367in}}{\pgfqpoint{0.027778in}{0.000000in}}%
\pgfpathcurveto{\pgfqpoint{0.027778in}{0.007367in}}{\pgfqpoint{0.024851in}{0.014433in}}{\pgfqpoint{0.019642in}{0.019642in}}%
\pgfpathcurveto{\pgfqpoint{0.014433in}{0.024851in}}{\pgfqpoint{0.007367in}{0.027778in}}{\pgfqpoint{0.000000in}{0.027778in}}%
\pgfpathcurveto{\pgfqpoint{-0.007367in}{0.027778in}}{\pgfqpoint{-0.014433in}{0.024851in}}{\pgfqpoint{-0.019642in}{0.019642in}}%
\pgfpathcurveto{\pgfqpoint{-0.024851in}{0.014433in}}{\pgfqpoint{-0.027778in}{0.007367in}}{\pgfqpoint{-0.027778in}{0.000000in}}%
\pgfpathcurveto{\pgfqpoint{-0.027778in}{-0.007367in}}{\pgfqpoint{-0.024851in}{-0.014433in}}{\pgfqpoint{-0.019642in}{-0.019642in}}%
\pgfpathcurveto{\pgfqpoint{-0.014433in}{-0.024851in}}{\pgfqpoint{-0.007367in}{-0.027778in}}{\pgfqpoint{0.000000in}{-0.027778in}}%
\pgfpathclose%
\pgfusepath{stroke,fill}%
}%
\begin{pgfscope}%
\pgfsys@transformshift{4.816969in}{1.776507in}%
\pgfsys@useobject{currentmarker}{}%
\end{pgfscope}%
\end{pgfscope}%
\begin{pgfscope}%
\pgfpathrectangle{\pgfqpoint{0.800000in}{0.528000in}}{\pgfqpoint{4.960000in}{3.696000in}}%
\pgfusepath{clip}%
\pgfsetbuttcap%
\pgfsetroundjoin%
\pgfsetlinewidth{1.505625pt}%
\definecolor{currentstroke}{rgb}{0.878431,0.000000,0.000000}%
\pgfsetstrokecolor{currentstroke}%
\pgfsetdash{{1.500000pt}{2.475000pt}}{0.000000pt}%
\pgfpathmoveto{\pgfqpoint{4.854795in}{0.892097in}}%
\pgfusepath{stroke}%
\end{pgfscope}%
\begin{pgfscope}%
\pgfpathrectangle{\pgfqpoint{0.800000in}{0.528000in}}{\pgfqpoint{4.960000in}{3.696000in}}%
\pgfusepath{clip}%
\pgfsetbuttcap%
\pgfsetroundjoin%
\definecolor{currentfill}{rgb}{0.878431,0.000000,0.000000}%
\pgfsetfillcolor{currentfill}%
\pgfsetlinewidth{0.501875pt}%
\definecolor{currentstroke}{rgb}{0.690196,0.000000,0.000000}%
\pgfsetstrokecolor{currentstroke}%
\pgfsetdash{}{0pt}%
\pgfsys@defobject{currentmarker}{\pgfqpoint{-0.027778in}{-0.027778in}}{\pgfqpoint{0.027778in}{0.027778in}}{%
\pgfpathmoveto{\pgfqpoint{0.000000in}{-0.027778in}}%
\pgfpathcurveto{\pgfqpoint{0.007367in}{-0.027778in}}{\pgfqpoint{0.014433in}{-0.024851in}}{\pgfqpoint{0.019642in}{-0.019642in}}%
\pgfpathcurveto{\pgfqpoint{0.024851in}{-0.014433in}}{\pgfqpoint{0.027778in}{-0.007367in}}{\pgfqpoint{0.027778in}{0.000000in}}%
\pgfpathcurveto{\pgfqpoint{0.027778in}{0.007367in}}{\pgfqpoint{0.024851in}{0.014433in}}{\pgfqpoint{0.019642in}{0.019642in}}%
\pgfpathcurveto{\pgfqpoint{0.014433in}{0.024851in}}{\pgfqpoint{0.007367in}{0.027778in}}{\pgfqpoint{0.000000in}{0.027778in}}%
\pgfpathcurveto{\pgfqpoint{-0.007367in}{0.027778in}}{\pgfqpoint{-0.014433in}{0.024851in}}{\pgfqpoint{-0.019642in}{0.019642in}}%
\pgfpathcurveto{\pgfqpoint{-0.024851in}{0.014433in}}{\pgfqpoint{-0.027778in}{0.007367in}}{\pgfqpoint{-0.027778in}{0.000000in}}%
\pgfpathcurveto{\pgfqpoint{-0.027778in}{-0.007367in}}{\pgfqpoint{-0.024851in}{-0.014433in}}{\pgfqpoint{-0.019642in}{-0.019642in}}%
\pgfpathcurveto{\pgfqpoint{-0.014433in}{-0.024851in}}{\pgfqpoint{-0.007367in}{-0.027778in}}{\pgfqpoint{0.000000in}{-0.027778in}}%
\pgfpathclose%
\pgfusepath{stroke,fill}%
}%
\begin{pgfscope}%
\pgfsys@transformshift{4.854795in}{0.892097in}%
\pgfsys@useobject{currentmarker}{}%
\end{pgfscope}%
\end{pgfscope}%
\begin{pgfscope}%
\pgfpathrectangle{\pgfqpoint{0.800000in}{0.528000in}}{\pgfqpoint{4.960000in}{3.696000in}}%
\pgfusepath{clip}%
\pgfsetbuttcap%
\pgfsetroundjoin%
\pgfsetlinewidth{1.505625pt}%
\definecolor{currentstroke}{rgb}{0.878431,0.000000,0.000000}%
\pgfsetstrokecolor{currentstroke}%
\pgfsetdash{{1.500000pt}{2.475000pt}}{0.000000pt}%
\pgfpathmoveto{\pgfqpoint{3.986616in}{1.363903in}}%
\pgfusepath{stroke}%
\end{pgfscope}%
\begin{pgfscope}%
\pgfpathrectangle{\pgfqpoint{0.800000in}{0.528000in}}{\pgfqpoint{4.960000in}{3.696000in}}%
\pgfusepath{clip}%
\pgfsetbuttcap%
\pgfsetroundjoin%
\definecolor{currentfill}{rgb}{0.878431,0.000000,0.000000}%
\pgfsetfillcolor{currentfill}%
\pgfsetlinewidth{0.501875pt}%
\definecolor{currentstroke}{rgb}{0.690196,0.000000,0.000000}%
\pgfsetstrokecolor{currentstroke}%
\pgfsetdash{}{0pt}%
\pgfsys@defobject{currentmarker}{\pgfqpoint{-0.027778in}{-0.027778in}}{\pgfqpoint{0.027778in}{0.027778in}}{%
\pgfpathmoveto{\pgfqpoint{0.000000in}{-0.027778in}}%
\pgfpathcurveto{\pgfqpoint{0.007367in}{-0.027778in}}{\pgfqpoint{0.014433in}{-0.024851in}}{\pgfqpoint{0.019642in}{-0.019642in}}%
\pgfpathcurveto{\pgfqpoint{0.024851in}{-0.014433in}}{\pgfqpoint{0.027778in}{-0.007367in}}{\pgfqpoint{0.027778in}{0.000000in}}%
\pgfpathcurveto{\pgfqpoint{0.027778in}{0.007367in}}{\pgfqpoint{0.024851in}{0.014433in}}{\pgfqpoint{0.019642in}{0.019642in}}%
\pgfpathcurveto{\pgfqpoint{0.014433in}{0.024851in}}{\pgfqpoint{0.007367in}{0.027778in}}{\pgfqpoint{0.000000in}{0.027778in}}%
\pgfpathcurveto{\pgfqpoint{-0.007367in}{0.027778in}}{\pgfqpoint{-0.014433in}{0.024851in}}{\pgfqpoint{-0.019642in}{0.019642in}}%
\pgfpathcurveto{\pgfqpoint{-0.024851in}{0.014433in}}{\pgfqpoint{-0.027778in}{0.007367in}}{\pgfqpoint{-0.027778in}{0.000000in}}%
\pgfpathcurveto{\pgfqpoint{-0.027778in}{-0.007367in}}{\pgfqpoint{-0.024851in}{-0.014433in}}{\pgfqpoint{-0.019642in}{-0.019642in}}%
\pgfpathcurveto{\pgfqpoint{-0.014433in}{-0.024851in}}{\pgfqpoint{-0.007367in}{-0.027778in}}{\pgfqpoint{0.000000in}{-0.027778in}}%
\pgfpathclose%
\pgfusepath{stroke,fill}%
}%
\begin{pgfscope}%
\pgfsys@transformshift{3.986616in}{1.363903in}%
\pgfsys@useobject{currentmarker}{}%
\end{pgfscope}%
\end{pgfscope}%
\begin{pgfscope}%
\pgfpathrectangle{\pgfqpoint{0.800000in}{0.528000in}}{\pgfqpoint{4.960000in}{3.696000in}}%
\pgfusepath{clip}%
\pgfsetbuttcap%
\pgfsetroundjoin%
\pgfsetlinewidth{1.505625pt}%
\definecolor{currentstroke}{rgb}{0.878431,0.000000,0.000000}%
\pgfsetstrokecolor{currentstroke}%
\pgfsetdash{{1.500000pt}{2.475000pt}}{0.000000pt}%
\pgfpathmoveto{\pgfqpoint{5.552996in}{2.003145in}}%
\pgfusepath{stroke}%
\end{pgfscope}%
\begin{pgfscope}%
\pgfpathrectangle{\pgfqpoint{0.800000in}{0.528000in}}{\pgfqpoint{4.960000in}{3.696000in}}%
\pgfusepath{clip}%
\pgfsetbuttcap%
\pgfsetroundjoin%
\definecolor{currentfill}{rgb}{0.878431,0.000000,0.000000}%
\pgfsetfillcolor{currentfill}%
\pgfsetlinewidth{0.501875pt}%
\definecolor{currentstroke}{rgb}{0.690196,0.000000,0.000000}%
\pgfsetstrokecolor{currentstroke}%
\pgfsetdash{}{0pt}%
\pgfsys@defobject{currentmarker}{\pgfqpoint{-0.027778in}{-0.027778in}}{\pgfqpoint{0.027778in}{0.027778in}}{%
\pgfpathmoveto{\pgfqpoint{0.000000in}{-0.027778in}}%
\pgfpathcurveto{\pgfqpoint{0.007367in}{-0.027778in}}{\pgfqpoint{0.014433in}{-0.024851in}}{\pgfqpoint{0.019642in}{-0.019642in}}%
\pgfpathcurveto{\pgfqpoint{0.024851in}{-0.014433in}}{\pgfqpoint{0.027778in}{-0.007367in}}{\pgfqpoint{0.027778in}{0.000000in}}%
\pgfpathcurveto{\pgfqpoint{0.027778in}{0.007367in}}{\pgfqpoint{0.024851in}{0.014433in}}{\pgfqpoint{0.019642in}{0.019642in}}%
\pgfpathcurveto{\pgfqpoint{0.014433in}{0.024851in}}{\pgfqpoint{0.007367in}{0.027778in}}{\pgfqpoint{0.000000in}{0.027778in}}%
\pgfpathcurveto{\pgfqpoint{-0.007367in}{0.027778in}}{\pgfqpoint{-0.014433in}{0.024851in}}{\pgfqpoint{-0.019642in}{0.019642in}}%
\pgfpathcurveto{\pgfqpoint{-0.024851in}{0.014433in}}{\pgfqpoint{-0.027778in}{0.007367in}}{\pgfqpoint{-0.027778in}{0.000000in}}%
\pgfpathcurveto{\pgfqpoint{-0.027778in}{-0.007367in}}{\pgfqpoint{-0.024851in}{-0.014433in}}{\pgfqpoint{-0.019642in}{-0.019642in}}%
\pgfpathcurveto{\pgfqpoint{-0.014433in}{-0.024851in}}{\pgfqpoint{-0.007367in}{-0.027778in}}{\pgfqpoint{0.000000in}{-0.027778in}}%
\pgfpathclose%
\pgfusepath{stroke,fill}%
}%
\begin{pgfscope}%
\pgfsys@transformshift{5.552996in}{2.003145in}%
\pgfsys@useobject{currentmarker}{}%
\end{pgfscope}%
\end{pgfscope}%
\begin{pgfscope}%
\pgfpathrectangle{\pgfqpoint{0.800000in}{0.528000in}}{\pgfqpoint{4.960000in}{3.696000in}}%
\pgfusepath{clip}%
\pgfsetbuttcap%
\pgfsetroundjoin%
\pgfsetlinewidth{1.505625pt}%
\definecolor{currentstroke}{rgb}{0.878431,0.000000,0.000000}%
\pgfsetstrokecolor{currentstroke}%
\pgfsetdash{{1.500000pt}{2.475000pt}}{0.000000pt}%
\pgfpathmoveto{\pgfqpoint{4.059422in}{1.997913in}}%
\pgfusepath{stroke}%
\end{pgfscope}%
\begin{pgfscope}%
\pgfpathrectangle{\pgfqpoint{0.800000in}{0.528000in}}{\pgfqpoint{4.960000in}{3.696000in}}%
\pgfusepath{clip}%
\pgfsetbuttcap%
\pgfsetroundjoin%
\definecolor{currentfill}{rgb}{0.878431,0.000000,0.000000}%
\pgfsetfillcolor{currentfill}%
\pgfsetlinewidth{0.501875pt}%
\definecolor{currentstroke}{rgb}{0.690196,0.000000,0.000000}%
\pgfsetstrokecolor{currentstroke}%
\pgfsetdash{}{0pt}%
\pgfsys@defobject{currentmarker}{\pgfqpoint{-0.027778in}{-0.027778in}}{\pgfqpoint{0.027778in}{0.027778in}}{%
\pgfpathmoveto{\pgfqpoint{0.000000in}{-0.027778in}}%
\pgfpathcurveto{\pgfqpoint{0.007367in}{-0.027778in}}{\pgfqpoint{0.014433in}{-0.024851in}}{\pgfqpoint{0.019642in}{-0.019642in}}%
\pgfpathcurveto{\pgfqpoint{0.024851in}{-0.014433in}}{\pgfqpoint{0.027778in}{-0.007367in}}{\pgfqpoint{0.027778in}{0.000000in}}%
\pgfpathcurveto{\pgfqpoint{0.027778in}{0.007367in}}{\pgfqpoint{0.024851in}{0.014433in}}{\pgfqpoint{0.019642in}{0.019642in}}%
\pgfpathcurveto{\pgfqpoint{0.014433in}{0.024851in}}{\pgfqpoint{0.007367in}{0.027778in}}{\pgfqpoint{0.000000in}{0.027778in}}%
\pgfpathcurveto{\pgfqpoint{-0.007367in}{0.027778in}}{\pgfqpoint{-0.014433in}{0.024851in}}{\pgfqpoint{-0.019642in}{0.019642in}}%
\pgfpathcurveto{\pgfqpoint{-0.024851in}{0.014433in}}{\pgfqpoint{-0.027778in}{0.007367in}}{\pgfqpoint{-0.027778in}{0.000000in}}%
\pgfpathcurveto{\pgfqpoint{-0.027778in}{-0.007367in}}{\pgfqpoint{-0.024851in}{-0.014433in}}{\pgfqpoint{-0.019642in}{-0.019642in}}%
\pgfpathcurveto{\pgfqpoint{-0.014433in}{-0.024851in}}{\pgfqpoint{-0.007367in}{-0.027778in}}{\pgfqpoint{0.000000in}{-0.027778in}}%
\pgfpathclose%
\pgfusepath{stroke,fill}%
}%
\begin{pgfscope}%
\pgfsys@transformshift{4.059422in}{1.997913in}%
\pgfsys@useobject{currentmarker}{}%
\end{pgfscope}%
\end{pgfscope}%
\begin{pgfscope}%
\pgfpathrectangle{\pgfqpoint{0.800000in}{0.528000in}}{\pgfqpoint{4.960000in}{3.696000in}}%
\pgfusepath{clip}%
\pgfsetbuttcap%
\pgfsetroundjoin%
\pgfsetlinewidth{1.505625pt}%
\definecolor{currentstroke}{rgb}{0.878431,0.000000,0.000000}%
\pgfsetstrokecolor{currentstroke}%
\pgfsetdash{{1.500000pt}{2.475000pt}}{0.000000pt}%
\pgfpathmoveto{\pgfqpoint{4.794767in}{1.326648in}}%
\pgfusepath{stroke}%
\end{pgfscope}%
\begin{pgfscope}%
\pgfpathrectangle{\pgfqpoint{0.800000in}{0.528000in}}{\pgfqpoint{4.960000in}{3.696000in}}%
\pgfusepath{clip}%
\pgfsetbuttcap%
\pgfsetroundjoin%
\definecolor{currentfill}{rgb}{0.878431,0.000000,0.000000}%
\pgfsetfillcolor{currentfill}%
\pgfsetlinewidth{0.501875pt}%
\definecolor{currentstroke}{rgb}{0.690196,0.000000,0.000000}%
\pgfsetstrokecolor{currentstroke}%
\pgfsetdash{}{0pt}%
\pgfsys@defobject{currentmarker}{\pgfqpoint{-0.027778in}{-0.027778in}}{\pgfqpoint{0.027778in}{0.027778in}}{%
\pgfpathmoveto{\pgfqpoint{0.000000in}{-0.027778in}}%
\pgfpathcurveto{\pgfqpoint{0.007367in}{-0.027778in}}{\pgfqpoint{0.014433in}{-0.024851in}}{\pgfqpoint{0.019642in}{-0.019642in}}%
\pgfpathcurveto{\pgfqpoint{0.024851in}{-0.014433in}}{\pgfqpoint{0.027778in}{-0.007367in}}{\pgfqpoint{0.027778in}{0.000000in}}%
\pgfpathcurveto{\pgfqpoint{0.027778in}{0.007367in}}{\pgfqpoint{0.024851in}{0.014433in}}{\pgfqpoint{0.019642in}{0.019642in}}%
\pgfpathcurveto{\pgfqpoint{0.014433in}{0.024851in}}{\pgfqpoint{0.007367in}{0.027778in}}{\pgfqpoint{0.000000in}{0.027778in}}%
\pgfpathcurveto{\pgfqpoint{-0.007367in}{0.027778in}}{\pgfqpoint{-0.014433in}{0.024851in}}{\pgfqpoint{-0.019642in}{0.019642in}}%
\pgfpathcurveto{\pgfqpoint{-0.024851in}{0.014433in}}{\pgfqpoint{-0.027778in}{0.007367in}}{\pgfqpoint{-0.027778in}{0.000000in}}%
\pgfpathcurveto{\pgfqpoint{-0.027778in}{-0.007367in}}{\pgfqpoint{-0.024851in}{-0.014433in}}{\pgfqpoint{-0.019642in}{-0.019642in}}%
\pgfpathcurveto{\pgfqpoint{-0.014433in}{-0.024851in}}{\pgfqpoint{-0.007367in}{-0.027778in}}{\pgfqpoint{0.000000in}{-0.027778in}}%
\pgfpathclose%
\pgfusepath{stroke,fill}%
}%
\begin{pgfscope}%
\pgfsys@transformshift{4.794767in}{1.326648in}%
\pgfsys@useobject{currentmarker}{}%
\end{pgfscope}%
\end{pgfscope}%
\begin{pgfscope}%
\pgfpathrectangle{\pgfqpoint{0.800000in}{0.528000in}}{\pgfqpoint{4.960000in}{3.696000in}}%
\pgfusepath{clip}%
\pgfsetbuttcap%
\pgfsetroundjoin%
\pgfsetlinewidth{1.505625pt}%
\definecolor{currentstroke}{rgb}{0.878431,0.000000,0.000000}%
\pgfsetstrokecolor{currentstroke}%
\pgfsetdash{{1.500000pt}{2.475000pt}}{0.000000pt}%
\pgfpathmoveto{\pgfqpoint{5.240260in}{1.543678in}}%
\pgfusepath{stroke}%
\end{pgfscope}%
\begin{pgfscope}%
\pgfpathrectangle{\pgfqpoint{0.800000in}{0.528000in}}{\pgfqpoint{4.960000in}{3.696000in}}%
\pgfusepath{clip}%
\pgfsetbuttcap%
\pgfsetroundjoin%
\definecolor{currentfill}{rgb}{0.878431,0.000000,0.000000}%
\pgfsetfillcolor{currentfill}%
\pgfsetlinewidth{0.501875pt}%
\definecolor{currentstroke}{rgb}{0.690196,0.000000,0.000000}%
\pgfsetstrokecolor{currentstroke}%
\pgfsetdash{}{0pt}%
\pgfsys@defobject{currentmarker}{\pgfqpoint{-0.027778in}{-0.027778in}}{\pgfqpoint{0.027778in}{0.027778in}}{%
\pgfpathmoveto{\pgfqpoint{0.000000in}{-0.027778in}}%
\pgfpathcurveto{\pgfqpoint{0.007367in}{-0.027778in}}{\pgfqpoint{0.014433in}{-0.024851in}}{\pgfqpoint{0.019642in}{-0.019642in}}%
\pgfpathcurveto{\pgfqpoint{0.024851in}{-0.014433in}}{\pgfqpoint{0.027778in}{-0.007367in}}{\pgfqpoint{0.027778in}{0.000000in}}%
\pgfpathcurveto{\pgfqpoint{0.027778in}{0.007367in}}{\pgfqpoint{0.024851in}{0.014433in}}{\pgfqpoint{0.019642in}{0.019642in}}%
\pgfpathcurveto{\pgfqpoint{0.014433in}{0.024851in}}{\pgfqpoint{0.007367in}{0.027778in}}{\pgfqpoint{0.000000in}{0.027778in}}%
\pgfpathcurveto{\pgfqpoint{-0.007367in}{0.027778in}}{\pgfqpoint{-0.014433in}{0.024851in}}{\pgfqpoint{-0.019642in}{0.019642in}}%
\pgfpathcurveto{\pgfqpoint{-0.024851in}{0.014433in}}{\pgfqpoint{-0.027778in}{0.007367in}}{\pgfqpoint{-0.027778in}{0.000000in}}%
\pgfpathcurveto{\pgfqpoint{-0.027778in}{-0.007367in}}{\pgfqpoint{-0.024851in}{-0.014433in}}{\pgfqpoint{-0.019642in}{-0.019642in}}%
\pgfpathcurveto{\pgfqpoint{-0.014433in}{-0.024851in}}{\pgfqpoint{-0.007367in}{-0.027778in}}{\pgfqpoint{0.000000in}{-0.027778in}}%
\pgfpathclose%
\pgfusepath{stroke,fill}%
}%
\begin{pgfscope}%
\pgfsys@transformshift{5.240260in}{1.543678in}%
\pgfsys@useobject{currentmarker}{}%
\end{pgfscope}%
\end{pgfscope}%
\begin{pgfscope}%
\pgfpathrectangle{\pgfqpoint{0.800000in}{0.528000in}}{\pgfqpoint{4.960000in}{3.696000in}}%
\pgfusepath{clip}%
\pgfsetbuttcap%
\pgfsetroundjoin%
\pgfsetlinewidth{1.505625pt}%
\definecolor{currentstroke}{rgb}{0.878431,0.000000,0.000000}%
\pgfsetstrokecolor{currentstroke}%
\pgfsetdash{{1.500000pt}{2.475000pt}}{0.000000pt}%
\pgfpathmoveto{\pgfqpoint{5.204472in}{1.554398in}}%
\pgfusepath{stroke}%
\end{pgfscope}%
\begin{pgfscope}%
\pgfpathrectangle{\pgfqpoint{0.800000in}{0.528000in}}{\pgfqpoint{4.960000in}{3.696000in}}%
\pgfusepath{clip}%
\pgfsetbuttcap%
\pgfsetroundjoin%
\definecolor{currentfill}{rgb}{0.878431,0.000000,0.000000}%
\pgfsetfillcolor{currentfill}%
\pgfsetlinewidth{0.501875pt}%
\definecolor{currentstroke}{rgb}{0.690196,0.000000,0.000000}%
\pgfsetstrokecolor{currentstroke}%
\pgfsetdash{}{0pt}%
\pgfsys@defobject{currentmarker}{\pgfqpoint{-0.027778in}{-0.027778in}}{\pgfqpoint{0.027778in}{0.027778in}}{%
\pgfpathmoveto{\pgfqpoint{0.000000in}{-0.027778in}}%
\pgfpathcurveto{\pgfqpoint{0.007367in}{-0.027778in}}{\pgfqpoint{0.014433in}{-0.024851in}}{\pgfqpoint{0.019642in}{-0.019642in}}%
\pgfpathcurveto{\pgfqpoint{0.024851in}{-0.014433in}}{\pgfqpoint{0.027778in}{-0.007367in}}{\pgfqpoint{0.027778in}{0.000000in}}%
\pgfpathcurveto{\pgfqpoint{0.027778in}{0.007367in}}{\pgfqpoint{0.024851in}{0.014433in}}{\pgfqpoint{0.019642in}{0.019642in}}%
\pgfpathcurveto{\pgfqpoint{0.014433in}{0.024851in}}{\pgfqpoint{0.007367in}{0.027778in}}{\pgfqpoint{0.000000in}{0.027778in}}%
\pgfpathcurveto{\pgfqpoint{-0.007367in}{0.027778in}}{\pgfqpoint{-0.014433in}{0.024851in}}{\pgfqpoint{-0.019642in}{0.019642in}}%
\pgfpathcurveto{\pgfqpoint{-0.024851in}{0.014433in}}{\pgfqpoint{-0.027778in}{0.007367in}}{\pgfqpoint{-0.027778in}{0.000000in}}%
\pgfpathcurveto{\pgfqpoint{-0.027778in}{-0.007367in}}{\pgfqpoint{-0.024851in}{-0.014433in}}{\pgfqpoint{-0.019642in}{-0.019642in}}%
\pgfpathcurveto{\pgfqpoint{-0.014433in}{-0.024851in}}{\pgfqpoint{-0.007367in}{-0.027778in}}{\pgfqpoint{0.000000in}{-0.027778in}}%
\pgfpathclose%
\pgfusepath{stroke,fill}%
}%
\begin{pgfscope}%
\pgfsys@transformshift{5.204472in}{1.554398in}%
\pgfsys@useobject{currentmarker}{}%
\end{pgfscope}%
\end{pgfscope}%
\begin{pgfscope}%
\pgfpathrectangle{\pgfqpoint{0.800000in}{0.528000in}}{\pgfqpoint{4.960000in}{3.696000in}}%
\pgfusepath{clip}%
\pgfsetbuttcap%
\pgfsetroundjoin%
\pgfsetlinewidth{1.505625pt}%
\definecolor{currentstroke}{rgb}{0.878431,0.000000,0.000000}%
\pgfsetstrokecolor{currentstroke}%
\pgfsetdash{{1.500000pt}{2.475000pt}}{0.000000pt}%
\pgfpathmoveto{\pgfqpoint{4.881477in}{2.095898in}}%
\pgfusepath{stroke}%
\end{pgfscope}%
\begin{pgfscope}%
\pgfpathrectangle{\pgfqpoint{0.800000in}{0.528000in}}{\pgfqpoint{4.960000in}{3.696000in}}%
\pgfusepath{clip}%
\pgfsetbuttcap%
\pgfsetroundjoin%
\definecolor{currentfill}{rgb}{0.878431,0.000000,0.000000}%
\pgfsetfillcolor{currentfill}%
\pgfsetlinewidth{0.501875pt}%
\definecolor{currentstroke}{rgb}{0.690196,0.000000,0.000000}%
\pgfsetstrokecolor{currentstroke}%
\pgfsetdash{}{0pt}%
\pgfsys@defobject{currentmarker}{\pgfqpoint{-0.027778in}{-0.027778in}}{\pgfqpoint{0.027778in}{0.027778in}}{%
\pgfpathmoveto{\pgfqpoint{0.000000in}{-0.027778in}}%
\pgfpathcurveto{\pgfqpoint{0.007367in}{-0.027778in}}{\pgfqpoint{0.014433in}{-0.024851in}}{\pgfqpoint{0.019642in}{-0.019642in}}%
\pgfpathcurveto{\pgfqpoint{0.024851in}{-0.014433in}}{\pgfqpoint{0.027778in}{-0.007367in}}{\pgfqpoint{0.027778in}{0.000000in}}%
\pgfpathcurveto{\pgfqpoint{0.027778in}{0.007367in}}{\pgfqpoint{0.024851in}{0.014433in}}{\pgfqpoint{0.019642in}{0.019642in}}%
\pgfpathcurveto{\pgfqpoint{0.014433in}{0.024851in}}{\pgfqpoint{0.007367in}{0.027778in}}{\pgfqpoint{0.000000in}{0.027778in}}%
\pgfpathcurveto{\pgfqpoint{-0.007367in}{0.027778in}}{\pgfqpoint{-0.014433in}{0.024851in}}{\pgfqpoint{-0.019642in}{0.019642in}}%
\pgfpathcurveto{\pgfqpoint{-0.024851in}{0.014433in}}{\pgfqpoint{-0.027778in}{0.007367in}}{\pgfqpoint{-0.027778in}{0.000000in}}%
\pgfpathcurveto{\pgfqpoint{-0.027778in}{-0.007367in}}{\pgfqpoint{-0.024851in}{-0.014433in}}{\pgfqpoint{-0.019642in}{-0.019642in}}%
\pgfpathcurveto{\pgfqpoint{-0.014433in}{-0.024851in}}{\pgfqpoint{-0.007367in}{-0.027778in}}{\pgfqpoint{0.000000in}{-0.027778in}}%
\pgfpathclose%
\pgfusepath{stroke,fill}%
}%
\begin{pgfscope}%
\pgfsys@transformshift{4.881477in}{2.095898in}%
\pgfsys@useobject{currentmarker}{}%
\end{pgfscope}%
\end{pgfscope}%
\begin{pgfscope}%
\pgfpathrectangle{\pgfqpoint{0.800000in}{0.528000in}}{\pgfqpoint{4.960000in}{3.696000in}}%
\pgfusepath{clip}%
\pgfsetbuttcap%
\pgfsetroundjoin%
\pgfsetlinewidth{1.505625pt}%
\definecolor{currentstroke}{rgb}{0.878431,0.000000,0.000000}%
\pgfsetstrokecolor{currentstroke}%
\pgfsetdash{{1.500000pt}{2.475000pt}}{0.000000pt}%
\pgfpathmoveto{\pgfqpoint{3.809853in}{2.328804in}}%
\pgfusepath{stroke}%
\end{pgfscope}%
\begin{pgfscope}%
\pgfpathrectangle{\pgfqpoint{0.800000in}{0.528000in}}{\pgfqpoint{4.960000in}{3.696000in}}%
\pgfusepath{clip}%
\pgfsetbuttcap%
\pgfsetroundjoin%
\definecolor{currentfill}{rgb}{0.878431,0.000000,0.000000}%
\pgfsetfillcolor{currentfill}%
\pgfsetlinewidth{0.501875pt}%
\definecolor{currentstroke}{rgb}{0.690196,0.000000,0.000000}%
\pgfsetstrokecolor{currentstroke}%
\pgfsetdash{}{0pt}%
\pgfsys@defobject{currentmarker}{\pgfqpoint{-0.027778in}{-0.027778in}}{\pgfqpoint{0.027778in}{0.027778in}}{%
\pgfpathmoveto{\pgfqpoint{0.000000in}{-0.027778in}}%
\pgfpathcurveto{\pgfqpoint{0.007367in}{-0.027778in}}{\pgfqpoint{0.014433in}{-0.024851in}}{\pgfqpoint{0.019642in}{-0.019642in}}%
\pgfpathcurveto{\pgfqpoint{0.024851in}{-0.014433in}}{\pgfqpoint{0.027778in}{-0.007367in}}{\pgfqpoint{0.027778in}{0.000000in}}%
\pgfpathcurveto{\pgfqpoint{0.027778in}{0.007367in}}{\pgfqpoint{0.024851in}{0.014433in}}{\pgfqpoint{0.019642in}{0.019642in}}%
\pgfpathcurveto{\pgfqpoint{0.014433in}{0.024851in}}{\pgfqpoint{0.007367in}{0.027778in}}{\pgfqpoint{0.000000in}{0.027778in}}%
\pgfpathcurveto{\pgfqpoint{-0.007367in}{0.027778in}}{\pgfqpoint{-0.014433in}{0.024851in}}{\pgfqpoint{-0.019642in}{0.019642in}}%
\pgfpathcurveto{\pgfqpoint{-0.024851in}{0.014433in}}{\pgfqpoint{-0.027778in}{0.007367in}}{\pgfqpoint{-0.027778in}{0.000000in}}%
\pgfpathcurveto{\pgfqpoint{-0.027778in}{-0.007367in}}{\pgfqpoint{-0.024851in}{-0.014433in}}{\pgfqpoint{-0.019642in}{-0.019642in}}%
\pgfpathcurveto{\pgfqpoint{-0.014433in}{-0.024851in}}{\pgfqpoint{-0.007367in}{-0.027778in}}{\pgfqpoint{0.000000in}{-0.027778in}}%
\pgfpathclose%
\pgfusepath{stroke,fill}%
}%
\begin{pgfscope}%
\pgfsys@transformshift{3.809853in}{2.328804in}%
\pgfsys@useobject{currentmarker}{}%
\end{pgfscope}%
\end{pgfscope}%
\begin{pgfscope}%
\pgfpathrectangle{\pgfqpoint{0.800000in}{0.528000in}}{\pgfqpoint{4.960000in}{3.696000in}}%
\pgfusepath{clip}%
\pgfsetbuttcap%
\pgfsetroundjoin%
\pgfsetlinewidth{1.505625pt}%
\definecolor{currentstroke}{rgb}{0.878431,0.000000,0.000000}%
\pgfsetstrokecolor{currentstroke}%
\pgfsetdash{{1.500000pt}{2.475000pt}}{0.000000pt}%
\pgfpathmoveto{\pgfqpoint{4.592108in}{1.504572in}}%
\pgfusepath{stroke}%
\end{pgfscope}%
\begin{pgfscope}%
\pgfpathrectangle{\pgfqpoint{0.800000in}{0.528000in}}{\pgfqpoint{4.960000in}{3.696000in}}%
\pgfusepath{clip}%
\pgfsetbuttcap%
\pgfsetroundjoin%
\definecolor{currentfill}{rgb}{0.878431,0.000000,0.000000}%
\pgfsetfillcolor{currentfill}%
\pgfsetlinewidth{0.501875pt}%
\definecolor{currentstroke}{rgb}{0.690196,0.000000,0.000000}%
\pgfsetstrokecolor{currentstroke}%
\pgfsetdash{}{0pt}%
\pgfsys@defobject{currentmarker}{\pgfqpoint{-0.027778in}{-0.027778in}}{\pgfqpoint{0.027778in}{0.027778in}}{%
\pgfpathmoveto{\pgfqpoint{0.000000in}{-0.027778in}}%
\pgfpathcurveto{\pgfqpoint{0.007367in}{-0.027778in}}{\pgfqpoint{0.014433in}{-0.024851in}}{\pgfqpoint{0.019642in}{-0.019642in}}%
\pgfpathcurveto{\pgfqpoint{0.024851in}{-0.014433in}}{\pgfqpoint{0.027778in}{-0.007367in}}{\pgfqpoint{0.027778in}{0.000000in}}%
\pgfpathcurveto{\pgfqpoint{0.027778in}{0.007367in}}{\pgfqpoint{0.024851in}{0.014433in}}{\pgfqpoint{0.019642in}{0.019642in}}%
\pgfpathcurveto{\pgfqpoint{0.014433in}{0.024851in}}{\pgfqpoint{0.007367in}{0.027778in}}{\pgfqpoint{0.000000in}{0.027778in}}%
\pgfpathcurveto{\pgfqpoint{-0.007367in}{0.027778in}}{\pgfqpoint{-0.014433in}{0.024851in}}{\pgfqpoint{-0.019642in}{0.019642in}}%
\pgfpathcurveto{\pgfqpoint{-0.024851in}{0.014433in}}{\pgfqpoint{-0.027778in}{0.007367in}}{\pgfqpoint{-0.027778in}{0.000000in}}%
\pgfpathcurveto{\pgfqpoint{-0.027778in}{-0.007367in}}{\pgfqpoint{-0.024851in}{-0.014433in}}{\pgfqpoint{-0.019642in}{-0.019642in}}%
\pgfpathcurveto{\pgfqpoint{-0.014433in}{-0.024851in}}{\pgfqpoint{-0.007367in}{-0.027778in}}{\pgfqpoint{0.000000in}{-0.027778in}}%
\pgfpathclose%
\pgfusepath{stroke,fill}%
}%
\begin{pgfscope}%
\pgfsys@transformshift{4.592108in}{1.504572in}%
\pgfsys@useobject{currentmarker}{}%
\end{pgfscope}%
\end{pgfscope}%
\begin{pgfscope}%
\pgfpathrectangle{\pgfqpoint{0.800000in}{0.528000in}}{\pgfqpoint{4.960000in}{3.696000in}}%
\pgfusepath{clip}%
\pgfsetbuttcap%
\pgfsetroundjoin%
\pgfsetlinewidth{1.505625pt}%
\definecolor{currentstroke}{rgb}{0.878431,0.000000,0.000000}%
\pgfsetstrokecolor{currentstroke}%
\pgfsetdash{{1.500000pt}{2.475000pt}}{0.000000pt}%
\pgfpathmoveto{\pgfqpoint{5.202797in}{0.621202in}}%
\pgfusepath{stroke}%
\end{pgfscope}%
\begin{pgfscope}%
\pgfpathrectangle{\pgfqpoint{0.800000in}{0.528000in}}{\pgfqpoint{4.960000in}{3.696000in}}%
\pgfusepath{clip}%
\pgfsetbuttcap%
\pgfsetroundjoin%
\definecolor{currentfill}{rgb}{0.878431,0.000000,0.000000}%
\pgfsetfillcolor{currentfill}%
\pgfsetlinewidth{0.501875pt}%
\definecolor{currentstroke}{rgb}{0.690196,0.000000,0.000000}%
\pgfsetstrokecolor{currentstroke}%
\pgfsetdash{}{0pt}%
\pgfsys@defobject{currentmarker}{\pgfqpoint{-0.027778in}{-0.027778in}}{\pgfqpoint{0.027778in}{0.027778in}}{%
\pgfpathmoveto{\pgfqpoint{0.000000in}{-0.027778in}}%
\pgfpathcurveto{\pgfqpoint{0.007367in}{-0.027778in}}{\pgfqpoint{0.014433in}{-0.024851in}}{\pgfqpoint{0.019642in}{-0.019642in}}%
\pgfpathcurveto{\pgfqpoint{0.024851in}{-0.014433in}}{\pgfqpoint{0.027778in}{-0.007367in}}{\pgfqpoint{0.027778in}{0.000000in}}%
\pgfpathcurveto{\pgfqpoint{0.027778in}{0.007367in}}{\pgfqpoint{0.024851in}{0.014433in}}{\pgfqpoint{0.019642in}{0.019642in}}%
\pgfpathcurveto{\pgfqpoint{0.014433in}{0.024851in}}{\pgfqpoint{0.007367in}{0.027778in}}{\pgfqpoint{0.000000in}{0.027778in}}%
\pgfpathcurveto{\pgfqpoint{-0.007367in}{0.027778in}}{\pgfqpoint{-0.014433in}{0.024851in}}{\pgfqpoint{-0.019642in}{0.019642in}}%
\pgfpathcurveto{\pgfqpoint{-0.024851in}{0.014433in}}{\pgfqpoint{-0.027778in}{0.007367in}}{\pgfqpoint{-0.027778in}{0.000000in}}%
\pgfpathcurveto{\pgfqpoint{-0.027778in}{-0.007367in}}{\pgfqpoint{-0.024851in}{-0.014433in}}{\pgfqpoint{-0.019642in}{-0.019642in}}%
\pgfpathcurveto{\pgfqpoint{-0.014433in}{-0.024851in}}{\pgfqpoint{-0.007367in}{-0.027778in}}{\pgfqpoint{0.000000in}{-0.027778in}}%
\pgfpathclose%
\pgfusepath{stroke,fill}%
}%
\begin{pgfscope}%
\pgfsys@transformshift{5.202797in}{0.621202in}%
\pgfsys@useobject{currentmarker}{}%
\end{pgfscope}%
\end{pgfscope}%
\begin{pgfscope}%
\pgfpathrectangle{\pgfqpoint{0.800000in}{0.528000in}}{\pgfqpoint{4.960000in}{3.696000in}}%
\pgfusepath{clip}%
\pgfsetbuttcap%
\pgfsetroundjoin%
\pgfsetlinewidth{1.505625pt}%
\definecolor{currentstroke}{rgb}{0.878431,0.000000,0.000000}%
\pgfsetstrokecolor{currentstroke}%
\pgfsetdash{{1.500000pt}{2.475000pt}}{0.000000pt}%
\pgfpathmoveto{\pgfqpoint{4.506756in}{1.889050in}}%
\pgfusepath{stroke}%
\end{pgfscope}%
\begin{pgfscope}%
\pgfpathrectangle{\pgfqpoint{0.800000in}{0.528000in}}{\pgfqpoint{4.960000in}{3.696000in}}%
\pgfusepath{clip}%
\pgfsetbuttcap%
\pgfsetroundjoin%
\definecolor{currentfill}{rgb}{0.878431,0.000000,0.000000}%
\pgfsetfillcolor{currentfill}%
\pgfsetlinewidth{0.501875pt}%
\definecolor{currentstroke}{rgb}{0.690196,0.000000,0.000000}%
\pgfsetstrokecolor{currentstroke}%
\pgfsetdash{}{0pt}%
\pgfsys@defobject{currentmarker}{\pgfqpoint{-0.027778in}{-0.027778in}}{\pgfqpoint{0.027778in}{0.027778in}}{%
\pgfpathmoveto{\pgfqpoint{0.000000in}{-0.027778in}}%
\pgfpathcurveto{\pgfqpoint{0.007367in}{-0.027778in}}{\pgfqpoint{0.014433in}{-0.024851in}}{\pgfqpoint{0.019642in}{-0.019642in}}%
\pgfpathcurveto{\pgfqpoint{0.024851in}{-0.014433in}}{\pgfqpoint{0.027778in}{-0.007367in}}{\pgfqpoint{0.027778in}{0.000000in}}%
\pgfpathcurveto{\pgfqpoint{0.027778in}{0.007367in}}{\pgfqpoint{0.024851in}{0.014433in}}{\pgfqpoint{0.019642in}{0.019642in}}%
\pgfpathcurveto{\pgfqpoint{0.014433in}{0.024851in}}{\pgfqpoint{0.007367in}{0.027778in}}{\pgfqpoint{0.000000in}{0.027778in}}%
\pgfpathcurveto{\pgfqpoint{-0.007367in}{0.027778in}}{\pgfqpoint{-0.014433in}{0.024851in}}{\pgfqpoint{-0.019642in}{0.019642in}}%
\pgfpathcurveto{\pgfqpoint{-0.024851in}{0.014433in}}{\pgfqpoint{-0.027778in}{0.007367in}}{\pgfqpoint{-0.027778in}{0.000000in}}%
\pgfpathcurveto{\pgfqpoint{-0.027778in}{-0.007367in}}{\pgfqpoint{-0.024851in}{-0.014433in}}{\pgfqpoint{-0.019642in}{-0.019642in}}%
\pgfpathcurveto{\pgfqpoint{-0.014433in}{-0.024851in}}{\pgfqpoint{-0.007367in}{-0.027778in}}{\pgfqpoint{0.000000in}{-0.027778in}}%
\pgfpathclose%
\pgfusepath{stroke,fill}%
}%
\begin{pgfscope}%
\pgfsys@transformshift{4.506756in}{1.889050in}%
\pgfsys@useobject{currentmarker}{}%
\end{pgfscope}%
\end{pgfscope}%
\begin{pgfscope}%
\pgfpathrectangle{\pgfqpoint{0.800000in}{0.528000in}}{\pgfqpoint{4.960000in}{3.696000in}}%
\pgfusepath{clip}%
\pgfsetbuttcap%
\pgfsetroundjoin%
\pgfsetlinewidth{1.505625pt}%
\definecolor{currentstroke}{rgb}{0.878431,0.000000,0.000000}%
\pgfsetstrokecolor{currentstroke}%
\pgfsetdash{{1.500000pt}{2.475000pt}}{0.000000pt}%
\pgfpathmoveto{\pgfqpoint{4.408860in}{2.188963in}}%
\pgfusepath{stroke}%
\end{pgfscope}%
\begin{pgfscope}%
\pgfpathrectangle{\pgfqpoint{0.800000in}{0.528000in}}{\pgfqpoint{4.960000in}{3.696000in}}%
\pgfusepath{clip}%
\pgfsetbuttcap%
\pgfsetroundjoin%
\definecolor{currentfill}{rgb}{0.878431,0.000000,0.000000}%
\pgfsetfillcolor{currentfill}%
\pgfsetlinewidth{0.501875pt}%
\definecolor{currentstroke}{rgb}{0.690196,0.000000,0.000000}%
\pgfsetstrokecolor{currentstroke}%
\pgfsetdash{}{0pt}%
\pgfsys@defobject{currentmarker}{\pgfqpoint{-0.027778in}{-0.027778in}}{\pgfqpoint{0.027778in}{0.027778in}}{%
\pgfpathmoveto{\pgfqpoint{0.000000in}{-0.027778in}}%
\pgfpathcurveto{\pgfqpoint{0.007367in}{-0.027778in}}{\pgfqpoint{0.014433in}{-0.024851in}}{\pgfqpoint{0.019642in}{-0.019642in}}%
\pgfpathcurveto{\pgfqpoint{0.024851in}{-0.014433in}}{\pgfqpoint{0.027778in}{-0.007367in}}{\pgfqpoint{0.027778in}{0.000000in}}%
\pgfpathcurveto{\pgfqpoint{0.027778in}{0.007367in}}{\pgfqpoint{0.024851in}{0.014433in}}{\pgfqpoint{0.019642in}{0.019642in}}%
\pgfpathcurveto{\pgfqpoint{0.014433in}{0.024851in}}{\pgfqpoint{0.007367in}{0.027778in}}{\pgfqpoint{0.000000in}{0.027778in}}%
\pgfpathcurveto{\pgfqpoint{-0.007367in}{0.027778in}}{\pgfqpoint{-0.014433in}{0.024851in}}{\pgfqpoint{-0.019642in}{0.019642in}}%
\pgfpathcurveto{\pgfqpoint{-0.024851in}{0.014433in}}{\pgfqpoint{-0.027778in}{0.007367in}}{\pgfqpoint{-0.027778in}{0.000000in}}%
\pgfpathcurveto{\pgfqpoint{-0.027778in}{-0.007367in}}{\pgfqpoint{-0.024851in}{-0.014433in}}{\pgfqpoint{-0.019642in}{-0.019642in}}%
\pgfpathcurveto{\pgfqpoint{-0.014433in}{-0.024851in}}{\pgfqpoint{-0.007367in}{-0.027778in}}{\pgfqpoint{0.000000in}{-0.027778in}}%
\pgfpathclose%
\pgfusepath{stroke,fill}%
}%
\begin{pgfscope}%
\pgfsys@transformshift{4.408860in}{2.188963in}%
\pgfsys@useobject{currentmarker}{}%
\end{pgfscope}%
\end{pgfscope}%
\begin{pgfscope}%
\pgfpathrectangle{\pgfqpoint{0.800000in}{0.528000in}}{\pgfqpoint{4.960000in}{3.696000in}}%
\pgfusepath{clip}%
\pgfsetbuttcap%
\pgfsetroundjoin%
\pgfsetlinewidth{1.505625pt}%
\definecolor{currentstroke}{rgb}{0.878431,0.000000,0.000000}%
\pgfsetstrokecolor{currentstroke}%
\pgfsetdash{{1.500000pt}{2.475000pt}}{0.000000pt}%
\pgfpathmoveto{\pgfqpoint{5.276373in}{1.429061in}}%
\pgfusepath{stroke}%
\end{pgfscope}%
\begin{pgfscope}%
\pgfpathrectangle{\pgfqpoint{0.800000in}{0.528000in}}{\pgfqpoint{4.960000in}{3.696000in}}%
\pgfusepath{clip}%
\pgfsetbuttcap%
\pgfsetroundjoin%
\definecolor{currentfill}{rgb}{0.878431,0.000000,0.000000}%
\pgfsetfillcolor{currentfill}%
\pgfsetlinewidth{0.501875pt}%
\definecolor{currentstroke}{rgb}{0.690196,0.000000,0.000000}%
\pgfsetstrokecolor{currentstroke}%
\pgfsetdash{}{0pt}%
\pgfsys@defobject{currentmarker}{\pgfqpoint{-0.027778in}{-0.027778in}}{\pgfqpoint{0.027778in}{0.027778in}}{%
\pgfpathmoveto{\pgfqpoint{0.000000in}{-0.027778in}}%
\pgfpathcurveto{\pgfqpoint{0.007367in}{-0.027778in}}{\pgfqpoint{0.014433in}{-0.024851in}}{\pgfqpoint{0.019642in}{-0.019642in}}%
\pgfpathcurveto{\pgfqpoint{0.024851in}{-0.014433in}}{\pgfqpoint{0.027778in}{-0.007367in}}{\pgfqpoint{0.027778in}{0.000000in}}%
\pgfpathcurveto{\pgfqpoint{0.027778in}{0.007367in}}{\pgfqpoint{0.024851in}{0.014433in}}{\pgfqpoint{0.019642in}{0.019642in}}%
\pgfpathcurveto{\pgfqpoint{0.014433in}{0.024851in}}{\pgfqpoint{0.007367in}{0.027778in}}{\pgfqpoint{0.000000in}{0.027778in}}%
\pgfpathcurveto{\pgfqpoint{-0.007367in}{0.027778in}}{\pgfqpoint{-0.014433in}{0.024851in}}{\pgfqpoint{-0.019642in}{0.019642in}}%
\pgfpathcurveto{\pgfqpoint{-0.024851in}{0.014433in}}{\pgfqpoint{-0.027778in}{0.007367in}}{\pgfqpoint{-0.027778in}{0.000000in}}%
\pgfpathcurveto{\pgfqpoint{-0.027778in}{-0.007367in}}{\pgfqpoint{-0.024851in}{-0.014433in}}{\pgfqpoint{-0.019642in}{-0.019642in}}%
\pgfpathcurveto{\pgfqpoint{-0.014433in}{-0.024851in}}{\pgfqpoint{-0.007367in}{-0.027778in}}{\pgfqpoint{0.000000in}{-0.027778in}}%
\pgfpathclose%
\pgfusepath{stroke,fill}%
}%
\begin{pgfscope}%
\pgfsys@transformshift{5.276373in}{1.429061in}%
\pgfsys@useobject{currentmarker}{}%
\end{pgfscope}%
\end{pgfscope}%
\begin{pgfscope}%
\pgfpathrectangle{\pgfqpoint{0.800000in}{0.528000in}}{\pgfqpoint{4.960000in}{3.696000in}}%
\pgfusepath{clip}%
\pgfsetbuttcap%
\pgfsetroundjoin%
\pgfsetlinewidth{1.505625pt}%
\definecolor{currentstroke}{rgb}{0.878431,0.000000,0.000000}%
\pgfsetstrokecolor{currentstroke}%
\pgfsetdash{{1.500000pt}{2.475000pt}}{0.000000pt}%
\pgfpathmoveto{\pgfqpoint{5.145409in}{1.874953in}}%
\pgfusepath{stroke}%
\end{pgfscope}%
\begin{pgfscope}%
\pgfpathrectangle{\pgfqpoint{0.800000in}{0.528000in}}{\pgfqpoint{4.960000in}{3.696000in}}%
\pgfusepath{clip}%
\pgfsetbuttcap%
\pgfsetroundjoin%
\definecolor{currentfill}{rgb}{0.878431,0.000000,0.000000}%
\pgfsetfillcolor{currentfill}%
\pgfsetlinewidth{0.501875pt}%
\definecolor{currentstroke}{rgb}{0.690196,0.000000,0.000000}%
\pgfsetstrokecolor{currentstroke}%
\pgfsetdash{}{0pt}%
\pgfsys@defobject{currentmarker}{\pgfqpoint{-0.027778in}{-0.027778in}}{\pgfqpoint{0.027778in}{0.027778in}}{%
\pgfpathmoveto{\pgfqpoint{0.000000in}{-0.027778in}}%
\pgfpathcurveto{\pgfqpoint{0.007367in}{-0.027778in}}{\pgfqpoint{0.014433in}{-0.024851in}}{\pgfqpoint{0.019642in}{-0.019642in}}%
\pgfpathcurveto{\pgfqpoint{0.024851in}{-0.014433in}}{\pgfqpoint{0.027778in}{-0.007367in}}{\pgfqpoint{0.027778in}{0.000000in}}%
\pgfpathcurveto{\pgfqpoint{0.027778in}{0.007367in}}{\pgfqpoint{0.024851in}{0.014433in}}{\pgfqpoint{0.019642in}{0.019642in}}%
\pgfpathcurveto{\pgfqpoint{0.014433in}{0.024851in}}{\pgfqpoint{0.007367in}{0.027778in}}{\pgfqpoint{0.000000in}{0.027778in}}%
\pgfpathcurveto{\pgfqpoint{-0.007367in}{0.027778in}}{\pgfqpoint{-0.014433in}{0.024851in}}{\pgfqpoint{-0.019642in}{0.019642in}}%
\pgfpathcurveto{\pgfqpoint{-0.024851in}{0.014433in}}{\pgfqpoint{-0.027778in}{0.007367in}}{\pgfqpoint{-0.027778in}{0.000000in}}%
\pgfpathcurveto{\pgfqpoint{-0.027778in}{-0.007367in}}{\pgfqpoint{-0.024851in}{-0.014433in}}{\pgfqpoint{-0.019642in}{-0.019642in}}%
\pgfpathcurveto{\pgfqpoint{-0.014433in}{-0.024851in}}{\pgfqpoint{-0.007367in}{-0.027778in}}{\pgfqpoint{0.000000in}{-0.027778in}}%
\pgfpathclose%
\pgfusepath{stroke,fill}%
}%
\begin{pgfscope}%
\pgfsys@transformshift{5.145409in}{1.874953in}%
\pgfsys@useobject{currentmarker}{}%
\end{pgfscope}%
\end{pgfscope}%
\begin{pgfscope}%
\pgfpathrectangle{\pgfqpoint{0.800000in}{0.528000in}}{\pgfqpoint{4.960000in}{3.696000in}}%
\pgfusepath{clip}%
\pgfsetbuttcap%
\pgfsetroundjoin%
\pgfsetlinewidth{1.505625pt}%
\definecolor{currentstroke}{rgb}{0.878431,0.000000,0.000000}%
\pgfsetstrokecolor{currentstroke}%
\pgfsetdash{{1.500000pt}{2.475000pt}}{0.000000pt}%
\pgfpathmoveto{\pgfqpoint{4.154470in}{1.559886in}}%
\pgfusepath{stroke}%
\end{pgfscope}%
\begin{pgfscope}%
\pgfpathrectangle{\pgfqpoint{0.800000in}{0.528000in}}{\pgfqpoint{4.960000in}{3.696000in}}%
\pgfusepath{clip}%
\pgfsetbuttcap%
\pgfsetroundjoin%
\definecolor{currentfill}{rgb}{0.878431,0.000000,0.000000}%
\pgfsetfillcolor{currentfill}%
\pgfsetlinewidth{0.501875pt}%
\definecolor{currentstroke}{rgb}{0.690196,0.000000,0.000000}%
\pgfsetstrokecolor{currentstroke}%
\pgfsetdash{}{0pt}%
\pgfsys@defobject{currentmarker}{\pgfqpoint{-0.027778in}{-0.027778in}}{\pgfqpoint{0.027778in}{0.027778in}}{%
\pgfpathmoveto{\pgfqpoint{0.000000in}{-0.027778in}}%
\pgfpathcurveto{\pgfqpoint{0.007367in}{-0.027778in}}{\pgfqpoint{0.014433in}{-0.024851in}}{\pgfqpoint{0.019642in}{-0.019642in}}%
\pgfpathcurveto{\pgfqpoint{0.024851in}{-0.014433in}}{\pgfqpoint{0.027778in}{-0.007367in}}{\pgfqpoint{0.027778in}{0.000000in}}%
\pgfpathcurveto{\pgfqpoint{0.027778in}{0.007367in}}{\pgfqpoint{0.024851in}{0.014433in}}{\pgfqpoint{0.019642in}{0.019642in}}%
\pgfpathcurveto{\pgfqpoint{0.014433in}{0.024851in}}{\pgfqpoint{0.007367in}{0.027778in}}{\pgfqpoint{0.000000in}{0.027778in}}%
\pgfpathcurveto{\pgfqpoint{-0.007367in}{0.027778in}}{\pgfqpoint{-0.014433in}{0.024851in}}{\pgfqpoint{-0.019642in}{0.019642in}}%
\pgfpathcurveto{\pgfqpoint{-0.024851in}{0.014433in}}{\pgfqpoint{-0.027778in}{0.007367in}}{\pgfqpoint{-0.027778in}{0.000000in}}%
\pgfpathcurveto{\pgfqpoint{-0.027778in}{-0.007367in}}{\pgfqpoint{-0.024851in}{-0.014433in}}{\pgfqpoint{-0.019642in}{-0.019642in}}%
\pgfpathcurveto{\pgfqpoint{-0.014433in}{-0.024851in}}{\pgfqpoint{-0.007367in}{-0.027778in}}{\pgfqpoint{0.000000in}{-0.027778in}}%
\pgfpathclose%
\pgfusepath{stroke,fill}%
}%
\begin{pgfscope}%
\pgfsys@transformshift{4.154470in}{1.559886in}%
\pgfsys@useobject{currentmarker}{}%
\end{pgfscope}%
\end{pgfscope}%
\begin{pgfscope}%
\pgfpathrectangle{\pgfqpoint{0.800000in}{0.528000in}}{\pgfqpoint{4.960000in}{3.696000in}}%
\pgfusepath{clip}%
\pgfsetbuttcap%
\pgfsetroundjoin%
\pgfsetlinewidth{1.505625pt}%
\definecolor{currentstroke}{rgb}{0.878431,0.000000,0.000000}%
\pgfsetstrokecolor{currentstroke}%
\pgfsetdash{{1.500000pt}{2.475000pt}}{0.000000pt}%
\pgfpathmoveto{\pgfqpoint{4.691059in}{1.003468in}}%
\pgfusepath{stroke}%
\end{pgfscope}%
\begin{pgfscope}%
\pgfpathrectangle{\pgfqpoint{0.800000in}{0.528000in}}{\pgfqpoint{4.960000in}{3.696000in}}%
\pgfusepath{clip}%
\pgfsetbuttcap%
\pgfsetroundjoin%
\definecolor{currentfill}{rgb}{0.878431,0.000000,0.000000}%
\pgfsetfillcolor{currentfill}%
\pgfsetlinewidth{0.501875pt}%
\definecolor{currentstroke}{rgb}{0.690196,0.000000,0.000000}%
\pgfsetstrokecolor{currentstroke}%
\pgfsetdash{}{0pt}%
\pgfsys@defobject{currentmarker}{\pgfqpoint{-0.027778in}{-0.027778in}}{\pgfqpoint{0.027778in}{0.027778in}}{%
\pgfpathmoveto{\pgfqpoint{0.000000in}{-0.027778in}}%
\pgfpathcurveto{\pgfqpoint{0.007367in}{-0.027778in}}{\pgfqpoint{0.014433in}{-0.024851in}}{\pgfqpoint{0.019642in}{-0.019642in}}%
\pgfpathcurveto{\pgfqpoint{0.024851in}{-0.014433in}}{\pgfqpoint{0.027778in}{-0.007367in}}{\pgfqpoint{0.027778in}{0.000000in}}%
\pgfpathcurveto{\pgfqpoint{0.027778in}{0.007367in}}{\pgfqpoint{0.024851in}{0.014433in}}{\pgfqpoint{0.019642in}{0.019642in}}%
\pgfpathcurveto{\pgfqpoint{0.014433in}{0.024851in}}{\pgfqpoint{0.007367in}{0.027778in}}{\pgfqpoint{0.000000in}{0.027778in}}%
\pgfpathcurveto{\pgfqpoint{-0.007367in}{0.027778in}}{\pgfqpoint{-0.014433in}{0.024851in}}{\pgfqpoint{-0.019642in}{0.019642in}}%
\pgfpathcurveto{\pgfqpoint{-0.024851in}{0.014433in}}{\pgfqpoint{-0.027778in}{0.007367in}}{\pgfqpoint{-0.027778in}{0.000000in}}%
\pgfpathcurveto{\pgfqpoint{-0.027778in}{-0.007367in}}{\pgfqpoint{-0.024851in}{-0.014433in}}{\pgfqpoint{-0.019642in}{-0.019642in}}%
\pgfpathcurveto{\pgfqpoint{-0.014433in}{-0.024851in}}{\pgfqpoint{-0.007367in}{-0.027778in}}{\pgfqpoint{0.000000in}{-0.027778in}}%
\pgfpathclose%
\pgfusepath{stroke,fill}%
}%
\begin{pgfscope}%
\pgfsys@transformshift{4.691059in}{1.003468in}%
\pgfsys@useobject{currentmarker}{}%
\end{pgfscope}%
\end{pgfscope}%
\begin{pgfscope}%
\pgfpathrectangle{\pgfqpoint{0.800000in}{0.528000in}}{\pgfqpoint{4.960000in}{3.696000in}}%
\pgfusepath{clip}%
\pgfsetbuttcap%
\pgfsetroundjoin%
\pgfsetlinewidth{1.505625pt}%
\definecolor{currentstroke}{rgb}{0.878431,0.000000,0.000000}%
\pgfsetstrokecolor{currentstroke}%
\pgfsetdash{{1.500000pt}{2.475000pt}}{0.000000pt}%
\pgfpathmoveto{\pgfqpoint{3.445635in}{0.769146in}}%
\pgfusepath{stroke}%
\end{pgfscope}%
\begin{pgfscope}%
\pgfpathrectangle{\pgfqpoint{0.800000in}{0.528000in}}{\pgfqpoint{4.960000in}{3.696000in}}%
\pgfusepath{clip}%
\pgfsetbuttcap%
\pgfsetroundjoin%
\definecolor{currentfill}{rgb}{0.878431,0.000000,0.000000}%
\pgfsetfillcolor{currentfill}%
\pgfsetlinewidth{0.501875pt}%
\definecolor{currentstroke}{rgb}{0.690196,0.000000,0.000000}%
\pgfsetstrokecolor{currentstroke}%
\pgfsetdash{}{0pt}%
\pgfsys@defobject{currentmarker}{\pgfqpoint{-0.027778in}{-0.027778in}}{\pgfqpoint{0.027778in}{0.027778in}}{%
\pgfpathmoveto{\pgfqpoint{0.000000in}{-0.027778in}}%
\pgfpathcurveto{\pgfqpoint{0.007367in}{-0.027778in}}{\pgfqpoint{0.014433in}{-0.024851in}}{\pgfqpoint{0.019642in}{-0.019642in}}%
\pgfpathcurveto{\pgfqpoint{0.024851in}{-0.014433in}}{\pgfqpoint{0.027778in}{-0.007367in}}{\pgfqpoint{0.027778in}{0.000000in}}%
\pgfpathcurveto{\pgfqpoint{0.027778in}{0.007367in}}{\pgfqpoint{0.024851in}{0.014433in}}{\pgfqpoint{0.019642in}{0.019642in}}%
\pgfpathcurveto{\pgfqpoint{0.014433in}{0.024851in}}{\pgfqpoint{0.007367in}{0.027778in}}{\pgfqpoint{0.000000in}{0.027778in}}%
\pgfpathcurveto{\pgfqpoint{-0.007367in}{0.027778in}}{\pgfqpoint{-0.014433in}{0.024851in}}{\pgfqpoint{-0.019642in}{0.019642in}}%
\pgfpathcurveto{\pgfqpoint{-0.024851in}{0.014433in}}{\pgfqpoint{-0.027778in}{0.007367in}}{\pgfqpoint{-0.027778in}{0.000000in}}%
\pgfpathcurveto{\pgfqpoint{-0.027778in}{-0.007367in}}{\pgfqpoint{-0.024851in}{-0.014433in}}{\pgfqpoint{-0.019642in}{-0.019642in}}%
\pgfpathcurveto{\pgfqpoint{-0.014433in}{-0.024851in}}{\pgfqpoint{-0.007367in}{-0.027778in}}{\pgfqpoint{0.000000in}{-0.027778in}}%
\pgfpathclose%
\pgfusepath{stroke,fill}%
}%
\begin{pgfscope}%
\pgfsys@transformshift{3.445635in}{0.769146in}%
\pgfsys@useobject{currentmarker}{}%
\end{pgfscope}%
\end{pgfscope}%
\begin{pgfscope}%
\pgfpathrectangle{\pgfqpoint{0.800000in}{0.528000in}}{\pgfqpoint{4.960000in}{3.696000in}}%
\pgfusepath{clip}%
\pgfsetbuttcap%
\pgfsetroundjoin%
\pgfsetlinewidth{0.501875pt}%
\definecolor{currentstroke}{rgb}{0.501961,0.501961,0.501961}%
\pgfsetstrokecolor{currentstroke}%
\pgfsetdash{{1.850000pt}{0.800000pt}}{0.000000pt}%
\pgfpathmoveto{\pgfqpoint{2.656645in}{0.518000in}}%
\pgfpathlineto{\pgfqpoint{3.903355in}{4.234000in}}%
\pgfpathlineto{\pgfqpoint{3.903355in}{4.234000in}}%
\pgfusepath{stroke}%
\end{pgfscope}%
\begin{pgfscope}%
\pgfsetrectcap%
\pgfsetmiterjoin%
\pgfsetlinewidth{0.803000pt}%
\definecolor{currentstroke}{rgb}{0.000000,0.000000,0.000000}%
\pgfsetstrokecolor{currentstroke}%
\pgfsetdash{}{0pt}%
\pgfpathmoveto{\pgfqpoint{0.800000in}{0.528000in}}%
\pgfpathlineto{\pgfqpoint{0.800000in}{4.224000in}}%
\pgfusepath{stroke}%
\end{pgfscope}%
\begin{pgfscope}%
\pgfsetrectcap%
\pgfsetmiterjoin%
\pgfsetlinewidth{0.803000pt}%
\definecolor{currentstroke}{rgb}{0.000000,0.000000,0.000000}%
\pgfsetstrokecolor{currentstroke}%
\pgfsetdash{}{0pt}%
\pgfpathmoveto{\pgfqpoint{5.760000in}{0.528000in}}%
\pgfpathlineto{\pgfqpoint{5.760000in}{4.224000in}}%
\pgfusepath{stroke}%
\end{pgfscope}%
\begin{pgfscope}%
\pgfsetrectcap%
\pgfsetmiterjoin%
\pgfsetlinewidth{0.803000pt}%
\definecolor{currentstroke}{rgb}{0.000000,0.000000,0.000000}%
\pgfsetstrokecolor{currentstroke}%
\pgfsetdash{}{0pt}%
\pgfpathmoveto{\pgfqpoint{0.800000in}{0.528000in}}%
\pgfpathlineto{\pgfqpoint{5.760000in}{0.528000in}}%
\pgfusepath{stroke}%
\end{pgfscope}%
\begin{pgfscope}%
\pgfsetrectcap%
\pgfsetmiterjoin%
\pgfsetlinewidth{0.803000pt}%
\definecolor{currentstroke}{rgb}{0.000000,0.000000,0.000000}%
\pgfsetstrokecolor{currentstroke}%
\pgfsetdash{}{0pt}%
\pgfpathmoveto{\pgfqpoint{0.800000in}{4.224000in}}%
\pgfpathlineto{\pgfqpoint{5.760000in}{4.224000in}}%
\pgfusepath{stroke}%
\end{pgfscope}%
\begin{pgfscope}%
\definecolor{textcolor}{rgb}{0.000000,0.000000,0.878431}%
\pgfsetstrokecolor{textcolor}%
\pgfsetfillcolor{textcolor}%
\pgftext[x=2.040000in,y=1.452000in,left,base]{\color{textcolor}\sffamily\fontsize{72.000000}{86.400000}\bfseries\selectfont -}%
\end{pgfscope}%
\begin{pgfscope}%
\definecolor{textcolor}{rgb}{0.878431,0.000000,0.000000}%
\pgfsetstrokecolor{textcolor}%
\pgfsetfillcolor{textcolor}%
\pgftext[x=4.520000in,y=3.300000in,left,base]{\color{textcolor}\sffamily\fontsize{48.000000}{57.600000}\bfseries\selectfont +}%
\end{pgfscope}%
\end{pgfpicture}%
\makeatother%
\endgroup%
}
					\end{center}
					\caption{Simple class shattering}
					\label{fig:evaluation_simple_class_shattering}
				\end{figure}
	
				\de{Die Funktion entspricht einem neuronalen Netz ohne verdeckte Schicht und enthält nur eine Eingangs- und eine Ausgangsschicht mit einem künstlichem Neuron ohne Aktivierungsfunktion. Die Dimension, die diese Funktion trennen kann, ist 2 und wird als VC-Dimension\footnote{Vapnik–Chervonenkis dimension: \url{https://en.wikipedia.org/wiki/Convolutional_neural_network}}. Diese Funktion kann genau 2 Klassen trennen. Aber was ist mit nichtlinearen Problemen? In diesem Fall schauen wir uns die folgenden Klassifikationen an:}
				\en{The function corresponds to a neural network without a hidden layer and contains only one input and one output layer with one artificial neuron without activation function. The dimension that this function can separate is 2 and is called VC dimension\footnote{Vapnik–Chervonenkis dimension: \url{https://en.wikipedia.org/wiki/Convolutional_neural_network}}. This function can separate exactly 2 classes. But what about nonlinear problems? In this case let's look at the following classifications:}
				
				\begin{figure}[H]
					\begin{center}
						\scalebox{1.0}{\includetex{classification}}
					\end{center}
					\caption{Linear vs. nonlinear classification}
					\label{fig:overview_linear_nonlinear_classification}
				\end{figure}
	
				\de{Für das zweite Problem können wir die Funktion noch anpassen. Für das dritte nichtlineare Problem ist der Klassifikationsraum nicht mehr ausreichend und erfordert einen anderen Algorithmus. Und hier kommen die neuronalen Netze ins Spiel. Ein Tool, um das Trennen von Daten zu visualisieren und die Funktionsweise der einzelnen Layer zu testen ist \url{https://playground.tensorflow.org}\footnote{Neural Network Right Here in Your Browser: \url{https://playground.tensorflow.org}}. Ein nichtlineares Problem kann im einfachsten Fall schon mit einer hinzugefügten versteckten Schicht mit drei weiteren Neuronen gelöst werden:}
				\en{For the second problem we can still adjust the function. For the third nonlinear problem the classification space is no longer sufficient and requires a different algorithm. And this is where the neural networks come into play. A tool to visualize the separation of data and to test the functionality of the individual layers is \url{https://playground.tensorflow.org}\footnote{Neural Network Right Here in Your Browser: \url{https://playground.tensorflow.org}}. In the simplest case, a nonlinear problem can be solved by adding a hidden layer with three additional neurons:}

				\begin{figure}[H]
					\centering
					\includegraphics[width=1.0\textwidth]{images/simple_neuronal_network}
					\caption[Simple neuronal network with one hidden layer]{Simple neuronal network with one hidden layer\footnotemark}
					\label{fig:beispiel4}
				\end{figure}
				\footnotetext{Source: \url{http://playground.tensorflow.org/}}
		
			\subsubsection{Convolutional Neuronal Network}
				\de{Ein neuronales Netzwerk verarbeitet einen Vektor und gibt einen neuen Vektor zurück. Das Problem bei Eingabedaten wie Bildern ist, dass sie auf den ersten Blick nicht erfolgreich als Vektor beschrieben werden können, um mit einem normalem neuronalem Netzwerk trainiert werden zu können. Man braucht einen Algorithmus, welcher auch matrizenähnliche Eingaben verarbeiten kann und in der Lage ist Muster zu erkennen. Dabei wurde in der Vergangenheit das Prinzip der Convolutional Layer entwickelt. Ein Convolutional Layer nimmt ein Matrix Eingang entgegen, transformiert diese und gibt wie auch bei den künstlichen Neuronen einen Ausgangswert zurück (in diesem Fall eine weitere Matrix). Dieser Ausgangswert wird danach an die nächste Schicht weitergegeben. Dabei beinhaltet ein Convolutional Layer eine Menge \(n\) an quadratischen Matrizen (meist 3x3 oder 5x5 Matrizen). Diese Matrizen werden Filter genannt. Jeder Filter\footnote{Filter, welche z.B. Kanten, Ecken, Quadrate, etc. erkennen können und in tieferen Layern Dinge wie Augen, Ohren, Haare, etc.} wird nun jeweils von links oben bis rechts unten über die Pixel des Bildes mittels Skalarprodukt miteinander verrechnet, wobei ein neues Bild entsteht. Die soganannte Feature Map. Bei einer Anzahl von \(n\) Filtern entstehen am Ende \(n\) Feature Maps und heben die in den Filtern definierten Merkmale jeweils im neu errechneten Bild hervor. Dieser Vorgang wird auch Faltung genannt\autocite{deeplizard2017CNNExplained}.}
				\en{A neural network processes a vector and returns a new vector. The problem with input data such as images is that at first view they cannot be successfully described as vectors to be trained with a normal neural network. You need an algorithm that can handle matrix-like inputs and that is able to recognize patterns. In the past the principle of the convolutional layer was developed. A convolutional layer receives a matrix input, transforms it and returns an output value (in this case another matrix). This output value is then passed on to the next layer. A convolutional layer contains a set \(n\) of square matrices (usually 3x3 or 5x5 matrices). These matrices are called filters. Each filter\footnote{Filters, which can recognize edges, corners, squares, etc. and in deeper layers things like eyes, ears, hair, etc.} is now calculated from top left to bottom right over the pixels of the image using a scalar product, which creates a new image. The so-called Feature Map. With a number of \(n\) filters, \(n\) feature maps are created at the end and highlight the features defined in the filters in the newly calculated image. This process is also called convolution\autocite{deeplizard2017CNNExplained}.}

				\de{Neuronale Netze, welche Gebrauch von Convolutional Layern machen, werden Convolutional Neuronal Networks genannt (kurz CNN) und haben einen entscheidenden Beitrag zum Fortschritt der Bildklassifizierung und auch in anderen Bereichen wie Spracherkennung geleistet. Neben den Convolutional Layern existieren in einem Convolutional Neuronal Network weitere spezielle Layer, welche sich von normalen Neuronalen Netzen unterscheiden: Z.B. die Pooling Layer. In einem Pooling Layer werden überflüssige Informationen verworfen und die Featuremaps verkleinert. Dieser Vorgang verringert den Speicherbedarf und erhöht Berechnungsgeschwindigkeit. Die Convolutional Layer und die Pooling Layer wechseln sich in aller Regel jeweils ab, bis am Ende statt einer \(n x n\) Matrix des Eingangsbildes ein großer Vektor entsteht, welcher von einem normalen neuronalem Netzwerk weiterverarbeitet werden kann und schlußendlich in dem schon beschriebenen One Hot Vektor endet.}
				\en{Neural networks that make use of convolutional layers are called convolutional neural networks (in short CNN) and have made a decisive contribution to the progress of image classification and also in other areas like speech recognition. In addition to the Convolutional Layers, a Convolutional Neural Network has other special layers that differ from normal neural networks: For example the Pooling Layer. In a pooling layer, unnecessary information is discarded and feature maps are reduced in size. This process reduces memory requirements and increases calculation speed. The convolutional layer and the pooling layer usually alternate until a large vector is created at the end instead of a \(n x n\) matrix of the input image, which can be further processed by a normal neural network and finally ends in the already described one hot vector.}

				\begin{figure}[H]
					\centering
					\begin{tikzpicture}
						\tikzmath{
							\inputImageWidth = 1.0;
							\layerOneDistance = 0.22;
							\layerOnePosition = 1.95;
							\layerOneWidth = 1.0;
							\layerTwoDistance = 0.12;
							\layerTwoPosition = 4.40;
							\layerTwoWidth = 0.6;
							\layerThreeDistance = 0.12;
							\layerThreePosition = 6.50;
							\layerThreeWidth = 0.6;
							\layerFourDistance = 0.052;
							\layerFourPosition = 8.5;
							\layerFourWidth = 0.36;
						}

						% input image
						\node at (0.5,-1){\begin{tabular}{c}input image\end{tabular}};
						\draw
							(0 * \inputImageWidth, 0 * \inputImageWidth) --
							(1 * \inputImageWidth, 0 * \inputImageWidth) --
							(1 * \inputImageWidth, 1 * \inputImageWidth) --
							(0 * \inputImageWidth, 1 * \inputImageWidth) --
							(0 * \inputImageWidth, 0 * \inputImageWidth);
						
						% layer 1
						\node at (3.0, 3.0){\begin{tabular}{c}convolutional layer\\layer $l = 1$\end{tabular}};
						\foreach \i in {0,...,5}
							\draw[fill=black, opacity=0.2, draw=black]
								(\layerOnePosition + 0 * \layerOneWidth + \layerOneDistance*\i, 0 * \layerOneWidth + \layerOneDistance*\i) --
								(\layerOnePosition + 1 * \layerOneWidth + \layerOneDistance*\i, 0 * \layerOneWidth + \layerOneDistance*\i) --
								(\layerOnePosition + 1 * \layerOneWidth + \layerOneDistance*\i, 1 * \layerOneWidth + \layerOneDistance*\i) --
								(\layerOnePosition + 0 * \layerOneWidth + \layerOneDistance*\i, 1 * \layerOneWidth + \layerOneDistance*\i) --
								(\layerOnePosition + 0 * \layerOneWidth + \layerOneDistance*\i, 0 * \layerOneWidth + \layerOneDistance*\i);

						% layer 2
						\node at (5.0,-1){\begin{tabular}{c}pooling layer\\layer $l = 2$\end{tabular}};
						\foreach \i in {0,...,5}
							\draw[fill=black, opacity=0.2, draw=black]
								(\layerTwoPosition + 0 * \layerTwoWidth + \layerTwoDistance*\i, 0 * \layerTwoWidth + \layerTwoDistance*\i) --
								(\layerTwoPosition + 1 * \layerTwoWidth + \layerTwoDistance*\i, 0 * \layerTwoWidth + \layerTwoDistance*\i) --
								(\layerTwoPosition + 1 * \layerTwoWidth + \layerTwoDistance*\i, 1 * \layerTwoWidth + \layerTwoDistance*\i) --
								(\layerTwoPosition + 0 * \layerTwoWidth + \layerTwoDistance*\i, 1 * \layerTwoWidth + \layerTwoDistance*\i) --
								(\layerTwoPosition + 0 * \layerTwoWidth + \layerTwoDistance*\i, 0 * \layerTwoWidth + \layerTwoDistance*\i);
						
						%layer 3
						\node at (7.5,3.0){\begin{tabular}{c}convolutional layer\\layer $l = 3$\end{tabular}};
						\foreach \i in {0,...,11}
							\draw[fill=black, opacity=0.2, draw=black]
								(\layerThreePosition + 0 * \layerThreeWidth + \layerThreeDistance*\i, 0 * \layerThreeWidth + \layerThreeDistance*\i) --
								(\layerThreePosition + 1 * \layerThreeWidth + \layerThreeDistance*\i, 0 * \layerThreeWidth + \layerThreeDistance*\i) --
								(\layerThreePosition + 1 * \layerThreeWidth + \layerThreeDistance*\i, 1 * \layerThreeWidth + \layerThreeDistance*\i) --
								(\layerThreePosition + 0 * \layerThreeWidth + \layerThreeDistance*\i, 1 * \layerThreeWidth + \layerThreeDistance*\i) --
								(\layerThreePosition + 0 * \layerThreeWidth + \layerThreeDistance*\i, 0 * \layerThreeWidth + \layerThreeDistance*\i);
						
						% layer 4
						\node at (9.0,-1){\begin{tabular}{c}pooling layer\\layer $l = 4$\end{tabular}};
						\foreach \i in {0,...,11}
							\draw[fill=black, opacity=0.2, draw=black]
								(\layerFourPosition + 0 * \layerFourWidth + \layerFourDistance*\i, 0 * \layerFourWidth + \layerFourDistance*\i) --
								(\layerFourPosition + 1 * \layerFourWidth + \layerFourDistance*\i, 0 * \layerFourWidth + \layerFourDistance*\i) --
								(\layerFourPosition + 1 * \layerFourWidth + \layerFourDistance*\i, 1 * \layerFourWidth + \layerFourDistance*\i) --
								(\layerFourPosition + 0 * \layerFourWidth + \layerFourDistance*\i, 1 * \layerFourWidth + \layerFourDistance*\i) --
								(\layerFourPosition + 0 * \layerFourWidth + \layerFourDistance*\i, 0 * \layerFourWidth + \layerFourDistance*\i);
						
						% layer 5
						\node at (12,3.0){\begin{tabular}{c}fully connected layer\\layer $l = 5$\end{tabular}};
						\draw[fill=black,draw=black,opacity=0.5]
							(10.5,0) --
							(11,0) --
							(12.75,1.75) --
							(12.25,1.75) --
							(10.5,0);
						
						% layer 6
						\node at (13,-1){\begin{tabular}{c}fully connected layer\\output layer $l = 6$\end{tabular}};
						\draw[fill=black,draw=black,opacity=0.5]
							(12.5,0.5) --
							(13,0.5) --
							(13.65,1.15) --
							(13.15,1.15) --
							(12.5,0.5);
					\end{tikzpicture}
					\caption[Architecture of a traditional convolutional neural network.]{Architecture of a traditional convolutional neural network.}
					\label{fig:traditional-convolutional-network}
				\end{figure}

				\de{Ein großer Vorteil von Convolutional neuronal networks soll nicht unerwähnt bleiben: Sie benötigen relativ wenig Vorverarbeitung im Vergleich zu anderen Bildklassifikationsalgorithmen. Dies bedeutet, dass das Netzwerk eigenständig die Filter lernt, die in herkömmlichen Algorithmen normalerweise von Hand entwickelt werden, wenn es mit ausreichender Schulung trainiert wird. Diese Eigenschaft dieser Netzwerke ist von großem Vorteil, da sie automatisiert durchgeführt werden können und sich bei Änderungen der Eingabedaten selbstständig ändern und keinem menschlichem Eingriff bedarf.}
				\en{A big advantage of convolutional neural networks should not remain unmentioned: They require relatively little preprocessing compared to other image classification algorithms. This means that the network independently learns the filters that are normally developed by hand in conventional algorithms, if trained with adequate training. This property of these networks is a great advantage because they can be automated and change independently when the input data changes and do not require human intervention.}
	
			\subsubsection{Transfer Learning}
				\de{Convolutional neuronal networks sind großartig und haben einen entscheidenden Beitrag zur Klassifizierung von Bilder beigetragen. Mit der Gründung der Forschungsdatenbank ImageNet im Jahre 2006 werden jährliche Wettbewerbe veranstaltet, um entwickelte Neuronale Netzwerke miteinander zu vergleichen. ImageNet ist eine Bilderdatenbank mit mehr als 14 Millionen Bildern. Ein CNN namens AlexNet im Jahr 2012 einen Top-5 Fehler von 15,3\% und erhöht sich aktuell stetig jedes Jahr. Aber die Architektur von einem CNN hat ein Problem. Alle Convolutional Layer sind vom Beginn an zufällig initialisiert und enthalten noch keine Muster. Damit sie zuverlässig funktioniert, muss sie mit vielen Bildern trainiert werden. Würde man vom Scratch an ein CNN selbst entwickeln und verwenden, so müssen alle Convolutional Layer auch vorab trainiert werden.}
				\en{Convolutional neural networks are great and have made a significant contribution to the classification of images. With the foundation of the research database ImageNet in 2006, annual competitions are organized to compare developed neural networks. ImageNet is an image database with more than 14 million images. A CNN called AlexNet in 2012 got a top-5 error of 15.3\% and is currently increasing steadily every year. But the architecture of a CNN has a problem. All convolutional layers are randomly initialized from the beginning and do not yet contain any patterns. For it to work reliably, it needs to be trained with many images. If one would develop and use a CNN from scratch, all convolutional layers have to be trained in advance.}

				\de{Die Convolutional Layer extrahieren Features wie Kanten, Quadrate, Kreise, etc. Diese sind so gut wie in jedem Bild vorhanden und es stellt sich die Frage, ob man diese nicht wiederverwenden kann, um den Trainingsaufwand zu verringern. Die Idee beim Transfer Learning ist es ein schon vortrainiertes CNN zu verwenden und lediglich das neuronale Netzwerk am Ende des Convolutional neuronal networks an die eigene Problemstellung anzupassen.}
				\en{The convolutional layers extract features such as edges, squares, circles, etc. These are present in almost every image and the question arises whether you can reuse them to reduce the training effort. The idea of Transfer Learning is to use an already pre-trained CNN and just adapt the neural network at the end of the Convolutional neural network to the own problem.}

				\hl{Put a picture in here.}

				\de{Welchen Vorteil ein vortrainiertes Netzwerk hat, kann man in dieser Arbeit im Kapitel ``\nameref{sec:section_use_of_the_transfer_learning_approach}" einsehen.}
				\en{The advantage of a pre-trained network can be seen in the chapter ``\nameref{sec:section_use_of_the_transfer_learning_approach}" of this thesis.}
			
			\subsubsection{Overview of current and known convolutional neural networks}
				\de{Zu guter Letzt folgen hier noch ein paar aktuelle und bekannte Convolutional Neuronal networks. Sie unterscheiden sich hauptsächlich in folgenden Metriken, wobei in Kombination jedes Netzwerk seine Vor- und Nachteile besitzt:}
				\en{Last but not least, here are a few current and well-known convolutional neural networks. They differ mainly in the following metrics, whereby in combination each network has its advantages and disadvantages:}

				\begin{itemize}
					\item the top-1 accuracy (based on the ImageNet image dataset)
					\item the computing operations which are required for a single forward pass (G-Ops)
					\item the model size (for comparison: the model size of InceptionV3 is about 180 MB)
				\end{itemize}
		
				\begin{figure}[H]
					\centering
					\includegraphics[width=1.0\textwidth]{images/tl_models}
					\caption[Overview of current and known convolutional neural networks.]{Overview of current and known convolutional neural networks.\footnotemark}
					\label{fig:beispiel5}
				\end{figure}
				\footnotetext{Source: \url{https://towardsdatascience.com/neural-network-architectures-156e5bad51ba}}
					
					
					
	% -------------------- %
	% Insufficient amount of data %
	% -------------------- %
	\section{Insufficient amount of data}
	\label{sec:section_insufficient_amount_of_data}
		\de{Gibt man einem Menschen einen Donut und erkärt ihm, dass es ein Donut ist, so ist dieser nach einigem Wiederholen selbstständig in der Lage diesen Donut in der Zukunft zu klassifizieren. Bei Machine Learning ist diese Problematik etwas komplexer. Wie bei den meisten machinellen Lernverfahren benötigt man eine große Menge an Daten. Wieviel ist nicht richtig belegt. Gerade wenn man es mit vielen vorherzusagenden Klassen zu tun hat, erhöht sich die Datenmenge erfahrungsgemäß. Einige Meinungen in Foren und Blogartikeln besagen (Hypothese), dass es mindestens 1000 Bilder pro Klasse sein müssen.\footnote{``Deep Learning for Image Classification with Less Data'', https://towardsdatascience.com, February 2, 2020, \url{https://towardsdatascience.com/deep-learning-for-image-classification-with-less-data-90e5df0a7b8e}}\textsuperscript{,}\footnote{``How many images do you need to train a neural network?'', https://petewarden.com, February 2, 2020, \url{https://petewarden.com/2017/12/14/how-many-images-do-you-need-to-train-a-neural-network/}}\textsuperscript{,}\footnote{``What is the minimum sample size required to train a Deep Learning model - CNN?'', https://www.researchgate.net, February 2, 2020, \url{https://www.researchgate.net/post/What_is_the_minimum_sample_size_required_to_train_a_Deep_Learning_model-CNN}}\textsuperscript{,}\autocite{krizhevsky2012imagenet}.}
		\en{If you give a person a donut and explain to him that it is a donut, then after some repetition he is able to classify this donut in the future. With Machine Learning this problem is a bit more complex. As with most machine learning methods, a large amount of data is required. How much is not properly documented. Especially when you are dealing with many classes to be predicted, experience shows that the amount of data increases. Some opinions in forums and blog articles say (hypothesis) that there must be at least 1000 pictures per class.\footnote{``Deep Learning for Image Classification with Less Data'', https://towardsdatascience.com, February 2, 2020, \url{https://towardsdatascience.com/deep-learning-for-image-classification-with-less-data-90e5df0a7b8e}}\textsuperscript{,}\footnote{``How many images do you need to train a neural network?'', https://petewarden.com, February 2, 2020, \url{https://petewarden.com/2017/12/14/how-many-images-do-you-need-to-train-a-neural-network/}}\textsuperscript{,}\footnote{``What is the minimum sample size required to train a Deep Learning model - CNN?'', https://www.researchgate.net, February 2, 2020, \url{https://www.researchgate.net/post/What_is_the_minimum_sample_size_required_to_train_a_Deep_Learning_model-CNN}}\textsuperscript{,}\autocite{krizhevsky2012imagenet}}

		\de{Je nach Anzahl von zu trainierenden Klassen gelangt man somit schnell zu einem benötigten Datenset, welches aus mehrere Gigabyte an Daten besteht. Mittels Transfer Learning ist es möglich diese Zahl nochmals etwas zu reduzieren, jedoch bleibt das Problem der vielen Daten bestehen. Ein Paper von Microsoft aus dem Jahre 2001 zeigte zur damaligen Zeit, dass einfache Algorithmen mit genügend Daten ähnliche Ergebnisse lieferten, wie komplexe Algorithmen auf Basis weniger Daten. Dabei bezogen sich die Forscher auf Daten, welche Sprachkonstrukte klassifizieren sollten:}
		\en{Depending on the number of classes to be trained, you will quickly arrive at the required data set, which consists of several gigabytes of data. With Transfer Learning it is possible to reduce this number a little bit, but the problem of the large amount of data remains. A paper  from Microsoft in 2001 showed at that time that simple algorithms with enough data gave similar results as complex algorithms based on less data. The researchers referred to data which should classify language constructs:}
		
		\blockquote{\textit{We have shown that for a prototypical natural language classification task, the performance of learners can benefit significantly from much larger training sets.}}\autocite{banko2001scaling}
		
		\de{In einem anderen Artikel nur wenige Jahre später wird dieses Thema ebenfalls aufgegriffen. Dabei bezog man sich auf Daten, welche von Texten lernen und man meist nur kleine oder mittelgroße Datensätze zur Verfügung hat. Um die Effizienz auch in diesem Fall zu verbessern, ist es eine gute Idee die Algorithmen und Methoden zu verbessern: \hl{Beispiele!}}
		\en{Another article only a few years later also addresses this issue. This referred to data that learn from texts and that usually only small or medium sized data sets are available. To improve the efficiency also in this case, it is a good idea to improve the algorithms and methods: \hl{Examples!}}

		\blockquote{\textit{...\autocite{halevy2009unreasonable}.}}



	% -------------------- %
	% Validation process %
	% -------------------- %
	\section{Related work}
		\hl{...}



	% -------------------- %
	% Validation process %
	% -------------------- %
	\section{Validation process}

		\hl{This is the part where I explain my approach.}
	
		\subsection{Preamble}
			In the following, the best possible accuracy is to be achieved by testing various parameters.
			A learning set with the following properties was used:
	
			\begin{itemize}
				\item 14865 images
				\item classified within 50 classes
				\item different number of images per class (unbalanced)
			\end{itemize}
	
			With the exception of the model tests, all tests were based on the following parameters
			(whereby one value of the parameters varied depending on the chapter):
		
			\begin{itemize}
				\item model: resnet18
				\item learning rate: 0,001 (decreases every 7 epochs to 10\% of the previous value)
				\item batch size: 48
				\item epochs: 21 (learning rate from epoch 15 to 21: 0,00001)
				\item image size: 224x224 pixels
				\item the entire training and validation set (14865 images)
			\end{itemize}
		
			Different models were tried out in chapter \flqq\nameref{usedModels}\footnote{
				see on page \pageref{usedModels}
				chapter \ref{usedModels}
				\flqq\nameref{usedModels}\frqq
			}\frqq{} with the same
			parameters as above:
		
			\begin{itemize}
				\item ResNet18
				\item ResNet50
				\item ResNet152
				\item AlexNet
				\item VGG
				\item SqueezeNet
				\item DenseNet
				\item Inception v3
			\end{itemize}
			
		\subsection{Working environment}
	
			\hl{Explain in this part of the thesis the frameworks, environments and hardware used, etc.}
			
		\subsection{Splitting and preparing the data}
		
			\subsubsection{Situation}
			
				We have 14866 images differently distributed in 50 classes (unbalanced). We would like to divide these into 80\% training and 20\% validation images.
				
			\subsubsection{Unbalanced}
			
				The unbalanced dispersion data set is divided exactly in the same ratio:
				
				\begin{itemize}
					\item 2953 images for the training
					\item 11913 images for validation
				\end{itemize}
				
				For training with different training elements, the validation dataset of 2953 images is retained for a comparable result. The number of training elements deviating from the total data set results from this:
				
				\begin{equation}
					n_{train} = k \cdot 500;  k \in 1 \dots 26
				\end{equation}
			
			\subsubsection{Balanced}
		
				\noindent ...
	
		\subsection{Performance}
		
			\noindent ...
		
		\pagebreak
	
		\subsection{Accuracy and evaluations}

			\noindent ...

			\subsubsection{Influence of number of trained images on accuracy}
	
				\noindent ...

				\begin{figure}[H]
					\begin{center}
						%% Creator: Matplotlib, PGF backend
%%
%% To include the figure in your LaTeX document, write
%%   \input{<filename>.pgf}
%%
%% Make sure the required packages are loaded in your preamble
%%   \usepackage{pgf}
%%
%% Figures using additional raster images can only be included by \input if
%% they are in the same directory as the main LaTeX file. For loading figures
%% from other directories you can use the `import` package
%%   \usepackage{import}
%% and then include the figures with
%%   \import{<path to file>}{<filename>.pgf}
%%
%% Matplotlib used the following preamble
%%   \usepackage{fontspec}
%%   \setmainfont{DejaVuSerif.ttf}[Path=C:/Users/bjoern/Anaconda3/envs/keras-gpu/lib/site-packages/matplotlib/mpl-data/fonts/ttf/]
%%   \setsansfont{DejaVuSans.ttf}[Path=C:/Users/bjoern/Anaconda3/envs/keras-gpu/lib/site-packages/matplotlib/mpl-data/fonts/ttf/]
%%   \setmonofont{DejaVuSansMono.ttf}[Path=C:/Users/bjoern/Anaconda3/envs/keras-gpu/lib/site-packages/matplotlib/mpl-data/fonts/ttf/]
%%
\begingroup%
\makeatletter%
\begin{pgfpicture}%
\pgfpathrectangle{\pgfpointorigin}{\pgfqpoint{8.000000in}{4.000000in}}%
\pgfusepath{use as bounding box, clip}%
\begin{pgfscope}%
\pgfsetbuttcap%
\pgfsetmiterjoin%
\pgfsetlinewidth{0.000000pt}%
\definecolor{currentstroke}{rgb}{1.000000,1.000000,1.000000}%
\pgfsetstrokecolor{currentstroke}%
\pgfsetstrokeopacity{0.000000}%
\pgfsetdash{}{0pt}%
\pgfpathmoveto{\pgfqpoint{0.000000in}{0.000000in}}%
\pgfpathlineto{\pgfqpoint{8.000000in}{0.000000in}}%
\pgfpathlineto{\pgfqpoint{8.000000in}{4.000000in}}%
\pgfpathlineto{\pgfqpoint{0.000000in}{4.000000in}}%
\pgfpathclose%
\pgfusepath{}%
\end{pgfscope}%
\begin{pgfscope}%
\pgfsetbuttcap%
\pgfsetmiterjoin%
\pgfsetlinewidth{0.000000pt}%
\definecolor{currentstroke}{rgb}{0.000000,0.000000,0.000000}%
\pgfsetstrokecolor{currentstroke}%
\pgfsetstrokeopacity{0.000000}%
\pgfsetdash{}{0pt}%
\pgfpathmoveto{\pgfqpoint{0.720000in}{0.480000in}}%
\pgfpathlineto{\pgfqpoint{7.760000in}{0.480000in}}%
\pgfpathlineto{\pgfqpoint{7.760000in}{3.600000in}}%
\pgfpathlineto{\pgfqpoint{0.720000in}{3.600000in}}%
\pgfpathclose%
\pgfusepath{}%
\end{pgfscope}%
\begin{pgfscope}%
\pgfpathrectangle{\pgfqpoint{0.720000in}{0.480000in}}{\pgfqpoint{7.040000in}{3.120000in}}%
\pgfusepath{clip}%
\pgfsetbuttcap%
\pgfsetmiterjoin%
\definecolor{currentfill}{rgb}{1.000000,1.000000,1.000000}%
\pgfsetfillcolor{currentfill}%
\pgfsetfillopacity{0.750000}%
\pgfsetlinewidth{0.501875pt}%
\definecolor{currentstroke}{rgb}{1.000000,0.250980,0.250980}%
\pgfsetstrokecolor{currentstroke}%
\pgfsetstrokeopacity{0.750000}%
\pgfsetdash{{1.850000pt}{0.800000pt}}{0.000000pt}%
\pgfpathmoveto{\pgfqpoint{0.785185in}{1.905185in}}%
\pgfpathlineto{\pgfqpoint{1.697778in}{1.905185in}}%
\pgfpathlineto{\pgfqpoint{1.697778in}{3.060741in}}%
\pgfpathlineto{\pgfqpoint{0.785185in}{3.060741in}}%
\pgfpathlineto{\pgfqpoint{0.785185in}{1.905185in}}%
\pgfpathclose%
\pgfusepath{stroke,fill}%
\end{pgfscope}%
\begin{pgfscope}%
\pgfpathrectangle{\pgfqpoint{0.720000in}{0.480000in}}{\pgfqpoint{7.040000in}{3.120000in}}%
\pgfusepath{clip}%
\pgfsetrectcap%
\pgfsetroundjoin%
\pgfsetlinewidth{0.803000pt}%
\definecolor{currentstroke}{rgb}{0.690196,0.690196,0.690196}%
\pgfsetstrokecolor{currentstroke}%
\pgfsetdash{}{0pt}%
\pgfpathmoveto{\pgfqpoint{0.720000in}{0.480000in}}%
\pgfpathlineto{\pgfqpoint{0.720000in}{3.600000in}}%
\pgfusepath{stroke}%
\end{pgfscope}%
\begin{pgfscope}%
\pgfsetbuttcap%
\pgfsetroundjoin%
\definecolor{currentfill}{rgb}{0.000000,0.000000,0.000000}%
\pgfsetfillcolor{currentfill}%
\pgfsetlinewidth{0.803000pt}%
\definecolor{currentstroke}{rgb}{0.000000,0.000000,0.000000}%
\pgfsetstrokecolor{currentstroke}%
\pgfsetdash{}{0pt}%
\pgfsys@defobject{currentmarker}{\pgfqpoint{0.000000in}{-0.048611in}}{\pgfqpoint{0.000000in}{0.000000in}}{%
\pgfpathmoveto{\pgfqpoint{0.000000in}{0.000000in}}%
\pgfpathlineto{\pgfqpoint{0.000000in}{-0.048611in}}%
\pgfusepath{stroke,fill}%
}%
\begin{pgfscope}%
\pgfsys@transformshift{0.720000in}{0.480000in}%
\pgfsys@useobject{currentmarker}{}%
\end{pgfscope}%
\end{pgfscope}%
\begin{pgfscope}%
\definecolor{textcolor}{rgb}{0.000000,0.000000,0.000000}%
\pgfsetstrokecolor{textcolor}%
\pgfsetfillcolor{textcolor}%
\pgftext[x=0.720000in,y=0.382778in,,top]{\color{textcolor}\sffamily\fontsize{10.000000}{12.000000}\selectfont 0}%
\end{pgfscope}%
\begin{pgfscope}%
\pgfpathrectangle{\pgfqpoint{0.720000in}{0.480000in}}{\pgfqpoint{7.040000in}{3.120000in}}%
\pgfusepath{clip}%
\pgfsetrectcap%
\pgfsetroundjoin%
\pgfsetlinewidth{0.803000pt}%
\definecolor{currentstroke}{rgb}{0.690196,0.690196,0.690196}%
\pgfsetstrokecolor{currentstroke}%
\pgfsetdash{}{0pt}%
\pgfpathmoveto{\pgfqpoint{3.066667in}{0.480000in}}%
\pgfpathlineto{\pgfqpoint{3.066667in}{3.600000in}}%
\pgfusepath{stroke}%
\end{pgfscope}%
\begin{pgfscope}%
\pgfsetbuttcap%
\pgfsetroundjoin%
\definecolor{currentfill}{rgb}{0.000000,0.000000,0.000000}%
\pgfsetfillcolor{currentfill}%
\pgfsetlinewidth{0.803000pt}%
\definecolor{currentstroke}{rgb}{0.000000,0.000000,0.000000}%
\pgfsetstrokecolor{currentstroke}%
\pgfsetdash{}{0pt}%
\pgfsys@defobject{currentmarker}{\pgfqpoint{0.000000in}{-0.048611in}}{\pgfqpoint{0.000000in}{0.000000in}}{%
\pgfpathmoveto{\pgfqpoint{0.000000in}{0.000000in}}%
\pgfpathlineto{\pgfqpoint{0.000000in}{-0.048611in}}%
\pgfusepath{stroke,fill}%
}%
\begin{pgfscope}%
\pgfsys@transformshift{3.066667in}{0.480000in}%
\pgfsys@useobject{currentmarker}{}%
\end{pgfscope}%
\end{pgfscope}%
\begin{pgfscope}%
\definecolor{textcolor}{rgb}{0.000000,0.000000,0.000000}%
\pgfsetstrokecolor{textcolor}%
\pgfsetfillcolor{textcolor}%
\pgftext[x=3.066667in,y=0.382778in,,top]{\color{textcolor}\sffamily\fontsize{10.000000}{12.000000}\selectfont 3600}%
\end{pgfscope}%
\begin{pgfscope}%
\pgfpathrectangle{\pgfqpoint{0.720000in}{0.480000in}}{\pgfqpoint{7.040000in}{3.120000in}}%
\pgfusepath{clip}%
\pgfsetrectcap%
\pgfsetroundjoin%
\pgfsetlinewidth{0.803000pt}%
\definecolor{currentstroke}{rgb}{0.690196,0.690196,0.690196}%
\pgfsetstrokecolor{currentstroke}%
\pgfsetdash{}{0pt}%
\pgfpathmoveto{\pgfqpoint{5.413333in}{0.480000in}}%
\pgfpathlineto{\pgfqpoint{5.413333in}{3.600000in}}%
\pgfusepath{stroke}%
\end{pgfscope}%
\begin{pgfscope}%
\pgfsetbuttcap%
\pgfsetroundjoin%
\definecolor{currentfill}{rgb}{0.000000,0.000000,0.000000}%
\pgfsetfillcolor{currentfill}%
\pgfsetlinewidth{0.803000pt}%
\definecolor{currentstroke}{rgb}{0.000000,0.000000,0.000000}%
\pgfsetstrokecolor{currentstroke}%
\pgfsetdash{}{0pt}%
\pgfsys@defobject{currentmarker}{\pgfqpoint{0.000000in}{-0.048611in}}{\pgfqpoint{0.000000in}{0.000000in}}{%
\pgfpathmoveto{\pgfqpoint{0.000000in}{0.000000in}}%
\pgfpathlineto{\pgfqpoint{0.000000in}{-0.048611in}}%
\pgfusepath{stroke,fill}%
}%
\begin{pgfscope}%
\pgfsys@transformshift{5.413333in}{0.480000in}%
\pgfsys@useobject{currentmarker}{}%
\end{pgfscope}%
\end{pgfscope}%
\begin{pgfscope}%
\definecolor{textcolor}{rgb}{0.000000,0.000000,0.000000}%
\pgfsetstrokecolor{textcolor}%
\pgfsetfillcolor{textcolor}%
\pgftext[x=5.413333in,y=0.382778in,,top]{\color{textcolor}\sffamily\fontsize{10.000000}{12.000000}\selectfont 7200}%
\end{pgfscope}%
\begin{pgfscope}%
\pgfpathrectangle{\pgfqpoint{0.720000in}{0.480000in}}{\pgfqpoint{7.040000in}{3.120000in}}%
\pgfusepath{clip}%
\pgfsetrectcap%
\pgfsetroundjoin%
\pgfsetlinewidth{0.803000pt}%
\definecolor{currentstroke}{rgb}{0.690196,0.690196,0.690196}%
\pgfsetstrokecolor{currentstroke}%
\pgfsetdash{}{0pt}%
\pgfpathmoveto{\pgfqpoint{7.760000in}{0.480000in}}%
\pgfpathlineto{\pgfqpoint{7.760000in}{3.600000in}}%
\pgfusepath{stroke}%
\end{pgfscope}%
\begin{pgfscope}%
\pgfsetbuttcap%
\pgfsetroundjoin%
\definecolor{currentfill}{rgb}{0.000000,0.000000,0.000000}%
\pgfsetfillcolor{currentfill}%
\pgfsetlinewidth{0.803000pt}%
\definecolor{currentstroke}{rgb}{0.000000,0.000000,0.000000}%
\pgfsetstrokecolor{currentstroke}%
\pgfsetdash{}{0pt}%
\pgfsys@defobject{currentmarker}{\pgfqpoint{0.000000in}{-0.048611in}}{\pgfqpoint{0.000000in}{0.000000in}}{%
\pgfpathmoveto{\pgfqpoint{0.000000in}{0.000000in}}%
\pgfpathlineto{\pgfqpoint{0.000000in}{-0.048611in}}%
\pgfusepath{stroke,fill}%
}%
\begin{pgfscope}%
\pgfsys@transformshift{7.760000in}{0.480000in}%
\pgfsys@useobject{currentmarker}{}%
\end{pgfscope}%
\end{pgfscope}%
\begin{pgfscope}%
\definecolor{textcolor}{rgb}{0.000000,0.000000,0.000000}%
\pgfsetstrokecolor{textcolor}%
\pgfsetfillcolor{textcolor}%
\pgftext[x=7.760000in,y=0.382778in,,top]{\color{textcolor}\sffamily\fontsize{10.000000}{12.000000}\selectfont 10800}%
\end{pgfscope}%
\begin{pgfscope}%
\definecolor{textcolor}{rgb}{0.000000,0.000000,0.000000}%
\pgfsetstrokecolor{textcolor}%
\pgfsetfillcolor{textcolor}%
\pgftext[x=4.240000in,y=0.192809in,,top]{\color{textcolor}\sffamily\fontsize{10.000000}{12.000000}\selectfont model training time [s]}%
\end{pgfscope}%
\begin{pgfscope}%
\pgfpathrectangle{\pgfqpoint{0.720000in}{0.480000in}}{\pgfqpoint{7.040000in}{3.120000in}}%
\pgfusepath{clip}%
\pgfsetrectcap%
\pgfsetroundjoin%
\pgfsetlinewidth{0.803000pt}%
\definecolor{currentstroke}{rgb}{0.690196,0.690196,0.690196}%
\pgfsetstrokecolor{currentstroke}%
\pgfsetdash{}{0pt}%
\pgfpathmoveto{\pgfqpoint{0.720000in}{0.672593in}}%
\pgfpathlineto{\pgfqpoint{7.760000in}{0.672593in}}%
\pgfusepath{stroke}%
\end{pgfscope}%
\begin{pgfscope}%
\pgfsetbuttcap%
\pgfsetroundjoin%
\definecolor{currentfill}{rgb}{0.000000,0.000000,0.000000}%
\pgfsetfillcolor{currentfill}%
\pgfsetlinewidth{0.803000pt}%
\definecolor{currentstroke}{rgb}{0.000000,0.000000,0.000000}%
\pgfsetstrokecolor{currentstroke}%
\pgfsetdash{}{0pt}%
\pgfsys@defobject{currentmarker}{\pgfqpoint{-0.048611in}{0.000000in}}{\pgfqpoint{0.000000in}{0.000000in}}{%
\pgfpathmoveto{\pgfqpoint{0.000000in}{0.000000in}}%
\pgfpathlineto{\pgfqpoint{-0.048611in}{0.000000in}}%
\pgfusepath{stroke,fill}%
}%
\begin{pgfscope}%
\pgfsys@transformshift{0.720000in}{0.672593in}%
\pgfsys@useobject{currentmarker}{}%
\end{pgfscope}%
\end{pgfscope}%
\begin{pgfscope}%
\definecolor{textcolor}{rgb}{0.000000,0.000000,0.000000}%
\pgfsetstrokecolor{textcolor}%
\pgfsetfillcolor{textcolor}%
\pgftext[x=0.314076in,y=0.619831in,left,base]{\color{textcolor}\sffamily\fontsize{10.000000}{12.000000}\selectfont 10\%}%
\end{pgfscope}%
\begin{pgfscope}%
\pgfpathrectangle{\pgfqpoint{0.720000in}{0.480000in}}{\pgfqpoint{7.040000in}{3.120000in}}%
\pgfusepath{clip}%
\pgfsetrectcap%
\pgfsetroundjoin%
\pgfsetlinewidth{0.803000pt}%
\definecolor{currentstroke}{rgb}{0.690196,0.690196,0.690196}%
\pgfsetstrokecolor{currentstroke}%
\pgfsetdash{}{0pt}%
\pgfpathmoveto{\pgfqpoint{0.720000in}{1.057778in}}%
\pgfpathlineto{\pgfqpoint{7.760000in}{1.057778in}}%
\pgfusepath{stroke}%
\end{pgfscope}%
\begin{pgfscope}%
\pgfsetbuttcap%
\pgfsetroundjoin%
\definecolor{currentfill}{rgb}{0.000000,0.000000,0.000000}%
\pgfsetfillcolor{currentfill}%
\pgfsetlinewidth{0.803000pt}%
\definecolor{currentstroke}{rgb}{0.000000,0.000000,0.000000}%
\pgfsetstrokecolor{currentstroke}%
\pgfsetdash{}{0pt}%
\pgfsys@defobject{currentmarker}{\pgfqpoint{-0.048611in}{0.000000in}}{\pgfqpoint{0.000000in}{0.000000in}}{%
\pgfpathmoveto{\pgfqpoint{0.000000in}{0.000000in}}%
\pgfpathlineto{\pgfqpoint{-0.048611in}{0.000000in}}%
\pgfusepath{stroke,fill}%
}%
\begin{pgfscope}%
\pgfsys@transformshift{0.720000in}{1.057778in}%
\pgfsys@useobject{currentmarker}{}%
\end{pgfscope}%
\end{pgfscope}%
\begin{pgfscope}%
\definecolor{textcolor}{rgb}{0.000000,0.000000,0.000000}%
\pgfsetstrokecolor{textcolor}%
\pgfsetfillcolor{textcolor}%
\pgftext[x=0.314076in,y=1.005016in,left,base]{\color{textcolor}\sffamily\fontsize{10.000000}{12.000000}\selectfont 20\%}%
\end{pgfscope}%
\begin{pgfscope}%
\pgfpathrectangle{\pgfqpoint{0.720000in}{0.480000in}}{\pgfqpoint{7.040000in}{3.120000in}}%
\pgfusepath{clip}%
\pgfsetrectcap%
\pgfsetroundjoin%
\pgfsetlinewidth{0.803000pt}%
\definecolor{currentstroke}{rgb}{0.690196,0.690196,0.690196}%
\pgfsetstrokecolor{currentstroke}%
\pgfsetdash{}{0pt}%
\pgfpathmoveto{\pgfqpoint{0.720000in}{1.442963in}}%
\pgfpathlineto{\pgfqpoint{7.760000in}{1.442963in}}%
\pgfusepath{stroke}%
\end{pgfscope}%
\begin{pgfscope}%
\pgfsetbuttcap%
\pgfsetroundjoin%
\definecolor{currentfill}{rgb}{0.000000,0.000000,0.000000}%
\pgfsetfillcolor{currentfill}%
\pgfsetlinewidth{0.803000pt}%
\definecolor{currentstroke}{rgb}{0.000000,0.000000,0.000000}%
\pgfsetstrokecolor{currentstroke}%
\pgfsetdash{}{0pt}%
\pgfsys@defobject{currentmarker}{\pgfqpoint{-0.048611in}{0.000000in}}{\pgfqpoint{0.000000in}{0.000000in}}{%
\pgfpathmoveto{\pgfqpoint{0.000000in}{0.000000in}}%
\pgfpathlineto{\pgfqpoint{-0.048611in}{0.000000in}}%
\pgfusepath{stroke,fill}%
}%
\begin{pgfscope}%
\pgfsys@transformshift{0.720000in}{1.442963in}%
\pgfsys@useobject{currentmarker}{}%
\end{pgfscope}%
\end{pgfscope}%
\begin{pgfscope}%
\definecolor{textcolor}{rgb}{0.000000,0.000000,0.000000}%
\pgfsetstrokecolor{textcolor}%
\pgfsetfillcolor{textcolor}%
\pgftext[x=0.314076in,y=1.390201in,left,base]{\color{textcolor}\sffamily\fontsize{10.000000}{12.000000}\selectfont 30\%}%
\end{pgfscope}%
\begin{pgfscope}%
\pgfpathrectangle{\pgfqpoint{0.720000in}{0.480000in}}{\pgfqpoint{7.040000in}{3.120000in}}%
\pgfusepath{clip}%
\pgfsetrectcap%
\pgfsetroundjoin%
\pgfsetlinewidth{0.803000pt}%
\definecolor{currentstroke}{rgb}{0.690196,0.690196,0.690196}%
\pgfsetstrokecolor{currentstroke}%
\pgfsetdash{}{0pt}%
\pgfpathmoveto{\pgfqpoint{0.720000in}{1.828148in}}%
\pgfpathlineto{\pgfqpoint{7.760000in}{1.828148in}}%
\pgfusepath{stroke}%
\end{pgfscope}%
\begin{pgfscope}%
\pgfsetbuttcap%
\pgfsetroundjoin%
\definecolor{currentfill}{rgb}{0.000000,0.000000,0.000000}%
\pgfsetfillcolor{currentfill}%
\pgfsetlinewidth{0.803000pt}%
\definecolor{currentstroke}{rgb}{0.000000,0.000000,0.000000}%
\pgfsetstrokecolor{currentstroke}%
\pgfsetdash{}{0pt}%
\pgfsys@defobject{currentmarker}{\pgfqpoint{-0.048611in}{0.000000in}}{\pgfqpoint{0.000000in}{0.000000in}}{%
\pgfpathmoveto{\pgfqpoint{0.000000in}{0.000000in}}%
\pgfpathlineto{\pgfqpoint{-0.048611in}{0.000000in}}%
\pgfusepath{stroke,fill}%
}%
\begin{pgfscope}%
\pgfsys@transformshift{0.720000in}{1.828148in}%
\pgfsys@useobject{currentmarker}{}%
\end{pgfscope}%
\end{pgfscope}%
\begin{pgfscope}%
\definecolor{textcolor}{rgb}{0.000000,0.000000,0.000000}%
\pgfsetstrokecolor{textcolor}%
\pgfsetfillcolor{textcolor}%
\pgftext[x=0.314076in,y=1.775387in,left,base]{\color{textcolor}\sffamily\fontsize{10.000000}{12.000000}\selectfont 40\%}%
\end{pgfscope}%
\begin{pgfscope}%
\pgfpathrectangle{\pgfqpoint{0.720000in}{0.480000in}}{\pgfqpoint{7.040000in}{3.120000in}}%
\pgfusepath{clip}%
\pgfsetrectcap%
\pgfsetroundjoin%
\pgfsetlinewidth{0.803000pt}%
\definecolor{currentstroke}{rgb}{0.690196,0.690196,0.690196}%
\pgfsetstrokecolor{currentstroke}%
\pgfsetdash{}{0pt}%
\pgfpathmoveto{\pgfqpoint{0.720000in}{2.213333in}}%
\pgfpathlineto{\pgfqpoint{7.760000in}{2.213333in}}%
\pgfusepath{stroke}%
\end{pgfscope}%
\begin{pgfscope}%
\pgfsetbuttcap%
\pgfsetroundjoin%
\definecolor{currentfill}{rgb}{0.000000,0.000000,0.000000}%
\pgfsetfillcolor{currentfill}%
\pgfsetlinewidth{0.803000pt}%
\definecolor{currentstroke}{rgb}{0.000000,0.000000,0.000000}%
\pgfsetstrokecolor{currentstroke}%
\pgfsetdash{}{0pt}%
\pgfsys@defobject{currentmarker}{\pgfqpoint{-0.048611in}{0.000000in}}{\pgfqpoint{0.000000in}{0.000000in}}{%
\pgfpathmoveto{\pgfqpoint{0.000000in}{0.000000in}}%
\pgfpathlineto{\pgfqpoint{-0.048611in}{0.000000in}}%
\pgfusepath{stroke,fill}%
}%
\begin{pgfscope}%
\pgfsys@transformshift{0.720000in}{2.213333in}%
\pgfsys@useobject{currentmarker}{}%
\end{pgfscope}%
\end{pgfscope}%
\begin{pgfscope}%
\definecolor{textcolor}{rgb}{0.000000,0.000000,0.000000}%
\pgfsetstrokecolor{textcolor}%
\pgfsetfillcolor{textcolor}%
\pgftext[x=0.314076in,y=2.160572in,left,base]{\color{textcolor}\sffamily\fontsize{10.000000}{12.000000}\selectfont 50\%}%
\end{pgfscope}%
\begin{pgfscope}%
\pgfpathrectangle{\pgfqpoint{0.720000in}{0.480000in}}{\pgfqpoint{7.040000in}{3.120000in}}%
\pgfusepath{clip}%
\pgfsetrectcap%
\pgfsetroundjoin%
\pgfsetlinewidth{0.803000pt}%
\definecolor{currentstroke}{rgb}{0.690196,0.690196,0.690196}%
\pgfsetstrokecolor{currentstroke}%
\pgfsetdash{}{0pt}%
\pgfpathmoveto{\pgfqpoint{0.720000in}{2.598519in}}%
\pgfpathlineto{\pgfqpoint{7.760000in}{2.598519in}}%
\pgfusepath{stroke}%
\end{pgfscope}%
\begin{pgfscope}%
\pgfsetbuttcap%
\pgfsetroundjoin%
\definecolor{currentfill}{rgb}{0.000000,0.000000,0.000000}%
\pgfsetfillcolor{currentfill}%
\pgfsetlinewidth{0.803000pt}%
\definecolor{currentstroke}{rgb}{0.000000,0.000000,0.000000}%
\pgfsetstrokecolor{currentstroke}%
\pgfsetdash{}{0pt}%
\pgfsys@defobject{currentmarker}{\pgfqpoint{-0.048611in}{0.000000in}}{\pgfqpoint{0.000000in}{0.000000in}}{%
\pgfpathmoveto{\pgfqpoint{0.000000in}{0.000000in}}%
\pgfpathlineto{\pgfqpoint{-0.048611in}{0.000000in}}%
\pgfusepath{stroke,fill}%
}%
\begin{pgfscope}%
\pgfsys@transformshift{0.720000in}{2.598519in}%
\pgfsys@useobject{currentmarker}{}%
\end{pgfscope}%
\end{pgfscope}%
\begin{pgfscope}%
\definecolor{textcolor}{rgb}{0.000000,0.000000,0.000000}%
\pgfsetstrokecolor{textcolor}%
\pgfsetfillcolor{textcolor}%
\pgftext[x=0.314076in,y=2.545757in,left,base]{\color{textcolor}\sffamily\fontsize{10.000000}{12.000000}\selectfont 60\%}%
\end{pgfscope}%
\begin{pgfscope}%
\pgfpathrectangle{\pgfqpoint{0.720000in}{0.480000in}}{\pgfqpoint{7.040000in}{3.120000in}}%
\pgfusepath{clip}%
\pgfsetrectcap%
\pgfsetroundjoin%
\pgfsetlinewidth{0.803000pt}%
\definecolor{currentstroke}{rgb}{0.690196,0.690196,0.690196}%
\pgfsetstrokecolor{currentstroke}%
\pgfsetdash{}{0pt}%
\pgfpathmoveto{\pgfqpoint{0.720000in}{2.983704in}}%
\pgfpathlineto{\pgfqpoint{7.760000in}{2.983704in}}%
\pgfusepath{stroke}%
\end{pgfscope}%
\begin{pgfscope}%
\pgfsetbuttcap%
\pgfsetroundjoin%
\definecolor{currentfill}{rgb}{0.000000,0.000000,0.000000}%
\pgfsetfillcolor{currentfill}%
\pgfsetlinewidth{0.803000pt}%
\definecolor{currentstroke}{rgb}{0.000000,0.000000,0.000000}%
\pgfsetstrokecolor{currentstroke}%
\pgfsetdash{}{0pt}%
\pgfsys@defobject{currentmarker}{\pgfqpoint{-0.048611in}{0.000000in}}{\pgfqpoint{0.000000in}{0.000000in}}{%
\pgfpathmoveto{\pgfqpoint{0.000000in}{0.000000in}}%
\pgfpathlineto{\pgfqpoint{-0.048611in}{0.000000in}}%
\pgfusepath{stroke,fill}%
}%
\begin{pgfscope}%
\pgfsys@transformshift{0.720000in}{2.983704in}%
\pgfsys@useobject{currentmarker}{}%
\end{pgfscope}%
\end{pgfscope}%
\begin{pgfscope}%
\definecolor{textcolor}{rgb}{0.000000,0.000000,0.000000}%
\pgfsetstrokecolor{textcolor}%
\pgfsetfillcolor{textcolor}%
\pgftext[x=0.314076in,y=2.930942in,left,base]{\color{textcolor}\sffamily\fontsize{10.000000}{12.000000}\selectfont 70\%}%
\end{pgfscope}%
\begin{pgfscope}%
\pgfpathrectangle{\pgfqpoint{0.720000in}{0.480000in}}{\pgfqpoint{7.040000in}{3.120000in}}%
\pgfusepath{clip}%
\pgfsetrectcap%
\pgfsetroundjoin%
\pgfsetlinewidth{0.803000pt}%
\definecolor{currentstroke}{rgb}{0.690196,0.690196,0.690196}%
\pgfsetstrokecolor{currentstroke}%
\pgfsetdash{}{0pt}%
\pgfpathmoveto{\pgfqpoint{0.720000in}{3.368889in}}%
\pgfpathlineto{\pgfqpoint{7.760000in}{3.368889in}}%
\pgfusepath{stroke}%
\end{pgfscope}%
\begin{pgfscope}%
\pgfsetbuttcap%
\pgfsetroundjoin%
\definecolor{currentfill}{rgb}{0.000000,0.000000,0.000000}%
\pgfsetfillcolor{currentfill}%
\pgfsetlinewidth{0.803000pt}%
\definecolor{currentstroke}{rgb}{0.000000,0.000000,0.000000}%
\pgfsetstrokecolor{currentstroke}%
\pgfsetdash{}{0pt}%
\pgfsys@defobject{currentmarker}{\pgfqpoint{-0.048611in}{0.000000in}}{\pgfqpoint{0.000000in}{0.000000in}}{%
\pgfpathmoveto{\pgfqpoint{0.000000in}{0.000000in}}%
\pgfpathlineto{\pgfqpoint{-0.048611in}{0.000000in}}%
\pgfusepath{stroke,fill}%
}%
\begin{pgfscope}%
\pgfsys@transformshift{0.720000in}{3.368889in}%
\pgfsys@useobject{currentmarker}{}%
\end{pgfscope}%
\end{pgfscope}%
\begin{pgfscope}%
\definecolor{textcolor}{rgb}{0.000000,0.000000,0.000000}%
\pgfsetstrokecolor{textcolor}%
\pgfsetfillcolor{textcolor}%
\pgftext[x=0.314076in,y=3.316127in,left,base]{\color{textcolor}\sffamily\fontsize{10.000000}{12.000000}\selectfont 80\%}%
\end{pgfscope}%
\begin{pgfscope}%
\definecolor{textcolor}{rgb}{0.000000,0.000000,0.000000}%
\pgfsetstrokecolor{textcolor}%
\pgfsetfillcolor{textcolor}%
\pgftext[x=0.258520in,y=2.040000in,,bottom,rotate=90.000000]{\color{textcolor}\sffamily\fontsize{10.000000}{12.000000}\selectfont accuracy validation top-1 [\%]}%
\end{pgfscope}%
\begin{pgfscope}%
\pgfpathrectangle{\pgfqpoint{0.720000in}{0.480000in}}{\pgfqpoint{7.040000in}{3.120000in}}%
\pgfusepath{clip}%
\pgfsetrectcap%
\pgfsetroundjoin%
\pgfsetlinewidth{0.752812pt}%
\definecolor{currentstroke}{rgb}{0.000000,0.000000,1.000000}%
\pgfsetstrokecolor{currentstroke}%
\pgfsetdash{}{0pt}%
\pgfpathmoveto{\pgfqpoint{0.763674in}{0.525452in}}%
\pgfpathlineto{\pgfqpoint{0.796919in}{0.756948in}}%
\pgfpathlineto{\pgfqpoint{0.828859in}{1.059704in}}%
\pgfpathlineto{\pgfqpoint{0.860800in}{1.221096in}}%
\pgfpathlineto{\pgfqpoint{0.892741in}{1.417926in}}%
\pgfpathlineto{\pgfqpoint{0.935111in}{1.442963in}}%
\pgfpathlineto{\pgfqpoint{0.967052in}{1.609363in}}%
\pgfpathlineto{\pgfqpoint{0.998993in}{1.773452in}}%
\pgfpathlineto{\pgfqpoint{1.030933in}{1.862430in}}%
\pgfpathlineto{\pgfqpoint{1.062874in}{1.806193in}}%
\pgfpathlineto{\pgfqpoint{1.094815in}{1.890163in}}%
\pgfpathlineto{\pgfqpoint{1.126756in}{1.959496in}}%
\pgfpathlineto{\pgfqpoint{1.158696in}{1.884770in}}%
\pgfpathlineto{\pgfqpoint{1.190637in}{1.938696in}}%
\pgfpathlineto{\pgfqpoint{1.222578in}{2.016119in}}%
\pgfpathlineto{\pgfqpoint{1.254519in}{1.988385in}}%
\pgfpathlineto{\pgfqpoint{1.286459in}{2.038074in}}%
\pgfpathlineto{\pgfqpoint{1.318400in}{1.980681in}}%
\pgfpathlineto{\pgfqpoint{1.350341in}{2.034222in}}%
\pgfpathlineto{\pgfqpoint{1.382281in}{2.050015in}}%
\pgfpathlineto{\pgfqpoint{1.414222in}{2.035763in}}%
\pgfusepath{stroke}%
\end{pgfscope}%
\begin{pgfscope}%
\pgfpathrectangle{\pgfqpoint{0.720000in}{0.480000in}}{\pgfqpoint{7.040000in}{3.120000in}}%
\pgfusepath{clip}%
\pgfsetbuttcap%
\pgfsetroundjoin%
\definecolor{currentfill}{rgb}{0.000000,0.000000,1.000000}%
\pgfsetfillcolor{currentfill}%
\pgfsetlinewidth{1.003750pt}%
\definecolor{currentstroke}{rgb}{0.000000,0.000000,1.000000}%
\pgfsetstrokecolor{currentstroke}%
\pgfsetdash{}{0pt}%
\pgfsys@defobject{currentmarker}{\pgfqpoint{-0.010417in}{-0.010417in}}{\pgfqpoint{0.010417in}{0.010417in}}{%
\pgfpathmoveto{\pgfqpoint{0.000000in}{-0.010417in}}%
\pgfpathcurveto{\pgfqpoint{0.002763in}{-0.010417in}}{\pgfqpoint{0.005412in}{-0.009319in}}{\pgfqpoint{0.007366in}{-0.007366in}}%
\pgfpathcurveto{\pgfqpoint{0.009319in}{-0.005412in}}{\pgfqpoint{0.010417in}{-0.002763in}}{\pgfqpoint{0.010417in}{0.000000in}}%
\pgfpathcurveto{\pgfqpoint{0.010417in}{0.002763in}}{\pgfqpoint{0.009319in}{0.005412in}}{\pgfqpoint{0.007366in}{0.007366in}}%
\pgfpathcurveto{\pgfqpoint{0.005412in}{0.009319in}}{\pgfqpoint{0.002763in}{0.010417in}}{\pgfqpoint{0.000000in}{0.010417in}}%
\pgfpathcurveto{\pgfqpoint{-0.002763in}{0.010417in}}{\pgfqpoint{-0.005412in}{0.009319in}}{\pgfqpoint{-0.007366in}{0.007366in}}%
\pgfpathcurveto{\pgfqpoint{-0.009319in}{0.005412in}}{\pgfqpoint{-0.010417in}{0.002763in}}{\pgfqpoint{-0.010417in}{0.000000in}}%
\pgfpathcurveto{\pgfqpoint{-0.010417in}{-0.002763in}}{\pgfqpoint{-0.009319in}{-0.005412in}}{\pgfqpoint{-0.007366in}{-0.007366in}}%
\pgfpathcurveto{\pgfqpoint{-0.005412in}{-0.009319in}}{\pgfqpoint{-0.002763in}{-0.010417in}}{\pgfqpoint{0.000000in}{-0.010417in}}%
\pgfpathclose%
\pgfusepath{stroke,fill}%
}%
\begin{pgfscope}%
\pgfsys@transformshift{0.763674in}{0.525452in}%
\pgfsys@useobject{currentmarker}{}%
\end{pgfscope}%
\begin{pgfscope}%
\pgfsys@transformshift{0.796919in}{0.756948in}%
\pgfsys@useobject{currentmarker}{}%
\end{pgfscope}%
\begin{pgfscope}%
\pgfsys@transformshift{0.828859in}{1.059704in}%
\pgfsys@useobject{currentmarker}{}%
\end{pgfscope}%
\begin{pgfscope}%
\pgfsys@transformshift{0.860800in}{1.221096in}%
\pgfsys@useobject{currentmarker}{}%
\end{pgfscope}%
\begin{pgfscope}%
\pgfsys@transformshift{0.892741in}{1.417926in}%
\pgfsys@useobject{currentmarker}{}%
\end{pgfscope}%
\begin{pgfscope}%
\pgfsys@transformshift{0.935111in}{1.442963in}%
\pgfsys@useobject{currentmarker}{}%
\end{pgfscope}%
\begin{pgfscope}%
\pgfsys@transformshift{0.967052in}{1.609363in}%
\pgfsys@useobject{currentmarker}{}%
\end{pgfscope}%
\begin{pgfscope}%
\pgfsys@transformshift{0.998993in}{1.773452in}%
\pgfsys@useobject{currentmarker}{}%
\end{pgfscope}%
\begin{pgfscope}%
\pgfsys@transformshift{1.030933in}{1.862430in}%
\pgfsys@useobject{currentmarker}{}%
\end{pgfscope}%
\begin{pgfscope}%
\pgfsys@transformshift{1.062874in}{1.806193in}%
\pgfsys@useobject{currentmarker}{}%
\end{pgfscope}%
\begin{pgfscope}%
\pgfsys@transformshift{1.094815in}{1.890163in}%
\pgfsys@useobject{currentmarker}{}%
\end{pgfscope}%
\begin{pgfscope}%
\pgfsys@transformshift{1.126756in}{1.959496in}%
\pgfsys@useobject{currentmarker}{}%
\end{pgfscope}%
\begin{pgfscope}%
\pgfsys@transformshift{1.158696in}{1.884770in}%
\pgfsys@useobject{currentmarker}{}%
\end{pgfscope}%
\begin{pgfscope}%
\pgfsys@transformshift{1.190637in}{1.938696in}%
\pgfsys@useobject{currentmarker}{}%
\end{pgfscope}%
\begin{pgfscope}%
\pgfsys@transformshift{1.222578in}{2.016119in}%
\pgfsys@useobject{currentmarker}{}%
\end{pgfscope}%
\begin{pgfscope}%
\pgfsys@transformshift{1.254519in}{1.988385in}%
\pgfsys@useobject{currentmarker}{}%
\end{pgfscope}%
\begin{pgfscope}%
\pgfsys@transformshift{1.286459in}{2.038074in}%
\pgfsys@useobject{currentmarker}{}%
\end{pgfscope}%
\begin{pgfscope}%
\pgfsys@transformshift{1.318400in}{1.980681in}%
\pgfsys@useobject{currentmarker}{}%
\end{pgfscope}%
\begin{pgfscope}%
\pgfsys@transformshift{1.350341in}{2.034222in}%
\pgfsys@useobject{currentmarker}{}%
\end{pgfscope}%
\begin{pgfscope}%
\pgfsys@transformshift{1.382281in}{2.050015in}%
\pgfsys@useobject{currentmarker}{}%
\end{pgfscope}%
\begin{pgfscope}%
\pgfsys@transformshift{1.414222in}{2.035763in}%
\pgfsys@useobject{currentmarker}{}%
\end{pgfscope}%
\end{pgfscope}%
\begin{pgfscope}%
\pgfpathrectangle{\pgfqpoint{0.720000in}{0.480000in}}{\pgfqpoint{7.040000in}{3.120000in}}%
\pgfusepath{clip}%
\pgfsetrectcap%
\pgfsetroundjoin%
\pgfsetlinewidth{0.752812pt}%
\definecolor{currentstroke}{rgb}{1.000000,0.647059,0.000000}%
\pgfsetstrokecolor{currentstroke}%
\pgfsetdash{}{0pt}%
\pgfpathmoveto{\pgfqpoint{0.773452in}{0.851319in}}%
\pgfpathlineto{\pgfqpoint{0.817778in}{1.349748in}}%
\pgfpathlineto{\pgfqpoint{0.860800in}{1.806193in}}%
\pgfpathlineto{\pgfqpoint{0.903822in}{2.059259in}}%
\pgfpathlineto{\pgfqpoint{0.946844in}{2.287289in}}%
\pgfpathlineto{\pgfqpoint{0.989867in}{2.425185in}}%
\pgfpathlineto{\pgfqpoint{1.032889in}{2.423644in}}%
\pgfpathlineto{\pgfqpoint{1.075911in}{2.507615in}}%
\pgfpathlineto{\pgfqpoint{1.118933in}{2.539200in}}%
\pgfpathlineto{\pgfqpoint{1.161956in}{2.606222in}}%
\pgfpathlineto{\pgfqpoint{1.204978in}{2.620474in}}%
\pgfpathlineto{\pgfqpoint{1.248000in}{2.632415in}}%
\pgfpathlineto{\pgfqpoint{1.291022in}{2.608919in}}%
\pgfpathlineto{\pgfqpoint{1.334044in}{2.585037in}}%
\pgfpathlineto{\pgfqpoint{1.377067in}{2.620474in}}%
\pgfpathlineto{\pgfqpoint{1.420089in}{2.628563in}}%
\pgfpathlineto{\pgfqpoint{1.463111in}{2.630874in}}%
\pgfpathlineto{\pgfqpoint{1.506133in}{2.653215in}}%
\pgfpathlineto{\pgfqpoint{1.549156in}{2.649363in}}%
\pgfpathlineto{\pgfqpoint{1.592178in}{2.642815in}}%
\pgfpathlineto{\pgfqpoint{1.635200in}{2.587733in}}%
\pgfusepath{stroke}%
\end{pgfscope}%
\begin{pgfscope}%
\pgfpathrectangle{\pgfqpoint{0.720000in}{0.480000in}}{\pgfqpoint{7.040000in}{3.120000in}}%
\pgfusepath{clip}%
\pgfsetbuttcap%
\pgfsetroundjoin%
\definecolor{currentfill}{rgb}{1.000000,0.647059,0.000000}%
\pgfsetfillcolor{currentfill}%
\pgfsetlinewidth{1.003750pt}%
\definecolor{currentstroke}{rgb}{1.000000,0.647059,0.000000}%
\pgfsetstrokecolor{currentstroke}%
\pgfsetdash{}{0pt}%
\pgfsys@defobject{currentmarker}{\pgfqpoint{-0.010417in}{-0.010417in}}{\pgfqpoint{0.010417in}{0.010417in}}{%
\pgfpathmoveto{\pgfqpoint{0.000000in}{-0.010417in}}%
\pgfpathcurveto{\pgfqpoint{0.002763in}{-0.010417in}}{\pgfqpoint{0.005412in}{-0.009319in}}{\pgfqpoint{0.007366in}{-0.007366in}}%
\pgfpathcurveto{\pgfqpoint{0.009319in}{-0.005412in}}{\pgfqpoint{0.010417in}{-0.002763in}}{\pgfqpoint{0.010417in}{0.000000in}}%
\pgfpathcurveto{\pgfqpoint{0.010417in}{0.002763in}}{\pgfqpoint{0.009319in}{0.005412in}}{\pgfqpoint{0.007366in}{0.007366in}}%
\pgfpathcurveto{\pgfqpoint{0.005412in}{0.009319in}}{\pgfqpoint{0.002763in}{0.010417in}}{\pgfqpoint{0.000000in}{0.010417in}}%
\pgfpathcurveto{\pgfqpoint{-0.002763in}{0.010417in}}{\pgfqpoint{-0.005412in}{0.009319in}}{\pgfqpoint{-0.007366in}{0.007366in}}%
\pgfpathcurveto{\pgfqpoint{-0.009319in}{0.005412in}}{\pgfqpoint{-0.010417in}{0.002763in}}{\pgfqpoint{-0.010417in}{0.000000in}}%
\pgfpathcurveto{\pgfqpoint{-0.010417in}{-0.002763in}}{\pgfqpoint{-0.009319in}{-0.005412in}}{\pgfqpoint{-0.007366in}{-0.007366in}}%
\pgfpathcurveto{\pgfqpoint{-0.005412in}{-0.009319in}}{\pgfqpoint{-0.002763in}{-0.010417in}}{\pgfqpoint{0.000000in}{-0.010417in}}%
\pgfpathclose%
\pgfusepath{stroke,fill}%
}%
\begin{pgfscope}%
\pgfsys@transformshift{0.773452in}{0.851319in}%
\pgfsys@useobject{currentmarker}{}%
\end{pgfscope}%
\begin{pgfscope}%
\pgfsys@transformshift{0.817778in}{1.349748in}%
\pgfsys@useobject{currentmarker}{}%
\end{pgfscope}%
\begin{pgfscope}%
\pgfsys@transformshift{0.860800in}{1.806193in}%
\pgfsys@useobject{currentmarker}{}%
\end{pgfscope}%
\begin{pgfscope}%
\pgfsys@transformshift{0.903822in}{2.059259in}%
\pgfsys@useobject{currentmarker}{}%
\end{pgfscope}%
\begin{pgfscope}%
\pgfsys@transformshift{0.946844in}{2.287289in}%
\pgfsys@useobject{currentmarker}{}%
\end{pgfscope}%
\begin{pgfscope}%
\pgfsys@transformshift{0.989867in}{2.425185in}%
\pgfsys@useobject{currentmarker}{}%
\end{pgfscope}%
\begin{pgfscope}%
\pgfsys@transformshift{1.032889in}{2.423644in}%
\pgfsys@useobject{currentmarker}{}%
\end{pgfscope}%
\begin{pgfscope}%
\pgfsys@transformshift{1.075911in}{2.507615in}%
\pgfsys@useobject{currentmarker}{}%
\end{pgfscope}%
\begin{pgfscope}%
\pgfsys@transformshift{1.118933in}{2.539200in}%
\pgfsys@useobject{currentmarker}{}%
\end{pgfscope}%
\begin{pgfscope}%
\pgfsys@transformshift{1.161956in}{2.606222in}%
\pgfsys@useobject{currentmarker}{}%
\end{pgfscope}%
\begin{pgfscope}%
\pgfsys@transformshift{1.204978in}{2.620474in}%
\pgfsys@useobject{currentmarker}{}%
\end{pgfscope}%
\begin{pgfscope}%
\pgfsys@transformshift{1.248000in}{2.632415in}%
\pgfsys@useobject{currentmarker}{}%
\end{pgfscope}%
\begin{pgfscope}%
\pgfsys@transformshift{1.291022in}{2.608919in}%
\pgfsys@useobject{currentmarker}{}%
\end{pgfscope}%
\begin{pgfscope}%
\pgfsys@transformshift{1.334044in}{2.585037in}%
\pgfsys@useobject{currentmarker}{}%
\end{pgfscope}%
\begin{pgfscope}%
\pgfsys@transformshift{1.377067in}{2.620474in}%
\pgfsys@useobject{currentmarker}{}%
\end{pgfscope}%
\begin{pgfscope}%
\pgfsys@transformshift{1.420089in}{2.628563in}%
\pgfsys@useobject{currentmarker}{}%
\end{pgfscope}%
\begin{pgfscope}%
\pgfsys@transformshift{1.463111in}{2.630874in}%
\pgfsys@useobject{currentmarker}{}%
\end{pgfscope}%
\begin{pgfscope}%
\pgfsys@transformshift{1.506133in}{2.653215in}%
\pgfsys@useobject{currentmarker}{}%
\end{pgfscope}%
\begin{pgfscope}%
\pgfsys@transformshift{1.549156in}{2.649363in}%
\pgfsys@useobject{currentmarker}{}%
\end{pgfscope}%
\begin{pgfscope}%
\pgfsys@transformshift{1.592178in}{2.642815in}%
\pgfsys@useobject{currentmarker}{}%
\end{pgfscope}%
\begin{pgfscope}%
\pgfsys@transformshift{1.635200in}{2.587733in}%
\pgfsys@useobject{currentmarker}{}%
\end{pgfscope}%
\end{pgfscope}%
\begin{pgfscope}%
\pgfpathrectangle{\pgfqpoint{0.720000in}{0.480000in}}{\pgfqpoint{7.040000in}{3.120000in}}%
\pgfusepath{clip}%
\pgfsetrectcap%
\pgfsetroundjoin%
\pgfsetlinewidth{0.752812pt}%
\definecolor{currentstroke}{rgb}{0.000000,0.501961,0.000000}%
\pgfsetstrokecolor{currentstroke}%
\pgfsetdash{}{0pt}%
\pgfpathmoveto{\pgfqpoint{0.794963in}{1.431022in}}%
\pgfpathlineto{\pgfqpoint{0.859496in}{2.115496in}}%
\pgfpathlineto{\pgfqpoint{0.924030in}{2.545748in}}%
\pgfpathlineto{\pgfqpoint{0.988563in}{2.682104in}}%
\pgfpathlineto{\pgfqpoint{1.053096in}{2.792267in}}%
\pgfpathlineto{\pgfqpoint{1.117630in}{2.808059in}}%
\pgfpathlineto{\pgfqpoint{1.182163in}{2.873541in}}%
\pgfpathlineto{\pgfqpoint{1.246696in}{2.935170in}}%
\pgfpathlineto{\pgfqpoint{1.311230in}{2.953659in}}%
\pgfpathlineto{\pgfqpoint{1.375763in}{2.930163in}}%
\pgfpathlineto{\pgfqpoint{1.440296in}{2.939407in}}%
\pgfpathlineto{\pgfqpoint{1.504830in}{2.970607in}}%
\pgfpathlineto{\pgfqpoint{1.569363in}{2.967911in}}%
\pgfpathlineto{\pgfqpoint{1.633244in}{3.001037in}}%
\pgfpathlineto{\pgfqpoint{1.697778in}{3.004889in}}%
\pgfpathlineto{\pgfqpoint{1.762311in}{2.985244in}}%
\pgfpathlineto{\pgfqpoint{1.826844in}{2.991793in}}%
\pgfpathlineto{\pgfqpoint{1.891378in}{2.983704in}}%
\pgfpathlineto{\pgfqpoint{1.955911in}{3.011437in}}%
\pgfpathlineto{\pgfqpoint{2.020444in}{2.999496in}}%
\pgfpathlineto{\pgfqpoint{2.084978in}{2.994489in}}%
\pgfusepath{stroke}%
\end{pgfscope}%
\begin{pgfscope}%
\pgfpathrectangle{\pgfqpoint{0.720000in}{0.480000in}}{\pgfqpoint{7.040000in}{3.120000in}}%
\pgfusepath{clip}%
\pgfsetbuttcap%
\pgfsetroundjoin%
\definecolor{currentfill}{rgb}{0.000000,0.501961,0.000000}%
\pgfsetfillcolor{currentfill}%
\pgfsetlinewidth{1.003750pt}%
\definecolor{currentstroke}{rgb}{0.000000,0.501961,0.000000}%
\pgfsetstrokecolor{currentstroke}%
\pgfsetdash{}{0pt}%
\pgfsys@defobject{currentmarker}{\pgfqpoint{-0.010417in}{-0.010417in}}{\pgfqpoint{0.010417in}{0.010417in}}{%
\pgfpathmoveto{\pgfqpoint{0.000000in}{-0.010417in}}%
\pgfpathcurveto{\pgfqpoint{0.002763in}{-0.010417in}}{\pgfqpoint{0.005412in}{-0.009319in}}{\pgfqpoint{0.007366in}{-0.007366in}}%
\pgfpathcurveto{\pgfqpoint{0.009319in}{-0.005412in}}{\pgfqpoint{0.010417in}{-0.002763in}}{\pgfqpoint{0.010417in}{0.000000in}}%
\pgfpathcurveto{\pgfqpoint{0.010417in}{0.002763in}}{\pgfqpoint{0.009319in}{0.005412in}}{\pgfqpoint{0.007366in}{0.007366in}}%
\pgfpathcurveto{\pgfqpoint{0.005412in}{0.009319in}}{\pgfqpoint{0.002763in}{0.010417in}}{\pgfqpoint{0.000000in}{0.010417in}}%
\pgfpathcurveto{\pgfqpoint{-0.002763in}{0.010417in}}{\pgfqpoint{-0.005412in}{0.009319in}}{\pgfqpoint{-0.007366in}{0.007366in}}%
\pgfpathcurveto{\pgfqpoint{-0.009319in}{0.005412in}}{\pgfqpoint{-0.010417in}{0.002763in}}{\pgfqpoint{-0.010417in}{0.000000in}}%
\pgfpathcurveto{\pgfqpoint{-0.010417in}{-0.002763in}}{\pgfqpoint{-0.009319in}{-0.005412in}}{\pgfqpoint{-0.007366in}{-0.007366in}}%
\pgfpathcurveto{\pgfqpoint{-0.005412in}{-0.009319in}}{\pgfqpoint{-0.002763in}{-0.010417in}}{\pgfqpoint{0.000000in}{-0.010417in}}%
\pgfpathclose%
\pgfusepath{stroke,fill}%
}%
\begin{pgfscope}%
\pgfsys@transformshift{0.794963in}{1.431022in}%
\pgfsys@useobject{currentmarker}{}%
\end{pgfscope}%
\begin{pgfscope}%
\pgfsys@transformshift{0.859496in}{2.115496in}%
\pgfsys@useobject{currentmarker}{}%
\end{pgfscope}%
\begin{pgfscope}%
\pgfsys@transformshift{0.924030in}{2.545748in}%
\pgfsys@useobject{currentmarker}{}%
\end{pgfscope}%
\begin{pgfscope}%
\pgfsys@transformshift{0.988563in}{2.682104in}%
\pgfsys@useobject{currentmarker}{}%
\end{pgfscope}%
\begin{pgfscope}%
\pgfsys@transformshift{1.053096in}{2.792267in}%
\pgfsys@useobject{currentmarker}{}%
\end{pgfscope}%
\begin{pgfscope}%
\pgfsys@transformshift{1.117630in}{2.808059in}%
\pgfsys@useobject{currentmarker}{}%
\end{pgfscope}%
\begin{pgfscope}%
\pgfsys@transformshift{1.182163in}{2.873541in}%
\pgfsys@useobject{currentmarker}{}%
\end{pgfscope}%
\begin{pgfscope}%
\pgfsys@transformshift{1.246696in}{2.935170in}%
\pgfsys@useobject{currentmarker}{}%
\end{pgfscope}%
\begin{pgfscope}%
\pgfsys@transformshift{1.311230in}{2.953659in}%
\pgfsys@useobject{currentmarker}{}%
\end{pgfscope}%
\begin{pgfscope}%
\pgfsys@transformshift{1.375763in}{2.930163in}%
\pgfsys@useobject{currentmarker}{}%
\end{pgfscope}%
\begin{pgfscope}%
\pgfsys@transformshift{1.440296in}{2.939407in}%
\pgfsys@useobject{currentmarker}{}%
\end{pgfscope}%
\begin{pgfscope}%
\pgfsys@transformshift{1.504830in}{2.970607in}%
\pgfsys@useobject{currentmarker}{}%
\end{pgfscope}%
\begin{pgfscope}%
\pgfsys@transformshift{1.569363in}{2.967911in}%
\pgfsys@useobject{currentmarker}{}%
\end{pgfscope}%
\begin{pgfscope}%
\pgfsys@transformshift{1.633244in}{3.001037in}%
\pgfsys@useobject{currentmarker}{}%
\end{pgfscope}%
\begin{pgfscope}%
\pgfsys@transformshift{1.697778in}{3.004889in}%
\pgfsys@useobject{currentmarker}{}%
\end{pgfscope}%
\begin{pgfscope}%
\pgfsys@transformshift{1.762311in}{2.985244in}%
\pgfsys@useobject{currentmarker}{}%
\end{pgfscope}%
\begin{pgfscope}%
\pgfsys@transformshift{1.826844in}{2.991793in}%
\pgfsys@useobject{currentmarker}{}%
\end{pgfscope}%
\begin{pgfscope}%
\pgfsys@transformshift{1.891378in}{2.983704in}%
\pgfsys@useobject{currentmarker}{}%
\end{pgfscope}%
\begin{pgfscope}%
\pgfsys@transformshift{1.955911in}{3.011437in}%
\pgfsys@useobject{currentmarker}{}%
\end{pgfscope}%
\begin{pgfscope}%
\pgfsys@transformshift{2.020444in}{2.999496in}%
\pgfsys@useobject{currentmarker}{}%
\end{pgfscope}%
\begin{pgfscope}%
\pgfsys@transformshift{2.084978in}{2.994489in}%
\pgfsys@useobject{currentmarker}{}%
\end{pgfscope}%
\end{pgfscope}%
\begin{pgfscope}%
\pgfpathrectangle{\pgfqpoint{0.720000in}{0.480000in}}{\pgfqpoint{7.040000in}{3.120000in}}%
\pgfusepath{clip}%
\pgfsetrectcap%
\pgfsetroundjoin%
\pgfsetlinewidth{0.752812pt}%
\definecolor{currentstroke}{rgb}{1.000000,0.000000,0.000000}%
\pgfsetstrokecolor{currentstroke}%
\pgfsetdash{}{0pt}%
\pgfpathmoveto{\pgfqpoint{0.816474in}{1.699111in}}%
\pgfpathlineto{\pgfqpoint{0.903822in}{2.484119in}}%
\pgfpathlineto{\pgfqpoint{0.989867in}{2.739881in}}%
\pgfpathlineto{\pgfqpoint{1.075911in}{2.817304in}}%
\pgfpathlineto{\pgfqpoint{1.161956in}{2.939407in}}%
\pgfpathlineto{\pgfqpoint{1.248000in}{2.989096in}}%
\pgfpathlineto{\pgfqpoint{1.334044in}{2.994489in}}%
\pgfpathlineto{\pgfqpoint{1.420089in}{3.086163in}}%
\pgfpathlineto{\pgfqpoint{1.506133in}{3.099259in}}%
\pgfpathlineto{\pgfqpoint{1.592178in}{3.079615in}}%
\pgfpathlineto{\pgfqpoint{1.678222in}{3.112356in}}%
\pgfpathlineto{\pgfqpoint{1.764267in}{3.104652in}}%
\pgfpathlineto{\pgfqpoint{1.850311in}{3.100415in}}%
\pgfpathlineto{\pgfqpoint{1.936356in}{3.111200in}}%
\pgfpathlineto{\pgfqpoint{2.022400in}{3.105807in}}%
\pgfpathlineto{\pgfqpoint{2.108444in}{3.150489in}}%
\pgfpathlineto{\pgfqpoint{2.194489in}{3.099259in}}%
\pgfpathlineto{\pgfqpoint{2.280533in}{3.133156in}}%
\pgfpathlineto{\pgfqpoint{2.366578in}{3.113511in}}%
\pgfpathlineto{\pgfqpoint{2.452622in}{3.105807in}}%
\pgfpathlineto{\pgfqpoint{2.538667in}{3.113511in}}%
\pgfusepath{stroke}%
\end{pgfscope}%
\begin{pgfscope}%
\pgfpathrectangle{\pgfqpoint{0.720000in}{0.480000in}}{\pgfqpoint{7.040000in}{3.120000in}}%
\pgfusepath{clip}%
\pgfsetbuttcap%
\pgfsetroundjoin%
\definecolor{currentfill}{rgb}{1.000000,0.000000,0.000000}%
\pgfsetfillcolor{currentfill}%
\pgfsetlinewidth{1.003750pt}%
\definecolor{currentstroke}{rgb}{1.000000,0.000000,0.000000}%
\pgfsetstrokecolor{currentstroke}%
\pgfsetdash{}{0pt}%
\pgfsys@defobject{currentmarker}{\pgfqpoint{-0.010417in}{-0.010417in}}{\pgfqpoint{0.010417in}{0.010417in}}{%
\pgfpathmoveto{\pgfqpoint{0.000000in}{-0.010417in}}%
\pgfpathcurveto{\pgfqpoint{0.002763in}{-0.010417in}}{\pgfqpoint{0.005412in}{-0.009319in}}{\pgfqpoint{0.007366in}{-0.007366in}}%
\pgfpathcurveto{\pgfqpoint{0.009319in}{-0.005412in}}{\pgfqpoint{0.010417in}{-0.002763in}}{\pgfqpoint{0.010417in}{0.000000in}}%
\pgfpathcurveto{\pgfqpoint{0.010417in}{0.002763in}}{\pgfqpoint{0.009319in}{0.005412in}}{\pgfqpoint{0.007366in}{0.007366in}}%
\pgfpathcurveto{\pgfqpoint{0.005412in}{0.009319in}}{\pgfqpoint{0.002763in}{0.010417in}}{\pgfqpoint{0.000000in}{0.010417in}}%
\pgfpathcurveto{\pgfqpoint{-0.002763in}{0.010417in}}{\pgfqpoint{-0.005412in}{0.009319in}}{\pgfqpoint{-0.007366in}{0.007366in}}%
\pgfpathcurveto{\pgfqpoint{-0.009319in}{0.005412in}}{\pgfqpoint{-0.010417in}{0.002763in}}{\pgfqpoint{-0.010417in}{0.000000in}}%
\pgfpathcurveto{\pgfqpoint{-0.010417in}{-0.002763in}}{\pgfqpoint{-0.009319in}{-0.005412in}}{\pgfqpoint{-0.007366in}{-0.007366in}}%
\pgfpathcurveto{\pgfqpoint{-0.005412in}{-0.009319in}}{\pgfqpoint{-0.002763in}{-0.010417in}}{\pgfqpoint{0.000000in}{-0.010417in}}%
\pgfpathclose%
\pgfusepath{stroke,fill}%
}%
\begin{pgfscope}%
\pgfsys@transformshift{0.816474in}{1.699111in}%
\pgfsys@useobject{currentmarker}{}%
\end{pgfscope}%
\begin{pgfscope}%
\pgfsys@transformshift{0.903822in}{2.484119in}%
\pgfsys@useobject{currentmarker}{}%
\end{pgfscope}%
\begin{pgfscope}%
\pgfsys@transformshift{0.989867in}{2.739881in}%
\pgfsys@useobject{currentmarker}{}%
\end{pgfscope}%
\begin{pgfscope}%
\pgfsys@transformshift{1.075911in}{2.817304in}%
\pgfsys@useobject{currentmarker}{}%
\end{pgfscope}%
\begin{pgfscope}%
\pgfsys@transformshift{1.161956in}{2.939407in}%
\pgfsys@useobject{currentmarker}{}%
\end{pgfscope}%
\begin{pgfscope}%
\pgfsys@transformshift{1.248000in}{2.989096in}%
\pgfsys@useobject{currentmarker}{}%
\end{pgfscope}%
\begin{pgfscope}%
\pgfsys@transformshift{1.334044in}{2.994489in}%
\pgfsys@useobject{currentmarker}{}%
\end{pgfscope}%
\begin{pgfscope}%
\pgfsys@transformshift{1.420089in}{3.086163in}%
\pgfsys@useobject{currentmarker}{}%
\end{pgfscope}%
\begin{pgfscope}%
\pgfsys@transformshift{1.506133in}{3.099259in}%
\pgfsys@useobject{currentmarker}{}%
\end{pgfscope}%
\begin{pgfscope}%
\pgfsys@transformshift{1.592178in}{3.079615in}%
\pgfsys@useobject{currentmarker}{}%
\end{pgfscope}%
\begin{pgfscope}%
\pgfsys@transformshift{1.678222in}{3.112356in}%
\pgfsys@useobject{currentmarker}{}%
\end{pgfscope}%
\begin{pgfscope}%
\pgfsys@transformshift{1.764267in}{3.104652in}%
\pgfsys@useobject{currentmarker}{}%
\end{pgfscope}%
\begin{pgfscope}%
\pgfsys@transformshift{1.850311in}{3.100415in}%
\pgfsys@useobject{currentmarker}{}%
\end{pgfscope}%
\begin{pgfscope}%
\pgfsys@transformshift{1.936356in}{3.111200in}%
\pgfsys@useobject{currentmarker}{}%
\end{pgfscope}%
\begin{pgfscope}%
\pgfsys@transformshift{2.022400in}{3.105807in}%
\pgfsys@useobject{currentmarker}{}%
\end{pgfscope}%
\begin{pgfscope}%
\pgfsys@transformshift{2.108444in}{3.150489in}%
\pgfsys@useobject{currentmarker}{}%
\end{pgfscope}%
\begin{pgfscope}%
\pgfsys@transformshift{2.194489in}{3.099259in}%
\pgfsys@useobject{currentmarker}{}%
\end{pgfscope}%
\begin{pgfscope}%
\pgfsys@transformshift{2.280533in}{3.133156in}%
\pgfsys@useobject{currentmarker}{}%
\end{pgfscope}%
\begin{pgfscope}%
\pgfsys@transformshift{2.366578in}{3.113511in}%
\pgfsys@useobject{currentmarker}{}%
\end{pgfscope}%
\begin{pgfscope}%
\pgfsys@transformshift{2.452622in}{3.105807in}%
\pgfsys@useobject{currentmarker}{}%
\end{pgfscope}%
\begin{pgfscope}%
\pgfsys@transformshift{2.538667in}{3.113511in}%
\pgfsys@useobject{currentmarker}{}%
\end{pgfscope}%
\end{pgfscope}%
\begin{pgfscope}%
\pgfpathrectangle{\pgfqpoint{0.720000in}{0.480000in}}{\pgfqpoint{7.040000in}{3.120000in}}%
\pgfusepath{clip}%
\pgfsetrectcap%
\pgfsetroundjoin%
\pgfsetlinewidth{0.752812pt}%
\definecolor{currentstroke}{rgb}{0.501961,0.000000,0.501961}%
\pgfsetstrokecolor{currentstroke}%
\pgfsetdash{}{0pt}%
\pgfpathmoveto{\pgfqpoint{0.837985in}{2.121659in}}%
\pgfpathlineto{\pgfqpoint{0.945541in}{2.714844in}}%
\pgfpathlineto{\pgfqpoint{1.053748in}{2.813452in}}%
\pgfpathlineto{\pgfqpoint{1.161304in}{2.939407in}}%
\pgfpathlineto{\pgfqpoint{1.268859in}{3.079615in}}%
\pgfpathlineto{\pgfqpoint{1.376415in}{3.034933in}}%
\pgfpathlineto{\pgfqpoint{1.483970in}{3.099259in}}%
\pgfpathlineto{\pgfqpoint{1.591526in}{3.164741in}}%
\pgfpathlineto{\pgfqpoint{1.699081in}{3.170133in}}%
\pgfpathlineto{\pgfqpoint{1.806637in}{3.163585in}}%
\pgfpathlineto{\pgfqpoint{1.914193in}{3.146637in}}%
\pgfpathlineto{\pgfqpoint{2.021748in}{3.184385in}}%
\pgfpathlineto{\pgfqpoint{2.129304in}{3.154341in}}%
\pgfpathlineto{\pgfqpoint{2.236859in}{3.157037in}}%
\pgfpathlineto{\pgfqpoint{2.344415in}{3.193630in}}%
\pgfpathlineto{\pgfqpoint{2.451970in}{3.173985in}}%
\pgfpathlineto{\pgfqpoint{2.559526in}{3.184385in}}%
\pgfpathlineto{\pgfqpoint{2.667081in}{3.195170in}}%
\pgfpathlineto{\pgfqpoint{2.774637in}{3.167437in}}%
\pgfpathlineto{\pgfqpoint{2.882844in}{3.168593in}}%
\pgfpathlineto{\pgfqpoint{2.990400in}{3.221363in}}%
\pgfusepath{stroke}%
\end{pgfscope}%
\begin{pgfscope}%
\pgfpathrectangle{\pgfqpoint{0.720000in}{0.480000in}}{\pgfqpoint{7.040000in}{3.120000in}}%
\pgfusepath{clip}%
\pgfsetbuttcap%
\pgfsetroundjoin%
\definecolor{currentfill}{rgb}{0.501961,0.000000,0.501961}%
\pgfsetfillcolor{currentfill}%
\pgfsetlinewidth{1.003750pt}%
\definecolor{currentstroke}{rgb}{0.501961,0.000000,0.501961}%
\pgfsetstrokecolor{currentstroke}%
\pgfsetdash{}{0pt}%
\pgfsys@defobject{currentmarker}{\pgfqpoint{-0.010417in}{-0.010417in}}{\pgfqpoint{0.010417in}{0.010417in}}{%
\pgfpathmoveto{\pgfqpoint{0.000000in}{-0.010417in}}%
\pgfpathcurveto{\pgfqpoint{0.002763in}{-0.010417in}}{\pgfqpoint{0.005412in}{-0.009319in}}{\pgfqpoint{0.007366in}{-0.007366in}}%
\pgfpathcurveto{\pgfqpoint{0.009319in}{-0.005412in}}{\pgfqpoint{0.010417in}{-0.002763in}}{\pgfqpoint{0.010417in}{0.000000in}}%
\pgfpathcurveto{\pgfqpoint{0.010417in}{0.002763in}}{\pgfqpoint{0.009319in}{0.005412in}}{\pgfqpoint{0.007366in}{0.007366in}}%
\pgfpathcurveto{\pgfqpoint{0.005412in}{0.009319in}}{\pgfqpoint{0.002763in}{0.010417in}}{\pgfqpoint{0.000000in}{0.010417in}}%
\pgfpathcurveto{\pgfqpoint{-0.002763in}{0.010417in}}{\pgfqpoint{-0.005412in}{0.009319in}}{\pgfqpoint{-0.007366in}{0.007366in}}%
\pgfpathcurveto{\pgfqpoint{-0.009319in}{0.005412in}}{\pgfqpoint{-0.010417in}{0.002763in}}{\pgfqpoint{-0.010417in}{0.000000in}}%
\pgfpathcurveto{\pgfqpoint{-0.010417in}{-0.002763in}}{\pgfqpoint{-0.009319in}{-0.005412in}}{\pgfqpoint{-0.007366in}{-0.007366in}}%
\pgfpathcurveto{\pgfqpoint{-0.005412in}{-0.009319in}}{\pgfqpoint{-0.002763in}{-0.010417in}}{\pgfqpoint{0.000000in}{-0.010417in}}%
\pgfpathclose%
\pgfusepath{stroke,fill}%
}%
\begin{pgfscope}%
\pgfsys@transformshift{0.837985in}{2.121659in}%
\pgfsys@useobject{currentmarker}{}%
\end{pgfscope}%
\begin{pgfscope}%
\pgfsys@transformshift{0.945541in}{2.714844in}%
\pgfsys@useobject{currentmarker}{}%
\end{pgfscope}%
\begin{pgfscope}%
\pgfsys@transformshift{1.053748in}{2.813452in}%
\pgfsys@useobject{currentmarker}{}%
\end{pgfscope}%
\begin{pgfscope}%
\pgfsys@transformshift{1.161304in}{2.939407in}%
\pgfsys@useobject{currentmarker}{}%
\end{pgfscope}%
\begin{pgfscope}%
\pgfsys@transformshift{1.268859in}{3.079615in}%
\pgfsys@useobject{currentmarker}{}%
\end{pgfscope}%
\begin{pgfscope}%
\pgfsys@transformshift{1.376415in}{3.034933in}%
\pgfsys@useobject{currentmarker}{}%
\end{pgfscope}%
\begin{pgfscope}%
\pgfsys@transformshift{1.483970in}{3.099259in}%
\pgfsys@useobject{currentmarker}{}%
\end{pgfscope}%
\begin{pgfscope}%
\pgfsys@transformshift{1.591526in}{3.164741in}%
\pgfsys@useobject{currentmarker}{}%
\end{pgfscope}%
\begin{pgfscope}%
\pgfsys@transformshift{1.699081in}{3.170133in}%
\pgfsys@useobject{currentmarker}{}%
\end{pgfscope}%
\begin{pgfscope}%
\pgfsys@transformshift{1.806637in}{3.163585in}%
\pgfsys@useobject{currentmarker}{}%
\end{pgfscope}%
\begin{pgfscope}%
\pgfsys@transformshift{1.914193in}{3.146637in}%
\pgfsys@useobject{currentmarker}{}%
\end{pgfscope}%
\begin{pgfscope}%
\pgfsys@transformshift{2.021748in}{3.184385in}%
\pgfsys@useobject{currentmarker}{}%
\end{pgfscope}%
\begin{pgfscope}%
\pgfsys@transformshift{2.129304in}{3.154341in}%
\pgfsys@useobject{currentmarker}{}%
\end{pgfscope}%
\begin{pgfscope}%
\pgfsys@transformshift{2.236859in}{3.157037in}%
\pgfsys@useobject{currentmarker}{}%
\end{pgfscope}%
\begin{pgfscope}%
\pgfsys@transformshift{2.344415in}{3.193630in}%
\pgfsys@useobject{currentmarker}{}%
\end{pgfscope}%
\begin{pgfscope}%
\pgfsys@transformshift{2.451970in}{3.173985in}%
\pgfsys@useobject{currentmarker}{}%
\end{pgfscope}%
\begin{pgfscope}%
\pgfsys@transformshift{2.559526in}{3.184385in}%
\pgfsys@useobject{currentmarker}{}%
\end{pgfscope}%
\begin{pgfscope}%
\pgfsys@transformshift{2.667081in}{3.195170in}%
\pgfsys@useobject{currentmarker}{}%
\end{pgfscope}%
\begin{pgfscope}%
\pgfsys@transformshift{2.774637in}{3.167437in}%
\pgfsys@useobject{currentmarker}{}%
\end{pgfscope}%
\begin{pgfscope}%
\pgfsys@transformshift{2.882844in}{3.168593in}%
\pgfsys@useobject{currentmarker}{}%
\end{pgfscope}%
\begin{pgfscope}%
\pgfsys@transformshift{2.990400in}{3.221363in}%
\pgfsys@useobject{currentmarker}{}%
\end{pgfscope}%
\end{pgfscope}%
\begin{pgfscope}%
\pgfpathrectangle{\pgfqpoint{0.720000in}{0.480000in}}{\pgfqpoint{7.040000in}{3.120000in}}%
\pgfusepath{clip}%
\pgfsetrectcap%
\pgfsetroundjoin%
\pgfsetlinewidth{0.752812pt}%
\definecolor{currentstroke}{rgb}{0.392157,0.584314,0.929412}%
\pgfsetstrokecolor{currentstroke}%
\pgfsetdash{}{0pt}%
\pgfpathmoveto{\pgfqpoint{0.858844in}{2.314252in}}%
\pgfpathlineto{\pgfqpoint{0.989215in}{2.857748in}}%
\pgfpathlineto{\pgfqpoint{1.118281in}{2.979852in}}%
\pgfpathlineto{\pgfqpoint{1.247348in}{3.009896in}}%
\pgfpathlineto{\pgfqpoint{1.376415in}{3.154341in}}%
\pgfpathlineto{\pgfqpoint{1.505481in}{3.059970in}}%
\pgfpathlineto{\pgfqpoint{1.634548in}{3.192474in}}%
\pgfpathlineto{\pgfqpoint{1.763615in}{3.264504in}}%
\pgfpathlineto{\pgfqpoint{1.892681in}{3.288000in}}%
\pgfpathlineto{\pgfqpoint{2.021748in}{3.243319in}}%
\pgfpathlineto{\pgfqpoint{2.150815in}{3.269896in}}%
\pgfpathlineto{\pgfqpoint{2.279881in}{3.271052in}}%
\pgfpathlineto{\pgfqpoint{2.408948in}{3.247556in}}%
\pgfpathlineto{\pgfqpoint{2.538015in}{3.285304in}}%
\pgfpathlineto{\pgfqpoint{2.667081in}{3.285304in}}%
\pgfpathlineto{\pgfqpoint{2.796148in}{3.272207in}}%
\pgfpathlineto{\pgfqpoint{2.925215in}{3.261807in}}%
\pgfpathlineto{\pgfqpoint{3.054281in}{3.307644in}}%
\pgfpathlineto{\pgfqpoint{3.183348in}{3.281452in}}%
\pgfpathlineto{\pgfqpoint{3.312415in}{3.260652in}}%
\pgfpathlineto{\pgfqpoint{3.441481in}{3.294548in}}%
\pgfusepath{stroke}%
\end{pgfscope}%
\begin{pgfscope}%
\pgfpathrectangle{\pgfqpoint{0.720000in}{0.480000in}}{\pgfqpoint{7.040000in}{3.120000in}}%
\pgfusepath{clip}%
\pgfsetbuttcap%
\pgfsetroundjoin%
\definecolor{currentfill}{rgb}{0.392157,0.584314,0.929412}%
\pgfsetfillcolor{currentfill}%
\pgfsetlinewidth{1.003750pt}%
\definecolor{currentstroke}{rgb}{0.392157,0.584314,0.929412}%
\pgfsetstrokecolor{currentstroke}%
\pgfsetdash{}{0pt}%
\pgfsys@defobject{currentmarker}{\pgfqpoint{-0.010417in}{-0.010417in}}{\pgfqpoint{0.010417in}{0.010417in}}{%
\pgfpathmoveto{\pgfqpoint{0.000000in}{-0.010417in}}%
\pgfpathcurveto{\pgfqpoint{0.002763in}{-0.010417in}}{\pgfqpoint{0.005412in}{-0.009319in}}{\pgfqpoint{0.007366in}{-0.007366in}}%
\pgfpathcurveto{\pgfqpoint{0.009319in}{-0.005412in}}{\pgfqpoint{0.010417in}{-0.002763in}}{\pgfqpoint{0.010417in}{0.000000in}}%
\pgfpathcurveto{\pgfqpoint{0.010417in}{0.002763in}}{\pgfqpoint{0.009319in}{0.005412in}}{\pgfqpoint{0.007366in}{0.007366in}}%
\pgfpathcurveto{\pgfqpoint{0.005412in}{0.009319in}}{\pgfqpoint{0.002763in}{0.010417in}}{\pgfqpoint{0.000000in}{0.010417in}}%
\pgfpathcurveto{\pgfqpoint{-0.002763in}{0.010417in}}{\pgfqpoint{-0.005412in}{0.009319in}}{\pgfqpoint{-0.007366in}{0.007366in}}%
\pgfpathcurveto{\pgfqpoint{-0.009319in}{0.005412in}}{\pgfqpoint{-0.010417in}{0.002763in}}{\pgfqpoint{-0.010417in}{0.000000in}}%
\pgfpathcurveto{\pgfqpoint{-0.010417in}{-0.002763in}}{\pgfqpoint{-0.009319in}{-0.005412in}}{\pgfqpoint{-0.007366in}{-0.007366in}}%
\pgfpathcurveto{\pgfqpoint{-0.005412in}{-0.009319in}}{\pgfqpoint{-0.002763in}{-0.010417in}}{\pgfqpoint{0.000000in}{-0.010417in}}%
\pgfpathclose%
\pgfusepath{stroke,fill}%
}%
\begin{pgfscope}%
\pgfsys@transformshift{0.858844in}{2.314252in}%
\pgfsys@useobject{currentmarker}{}%
\end{pgfscope}%
\begin{pgfscope}%
\pgfsys@transformshift{0.989215in}{2.857748in}%
\pgfsys@useobject{currentmarker}{}%
\end{pgfscope}%
\begin{pgfscope}%
\pgfsys@transformshift{1.118281in}{2.979852in}%
\pgfsys@useobject{currentmarker}{}%
\end{pgfscope}%
\begin{pgfscope}%
\pgfsys@transformshift{1.247348in}{3.009896in}%
\pgfsys@useobject{currentmarker}{}%
\end{pgfscope}%
\begin{pgfscope}%
\pgfsys@transformshift{1.376415in}{3.154341in}%
\pgfsys@useobject{currentmarker}{}%
\end{pgfscope}%
\begin{pgfscope}%
\pgfsys@transformshift{1.505481in}{3.059970in}%
\pgfsys@useobject{currentmarker}{}%
\end{pgfscope}%
\begin{pgfscope}%
\pgfsys@transformshift{1.634548in}{3.192474in}%
\pgfsys@useobject{currentmarker}{}%
\end{pgfscope}%
\begin{pgfscope}%
\pgfsys@transformshift{1.763615in}{3.264504in}%
\pgfsys@useobject{currentmarker}{}%
\end{pgfscope}%
\begin{pgfscope}%
\pgfsys@transformshift{1.892681in}{3.288000in}%
\pgfsys@useobject{currentmarker}{}%
\end{pgfscope}%
\begin{pgfscope}%
\pgfsys@transformshift{2.021748in}{3.243319in}%
\pgfsys@useobject{currentmarker}{}%
\end{pgfscope}%
\begin{pgfscope}%
\pgfsys@transformshift{2.150815in}{3.269896in}%
\pgfsys@useobject{currentmarker}{}%
\end{pgfscope}%
\begin{pgfscope}%
\pgfsys@transformshift{2.279881in}{3.271052in}%
\pgfsys@useobject{currentmarker}{}%
\end{pgfscope}%
\begin{pgfscope}%
\pgfsys@transformshift{2.408948in}{3.247556in}%
\pgfsys@useobject{currentmarker}{}%
\end{pgfscope}%
\begin{pgfscope}%
\pgfsys@transformshift{2.538015in}{3.285304in}%
\pgfsys@useobject{currentmarker}{}%
\end{pgfscope}%
\begin{pgfscope}%
\pgfsys@transformshift{2.667081in}{3.285304in}%
\pgfsys@useobject{currentmarker}{}%
\end{pgfscope}%
\begin{pgfscope}%
\pgfsys@transformshift{2.796148in}{3.272207in}%
\pgfsys@useobject{currentmarker}{}%
\end{pgfscope}%
\begin{pgfscope}%
\pgfsys@transformshift{2.925215in}{3.261807in}%
\pgfsys@useobject{currentmarker}{}%
\end{pgfscope}%
\begin{pgfscope}%
\pgfsys@transformshift{3.054281in}{3.307644in}%
\pgfsys@useobject{currentmarker}{}%
\end{pgfscope}%
\begin{pgfscope}%
\pgfsys@transformshift{3.183348in}{3.281452in}%
\pgfsys@useobject{currentmarker}{}%
\end{pgfscope}%
\begin{pgfscope}%
\pgfsys@transformshift{3.312415in}{3.260652in}%
\pgfsys@useobject{currentmarker}{}%
\end{pgfscope}%
\begin{pgfscope}%
\pgfsys@transformshift{3.441481in}{3.294548in}%
\pgfsys@useobject{currentmarker}{}%
\end{pgfscope}%
\end{pgfscope}%
\begin{pgfscope}%
\pgfpathrectangle{\pgfqpoint{0.720000in}{0.480000in}}{\pgfqpoint{7.040000in}{3.120000in}}%
\pgfusepath{clip}%
\pgfsetrectcap%
\pgfsetroundjoin%
\pgfsetlinewidth{0.752812pt}%
\definecolor{currentstroke}{rgb}{1.000000,0.752941,0.796078}%
\pgfsetstrokecolor{currentstroke}%
\pgfsetdash{}{0pt}%
\pgfpathmoveto{\pgfqpoint{0.881659in}{2.438281in}}%
\pgfpathlineto{\pgfqpoint{1.032889in}{2.769926in}}%
\pgfpathlineto{\pgfqpoint{1.184119in}{3.083467in}}%
\pgfpathlineto{\pgfqpoint{1.335348in}{3.145096in}}%
\pgfpathlineto{\pgfqpoint{1.486578in}{3.076919in}}%
\pgfpathlineto{\pgfqpoint{1.637807in}{3.122756in}}%
\pgfpathlineto{\pgfqpoint{1.789037in}{3.146637in}}%
\pgfpathlineto{\pgfqpoint{1.940267in}{3.251407in}}%
\pgfpathlineto{\pgfqpoint{2.091496in}{3.294548in}}%
\pgfpathlineto{\pgfqpoint{2.242726in}{3.291852in}}%
\pgfpathlineto{\pgfqpoint{2.393956in}{3.316889in}}%
\pgfpathlineto{\pgfqpoint{2.545185in}{3.290696in}}%
\pgfpathlineto{\pgfqpoint{2.696415in}{3.299941in}}%
\pgfpathlineto{\pgfqpoint{2.847644in}{3.332681in}}%
\pgfpathlineto{\pgfqpoint{2.998874in}{3.335378in}}%
\pgfpathlineto{\pgfqpoint{3.150104in}{3.341926in}}%
\pgfpathlineto{\pgfqpoint{3.301333in}{3.314193in}}%
\pgfpathlineto{\pgfqpoint{3.452563in}{3.319585in}}%
\pgfpathlineto{\pgfqpoint{3.603793in}{3.351170in}}%
\pgfpathlineto{\pgfqpoint{3.755022in}{3.309185in}}%
\pgfpathlineto{\pgfqpoint{3.906252in}{3.327289in}}%
\pgfusepath{stroke}%
\end{pgfscope}%
\begin{pgfscope}%
\pgfpathrectangle{\pgfqpoint{0.720000in}{0.480000in}}{\pgfqpoint{7.040000in}{3.120000in}}%
\pgfusepath{clip}%
\pgfsetbuttcap%
\pgfsetroundjoin%
\definecolor{currentfill}{rgb}{1.000000,0.752941,0.796078}%
\pgfsetfillcolor{currentfill}%
\pgfsetlinewidth{1.003750pt}%
\definecolor{currentstroke}{rgb}{1.000000,0.752941,0.796078}%
\pgfsetstrokecolor{currentstroke}%
\pgfsetdash{}{0pt}%
\pgfsys@defobject{currentmarker}{\pgfqpoint{-0.010417in}{-0.010417in}}{\pgfqpoint{0.010417in}{0.010417in}}{%
\pgfpathmoveto{\pgfqpoint{0.000000in}{-0.010417in}}%
\pgfpathcurveto{\pgfqpoint{0.002763in}{-0.010417in}}{\pgfqpoint{0.005412in}{-0.009319in}}{\pgfqpoint{0.007366in}{-0.007366in}}%
\pgfpathcurveto{\pgfqpoint{0.009319in}{-0.005412in}}{\pgfqpoint{0.010417in}{-0.002763in}}{\pgfqpoint{0.010417in}{0.000000in}}%
\pgfpathcurveto{\pgfqpoint{0.010417in}{0.002763in}}{\pgfqpoint{0.009319in}{0.005412in}}{\pgfqpoint{0.007366in}{0.007366in}}%
\pgfpathcurveto{\pgfqpoint{0.005412in}{0.009319in}}{\pgfqpoint{0.002763in}{0.010417in}}{\pgfqpoint{0.000000in}{0.010417in}}%
\pgfpathcurveto{\pgfqpoint{-0.002763in}{0.010417in}}{\pgfqpoint{-0.005412in}{0.009319in}}{\pgfqpoint{-0.007366in}{0.007366in}}%
\pgfpathcurveto{\pgfqpoint{-0.009319in}{0.005412in}}{\pgfqpoint{-0.010417in}{0.002763in}}{\pgfqpoint{-0.010417in}{0.000000in}}%
\pgfpathcurveto{\pgfqpoint{-0.010417in}{-0.002763in}}{\pgfqpoint{-0.009319in}{-0.005412in}}{\pgfqpoint{-0.007366in}{-0.007366in}}%
\pgfpathcurveto{\pgfqpoint{-0.005412in}{-0.009319in}}{\pgfqpoint{-0.002763in}{-0.010417in}}{\pgfqpoint{0.000000in}{-0.010417in}}%
\pgfpathclose%
\pgfusepath{stroke,fill}%
}%
\begin{pgfscope}%
\pgfsys@transformshift{0.881659in}{2.438281in}%
\pgfsys@useobject{currentmarker}{}%
\end{pgfscope}%
\begin{pgfscope}%
\pgfsys@transformshift{1.032889in}{2.769926in}%
\pgfsys@useobject{currentmarker}{}%
\end{pgfscope}%
\begin{pgfscope}%
\pgfsys@transformshift{1.184119in}{3.083467in}%
\pgfsys@useobject{currentmarker}{}%
\end{pgfscope}%
\begin{pgfscope}%
\pgfsys@transformshift{1.335348in}{3.145096in}%
\pgfsys@useobject{currentmarker}{}%
\end{pgfscope}%
\begin{pgfscope}%
\pgfsys@transformshift{1.486578in}{3.076919in}%
\pgfsys@useobject{currentmarker}{}%
\end{pgfscope}%
\begin{pgfscope}%
\pgfsys@transformshift{1.637807in}{3.122756in}%
\pgfsys@useobject{currentmarker}{}%
\end{pgfscope}%
\begin{pgfscope}%
\pgfsys@transformshift{1.789037in}{3.146637in}%
\pgfsys@useobject{currentmarker}{}%
\end{pgfscope}%
\begin{pgfscope}%
\pgfsys@transformshift{1.940267in}{3.251407in}%
\pgfsys@useobject{currentmarker}{}%
\end{pgfscope}%
\begin{pgfscope}%
\pgfsys@transformshift{2.091496in}{3.294548in}%
\pgfsys@useobject{currentmarker}{}%
\end{pgfscope}%
\begin{pgfscope}%
\pgfsys@transformshift{2.242726in}{3.291852in}%
\pgfsys@useobject{currentmarker}{}%
\end{pgfscope}%
\begin{pgfscope}%
\pgfsys@transformshift{2.393956in}{3.316889in}%
\pgfsys@useobject{currentmarker}{}%
\end{pgfscope}%
\begin{pgfscope}%
\pgfsys@transformshift{2.545185in}{3.290696in}%
\pgfsys@useobject{currentmarker}{}%
\end{pgfscope}%
\begin{pgfscope}%
\pgfsys@transformshift{2.696415in}{3.299941in}%
\pgfsys@useobject{currentmarker}{}%
\end{pgfscope}%
\begin{pgfscope}%
\pgfsys@transformshift{2.847644in}{3.332681in}%
\pgfsys@useobject{currentmarker}{}%
\end{pgfscope}%
\begin{pgfscope}%
\pgfsys@transformshift{2.998874in}{3.335378in}%
\pgfsys@useobject{currentmarker}{}%
\end{pgfscope}%
\begin{pgfscope}%
\pgfsys@transformshift{3.150104in}{3.341926in}%
\pgfsys@useobject{currentmarker}{}%
\end{pgfscope}%
\begin{pgfscope}%
\pgfsys@transformshift{3.301333in}{3.314193in}%
\pgfsys@useobject{currentmarker}{}%
\end{pgfscope}%
\begin{pgfscope}%
\pgfsys@transformshift{3.452563in}{3.319585in}%
\pgfsys@useobject{currentmarker}{}%
\end{pgfscope}%
\begin{pgfscope}%
\pgfsys@transformshift{3.603793in}{3.351170in}%
\pgfsys@useobject{currentmarker}{}%
\end{pgfscope}%
\begin{pgfscope}%
\pgfsys@transformshift{3.755022in}{3.309185in}%
\pgfsys@useobject{currentmarker}{}%
\end{pgfscope}%
\begin{pgfscope}%
\pgfsys@transformshift{3.906252in}{3.327289in}%
\pgfsys@useobject{currentmarker}{}%
\end{pgfscope}%
\end{pgfscope}%
\begin{pgfscope}%
\pgfpathrectangle{\pgfqpoint{0.720000in}{0.480000in}}{\pgfqpoint{7.040000in}{3.120000in}}%
\pgfusepath{clip}%
\pgfsetrectcap%
\pgfsetroundjoin%
\pgfsetlinewidth{0.752812pt}%
\definecolor{currentstroke}{rgb}{0.501961,0.501961,0.501961}%
\pgfsetstrokecolor{currentstroke}%
\pgfsetdash{}{0pt}%
\pgfpathmoveto{\pgfqpoint{0.905126in}{2.608533in}}%
\pgfpathlineto{\pgfqpoint{1.078519in}{2.934015in}}%
\pgfpathlineto{\pgfqpoint{1.249956in}{3.033778in}}%
\pgfpathlineto{\pgfqpoint{1.422044in}{3.122756in}}%
\pgfpathlineto{\pgfqpoint{1.593481in}{3.177837in}}%
\pgfpathlineto{\pgfqpoint{1.765570in}{3.193630in}}%
\pgfpathlineto{\pgfqpoint{1.937007in}{3.160889in}}%
\pgfpathlineto{\pgfqpoint{2.108444in}{3.309185in}}%
\pgfpathlineto{\pgfqpoint{2.280533in}{3.319585in}}%
\pgfpathlineto{\pgfqpoint{2.452622in}{3.336533in}}%
\pgfpathlineto{\pgfqpoint{2.624711in}{3.316889in}}%
\pgfpathlineto{\pgfqpoint{2.796800in}{3.364267in}}%
\pgfpathlineto{\pgfqpoint{2.968237in}{3.331526in}}%
\pgfpathlineto{\pgfqpoint{3.140326in}{3.371970in}}%
\pgfpathlineto{\pgfqpoint{3.312415in}{3.371970in}}%
\pgfpathlineto{\pgfqpoint{3.484504in}{3.369274in}}%
\pgfpathlineto{\pgfqpoint{3.656593in}{3.366963in}}%
\pgfpathlineto{\pgfqpoint{3.828681in}{3.393156in}}%
\pgfpathlineto{\pgfqpoint{4.000770in}{3.353867in}}%
\pgfpathlineto{\pgfqpoint{4.172859in}{3.380059in}}%
\pgfpathlineto{\pgfqpoint{4.344948in}{3.361570in}}%
\pgfusepath{stroke}%
\end{pgfscope}%
\begin{pgfscope}%
\pgfpathrectangle{\pgfqpoint{0.720000in}{0.480000in}}{\pgfqpoint{7.040000in}{3.120000in}}%
\pgfusepath{clip}%
\pgfsetbuttcap%
\pgfsetroundjoin%
\definecolor{currentfill}{rgb}{0.501961,0.501961,0.501961}%
\pgfsetfillcolor{currentfill}%
\pgfsetlinewidth{1.003750pt}%
\definecolor{currentstroke}{rgb}{0.501961,0.501961,0.501961}%
\pgfsetstrokecolor{currentstroke}%
\pgfsetdash{}{0pt}%
\pgfsys@defobject{currentmarker}{\pgfqpoint{-0.010417in}{-0.010417in}}{\pgfqpoint{0.010417in}{0.010417in}}{%
\pgfpathmoveto{\pgfqpoint{0.000000in}{-0.010417in}}%
\pgfpathcurveto{\pgfqpoint{0.002763in}{-0.010417in}}{\pgfqpoint{0.005412in}{-0.009319in}}{\pgfqpoint{0.007366in}{-0.007366in}}%
\pgfpathcurveto{\pgfqpoint{0.009319in}{-0.005412in}}{\pgfqpoint{0.010417in}{-0.002763in}}{\pgfqpoint{0.010417in}{0.000000in}}%
\pgfpathcurveto{\pgfqpoint{0.010417in}{0.002763in}}{\pgfqpoint{0.009319in}{0.005412in}}{\pgfqpoint{0.007366in}{0.007366in}}%
\pgfpathcurveto{\pgfqpoint{0.005412in}{0.009319in}}{\pgfqpoint{0.002763in}{0.010417in}}{\pgfqpoint{0.000000in}{0.010417in}}%
\pgfpathcurveto{\pgfqpoint{-0.002763in}{0.010417in}}{\pgfqpoint{-0.005412in}{0.009319in}}{\pgfqpoint{-0.007366in}{0.007366in}}%
\pgfpathcurveto{\pgfqpoint{-0.009319in}{0.005412in}}{\pgfqpoint{-0.010417in}{0.002763in}}{\pgfqpoint{-0.010417in}{0.000000in}}%
\pgfpathcurveto{\pgfqpoint{-0.010417in}{-0.002763in}}{\pgfqpoint{-0.009319in}{-0.005412in}}{\pgfqpoint{-0.007366in}{-0.007366in}}%
\pgfpathcurveto{\pgfqpoint{-0.005412in}{-0.009319in}}{\pgfqpoint{-0.002763in}{-0.010417in}}{\pgfqpoint{0.000000in}{-0.010417in}}%
\pgfpathclose%
\pgfusepath{stroke,fill}%
}%
\begin{pgfscope}%
\pgfsys@transformshift{0.905126in}{2.608533in}%
\pgfsys@useobject{currentmarker}{}%
\end{pgfscope}%
\begin{pgfscope}%
\pgfsys@transformshift{1.078519in}{2.934015in}%
\pgfsys@useobject{currentmarker}{}%
\end{pgfscope}%
\begin{pgfscope}%
\pgfsys@transformshift{1.249956in}{3.033778in}%
\pgfsys@useobject{currentmarker}{}%
\end{pgfscope}%
\begin{pgfscope}%
\pgfsys@transformshift{1.422044in}{3.122756in}%
\pgfsys@useobject{currentmarker}{}%
\end{pgfscope}%
\begin{pgfscope}%
\pgfsys@transformshift{1.593481in}{3.177837in}%
\pgfsys@useobject{currentmarker}{}%
\end{pgfscope}%
\begin{pgfscope}%
\pgfsys@transformshift{1.765570in}{3.193630in}%
\pgfsys@useobject{currentmarker}{}%
\end{pgfscope}%
\begin{pgfscope}%
\pgfsys@transformshift{1.937007in}{3.160889in}%
\pgfsys@useobject{currentmarker}{}%
\end{pgfscope}%
\begin{pgfscope}%
\pgfsys@transformshift{2.108444in}{3.309185in}%
\pgfsys@useobject{currentmarker}{}%
\end{pgfscope}%
\begin{pgfscope}%
\pgfsys@transformshift{2.280533in}{3.319585in}%
\pgfsys@useobject{currentmarker}{}%
\end{pgfscope}%
\begin{pgfscope}%
\pgfsys@transformshift{2.452622in}{3.336533in}%
\pgfsys@useobject{currentmarker}{}%
\end{pgfscope}%
\begin{pgfscope}%
\pgfsys@transformshift{2.624711in}{3.316889in}%
\pgfsys@useobject{currentmarker}{}%
\end{pgfscope}%
\begin{pgfscope}%
\pgfsys@transformshift{2.796800in}{3.364267in}%
\pgfsys@useobject{currentmarker}{}%
\end{pgfscope}%
\begin{pgfscope}%
\pgfsys@transformshift{2.968237in}{3.331526in}%
\pgfsys@useobject{currentmarker}{}%
\end{pgfscope}%
\begin{pgfscope}%
\pgfsys@transformshift{3.140326in}{3.371970in}%
\pgfsys@useobject{currentmarker}{}%
\end{pgfscope}%
\begin{pgfscope}%
\pgfsys@transformshift{3.312415in}{3.371970in}%
\pgfsys@useobject{currentmarker}{}%
\end{pgfscope}%
\begin{pgfscope}%
\pgfsys@transformshift{3.484504in}{3.369274in}%
\pgfsys@useobject{currentmarker}{}%
\end{pgfscope}%
\begin{pgfscope}%
\pgfsys@transformshift{3.656593in}{3.366963in}%
\pgfsys@useobject{currentmarker}{}%
\end{pgfscope}%
\begin{pgfscope}%
\pgfsys@transformshift{3.828681in}{3.393156in}%
\pgfsys@useobject{currentmarker}{}%
\end{pgfscope}%
\begin{pgfscope}%
\pgfsys@transformshift{4.000770in}{3.353867in}%
\pgfsys@useobject{currentmarker}{}%
\end{pgfscope}%
\begin{pgfscope}%
\pgfsys@transformshift{4.172859in}{3.380059in}%
\pgfsys@useobject{currentmarker}{}%
\end{pgfscope}%
\begin{pgfscope}%
\pgfsys@transformshift{4.344948in}{3.361570in}%
\pgfsys@useobject{currentmarker}{}%
\end{pgfscope}%
\end{pgfscope}%
\begin{pgfscope}%
\pgfpathrectangle{\pgfqpoint{0.720000in}{0.480000in}}{\pgfqpoint{7.040000in}{3.120000in}}%
\pgfusepath{clip}%
\pgfsetrectcap%
\pgfsetroundjoin%
\pgfsetlinewidth{0.752812pt}%
\definecolor{currentstroke}{rgb}{0.501961,0.501961,0.000000}%
\pgfsetstrokecolor{currentstroke}%
\pgfsetdash{}{0pt}%
\pgfpathmoveto{\pgfqpoint{0.924030in}{2.648978in}}%
\pgfpathlineto{\pgfqpoint{1.118281in}{3.062667in}}%
\pgfpathlineto{\pgfqpoint{1.312533in}{3.115052in}}%
\pgfpathlineto{\pgfqpoint{1.506785in}{3.133156in}}%
\pgfpathlineto{\pgfqpoint{1.701037in}{3.208267in}}%
\pgfpathlineto{\pgfqpoint{1.895289in}{3.117748in}}%
\pgfpathlineto{\pgfqpoint{2.089541in}{3.209422in}}%
\pgfpathlineto{\pgfqpoint{2.283793in}{3.370815in}}%
\pgfpathlineto{\pgfqpoint{2.478044in}{3.356178in}}%
\pgfpathlineto{\pgfqpoint{2.672296in}{3.399704in}}%
\pgfpathlineto{\pgfqpoint{2.866548in}{3.380059in}}%
\pgfpathlineto{\pgfqpoint{3.060800in}{3.341926in}}%
\pgfpathlineto{\pgfqpoint{3.255052in}{3.358874in}}%
\pgfpathlineto{\pgfqpoint{3.449304in}{3.388919in}}%
\pgfpathlineto{\pgfqpoint{3.642904in}{3.358874in}}%
\pgfpathlineto{\pgfqpoint{3.837156in}{3.416652in}}%
\pgfpathlineto{\pgfqpoint{4.031407in}{3.386607in}}%
\pgfpathlineto{\pgfqpoint{4.225659in}{3.388919in}}%
\pgfpathlineto{\pgfqpoint{4.419911in}{3.400859in}}%
\pgfpathlineto{\pgfqpoint{4.614163in}{3.382370in}}%
\pgfpathlineto{\pgfqpoint{4.808415in}{3.403556in}}%
\pgfusepath{stroke}%
\end{pgfscope}%
\begin{pgfscope}%
\pgfpathrectangle{\pgfqpoint{0.720000in}{0.480000in}}{\pgfqpoint{7.040000in}{3.120000in}}%
\pgfusepath{clip}%
\pgfsetbuttcap%
\pgfsetroundjoin%
\definecolor{currentfill}{rgb}{0.501961,0.501961,0.000000}%
\pgfsetfillcolor{currentfill}%
\pgfsetlinewidth{1.003750pt}%
\definecolor{currentstroke}{rgb}{0.501961,0.501961,0.000000}%
\pgfsetstrokecolor{currentstroke}%
\pgfsetdash{}{0pt}%
\pgfsys@defobject{currentmarker}{\pgfqpoint{-0.010417in}{-0.010417in}}{\pgfqpoint{0.010417in}{0.010417in}}{%
\pgfpathmoveto{\pgfqpoint{0.000000in}{-0.010417in}}%
\pgfpathcurveto{\pgfqpoint{0.002763in}{-0.010417in}}{\pgfqpoint{0.005412in}{-0.009319in}}{\pgfqpoint{0.007366in}{-0.007366in}}%
\pgfpathcurveto{\pgfqpoint{0.009319in}{-0.005412in}}{\pgfqpoint{0.010417in}{-0.002763in}}{\pgfqpoint{0.010417in}{0.000000in}}%
\pgfpathcurveto{\pgfqpoint{0.010417in}{0.002763in}}{\pgfqpoint{0.009319in}{0.005412in}}{\pgfqpoint{0.007366in}{0.007366in}}%
\pgfpathcurveto{\pgfqpoint{0.005412in}{0.009319in}}{\pgfqpoint{0.002763in}{0.010417in}}{\pgfqpoint{0.000000in}{0.010417in}}%
\pgfpathcurveto{\pgfqpoint{-0.002763in}{0.010417in}}{\pgfqpoint{-0.005412in}{0.009319in}}{\pgfqpoint{-0.007366in}{0.007366in}}%
\pgfpathcurveto{\pgfqpoint{-0.009319in}{0.005412in}}{\pgfqpoint{-0.010417in}{0.002763in}}{\pgfqpoint{-0.010417in}{0.000000in}}%
\pgfpathcurveto{\pgfqpoint{-0.010417in}{-0.002763in}}{\pgfqpoint{-0.009319in}{-0.005412in}}{\pgfqpoint{-0.007366in}{-0.007366in}}%
\pgfpathcurveto{\pgfqpoint{-0.005412in}{-0.009319in}}{\pgfqpoint{-0.002763in}{-0.010417in}}{\pgfqpoint{0.000000in}{-0.010417in}}%
\pgfpathclose%
\pgfusepath{stroke,fill}%
}%
\begin{pgfscope}%
\pgfsys@transformshift{0.924030in}{2.648978in}%
\pgfsys@useobject{currentmarker}{}%
\end{pgfscope}%
\begin{pgfscope}%
\pgfsys@transformshift{1.118281in}{3.062667in}%
\pgfsys@useobject{currentmarker}{}%
\end{pgfscope}%
\begin{pgfscope}%
\pgfsys@transformshift{1.312533in}{3.115052in}%
\pgfsys@useobject{currentmarker}{}%
\end{pgfscope}%
\begin{pgfscope}%
\pgfsys@transformshift{1.506785in}{3.133156in}%
\pgfsys@useobject{currentmarker}{}%
\end{pgfscope}%
\begin{pgfscope}%
\pgfsys@transformshift{1.701037in}{3.208267in}%
\pgfsys@useobject{currentmarker}{}%
\end{pgfscope}%
\begin{pgfscope}%
\pgfsys@transformshift{1.895289in}{3.117748in}%
\pgfsys@useobject{currentmarker}{}%
\end{pgfscope}%
\begin{pgfscope}%
\pgfsys@transformshift{2.089541in}{3.209422in}%
\pgfsys@useobject{currentmarker}{}%
\end{pgfscope}%
\begin{pgfscope}%
\pgfsys@transformshift{2.283793in}{3.370815in}%
\pgfsys@useobject{currentmarker}{}%
\end{pgfscope}%
\begin{pgfscope}%
\pgfsys@transformshift{2.478044in}{3.356178in}%
\pgfsys@useobject{currentmarker}{}%
\end{pgfscope}%
\begin{pgfscope}%
\pgfsys@transformshift{2.672296in}{3.399704in}%
\pgfsys@useobject{currentmarker}{}%
\end{pgfscope}%
\begin{pgfscope}%
\pgfsys@transformshift{2.866548in}{3.380059in}%
\pgfsys@useobject{currentmarker}{}%
\end{pgfscope}%
\begin{pgfscope}%
\pgfsys@transformshift{3.060800in}{3.341926in}%
\pgfsys@useobject{currentmarker}{}%
\end{pgfscope}%
\begin{pgfscope}%
\pgfsys@transformshift{3.255052in}{3.358874in}%
\pgfsys@useobject{currentmarker}{}%
\end{pgfscope}%
\begin{pgfscope}%
\pgfsys@transformshift{3.449304in}{3.388919in}%
\pgfsys@useobject{currentmarker}{}%
\end{pgfscope}%
\begin{pgfscope}%
\pgfsys@transformshift{3.642904in}{3.358874in}%
\pgfsys@useobject{currentmarker}{}%
\end{pgfscope}%
\begin{pgfscope}%
\pgfsys@transformshift{3.837156in}{3.416652in}%
\pgfsys@useobject{currentmarker}{}%
\end{pgfscope}%
\begin{pgfscope}%
\pgfsys@transformshift{4.031407in}{3.386607in}%
\pgfsys@useobject{currentmarker}{}%
\end{pgfscope}%
\begin{pgfscope}%
\pgfsys@transformshift{4.225659in}{3.388919in}%
\pgfsys@useobject{currentmarker}{}%
\end{pgfscope}%
\begin{pgfscope}%
\pgfsys@transformshift{4.419911in}{3.400859in}%
\pgfsys@useobject{currentmarker}{}%
\end{pgfscope}%
\begin{pgfscope}%
\pgfsys@transformshift{4.614163in}{3.382370in}%
\pgfsys@useobject{currentmarker}{}%
\end{pgfscope}%
\begin{pgfscope}%
\pgfsys@transformshift{4.808415in}{3.403556in}%
\pgfsys@useobject{currentmarker}{}%
\end{pgfscope}%
\end{pgfscope}%
\begin{pgfscope}%
\pgfpathrectangle{\pgfqpoint{0.720000in}{0.480000in}}{\pgfqpoint{7.040000in}{3.120000in}}%
\pgfusepath{clip}%
\pgfsetrectcap%
\pgfsetroundjoin%
\pgfsetlinewidth{0.752812pt}%
\definecolor{currentstroke}{rgb}{0.000000,1.000000,1.000000}%
\pgfsetstrokecolor{currentstroke}%
\pgfsetdash{}{0pt}%
\pgfpathmoveto{\pgfqpoint{0.945541in}{2.731407in}}%
\pgfpathlineto{\pgfqpoint{1.162607in}{2.880089in}}%
\pgfpathlineto{\pgfqpoint{1.378370in}{3.183230in}}%
\pgfpathlineto{\pgfqpoint{1.594133in}{3.208267in}}%
\pgfpathlineto{\pgfqpoint{1.809896in}{3.254104in}}%
\pgfpathlineto{\pgfqpoint{2.025659in}{3.288000in}}%
\pgfpathlineto{\pgfqpoint{2.241422in}{3.301096in}}%
\pgfpathlineto{\pgfqpoint{2.457185in}{3.398163in}}%
\pgfpathlineto{\pgfqpoint{2.672948in}{3.413956in}}%
\pgfpathlineto{\pgfqpoint{2.888711in}{3.388919in}}%
\pgfpathlineto{\pgfqpoint{3.104474in}{3.388919in}}%
\pgfpathlineto{\pgfqpoint{3.320237in}{3.352326in}}%
\pgfpathlineto{\pgfqpoint{3.536000in}{3.398163in}}%
\pgfpathlineto{\pgfqpoint{3.751763in}{3.368119in}}%
\pgfpathlineto{\pgfqpoint{3.967526in}{3.408948in}}%
\pgfpathlineto{\pgfqpoint{4.183289in}{3.412800in}}%
\pgfpathlineto{\pgfqpoint{4.399052in}{3.412800in}}%
\pgfpathlineto{\pgfqpoint{4.614815in}{3.394311in}}%
\pgfpathlineto{\pgfqpoint{4.830578in}{3.419348in}}%
\pgfpathlineto{\pgfqpoint{5.046341in}{3.382370in}}%
\pgfpathlineto{\pgfqpoint{5.262756in}{3.415496in}}%
\pgfusepath{stroke}%
\end{pgfscope}%
\begin{pgfscope}%
\pgfpathrectangle{\pgfqpoint{0.720000in}{0.480000in}}{\pgfqpoint{7.040000in}{3.120000in}}%
\pgfusepath{clip}%
\pgfsetbuttcap%
\pgfsetroundjoin%
\definecolor{currentfill}{rgb}{0.000000,1.000000,1.000000}%
\pgfsetfillcolor{currentfill}%
\pgfsetlinewidth{1.003750pt}%
\definecolor{currentstroke}{rgb}{0.000000,1.000000,1.000000}%
\pgfsetstrokecolor{currentstroke}%
\pgfsetdash{}{0pt}%
\pgfsys@defobject{currentmarker}{\pgfqpoint{-0.010417in}{-0.010417in}}{\pgfqpoint{0.010417in}{0.010417in}}{%
\pgfpathmoveto{\pgfqpoint{0.000000in}{-0.010417in}}%
\pgfpathcurveto{\pgfqpoint{0.002763in}{-0.010417in}}{\pgfqpoint{0.005412in}{-0.009319in}}{\pgfqpoint{0.007366in}{-0.007366in}}%
\pgfpathcurveto{\pgfqpoint{0.009319in}{-0.005412in}}{\pgfqpoint{0.010417in}{-0.002763in}}{\pgfqpoint{0.010417in}{0.000000in}}%
\pgfpathcurveto{\pgfqpoint{0.010417in}{0.002763in}}{\pgfqpoint{0.009319in}{0.005412in}}{\pgfqpoint{0.007366in}{0.007366in}}%
\pgfpathcurveto{\pgfqpoint{0.005412in}{0.009319in}}{\pgfqpoint{0.002763in}{0.010417in}}{\pgfqpoint{0.000000in}{0.010417in}}%
\pgfpathcurveto{\pgfqpoint{-0.002763in}{0.010417in}}{\pgfqpoint{-0.005412in}{0.009319in}}{\pgfqpoint{-0.007366in}{0.007366in}}%
\pgfpathcurveto{\pgfqpoint{-0.009319in}{0.005412in}}{\pgfqpoint{-0.010417in}{0.002763in}}{\pgfqpoint{-0.010417in}{0.000000in}}%
\pgfpathcurveto{\pgfqpoint{-0.010417in}{-0.002763in}}{\pgfqpoint{-0.009319in}{-0.005412in}}{\pgfqpoint{-0.007366in}{-0.007366in}}%
\pgfpathcurveto{\pgfqpoint{-0.005412in}{-0.009319in}}{\pgfqpoint{-0.002763in}{-0.010417in}}{\pgfqpoint{0.000000in}{-0.010417in}}%
\pgfpathclose%
\pgfusepath{stroke,fill}%
}%
\begin{pgfscope}%
\pgfsys@transformshift{0.945541in}{2.731407in}%
\pgfsys@useobject{currentmarker}{}%
\end{pgfscope}%
\begin{pgfscope}%
\pgfsys@transformshift{1.162607in}{2.880089in}%
\pgfsys@useobject{currentmarker}{}%
\end{pgfscope}%
\begin{pgfscope}%
\pgfsys@transformshift{1.378370in}{3.183230in}%
\pgfsys@useobject{currentmarker}{}%
\end{pgfscope}%
\begin{pgfscope}%
\pgfsys@transformshift{1.594133in}{3.208267in}%
\pgfsys@useobject{currentmarker}{}%
\end{pgfscope}%
\begin{pgfscope}%
\pgfsys@transformshift{1.809896in}{3.254104in}%
\pgfsys@useobject{currentmarker}{}%
\end{pgfscope}%
\begin{pgfscope}%
\pgfsys@transformshift{2.025659in}{3.288000in}%
\pgfsys@useobject{currentmarker}{}%
\end{pgfscope}%
\begin{pgfscope}%
\pgfsys@transformshift{2.241422in}{3.301096in}%
\pgfsys@useobject{currentmarker}{}%
\end{pgfscope}%
\begin{pgfscope}%
\pgfsys@transformshift{2.457185in}{3.398163in}%
\pgfsys@useobject{currentmarker}{}%
\end{pgfscope}%
\begin{pgfscope}%
\pgfsys@transformshift{2.672948in}{3.413956in}%
\pgfsys@useobject{currentmarker}{}%
\end{pgfscope}%
\begin{pgfscope}%
\pgfsys@transformshift{2.888711in}{3.388919in}%
\pgfsys@useobject{currentmarker}{}%
\end{pgfscope}%
\begin{pgfscope}%
\pgfsys@transformshift{3.104474in}{3.388919in}%
\pgfsys@useobject{currentmarker}{}%
\end{pgfscope}%
\begin{pgfscope}%
\pgfsys@transformshift{3.320237in}{3.352326in}%
\pgfsys@useobject{currentmarker}{}%
\end{pgfscope}%
\begin{pgfscope}%
\pgfsys@transformshift{3.536000in}{3.398163in}%
\pgfsys@useobject{currentmarker}{}%
\end{pgfscope}%
\begin{pgfscope}%
\pgfsys@transformshift{3.751763in}{3.368119in}%
\pgfsys@useobject{currentmarker}{}%
\end{pgfscope}%
\begin{pgfscope}%
\pgfsys@transformshift{3.967526in}{3.408948in}%
\pgfsys@useobject{currentmarker}{}%
\end{pgfscope}%
\begin{pgfscope}%
\pgfsys@transformshift{4.183289in}{3.412800in}%
\pgfsys@useobject{currentmarker}{}%
\end{pgfscope}%
\begin{pgfscope}%
\pgfsys@transformshift{4.399052in}{3.412800in}%
\pgfsys@useobject{currentmarker}{}%
\end{pgfscope}%
\begin{pgfscope}%
\pgfsys@transformshift{4.614815in}{3.394311in}%
\pgfsys@useobject{currentmarker}{}%
\end{pgfscope}%
\begin{pgfscope}%
\pgfsys@transformshift{4.830578in}{3.419348in}%
\pgfsys@useobject{currentmarker}{}%
\end{pgfscope}%
\begin{pgfscope}%
\pgfsys@transformshift{5.046341in}{3.382370in}%
\pgfsys@useobject{currentmarker}{}%
\end{pgfscope}%
\begin{pgfscope}%
\pgfsys@transformshift{5.262756in}{3.415496in}%
\pgfsys@useobject{currentmarker}{}%
\end{pgfscope}%
\end{pgfscope}%
\begin{pgfscope}%
\pgfpathrectangle{\pgfqpoint{0.720000in}{0.480000in}}{\pgfqpoint{7.040000in}{3.120000in}}%
\pgfusepath{clip}%
\pgfsetrectcap%
\pgfsetroundjoin%
\pgfsetlinewidth{0.752812pt}%
\definecolor{currentstroke}{rgb}{0.698039,0.133333,0.133333}%
\pgfsetstrokecolor{currentstroke}%
\pgfsetdash{}{0pt}%
\pgfpathmoveto{\pgfqpoint{0.973570in}{2.910519in}}%
\pgfpathlineto{\pgfqpoint{1.210844in}{3.125452in}}%
\pgfpathlineto{\pgfqpoint{1.448119in}{3.142400in}}%
\pgfpathlineto{\pgfqpoint{1.685393in}{3.259111in}}%
\pgfpathlineto{\pgfqpoint{1.922667in}{3.244859in}}%
\pgfpathlineto{\pgfqpoint{2.159941in}{3.305333in}}%
\pgfpathlineto{\pgfqpoint{2.442193in}{3.256800in}}%
\pgfpathlineto{\pgfqpoint{2.679467in}{3.413956in}}%
\pgfpathlineto{\pgfqpoint{2.916741in}{3.394311in}}%
\pgfpathlineto{\pgfqpoint{3.154015in}{3.423200in}}%
\pgfpathlineto{\pgfqpoint{3.391289in}{3.453244in}}%
\pgfpathlineto{\pgfqpoint{3.628563in}{3.403556in}}%
\pgfpathlineto{\pgfqpoint{3.865837in}{3.408948in}}%
\pgfpathlineto{\pgfqpoint{4.103111in}{3.427052in}}%
\pgfpathlineto{\pgfqpoint{4.340385in}{3.464030in}}%
\pgfpathlineto{\pgfqpoint{4.577659in}{3.445541in}}%
\pgfpathlineto{\pgfqpoint{4.814933in}{3.453244in}}%
\pgfpathlineto{\pgfqpoint{5.052207in}{3.449393in}}%
\pgfpathlineto{\pgfqpoint{5.289481in}{3.477126in}}%
\pgfpathlineto{\pgfqpoint{5.526756in}{3.432444in}}%
\pgfpathlineto{\pgfqpoint{5.764030in}{3.442844in}}%
\pgfusepath{stroke}%
\end{pgfscope}%
\begin{pgfscope}%
\pgfpathrectangle{\pgfqpoint{0.720000in}{0.480000in}}{\pgfqpoint{7.040000in}{3.120000in}}%
\pgfusepath{clip}%
\pgfsetbuttcap%
\pgfsetroundjoin%
\definecolor{currentfill}{rgb}{0.698039,0.133333,0.133333}%
\pgfsetfillcolor{currentfill}%
\pgfsetlinewidth{1.003750pt}%
\definecolor{currentstroke}{rgb}{0.698039,0.133333,0.133333}%
\pgfsetstrokecolor{currentstroke}%
\pgfsetdash{}{0pt}%
\pgfsys@defobject{currentmarker}{\pgfqpoint{-0.010417in}{-0.010417in}}{\pgfqpoint{0.010417in}{0.010417in}}{%
\pgfpathmoveto{\pgfqpoint{0.000000in}{-0.010417in}}%
\pgfpathcurveto{\pgfqpoint{0.002763in}{-0.010417in}}{\pgfqpoint{0.005412in}{-0.009319in}}{\pgfqpoint{0.007366in}{-0.007366in}}%
\pgfpathcurveto{\pgfqpoint{0.009319in}{-0.005412in}}{\pgfqpoint{0.010417in}{-0.002763in}}{\pgfqpoint{0.010417in}{0.000000in}}%
\pgfpathcurveto{\pgfqpoint{0.010417in}{0.002763in}}{\pgfqpoint{0.009319in}{0.005412in}}{\pgfqpoint{0.007366in}{0.007366in}}%
\pgfpathcurveto{\pgfqpoint{0.005412in}{0.009319in}}{\pgfqpoint{0.002763in}{0.010417in}}{\pgfqpoint{0.000000in}{0.010417in}}%
\pgfpathcurveto{\pgfqpoint{-0.002763in}{0.010417in}}{\pgfqpoint{-0.005412in}{0.009319in}}{\pgfqpoint{-0.007366in}{0.007366in}}%
\pgfpathcurveto{\pgfqpoint{-0.009319in}{0.005412in}}{\pgfqpoint{-0.010417in}{0.002763in}}{\pgfqpoint{-0.010417in}{0.000000in}}%
\pgfpathcurveto{\pgfqpoint{-0.010417in}{-0.002763in}}{\pgfqpoint{-0.009319in}{-0.005412in}}{\pgfqpoint{-0.007366in}{-0.007366in}}%
\pgfpathcurveto{\pgfqpoint{-0.005412in}{-0.009319in}}{\pgfqpoint{-0.002763in}{-0.010417in}}{\pgfqpoint{0.000000in}{-0.010417in}}%
\pgfpathclose%
\pgfusepath{stroke,fill}%
}%
\begin{pgfscope}%
\pgfsys@transformshift{0.973570in}{2.910519in}%
\pgfsys@useobject{currentmarker}{}%
\end{pgfscope}%
\begin{pgfscope}%
\pgfsys@transformshift{1.210844in}{3.125452in}%
\pgfsys@useobject{currentmarker}{}%
\end{pgfscope}%
\begin{pgfscope}%
\pgfsys@transformshift{1.448119in}{3.142400in}%
\pgfsys@useobject{currentmarker}{}%
\end{pgfscope}%
\begin{pgfscope}%
\pgfsys@transformshift{1.685393in}{3.259111in}%
\pgfsys@useobject{currentmarker}{}%
\end{pgfscope}%
\begin{pgfscope}%
\pgfsys@transformshift{1.922667in}{3.244859in}%
\pgfsys@useobject{currentmarker}{}%
\end{pgfscope}%
\begin{pgfscope}%
\pgfsys@transformshift{2.159941in}{3.305333in}%
\pgfsys@useobject{currentmarker}{}%
\end{pgfscope}%
\begin{pgfscope}%
\pgfsys@transformshift{2.442193in}{3.256800in}%
\pgfsys@useobject{currentmarker}{}%
\end{pgfscope}%
\begin{pgfscope}%
\pgfsys@transformshift{2.679467in}{3.413956in}%
\pgfsys@useobject{currentmarker}{}%
\end{pgfscope}%
\begin{pgfscope}%
\pgfsys@transformshift{2.916741in}{3.394311in}%
\pgfsys@useobject{currentmarker}{}%
\end{pgfscope}%
\begin{pgfscope}%
\pgfsys@transformshift{3.154015in}{3.423200in}%
\pgfsys@useobject{currentmarker}{}%
\end{pgfscope}%
\begin{pgfscope}%
\pgfsys@transformshift{3.391289in}{3.453244in}%
\pgfsys@useobject{currentmarker}{}%
\end{pgfscope}%
\begin{pgfscope}%
\pgfsys@transformshift{3.628563in}{3.403556in}%
\pgfsys@useobject{currentmarker}{}%
\end{pgfscope}%
\begin{pgfscope}%
\pgfsys@transformshift{3.865837in}{3.408948in}%
\pgfsys@useobject{currentmarker}{}%
\end{pgfscope}%
\begin{pgfscope}%
\pgfsys@transformshift{4.103111in}{3.427052in}%
\pgfsys@useobject{currentmarker}{}%
\end{pgfscope}%
\begin{pgfscope}%
\pgfsys@transformshift{4.340385in}{3.464030in}%
\pgfsys@useobject{currentmarker}{}%
\end{pgfscope}%
\begin{pgfscope}%
\pgfsys@transformshift{4.577659in}{3.445541in}%
\pgfsys@useobject{currentmarker}{}%
\end{pgfscope}%
\begin{pgfscope}%
\pgfsys@transformshift{4.814933in}{3.453244in}%
\pgfsys@useobject{currentmarker}{}%
\end{pgfscope}%
\begin{pgfscope}%
\pgfsys@transformshift{5.052207in}{3.449393in}%
\pgfsys@useobject{currentmarker}{}%
\end{pgfscope}%
\begin{pgfscope}%
\pgfsys@transformshift{5.289481in}{3.477126in}%
\pgfsys@useobject{currentmarker}{}%
\end{pgfscope}%
\begin{pgfscope}%
\pgfsys@transformshift{5.526756in}{3.432444in}%
\pgfsys@useobject{currentmarker}{}%
\end{pgfscope}%
\begin{pgfscope}%
\pgfsys@transformshift{5.764030in}{3.442844in}%
\pgfsys@useobject{currentmarker}{}%
\end{pgfscope}%
\end{pgfscope}%
\begin{pgfscope}%
\pgfpathrectangle{\pgfqpoint{0.720000in}{0.480000in}}{\pgfqpoint{7.040000in}{3.120000in}}%
\pgfusepath{clip}%
\pgfsetrectcap%
\pgfsetroundjoin%
\pgfsetlinewidth{0.752812pt}%
\definecolor{currentstroke}{rgb}{0.000000,0.750000,0.750000}%
\pgfsetstrokecolor{currentstroke}%
\pgfsetdash{}{0pt}%
\pgfpathmoveto{\pgfqpoint{1.009422in}{2.939407in}}%
\pgfpathlineto{\pgfqpoint{1.288415in}{3.022993in}}%
\pgfpathlineto{\pgfqpoint{1.566756in}{3.310341in}}%
\pgfpathlineto{\pgfqpoint{1.845096in}{3.329985in}}%
\pgfpathlineto{\pgfqpoint{2.124089in}{3.264504in}}%
\pgfpathlineto{\pgfqpoint{2.402430in}{3.345778in}}%
\pgfpathlineto{\pgfqpoint{2.680770in}{3.397007in}}%
\pgfpathlineto{\pgfqpoint{2.959763in}{3.474430in}}%
\pgfpathlineto{\pgfqpoint{3.238104in}{3.452089in}}%
\pgfpathlineto{\pgfqpoint{3.516444in}{3.442844in}}%
\pgfpathlineto{\pgfqpoint{3.794785in}{3.479437in}}%
\pgfpathlineto{\pgfqpoint{4.073126in}{3.496770in}}%
\pgfpathlineto{\pgfqpoint{4.351467in}{3.452089in}}%
\pgfpathlineto{\pgfqpoint{4.629807in}{3.472889in}}%
\pgfpathlineto{\pgfqpoint{4.907496in}{3.487526in}}%
\pgfpathlineto{\pgfqpoint{5.187141in}{3.472889in}}%
\pgfpathlineto{\pgfqpoint{5.465481in}{3.503319in}}%
\pgfpathlineto{\pgfqpoint{5.764681in}{3.487526in}}%
\pgfpathlineto{\pgfqpoint{6.043022in}{3.497926in}}%
\pgfpathlineto{\pgfqpoint{6.321363in}{3.496770in}}%
\pgfpathlineto{\pgfqpoint{6.599704in}{3.507170in}}%
\pgfusepath{stroke}%
\end{pgfscope}%
\begin{pgfscope}%
\pgfpathrectangle{\pgfqpoint{0.720000in}{0.480000in}}{\pgfqpoint{7.040000in}{3.120000in}}%
\pgfusepath{clip}%
\pgfsetbuttcap%
\pgfsetroundjoin%
\definecolor{currentfill}{rgb}{0.000000,0.750000,0.750000}%
\pgfsetfillcolor{currentfill}%
\pgfsetlinewidth{1.003750pt}%
\definecolor{currentstroke}{rgb}{0.000000,0.750000,0.750000}%
\pgfsetstrokecolor{currentstroke}%
\pgfsetdash{}{0pt}%
\pgfsys@defobject{currentmarker}{\pgfqpoint{-0.010417in}{-0.010417in}}{\pgfqpoint{0.010417in}{0.010417in}}{%
\pgfpathmoveto{\pgfqpoint{0.000000in}{-0.010417in}}%
\pgfpathcurveto{\pgfqpoint{0.002763in}{-0.010417in}}{\pgfqpoint{0.005412in}{-0.009319in}}{\pgfqpoint{0.007366in}{-0.007366in}}%
\pgfpathcurveto{\pgfqpoint{0.009319in}{-0.005412in}}{\pgfqpoint{0.010417in}{-0.002763in}}{\pgfqpoint{0.010417in}{0.000000in}}%
\pgfpathcurveto{\pgfqpoint{0.010417in}{0.002763in}}{\pgfqpoint{0.009319in}{0.005412in}}{\pgfqpoint{0.007366in}{0.007366in}}%
\pgfpathcurveto{\pgfqpoint{0.005412in}{0.009319in}}{\pgfqpoint{0.002763in}{0.010417in}}{\pgfqpoint{0.000000in}{0.010417in}}%
\pgfpathcurveto{\pgfqpoint{-0.002763in}{0.010417in}}{\pgfqpoint{-0.005412in}{0.009319in}}{\pgfqpoint{-0.007366in}{0.007366in}}%
\pgfpathcurveto{\pgfqpoint{-0.009319in}{0.005412in}}{\pgfqpoint{-0.010417in}{0.002763in}}{\pgfqpoint{-0.010417in}{0.000000in}}%
\pgfpathcurveto{\pgfqpoint{-0.010417in}{-0.002763in}}{\pgfqpoint{-0.009319in}{-0.005412in}}{\pgfqpoint{-0.007366in}{-0.007366in}}%
\pgfpathcurveto{\pgfqpoint{-0.005412in}{-0.009319in}}{\pgfqpoint{-0.002763in}{-0.010417in}}{\pgfqpoint{0.000000in}{-0.010417in}}%
\pgfpathclose%
\pgfusepath{stroke,fill}%
}%
\begin{pgfscope}%
\pgfsys@transformshift{1.009422in}{2.939407in}%
\pgfsys@useobject{currentmarker}{}%
\end{pgfscope}%
\begin{pgfscope}%
\pgfsys@transformshift{1.288415in}{3.022993in}%
\pgfsys@useobject{currentmarker}{}%
\end{pgfscope}%
\begin{pgfscope}%
\pgfsys@transformshift{1.566756in}{3.310341in}%
\pgfsys@useobject{currentmarker}{}%
\end{pgfscope}%
\begin{pgfscope}%
\pgfsys@transformshift{1.845096in}{3.329985in}%
\pgfsys@useobject{currentmarker}{}%
\end{pgfscope}%
\begin{pgfscope}%
\pgfsys@transformshift{2.124089in}{3.264504in}%
\pgfsys@useobject{currentmarker}{}%
\end{pgfscope}%
\begin{pgfscope}%
\pgfsys@transformshift{2.402430in}{3.345778in}%
\pgfsys@useobject{currentmarker}{}%
\end{pgfscope}%
\begin{pgfscope}%
\pgfsys@transformshift{2.680770in}{3.397007in}%
\pgfsys@useobject{currentmarker}{}%
\end{pgfscope}%
\begin{pgfscope}%
\pgfsys@transformshift{2.959763in}{3.474430in}%
\pgfsys@useobject{currentmarker}{}%
\end{pgfscope}%
\begin{pgfscope}%
\pgfsys@transformshift{3.238104in}{3.452089in}%
\pgfsys@useobject{currentmarker}{}%
\end{pgfscope}%
\begin{pgfscope}%
\pgfsys@transformshift{3.516444in}{3.442844in}%
\pgfsys@useobject{currentmarker}{}%
\end{pgfscope}%
\begin{pgfscope}%
\pgfsys@transformshift{3.794785in}{3.479437in}%
\pgfsys@useobject{currentmarker}{}%
\end{pgfscope}%
\begin{pgfscope}%
\pgfsys@transformshift{4.073126in}{3.496770in}%
\pgfsys@useobject{currentmarker}{}%
\end{pgfscope}%
\begin{pgfscope}%
\pgfsys@transformshift{4.351467in}{3.452089in}%
\pgfsys@useobject{currentmarker}{}%
\end{pgfscope}%
\begin{pgfscope}%
\pgfsys@transformshift{4.629807in}{3.472889in}%
\pgfsys@useobject{currentmarker}{}%
\end{pgfscope}%
\begin{pgfscope}%
\pgfsys@transformshift{4.907496in}{3.487526in}%
\pgfsys@useobject{currentmarker}{}%
\end{pgfscope}%
\begin{pgfscope}%
\pgfsys@transformshift{5.187141in}{3.472889in}%
\pgfsys@useobject{currentmarker}{}%
\end{pgfscope}%
\begin{pgfscope}%
\pgfsys@transformshift{5.465481in}{3.503319in}%
\pgfsys@useobject{currentmarker}{}%
\end{pgfscope}%
\begin{pgfscope}%
\pgfsys@transformshift{5.764681in}{3.487526in}%
\pgfsys@useobject{currentmarker}{}%
\end{pgfscope}%
\begin{pgfscope}%
\pgfsys@transformshift{6.043022in}{3.497926in}%
\pgfsys@useobject{currentmarker}{}%
\end{pgfscope}%
\begin{pgfscope}%
\pgfsys@transformshift{6.321363in}{3.496770in}%
\pgfsys@useobject{currentmarker}{}%
\end{pgfscope}%
\begin{pgfscope}%
\pgfsys@transformshift{6.599704in}{3.507170in}%
\pgfsys@useobject{currentmarker}{}%
\end{pgfscope}%
\end{pgfscope}%
\begin{pgfscope}%
\pgfsetrectcap%
\pgfsetmiterjoin%
\pgfsetlinewidth{0.803000pt}%
\definecolor{currentstroke}{rgb}{0.000000,0.000000,0.000000}%
\pgfsetstrokecolor{currentstroke}%
\pgfsetdash{}{0pt}%
\pgfpathmoveto{\pgfqpoint{0.720000in}{0.480000in}}%
\pgfpathlineto{\pgfqpoint{0.720000in}{3.600000in}}%
\pgfusepath{stroke}%
\end{pgfscope}%
\begin{pgfscope}%
\pgfsetrectcap%
\pgfsetmiterjoin%
\pgfsetlinewidth{0.803000pt}%
\definecolor{currentstroke}{rgb}{0.000000,0.000000,0.000000}%
\pgfsetstrokecolor{currentstroke}%
\pgfsetdash{}{0pt}%
\pgfpathmoveto{\pgfqpoint{7.760000in}{0.480000in}}%
\pgfpathlineto{\pgfqpoint{7.760000in}{3.600000in}}%
\pgfusepath{stroke}%
\end{pgfscope}%
\begin{pgfscope}%
\pgfsetrectcap%
\pgfsetmiterjoin%
\pgfsetlinewidth{0.803000pt}%
\definecolor{currentstroke}{rgb}{0.000000,0.000000,0.000000}%
\pgfsetstrokecolor{currentstroke}%
\pgfsetdash{}{0pt}%
\pgfpathmoveto{\pgfqpoint{0.720000in}{0.480000in}}%
\pgfpathlineto{\pgfqpoint{7.760000in}{0.480000in}}%
\pgfusepath{stroke}%
\end{pgfscope}%
\begin{pgfscope}%
\pgfsetrectcap%
\pgfsetmiterjoin%
\pgfsetlinewidth{0.803000pt}%
\definecolor{currentstroke}{rgb}{0.000000,0.000000,0.000000}%
\pgfsetstrokecolor{currentstroke}%
\pgfsetdash{}{0pt}%
\pgfpathmoveto{\pgfqpoint{0.720000in}{3.600000in}}%
\pgfpathlineto{\pgfqpoint{7.760000in}{3.600000in}}%
\pgfusepath{stroke}%
\end{pgfscope}%
\begin{pgfscope}%
\definecolor{textcolor}{rgb}{0.000000,0.000000,0.000000}%
\pgfsetstrokecolor{textcolor}%
\pgfsetfillcolor{textcolor}%
\pgftext[x=1.437910in,y=3.838851in,left,base]{\color{textcolor}\sffamily\fontsize{10.000000}{12.000000}\selectfont model training "food-50"; InceptionV3; 21 epochs; min. 00:17:45; max 02:30:20}%
\end{pgfscope}%
\begin{pgfscope}%
\definecolor{textcolor}{rgb}{0.000000,0.000000,0.000000}%
\pgfsetstrokecolor{textcolor}%
\pgfsetfillcolor{textcolor}%
\pgftext[x=3.389611in,y=3.683333in,left,base]{\color{textcolor}\sffamily\fontsize{10.000000}{12.000000}\selectfont (GPU - Nvidia GTX 1060)}%
\end{pgfscope}%
\begin{pgfscope}%
\pgfsetbuttcap%
\pgfsetmiterjoin%
\definecolor{currentfill}{rgb}{1.000000,1.000000,1.000000}%
\pgfsetfillcolor{currentfill}%
\pgfsetfillopacity{0.800000}%
\pgfsetlinewidth{1.003750pt}%
\definecolor{currentstroke}{rgb}{0.800000,0.800000,0.800000}%
\pgfsetstrokecolor{currentstroke}%
\pgfsetstrokeopacity{0.800000}%
\pgfsetdash{}{0pt}%
\pgfpathmoveto{\pgfqpoint{5.574990in}{0.542500in}}%
\pgfpathlineto{\pgfqpoint{7.672500in}{0.542500in}}%
\pgfpathquadraticcurveto{\pgfqpoint{7.697500in}{0.542500in}}{\pgfqpoint{7.697500in}{0.567500in}}%
\pgfpathlineto{\pgfqpoint{7.697500in}{2.756658in}}%
\pgfpathquadraticcurveto{\pgfqpoint{7.697500in}{2.781658in}}{\pgfqpoint{7.672500in}{2.781658in}}%
\pgfpathlineto{\pgfqpoint{5.574990in}{2.781658in}}%
\pgfpathquadraticcurveto{\pgfqpoint{5.549990in}{2.781658in}}{\pgfqpoint{5.549990in}{2.756658in}}%
\pgfpathlineto{\pgfqpoint{5.549990in}{0.567500in}}%
\pgfpathquadraticcurveto{\pgfqpoint{5.549990in}{0.542500in}}{\pgfqpoint{5.574990in}{0.542500in}}%
\pgfpathclose%
\pgfusepath{stroke,fill}%
\end{pgfscope}%
\begin{pgfscope}%
\pgfsetrectcap%
\pgfsetroundjoin%
\pgfsetlinewidth{0.752812pt}%
\definecolor{currentstroke}{rgb}{0.000000,0.000000,1.000000}%
\pgfsetstrokecolor{currentstroke}%
\pgfsetdash{}{0pt}%
\pgfpathmoveto{\pgfqpoint{5.599990in}{2.680438in}}%
\pgfpathlineto{\pgfqpoint{5.849990in}{2.680438in}}%
\pgfusepath{stroke}%
\end{pgfscope}%
\begin{pgfscope}%
\pgfsetbuttcap%
\pgfsetroundjoin%
\definecolor{currentfill}{rgb}{0.000000,0.000000,1.000000}%
\pgfsetfillcolor{currentfill}%
\pgfsetlinewidth{1.003750pt}%
\definecolor{currentstroke}{rgb}{0.000000,0.000000,1.000000}%
\pgfsetstrokecolor{currentstroke}%
\pgfsetdash{}{0pt}%
\pgfsys@defobject{currentmarker}{\pgfqpoint{-0.010417in}{-0.010417in}}{\pgfqpoint{0.010417in}{0.010417in}}{%
\pgfpathmoveto{\pgfqpoint{0.000000in}{-0.010417in}}%
\pgfpathcurveto{\pgfqpoint{0.002763in}{-0.010417in}}{\pgfqpoint{0.005412in}{-0.009319in}}{\pgfqpoint{0.007366in}{-0.007366in}}%
\pgfpathcurveto{\pgfqpoint{0.009319in}{-0.005412in}}{\pgfqpoint{0.010417in}{-0.002763in}}{\pgfqpoint{0.010417in}{0.000000in}}%
\pgfpathcurveto{\pgfqpoint{0.010417in}{0.002763in}}{\pgfqpoint{0.009319in}{0.005412in}}{\pgfqpoint{0.007366in}{0.007366in}}%
\pgfpathcurveto{\pgfqpoint{0.005412in}{0.009319in}}{\pgfqpoint{0.002763in}{0.010417in}}{\pgfqpoint{0.000000in}{0.010417in}}%
\pgfpathcurveto{\pgfqpoint{-0.002763in}{0.010417in}}{\pgfqpoint{-0.005412in}{0.009319in}}{\pgfqpoint{-0.007366in}{0.007366in}}%
\pgfpathcurveto{\pgfqpoint{-0.009319in}{0.005412in}}{\pgfqpoint{-0.010417in}{0.002763in}}{\pgfqpoint{-0.010417in}{0.000000in}}%
\pgfpathcurveto{\pgfqpoint{-0.010417in}{-0.002763in}}{\pgfqpoint{-0.009319in}{-0.005412in}}{\pgfqpoint{-0.007366in}{-0.007366in}}%
\pgfpathcurveto{\pgfqpoint{-0.005412in}{-0.009319in}}{\pgfqpoint{-0.002763in}{-0.010417in}}{\pgfqpoint{0.000000in}{-0.010417in}}%
\pgfpathclose%
\pgfusepath{stroke,fill}%
}%
\begin{pgfscope}%
\pgfsys@transformshift{5.724990in}{2.680438in}%
\pgfsys@useobject{currentmarker}{}%
\end{pgfscope}%
\end{pgfscope}%
\begin{pgfscope}%
\definecolor{textcolor}{rgb}{0.000000,0.000000,0.000000}%
\pgfsetstrokecolor{textcolor}%
\pgfsetfillcolor{textcolor}%
\pgftext[x=5.949990in,y=2.636688in,left,base]{\color{textcolor}\sffamily\fontsize{9.000000}{10.800000}\selectfont 00500 (45.76\% - 00:17:45)}%
\end{pgfscope}%
\begin{pgfscope}%
\pgfsetrectcap%
\pgfsetroundjoin%
\pgfsetlinewidth{0.752812pt}%
\definecolor{currentstroke}{rgb}{1.000000,0.647059,0.000000}%
\pgfsetstrokecolor{currentstroke}%
\pgfsetdash{}{0pt}%
\pgfpathmoveto{\pgfqpoint{5.599990in}{2.496966in}}%
\pgfpathlineto{\pgfqpoint{5.849990in}{2.496966in}}%
\pgfusepath{stroke}%
\end{pgfscope}%
\begin{pgfscope}%
\pgfsetbuttcap%
\pgfsetroundjoin%
\definecolor{currentfill}{rgb}{1.000000,0.647059,0.000000}%
\pgfsetfillcolor{currentfill}%
\pgfsetlinewidth{1.003750pt}%
\definecolor{currentstroke}{rgb}{1.000000,0.647059,0.000000}%
\pgfsetstrokecolor{currentstroke}%
\pgfsetdash{}{0pt}%
\pgfsys@defobject{currentmarker}{\pgfqpoint{-0.010417in}{-0.010417in}}{\pgfqpoint{0.010417in}{0.010417in}}{%
\pgfpathmoveto{\pgfqpoint{0.000000in}{-0.010417in}}%
\pgfpathcurveto{\pgfqpoint{0.002763in}{-0.010417in}}{\pgfqpoint{0.005412in}{-0.009319in}}{\pgfqpoint{0.007366in}{-0.007366in}}%
\pgfpathcurveto{\pgfqpoint{0.009319in}{-0.005412in}}{\pgfqpoint{0.010417in}{-0.002763in}}{\pgfqpoint{0.010417in}{0.000000in}}%
\pgfpathcurveto{\pgfqpoint{0.010417in}{0.002763in}}{\pgfqpoint{0.009319in}{0.005412in}}{\pgfqpoint{0.007366in}{0.007366in}}%
\pgfpathcurveto{\pgfqpoint{0.005412in}{0.009319in}}{\pgfqpoint{0.002763in}{0.010417in}}{\pgfqpoint{0.000000in}{0.010417in}}%
\pgfpathcurveto{\pgfqpoint{-0.002763in}{0.010417in}}{\pgfqpoint{-0.005412in}{0.009319in}}{\pgfqpoint{-0.007366in}{0.007366in}}%
\pgfpathcurveto{\pgfqpoint{-0.009319in}{0.005412in}}{\pgfqpoint{-0.010417in}{0.002763in}}{\pgfqpoint{-0.010417in}{0.000000in}}%
\pgfpathcurveto{\pgfqpoint{-0.010417in}{-0.002763in}}{\pgfqpoint{-0.009319in}{-0.005412in}}{\pgfqpoint{-0.007366in}{-0.007366in}}%
\pgfpathcurveto{\pgfqpoint{-0.005412in}{-0.009319in}}{\pgfqpoint{-0.002763in}{-0.010417in}}{\pgfqpoint{0.000000in}{-0.010417in}}%
\pgfpathclose%
\pgfusepath{stroke,fill}%
}%
\begin{pgfscope}%
\pgfsys@transformshift{5.724990in}{2.496966in}%
\pgfsys@useobject{currentmarker}{}%
\end{pgfscope}%
\end{pgfscope}%
\begin{pgfscope}%
\definecolor{textcolor}{rgb}{0.000000,0.000000,0.000000}%
\pgfsetstrokecolor{textcolor}%
\pgfsetfillcolor{textcolor}%
\pgftext[x=5.949990in,y=2.453216in,left,base]{\color{textcolor}\sffamily\fontsize{9.000000}{10.800000}\selectfont 01000 (61.42\% - 00:23:24)}%
\end{pgfscope}%
\begin{pgfscope}%
\pgfsetrectcap%
\pgfsetroundjoin%
\pgfsetlinewidth{0.752812pt}%
\definecolor{currentstroke}{rgb}{0.000000,0.501961,0.000000}%
\pgfsetstrokecolor{currentstroke}%
\pgfsetdash{}{0pt}%
\pgfpathmoveto{\pgfqpoint{5.599990in}{2.313495in}}%
\pgfpathlineto{\pgfqpoint{5.849990in}{2.313495in}}%
\pgfusepath{stroke}%
\end{pgfscope}%
\begin{pgfscope}%
\pgfsetbuttcap%
\pgfsetroundjoin%
\definecolor{currentfill}{rgb}{0.000000,0.501961,0.000000}%
\pgfsetfillcolor{currentfill}%
\pgfsetlinewidth{1.003750pt}%
\definecolor{currentstroke}{rgb}{0.000000,0.501961,0.000000}%
\pgfsetstrokecolor{currentstroke}%
\pgfsetdash{}{0pt}%
\pgfsys@defobject{currentmarker}{\pgfqpoint{-0.010417in}{-0.010417in}}{\pgfqpoint{0.010417in}{0.010417in}}{%
\pgfpathmoveto{\pgfqpoint{0.000000in}{-0.010417in}}%
\pgfpathcurveto{\pgfqpoint{0.002763in}{-0.010417in}}{\pgfqpoint{0.005412in}{-0.009319in}}{\pgfqpoint{0.007366in}{-0.007366in}}%
\pgfpathcurveto{\pgfqpoint{0.009319in}{-0.005412in}}{\pgfqpoint{0.010417in}{-0.002763in}}{\pgfqpoint{0.010417in}{0.000000in}}%
\pgfpathcurveto{\pgfqpoint{0.010417in}{0.002763in}}{\pgfqpoint{0.009319in}{0.005412in}}{\pgfqpoint{0.007366in}{0.007366in}}%
\pgfpathcurveto{\pgfqpoint{0.005412in}{0.009319in}}{\pgfqpoint{0.002763in}{0.010417in}}{\pgfqpoint{0.000000in}{0.010417in}}%
\pgfpathcurveto{\pgfqpoint{-0.002763in}{0.010417in}}{\pgfqpoint{-0.005412in}{0.009319in}}{\pgfqpoint{-0.007366in}{0.007366in}}%
\pgfpathcurveto{\pgfqpoint{-0.009319in}{0.005412in}}{\pgfqpoint{-0.010417in}{0.002763in}}{\pgfqpoint{-0.010417in}{0.000000in}}%
\pgfpathcurveto{\pgfqpoint{-0.010417in}{-0.002763in}}{\pgfqpoint{-0.009319in}{-0.005412in}}{\pgfqpoint{-0.007366in}{-0.007366in}}%
\pgfpathcurveto{\pgfqpoint{-0.005412in}{-0.009319in}}{\pgfqpoint{-0.002763in}{-0.010417in}}{\pgfqpoint{0.000000in}{-0.010417in}}%
\pgfpathclose%
\pgfusepath{stroke,fill}%
}%
\begin{pgfscope}%
\pgfsys@transformshift{5.724990in}{2.313495in}%
\pgfsys@useobject{currentmarker}{}%
\end{pgfscope}%
\end{pgfscope}%
\begin{pgfscope}%
\definecolor{textcolor}{rgb}{0.000000,0.000000,0.000000}%
\pgfsetstrokecolor{textcolor}%
\pgfsetfillcolor{textcolor}%
\pgftext[x=5.949990in,y=2.269745in,left,base]{\color{textcolor}\sffamily\fontsize{9.000000}{10.800000}\selectfont 02000 (70.72\% - 00:34:54)}%
\end{pgfscope}%
\begin{pgfscope}%
\pgfsetrectcap%
\pgfsetroundjoin%
\pgfsetlinewidth{0.752812pt}%
\definecolor{currentstroke}{rgb}{1.000000,0.000000,0.000000}%
\pgfsetstrokecolor{currentstroke}%
\pgfsetdash{}{0pt}%
\pgfpathmoveto{\pgfqpoint{5.599990in}{2.130023in}}%
\pgfpathlineto{\pgfqpoint{5.849990in}{2.130023in}}%
\pgfusepath{stroke}%
\end{pgfscope}%
\begin{pgfscope}%
\pgfsetbuttcap%
\pgfsetroundjoin%
\definecolor{currentfill}{rgb}{1.000000,0.000000,0.000000}%
\pgfsetfillcolor{currentfill}%
\pgfsetlinewidth{1.003750pt}%
\definecolor{currentstroke}{rgb}{1.000000,0.000000,0.000000}%
\pgfsetstrokecolor{currentstroke}%
\pgfsetdash{}{0pt}%
\pgfsys@defobject{currentmarker}{\pgfqpoint{-0.010417in}{-0.010417in}}{\pgfqpoint{0.010417in}{0.010417in}}{%
\pgfpathmoveto{\pgfqpoint{0.000000in}{-0.010417in}}%
\pgfpathcurveto{\pgfqpoint{0.002763in}{-0.010417in}}{\pgfqpoint{0.005412in}{-0.009319in}}{\pgfqpoint{0.007366in}{-0.007366in}}%
\pgfpathcurveto{\pgfqpoint{0.009319in}{-0.005412in}}{\pgfqpoint{0.010417in}{-0.002763in}}{\pgfqpoint{0.010417in}{0.000000in}}%
\pgfpathcurveto{\pgfqpoint{0.010417in}{0.002763in}}{\pgfqpoint{0.009319in}{0.005412in}}{\pgfqpoint{0.007366in}{0.007366in}}%
\pgfpathcurveto{\pgfqpoint{0.005412in}{0.009319in}}{\pgfqpoint{0.002763in}{0.010417in}}{\pgfqpoint{0.000000in}{0.010417in}}%
\pgfpathcurveto{\pgfqpoint{-0.002763in}{0.010417in}}{\pgfqpoint{-0.005412in}{0.009319in}}{\pgfqpoint{-0.007366in}{0.007366in}}%
\pgfpathcurveto{\pgfqpoint{-0.009319in}{0.005412in}}{\pgfqpoint{-0.010417in}{0.002763in}}{\pgfqpoint{-0.010417in}{0.000000in}}%
\pgfpathcurveto{\pgfqpoint{-0.010417in}{-0.002763in}}{\pgfqpoint{-0.009319in}{-0.005412in}}{\pgfqpoint{-0.007366in}{-0.007366in}}%
\pgfpathcurveto{\pgfqpoint{-0.005412in}{-0.009319in}}{\pgfqpoint{-0.002763in}{-0.010417in}}{\pgfqpoint{0.000000in}{-0.010417in}}%
\pgfpathclose%
\pgfusepath{stroke,fill}%
}%
\begin{pgfscope}%
\pgfsys@transformshift{5.724990in}{2.130023in}%
\pgfsys@useobject{currentmarker}{}%
\end{pgfscope}%
\end{pgfscope}%
\begin{pgfscope}%
\definecolor{textcolor}{rgb}{0.000000,0.000000,0.000000}%
\pgfsetstrokecolor{textcolor}%
\pgfsetfillcolor{textcolor}%
\pgftext[x=5.949990in,y=2.086273in,left,base]{\color{textcolor}\sffamily\fontsize{9.000000}{10.800000}\selectfont 03000 (74.33\% - 00:46:30)}%
\end{pgfscope}%
\begin{pgfscope}%
\pgfsetrectcap%
\pgfsetroundjoin%
\pgfsetlinewidth{0.752812pt}%
\definecolor{currentstroke}{rgb}{0.501961,0.000000,0.501961}%
\pgfsetstrokecolor{currentstroke}%
\pgfsetdash{}{0pt}%
\pgfpathmoveto{\pgfqpoint{5.599990in}{1.946552in}}%
\pgfpathlineto{\pgfqpoint{5.849990in}{1.946552in}}%
\pgfusepath{stroke}%
\end{pgfscope}%
\begin{pgfscope}%
\pgfsetbuttcap%
\pgfsetroundjoin%
\definecolor{currentfill}{rgb}{0.501961,0.000000,0.501961}%
\pgfsetfillcolor{currentfill}%
\pgfsetlinewidth{1.003750pt}%
\definecolor{currentstroke}{rgb}{0.501961,0.000000,0.501961}%
\pgfsetstrokecolor{currentstroke}%
\pgfsetdash{}{0pt}%
\pgfsys@defobject{currentmarker}{\pgfqpoint{-0.010417in}{-0.010417in}}{\pgfqpoint{0.010417in}{0.010417in}}{%
\pgfpathmoveto{\pgfqpoint{0.000000in}{-0.010417in}}%
\pgfpathcurveto{\pgfqpoint{0.002763in}{-0.010417in}}{\pgfqpoint{0.005412in}{-0.009319in}}{\pgfqpoint{0.007366in}{-0.007366in}}%
\pgfpathcurveto{\pgfqpoint{0.009319in}{-0.005412in}}{\pgfqpoint{0.010417in}{-0.002763in}}{\pgfqpoint{0.010417in}{0.000000in}}%
\pgfpathcurveto{\pgfqpoint{0.010417in}{0.002763in}}{\pgfqpoint{0.009319in}{0.005412in}}{\pgfqpoint{0.007366in}{0.007366in}}%
\pgfpathcurveto{\pgfqpoint{0.005412in}{0.009319in}}{\pgfqpoint{0.002763in}{0.010417in}}{\pgfqpoint{0.000000in}{0.010417in}}%
\pgfpathcurveto{\pgfqpoint{-0.002763in}{0.010417in}}{\pgfqpoint{-0.005412in}{0.009319in}}{\pgfqpoint{-0.007366in}{0.007366in}}%
\pgfpathcurveto{\pgfqpoint{-0.009319in}{0.005412in}}{\pgfqpoint{-0.010417in}{0.002763in}}{\pgfqpoint{-0.010417in}{0.000000in}}%
\pgfpathcurveto{\pgfqpoint{-0.010417in}{-0.002763in}}{\pgfqpoint{-0.009319in}{-0.005412in}}{\pgfqpoint{-0.007366in}{-0.007366in}}%
\pgfpathcurveto{\pgfqpoint{-0.005412in}{-0.009319in}}{\pgfqpoint{-0.002763in}{-0.010417in}}{\pgfqpoint{0.000000in}{-0.010417in}}%
\pgfpathclose%
\pgfusepath{stroke,fill}%
}%
\begin{pgfscope}%
\pgfsys@transformshift{5.724990in}{1.946552in}%
\pgfsys@useobject{currentmarker}{}%
\end{pgfscope}%
\end{pgfscope}%
\begin{pgfscope}%
\definecolor{textcolor}{rgb}{0.000000,0.000000,0.000000}%
\pgfsetstrokecolor{textcolor}%
\pgfsetfillcolor{textcolor}%
\pgftext[x=5.949990in,y=1.902802in,left,base]{\color{textcolor}\sffamily\fontsize{9.000000}{10.800000}\selectfont 04000 (76.17\% - 00:58:03)}%
\end{pgfscope}%
\begin{pgfscope}%
\pgfsetrectcap%
\pgfsetroundjoin%
\pgfsetlinewidth{0.752812pt}%
\definecolor{currentstroke}{rgb}{0.392157,0.584314,0.929412}%
\pgfsetstrokecolor{currentstroke}%
\pgfsetdash{}{0pt}%
\pgfpathmoveto{\pgfqpoint{5.599990in}{1.763080in}}%
\pgfpathlineto{\pgfqpoint{5.849990in}{1.763080in}}%
\pgfusepath{stroke}%
\end{pgfscope}%
\begin{pgfscope}%
\pgfsetbuttcap%
\pgfsetroundjoin%
\definecolor{currentfill}{rgb}{0.392157,0.584314,0.929412}%
\pgfsetfillcolor{currentfill}%
\pgfsetlinewidth{1.003750pt}%
\definecolor{currentstroke}{rgb}{0.392157,0.584314,0.929412}%
\pgfsetstrokecolor{currentstroke}%
\pgfsetdash{}{0pt}%
\pgfsys@defobject{currentmarker}{\pgfqpoint{-0.010417in}{-0.010417in}}{\pgfqpoint{0.010417in}{0.010417in}}{%
\pgfpathmoveto{\pgfqpoint{0.000000in}{-0.010417in}}%
\pgfpathcurveto{\pgfqpoint{0.002763in}{-0.010417in}}{\pgfqpoint{0.005412in}{-0.009319in}}{\pgfqpoint{0.007366in}{-0.007366in}}%
\pgfpathcurveto{\pgfqpoint{0.009319in}{-0.005412in}}{\pgfqpoint{0.010417in}{-0.002763in}}{\pgfqpoint{0.010417in}{0.000000in}}%
\pgfpathcurveto{\pgfqpoint{0.010417in}{0.002763in}}{\pgfqpoint{0.009319in}{0.005412in}}{\pgfqpoint{0.007366in}{0.007366in}}%
\pgfpathcurveto{\pgfqpoint{0.005412in}{0.009319in}}{\pgfqpoint{0.002763in}{0.010417in}}{\pgfqpoint{0.000000in}{0.010417in}}%
\pgfpathcurveto{\pgfqpoint{-0.002763in}{0.010417in}}{\pgfqpoint{-0.005412in}{0.009319in}}{\pgfqpoint{-0.007366in}{0.007366in}}%
\pgfpathcurveto{\pgfqpoint{-0.009319in}{0.005412in}}{\pgfqpoint{-0.010417in}{0.002763in}}{\pgfqpoint{-0.010417in}{0.000000in}}%
\pgfpathcurveto{\pgfqpoint{-0.010417in}{-0.002763in}}{\pgfqpoint{-0.009319in}{-0.005412in}}{\pgfqpoint{-0.007366in}{-0.007366in}}%
\pgfpathcurveto{\pgfqpoint{-0.005412in}{-0.009319in}}{\pgfqpoint{-0.002763in}{-0.010417in}}{\pgfqpoint{0.000000in}{-0.010417in}}%
\pgfpathclose%
\pgfusepath{stroke,fill}%
}%
\begin{pgfscope}%
\pgfsys@transformshift{5.724990in}{1.763080in}%
\pgfsys@useobject{currentmarker}{}%
\end{pgfscope}%
\end{pgfscope}%
\begin{pgfscope}%
\definecolor{textcolor}{rgb}{0.000000,0.000000,0.000000}%
\pgfsetstrokecolor{textcolor}%
\pgfsetfillcolor{textcolor}%
\pgftext[x=5.949990in,y=1.719330in,left,base]{\color{textcolor}\sffamily\fontsize{9.000000}{10.800000}\selectfont 05000 (78.41\% - 01:09:35)}%
\end{pgfscope}%
\begin{pgfscope}%
\pgfsetrectcap%
\pgfsetroundjoin%
\pgfsetlinewidth{0.752812pt}%
\definecolor{currentstroke}{rgb}{1.000000,0.752941,0.796078}%
\pgfsetstrokecolor{currentstroke}%
\pgfsetdash{}{0pt}%
\pgfpathmoveto{\pgfqpoint{5.599990in}{1.579608in}}%
\pgfpathlineto{\pgfqpoint{5.849990in}{1.579608in}}%
\pgfusepath{stroke}%
\end{pgfscope}%
\begin{pgfscope}%
\pgfsetbuttcap%
\pgfsetroundjoin%
\definecolor{currentfill}{rgb}{1.000000,0.752941,0.796078}%
\pgfsetfillcolor{currentfill}%
\pgfsetlinewidth{1.003750pt}%
\definecolor{currentstroke}{rgb}{1.000000,0.752941,0.796078}%
\pgfsetstrokecolor{currentstroke}%
\pgfsetdash{}{0pt}%
\pgfsys@defobject{currentmarker}{\pgfqpoint{-0.010417in}{-0.010417in}}{\pgfqpoint{0.010417in}{0.010417in}}{%
\pgfpathmoveto{\pgfqpoint{0.000000in}{-0.010417in}}%
\pgfpathcurveto{\pgfqpoint{0.002763in}{-0.010417in}}{\pgfqpoint{0.005412in}{-0.009319in}}{\pgfqpoint{0.007366in}{-0.007366in}}%
\pgfpathcurveto{\pgfqpoint{0.009319in}{-0.005412in}}{\pgfqpoint{0.010417in}{-0.002763in}}{\pgfqpoint{0.010417in}{0.000000in}}%
\pgfpathcurveto{\pgfqpoint{0.010417in}{0.002763in}}{\pgfqpoint{0.009319in}{0.005412in}}{\pgfqpoint{0.007366in}{0.007366in}}%
\pgfpathcurveto{\pgfqpoint{0.005412in}{0.009319in}}{\pgfqpoint{0.002763in}{0.010417in}}{\pgfqpoint{0.000000in}{0.010417in}}%
\pgfpathcurveto{\pgfqpoint{-0.002763in}{0.010417in}}{\pgfqpoint{-0.005412in}{0.009319in}}{\pgfqpoint{-0.007366in}{0.007366in}}%
\pgfpathcurveto{\pgfqpoint{-0.009319in}{0.005412in}}{\pgfqpoint{-0.010417in}{0.002763in}}{\pgfqpoint{-0.010417in}{0.000000in}}%
\pgfpathcurveto{\pgfqpoint{-0.010417in}{-0.002763in}}{\pgfqpoint{-0.009319in}{-0.005412in}}{\pgfqpoint{-0.007366in}{-0.007366in}}%
\pgfpathcurveto{\pgfqpoint{-0.005412in}{-0.009319in}}{\pgfqpoint{-0.002763in}{-0.010417in}}{\pgfqpoint{0.000000in}{-0.010417in}}%
\pgfpathclose%
\pgfusepath{stroke,fill}%
}%
\begin{pgfscope}%
\pgfsys@transformshift{5.724990in}{1.579608in}%
\pgfsys@useobject{currentmarker}{}%
\end{pgfscope}%
\end{pgfscope}%
\begin{pgfscope}%
\definecolor{textcolor}{rgb}{0.000000,0.000000,0.000000}%
\pgfsetstrokecolor{textcolor}%
\pgfsetfillcolor{textcolor}%
\pgftext[x=5.949990in,y=1.535858in,left,base]{\color{textcolor}\sffamily\fontsize{9.000000}{10.800000}\selectfont 06000 (79.54\% - 01:21:28)}%
\end{pgfscope}%
\begin{pgfscope}%
\pgfsetrectcap%
\pgfsetroundjoin%
\pgfsetlinewidth{0.752812pt}%
\definecolor{currentstroke}{rgb}{0.501961,0.501961,0.501961}%
\pgfsetstrokecolor{currentstroke}%
\pgfsetdash{}{0pt}%
\pgfpathmoveto{\pgfqpoint{5.599990in}{1.396137in}}%
\pgfpathlineto{\pgfqpoint{5.849990in}{1.396137in}}%
\pgfusepath{stroke}%
\end{pgfscope}%
\begin{pgfscope}%
\pgfsetbuttcap%
\pgfsetroundjoin%
\definecolor{currentfill}{rgb}{0.501961,0.501961,0.501961}%
\pgfsetfillcolor{currentfill}%
\pgfsetlinewidth{1.003750pt}%
\definecolor{currentstroke}{rgb}{0.501961,0.501961,0.501961}%
\pgfsetstrokecolor{currentstroke}%
\pgfsetdash{}{0pt}%
\pgfsys@defobject{currentmarker}{\pgfqpoint{-0.010417in}{-0.010417in}}{\pgfqpoint{0.010417in}{0.010417in}}{%
\pgfpathmoveto{\pgfqpoint{0.000000in}{-0.010417in}}%
\pgfpathcurveto{\pgfqpoint{0.002763in}{-0.010417in}}{\pgfqpoint{0.005412in}{-0.009319in}}{\pgfqpoint{0.007366in}{-0.007366in}}%
\pgfpathcurveto{\pgfqpoint{0.009319in}{-0.005412in}}{\pgfqpoint{0.010417in}{-0.002763in}}{\pgfqpoint{0.010417in}{0.000000in}}%
\pgfpathcurveto{\pgfqpoint{0.010417in}{0.002763in}}{\pgfqpoint{0.009319in}{0.005412in}}{\pgfqpoint{0.007366in}{0.007366in}}%
\pgfpathcurveto{\pgfqpoint{0.005412in}{0.009319in}}{\pgfqpoint{0.002763in}{0.010417in}}{\pgfqpoint{0.000000in}{0.010417in}}%
\pgfpathcurveto{\pgfqpoint{-0.002763in}{0.010417in}}{\pgfqpoint{-0.005412in}{0.009319in}}{\pgfqpoint{-0.007366in}{0.007366in}}%
\pgfpathcurveto{\pgfqpoint{-0.009319in}{0.005412in}}{\pgfqpoint{-0.010417in}{0.002763in}}{\pgfqpoint{-0.010417in}{0.000000in}}%
\pgfpathcurveto{\pgfqpoint{-0.010417in}{-0.002763in}}{\pgfqpoint{-0.009319in}{-0.005412in}}{\pgfqpoint{-0.007366in}{-0.007366in}}%
\pgfpathcurveto{\pgfqpoint{-0.005412in}{-0.009319in}}{\pgfqpoint{-0.002763in}{-0.010417in}}{\pgfqpoint{0.000000in}{-0.010417in}}%
\pgfpathclose%
\pgfusepath{stroke,fill}%
}%
\begin{pgfscope}%
\pgfsys@transformshift{5.724990in}{1.396137in}%
\pgfsys@useobject{currentmarker}{}%
\end{pgfscope}%
\end{pgfscope}%
\begin{pgfscope}%
\definecolor{textcolor}{rgb}{0.000000,0.000000,0.000000}%
\pgfsetstrokecolor{textcolor}%
\pgfsetfillcolor{textcolor}%
\pgftext[x=5.949990in,y=1.352387in,left,base]{\color{textcolor}\sffamily\fontsize{9.000000}{10.800000}\selectfont 07000 (80.63\% - 01:32:41)}%
\end{pgfscope}%
\begin{pgfscope}%
\pgfsetrectcap%
\pgfsetroundjoin%
\pgfsetlinewidth{0.752812pt}%
\definecolor{currentstroke}{rgb}{0.501961,0.501961,0.000000}%
\pgfsetstrokecolor{currentstroke}%
\pgfsetdash{}{0pt}%
\pgfpathmoveto{\pgfqpoint{5.599990in}{1.212665in}}%
\pgfpathlineto{\pgfqpoint{5.849990in}{1.212665in}}%
\pgfusepath{stroke}%
\end{pgfscope}%
\begin{pgfscope}%
\pgfsetbuttcap%
\pgfsetroundjoin%
\definecolor{currentfill}{rgb}{0.501961,0.501961,0.000000}%
\pgfsetfillcolor{currentfill}%
\pgfsetlinewidth{1.003750pt}%
\definecolor{currentstroke}{rgb}{0.501961,0.501961,0.000000}%
\pgfsetstrokecolor{currentstroke}%
\pgfsetdash{}{0pt}%
\pgfsys@defobject{currentmarker}{\pgfqpoint{-0.010417in}{-0.010417in}}{\pgfqpoint{0.010417in}{0.010417in}}{%
\pgfpathmoveto{\pgfqpoint{0.000000in}{-0.010417in}}%
\pgfpathcurveto{\pgfqpoint{0.002763in}{-0.010417in}}{\pgfqpoint{0.005412in}{-0.009319in}}{\pgfqpoint{0.007366in}{-0.007366in}}%
\pgfpathcurveto{\pgfqpoint{0.009319in}{-0.005412in}}{\pgfqpoint{0.010417in}{-0.002763in}}{\pgfqpoint{0.010417in}{0.000000in}}%
\pgfpathcurveto{\pgfqpoint{0.010417in}{0.002763in}}{\pgfqpoint{0.009319in}{0.005412in}}{\pgfqpoint{0.007366in}{0.007366in}}%
\pgfpathcurveto{\pgfqpoint{0.005412in}{0.009319in}}{\pgfqpoint{0.002763in}{0.010417in}}{\pgfqpoint{0.000000in}{0.010417in}}%
\pgfpathcurveto{\pgfqpoint{-0.002763in}{0.010417in}}{\pgfqpoint{-0.005412in}{0.009319in}}{\pgfqpoint{-0.007366in}{0.007366in}}%
\pgfpathcurveto{\pgfqpoint{-0.009319in}{0.005412in}}{\pgfqpoint{-0.010417in}{0.002763in}}{\pgfqpoint{-0.010417in}{0.000000in}}%
\pgfpathcurveto{\pgfqpoint{-0.010417in}{-0.002763in}}{\pgfqpoint{-0.009319in}{-0.005412in}}{\pgfqpoint{-0.007366in}{-0.007366in}}%
\pgfpathcurveto{\pgfqpoint{-0.005412in}{-0.009319in}}{\pgfqpoint{-0.002763in}{-0.010417in}}{\pgfqpoint{0.000000in}{-0.010417in}}%
\pgfpathclose%
\pgfusepath{stroke,fill}%
}%
\begin{pgfscope}%
\pgfsys@transformshift{5.724990in}{1.212665in}%
\pgfsys@useobject{currentmarker}{}%
\end{pgfscope}%
\end{pgfscope}%
\begin{pgfscope}%
\definecolor{textcolor}{rgb}{0.000000,0.000000,0.000000}%
\pgfsetstrokecolor{textcolor}%
\pgfsetfillcolor{textcolor}%
\pgftext[x=5.949990in,y=1.168915in,left,base]{\color{textcolor}\sffamily\fontsize{9.000000}{10.800000}\selectfont 08000 (81.24\% - 01:44:32)}%
\end{pgfscope}%
\begin{pgfscope}%
\pgfsetrectcap%
\pgfsetroundjoin%
\pgfsetlinewidth{0.752812pt}%
\definecolor{currentstroke}{rgb}{0.000000,1.000000,1.000000}%
\pgfsetstrokecolor{currentstroke}%
\pgfsetdash{}{0pt}%
\pgfpathmoveto{\pgfqpoint{5.599990in}{1.029194in}}%
\pgfpathlineto{\pgfqpoint{5.849990in}{1.029194in}}%
\pgfusepath{stroke}%
\end{pgfscope}%
\begin{pgfscope}%
\pgfsetbuttcap%
\pgfsetroundjoin%
\definecolor{currentfill}{rgb}{0.000000,1.000000,1.000000}%
\pgfsetfillcolor{currentfill}%
\pgfsetlinewidth{1.003750pt}%
\definecolor{currentstroke}{rgb}{0.000000,1.000000,1.000000}%
\pgfsetstrokecolor{currentstroke}%
\pgfsetdash{}{0pt}%
\pgfsys@defobject{currentmarker}{\pgfqpoint{-0.010417in}{-0.010417in}}{\pgfqpoint{0.010417in}{0.010417in}}{%
\pgfpathmoveto{\pgfqpoint{0.000000in}{-0.010417in}}%
\pgfpathcurveto{\pgfqpoint{0.002763in}{-0.010417in}}{\pgfqpoint{0.005412in}{-0.009319in}}{\pgfqpoint{0.007366in}{-0.007366in}}%
\pgfpathcurveto{\pgfqpoint{0.009319in}{-0.005412in}}{\pgfqpoint{0.010417in}{-0.002763in}}{\pgfqpoint{0.010417in}{0.000000in}}%
\pgfpathcurveto{\pgfqpoint{0.010417in}{0.002763in}}{\pgfqpoint{0.009319in}{0.005412in}}{\pgfqpoint{0.007366in}{0.007366in}}%
\pgfpathcurveto{\pgfqpoint{0.005412in}{0.009319in}}{\pgfqpoint{0.002763in}{0.010417in}}{\pgfqpoint{0.000000in}{0.010417in}}%
\pgfpathcurveto{\pgfqpoint{-0.002763in}{0.010417in}}{\pgfqpoint{-0.005412in}{0.009319in}}{\pgfqpoint{-0.007366in}{0.007366in}}%
\pgfpathcurveto{\pgfqpoint{-0.009319in}{0.005412in}}{\pgfqpoint{-0.010417in}{0.002763in}}{\pgfqpoint{-0.010417in}{0.000000in}}%
\pgfpathcurveto{\pgfqpoint{-0.010417in}{-0.002763in}}{\pgfqpoint{-0.009319in}{-0.005412in}}{\pgfqpoint{-0.007366in}{-0.007366in}}%
\pgfpathcurveto{\pgfqpoint{-0.005412in}{-0.009319in}}{\pgfqpoint{-0.002763in}{-0.010417in}}{\pgfqpoint{0.000000in}{-0.010417in}}%
\pgfpathclose%
\pgfusepath{stroke,fill}%
}%
\begin{pgfscope}%
\pgfsys@transformshift{5.724990in}{1.029194in}%
\pgfsys@useobject{currentmarker}{}%
\end{pgfscope}%
\end{pgfscope}%
\begin{pgfscope}%
\definecolor{textcolor}{rgb}{0.000000,0.000000,0.000000}%
\pgfsetstrokecolor{textcolor}%
\pgfsetfillcolor{textcolor}%
\pgftext[x=5.949990in,y=0.985444in,left,base]{\color{textcolor}\sffamily\fontsize{9.000000}{10.800000}\selectfont 09000 (81.31\% - 01:56:09)}%
\end{pgfscope}%
\begin{pgfscope}%
\pgfsetrectcap%
\pgfsetroundjoin%
\pgfsetlinewidth{0.752812pt}%
\definecolor{currentstroke}{rgb}{0.698039,0.133333,0.133333}%
\pgfsetstrokecolor{currentstroke}%
\pgfsetdash{}{0pt}%
\pgfpathmoveto{\pgfqpoint{5.599990in}{0.845722in}}%
\pgfpathlineto{\pgfqpoint{5.849990in}{0.845722in}}%
\pgfusepath{stroke}%
\end{pgfscope}%
\begin{pgfscope}%
\pgfsetbuttcap%
\pgfsetroundjoin%
\definecolor{currentfill}{rgb}{0.698039,0.133333,0.133333}%
\pgfsetfillcolor{currentfill}%
\pgfsetlinewidth{1.003750pt}%
\definecolor{currentstroke}{rgb}{0.698039,0.133333,0.133333}%
\pgfsetstrokecolor{currentstroke}%
\pgfsetdash{}{0pt}%
\pgfsys@defobject{currentmarker}{\pgfqpoint{-0.010417in}{-0.010417in}}{\pgfqpoint{0.010417in}{0.010417in}}{%
\pgfpathmoveto{\pgfqpoint{0.000000in}{-0.010417in}}%
\pgfpathcurveto{\pgfqpoint{0.002763in}{-0.010417in}}{\pgfqpoint{0.005412in}{-0.009319in}}{\pgfqpoint{0.007366in}{-0.007366in}}%
\pgfpathcurveto{\pgfqpoint{0.009319in}{-0.005412in}}{\pgfqpoint{0.010417in}{-0.002763in}}{\pgfqpoint{0.010417in}{0.000000in}}%
\pgfpathcurveto{\pgfqpoint{0.010417in}{0.002763in}}{\pgfqpoint{0.009319in}{0.005412in}}{\pgfqpoint{0.007366in}{0.007366in}}%
\pgfpathcurveto{\pgfqpoint{0.005412in}{0.009319in}}{\pgfqpoint{0.002763in}{0.010417in}}{\pgfqpoint{0.000000in}{0.010417in}}%
\pgfpathcurveto{\pgfqpoint{-0.002763in}{0.010417in}}{\pgfqpoint{-0.005412in}{0.009319in}}{\pgfqpoint{-0.007366in}{0.007366in}}%
\pgfpathcurveto{\pgfqpoint{-0.009319in}{0.005412in}}{\pgfqpoint{-0.010417in}{0.002763in}}{\pgfqpoint{-0.010417in}{0.000000in}}%
\pgfpathcurveto{\pgfqpoint{-0.010417in}{-0.002763in}}{\pgfqpoint{-0.009319in}{-0.005412in}}{\pgfqpoint{-0.007366in}{-0.007366in}}%
\pgfpathcurveto{\pgfqpoint{-0.005412in}{-0.009319in}}{\pgfqpoint{-0.002763in}{-0.010417in}}{\pgfqpoint{0.000000in}{-0.010417in}}%
\pgfpathclose%
\pgfusepath{stroke,fill}%
}%
\begin{pgfscope}%
\pgfsys@transformshift{5.724990in}{0.845722in}%
\pgfsys@useobject{currentmarker}{}%
\end{pgfscope}%
\end{pgfscope}%
\begin{pgfscope}%
\definecolor{textcolor}{rgb}{0.000000,0.000000,0.000000}%
\pgfsetstrokecolor{textcolor}%
\pgfsetfillcolor{textcolor}%
\pgftext[x=5.949990in,y=0.801972in,left,base]{\color{textcolor}\sffamily\fontsize{9.000000}{10.800000}\selectfont 10000 (82.81\% - 02:08:58)}%
\end{pgfscope}%
\begin{pgfscope}%
\pgfsetrectcap%
\pgfsetroundjoin%
\pgfsetlinewidth{0.752812pt}%
\definecolor{currentstroke}{rgb}{0.000000,0.750000,0.750000}%
\pgfsetstrokecolor{currentstroke}%
\pgfsetdash{}{0pt}%
\pgfpathmoveto{\pgfqpoint{5.599990in}{0.662251in}}%
\pgfpathlineto{\pgfqpoint{5.849990in}{0.662251in}}%
\pgfusepath{stroke}%
\end{pgfscope}%
\begin{pgfscope}%
\pgfsetbuttcap%
\pgfsetroundjoin%
\definecolor{currentfill}{rgb}{0.000000,0.750000,0.750000}%
\pgfsetfillcolor{currentfill}%
\pgfsetlinewidth{1.003750pt}%
\definecolor{currentstroke}{rgb}{0.000000,0.750000,0.750000}%
\pgfsetstrokecolor{currentstroke}%
\pgfsetdash{}{0pt}%
\pgfsys@defobject{currentmarker}{\pgfqpoint{-0.010417in}{-0.010417in}}{\pgfqpoint{0.010417in}{0.010417in}}{%
\pgfpathmoveto{\pgfqpoint{0.000000in}{-0.010417in}}%
\pgfpathcurveto{\pgfqpoint{0.002763in}{-0.010417in}}{\pgfqpoint{0.005412in}{-0.009319in}}{\pgfqpoint{0.007366in}{-0.007366in}}%
\pgfpathcurveto{\pgfqpoint{0.009319in}{-0.005412in}}{\pgfqpoint{0.010417in}{-0.002763in}}{\pgfqpoint{0.010417in}{0.000000in}}%
\pgfpathcurveto{\pgfqpoint{0.010417in}{0.002763in}}{\pgfqpoint{0.009319in}{0.005412in}}{\pgfqpoint{0.007366in}{0.007366in}}%
\pgfpathcurveto{\pgfqpoint{0.005412in}{0.009319in}}{\pgfqpoint{0.002763in}{0.010417in}}{\pgfqpoint{0.000000in}{0.010417in}}%
\pgfpathcurveto{\pgfqpoint{-0.002763in}{0.010417in}}{\pgfqpoint{-0.005412in}{0.009319in}}{\pgfqpoint{-0.007366in}{0.007366in}}%
\pgfpathcurveto{\pgfqpoint{-0.009319in}{0.005412in}}{\pgfqpoint{-0.010417in}{0.002763in}}{\pgfqpoint{-0.010417in}{0.000000in}}%
\pgfpathcurveto{\pgfqpoint{-0.010417in}{-0.002763in}}{\pgfqpoint{-0.009319in}{-0.005412in}}{\pgfqpoint{-0.007366in}{-0.007366in}}%
\pgfpathcurveto{\pgfqpoint{-0.005412in}{-0.009319in}}{\pgfqpoint{-0.002763in}{-0.010417in}}{\pgfqpoint{0.000000in}{-0.010417in}}%
\pgfpathclose%
\pgfusepath{stroke,fill}%
}%
\begin{pgfscope}%
\pgfsys@transformshift{5.724990in}{0.662251in}%
\pgfsys@useobject{currentmarker}{}%
\end{pgfscope}%
\end{pgfscope}%
\begin{pgfscope}%
\definecolor{textcolor}{rgb}{0.000000,0.000000,0.000000}%
\pgfsetstrokecolor{textcolor}%
\pgfsetfillcolor{textcolor}%
\pgftext[x=5.949990in,y=0.618501in,left,base]{\color{textcolor}\sffamily\fontsize{9.000000}{10.800000}\selectfont 11913 (83.59\% - 02:30:20)}%
\end{pgfscope}%
\begin{pgfscope}%
\pgfsetbuttcap%
\pgfsetmiterjoin%
\definecolor{currentfill}{rgb}{1.000000,1.000000,1.000000}%
\pgfsetfillcolor{currentfill}%
\pgfsetfillopacity{0.750000}%
\pgfsetlinewidth{0.000000pt}%
\definecolor{currentstroke}{rgb}{0.000000,0.000000,0.000000}%
\pgfsetstrokecolor{currentstroke}%
\pgfsetstrokeopacity{0.750000}%
\pgfsetdash{}{0pt}%
\pgfpathmoveto{\pgfqpoint{3.069444in}{0.769444in}}%
\pgfpathlineto{\pgfqpoint{4.894630in}{0.769444in}}%
\pgfpathlineto{\pgfqpoint{4.894630in}{3.080556in}}%
\pgfpathlineto{\pgfqpoint{3.069444in}{3.080556in}}%
\pgfpathclose%
\pgfusepath{fill}%
\end{pgfscope}%
\begin{pgfscope}%
\pgfsetbuttcap%
\pgfsetmiterjoin%
\definecolor{currentfill}{rgb}{1.000000,1.000000,1.000000}%
\pgfsetfillcolor{currentfill}%
\pgfsetfillopacity{0.750000}%
\pgfsetlinewidth{0.501875pt}%
\definecolor{currentstroke}{rgb}{1.000000,0.250980,0.250980}%
\pgfsetstrokecolor{currentstroke}%
\pgfsetstrokeopacity{0.750000}%
\pgfsetdash{{1.850000pt}{0.800000pt}}{0.000000pt}%
\pgfpathmoveto{\pgfqpoint{3.069444in}{3.080556in}}%
\pgfpathlineto{\pgfqpoint{0.785185in}{3.060741in}}%
\pgfusepath{stroke,fill}%
\end{pgfscope}%
\begin{pgfscope}%
\pgfsetbuttcap%
\pgfsetmiterjoin%
\definecolor{currentfill}{rgb}{1.000000,1.000000,1.000000}%
\pgfsetfillcolor{currentfill}%
\pgfsetfillopacity{0.750000}%
\pgfsetlinewidth{0.501875pt}%
\definecolor{currentstroke}{rgb}{1.000000,0.250980,0.250980}%
\pgfsetstrokecolor{currentstroke}%
\pgfsetstrokeopacity{0.750000}%
\pgfsetdash{{1.850000pt}{0.800000pt}}{0.000000pt}%
\pgfpathmoveto{\pgfqpoint{3.069444in}{0.769444in}}%
\pgfpathlineto{\pgfqpoint{0.785185in}{1.905185in}}%
\pgfusepath{stroke,fill}%
\end{pgfscope}%
\begin{pgfscope}%
\pgfsetbuttcap%
\pgfsetroundjoin%
\definecolor{currentfill}{rgb}{0.000000,0.000000,0.000000}%
\pgfsetfillcolor{currentfill}%
\pgfsetlinewidth{0.803000pt}%
\definecolor{currentstroke}{rgb}{0.000000,0.000000,0.000000}%
\pgfsetstrokecolor{currentstroke}%
\pgfsetdash{}{0pt}%
\pgfsys@defobject{currentmarker}{\pgfqpoint{0.000000in}{-0.048611in}}{\pgfqpoint{0.000000in}{0.000000in}}{%
\pgfpathmoveto{\pgfqpoint{0.000000in}{0.000000in}}%
\pgfpathlineto{\pgfqpoint{0.000000in}{-0.048611in}}%
\pgfusepath{stroke,fill}%
}%
\begin{pgfscope}%
\pgfsys@transformshift{3.069444in}{0.769444in}%
\pgfsys@useobject{currentmarker}{}%
\end{pgfscope}%
\end{pgfscope}%
\begin{pgfscope}%
\definecolor{textcolor}{rgb}{0.000000,0.000000,0.000000}%
\pgfsetstrokecolor{textcolor}%
\pgfsetfillcolor{textcolor}%
\pgftext[x=3.069444in,y=0.672222in,,top]{\color{textcolor}\sffamily\fontsize{7.000000}{8.400000}\selectfont 100}%
\end{pgfscope}%
\begin{pgfscope}%
\pgfsetbuttcap%
\pgfsetroundjoin%
\definecolor{currentfill}{rgb}{0.000000,0.000000,0.000000}%
\pgfsetfillcolor{currentfill}%
\pgfsetlinewidth{0.803000pt}%
\definecolor{currentstroke}{rgb}{0.000000,0.000000,0.000000}%
\pgfsetstrokecolor{currentstroke}%
\pgfsetdash{}{0pt}%
\pgfsys@defobject{currentmarker}{\pgfqpoint{0.000000in}{-0.048611in}}{\pgfqpoint{0.000000in}{0.000000in}}{%
\pgfpathmoveto{\pgfqpoint{0.000000in}{0.000000in}}%
\pgfpathlineto{\pgfqpoint{0.000000in}{-0.048611in}}%
\pgfusepath{stroke,fill}%
}%
\begin{pgfscope}%
\pgfsys@transformshift{3.330185in}{0.769444in}%
\pgfsys@useobject{currentmarker}{}%
\end{pgfscope}%
\end{pgfscope}%
\begin{pgfscope}%
\definecolor{textcolor}{rgb}{0.000000,0.000000,0.000000}%
\pgfsetstrokecolor{textcolor}%
\pgfsetfillcolor{textcolor}%
\pgftext[x=3.330185in,y=0.672222in,,top]{\color{textcolor}\sffamily\fontsize{7.000000}{8.400000}\selectfont 300}%
\end{pgfscope}%
\begin{pgfscope}%
\pgfsetbuttcap%
\pgfsetroundjoin%
\definecolor{currentfill}{rgb}{0.000000,0.000000,0.000000}%
\pgfsetfillcolor{currentfill}%
\pgfsetlinewidth{0.803000pt}%
\definecolor{currentstroke}{rgb}{0.000000,0.000000,0.000000}%
\pgfsetstrokecolor{currentstroke}%
\pgfsetdash{}{0pt}%
\pgfsys@defobject{currentmarker}{\pgfqpoint{0.000000in}{-0.048611in}}{\pgfqpoint{0.000000in}{0.000000in}}{%
\pgfpathmoveto{\pgfqpoint{0.000000in}{0.000000in}}%
\pgfpathlineto{\pgfqpoint{0.000000in}{-0.048611in}}%
\pgfusepath{stroke,fill}%
}%
\begin{pgfscope}%
\pgfsys@transformshift{3.590926in}{0.769444in}%
\pgfsys@useobject{currentmarker}{}%
\end{pgfscope}%
\end{pgfscope}%
\begin{pgfscope}%
\definecolor{textcolor}{rgb}{0.000000,0.000000,0.000000}%
\pgfsetstrokecolor{textcolor}%
\pgfsetfillcolor{textcolor}%
\pgftext[x=3.590926in,y=0.672222in,,top]{\color{textcolor}\sffamily\fontsize{7.000000}{8.400000}\selectfont 500}%
\end{pgfscope}%
\begin{pgfscope}%
\pgfsetbuttcap%
\pgfsetroundjoin%
\definecolor{currentfill}{rgb}{0.000000,0.000000,0.000000}%
\pgfsetfillcolor{currentfill}%
\pgfsetlinewidth{0.803000pt}%
\definecolor{currentstroke}{rgb}{0.000000,0.000000,0.000000}%
\pgfsetstrokecolor{currentstroke}%
\pgfsetdash{}{0pt}%
\pgfsys@defobject{currentmarker}{\pgfqpoint{0.000000in}{-0.048611in}}{\pgfqpoint{0.000000in}{0.000000in}}{%
\pgfpathmoveto{\pgfqpoint{0.000000in}{0.000000in}}%
\pgfpathlineto{\pgfqpoint{0.000000in}{-0.048611in}}%
\pgfusepath{stroke,fill}%
}%
\begin{pgfscope}%
\pgfsys@transformshift{3.851667in}{0.769444in}%
\pgfsys@useobject{currentmarker}{}%
\end{pgfscope}%
\end{pgfscope}%
\begin{pgfscope}%
\definecolor{textcolor}{rgb}{0.000000,0.000000,0.000000}%
\pgfsetstrokecolor{textcolor}%
\pgfsetfillcolor{textcolor}%
\pgftext[x=3.851667in,y=0.672222in,,top]{\color{textcolor}\sffamily\fontsize{7.000000}{8.400000}\selectfont 700}%
\end{pgfscope}%
\begin{pgfscope}%
\pgfsetbuttcap%
\pgfsetroundjoin%
\definecolor{currentfill}{rgb}{0.000000,0.000000,0.000000}%
\pgfsetfillcolor{currentfill}%
\pgfsetlinewidth{0.803000pt}%
\definecolor{currentstroke}{rgb}{0.000000,0.000000,0.000000}%
\pgfsetstrokecolor{currentstroke}%
\pgfsetdash{}{0pt}%
\pgfsys@defobject{currentmarker}{\pgfqpoint{0.000000in}{-0.048611in}}{\pgfqpoint{0.000000in}{0.000000in}}{%
\pgfpathmoveto{\pgfqpoint{0.000000in}{0.000000in}}%
\pgfpathlineto{\pgfqpoint{0.000000in}{-0.048611in}}%
\pgfusepath{stroke,fill}%
}%
\begin{pgfscope}%
\pgfsys@transformshift{4.112407in}{0.769444in}%
\pgfsys@useobject{currentmarker}{}%
\end{pgfscope}%
\end{pgfscope}%
\begin{pgfscope}%
\definecolor{textcolor}{rgb}{0.000000,0.000000,0.000000}%
\pgfsetstrokecolor{textcolor}%
\pgfsetfillcolor{textcolor}%
\pgftext[x=4.112407in,y=0.672222in,,top]{\color{textcolor}\sffamily\fontsize{7.000000}{8.400000}\selectfont 900}%
\end{pgfscope}%
\begin{pgfscope}%
\pgfsetbuttcap%
\pgfsetroundjoin%
\definecolor{currentfill}{rgb}{0.000000,0.000000,0.000000}%
\pgfsetfillcolor{currentfill}%
\pgfsetlinewidth{0.803000pt}%
\definecolor{currentstroke}{rgb}{0.000000,0.000000,0.000000}%
\pgfsetstrokecolor{currentstroke}%
\pgfsetdash{}{0pt}%
\pgfsys@defobject{currentmarker}{\pgfqpoint{0.000000in}{-0.048611in}}{\pgfqpoint{0.000000in}{0.000000in}}{%
\pgfpathmoveto{\pgfqpoint{0.000000in}{0.000000in}}%
\pgfpathlineto{\pgfqpoint{0.000000in}{-0.048611in}}%
\pgfusepath{stroke,fill}%
}%
\begin{pgfscope}%
\pgfsys@transformshift{4.373148in}{0.769444in}%
\pgfsys@useobject{currentmarker}{}%
\end{pgfscope}%
\end{pgfscope}%
\begin{pgfscope}%
\definecolor{textcolor}{rgb}{0.000000,0.000000,0.000000}%
\pgfsetstrokecolor{textcolor}%
\pgfsetfillcolor{textcolor}%
\pgftext[x=4.373148in,y=0.672222in,,top]{\color{textcolor}\sffamily\fontsize{7.000000}{8.400000}\selectfont 1100}%
\end{pgfscope}%
\begin{pgfscope}%
\pgfsetbuttcap%
\pgfsetroundjoin%
\definecolor{currentfill}{rgb}{0.000000,0.000000,0.000000}%
\pgfsetfillcolor{currentfill}%
\pgfsetlinewidth{0.803000pt}%
\definecolor{currentstroke}{rgb}{0.000000,0.000000,0.000000}%
\pgfsetstrokecolor{currentstroke}%
\pgfsetdash{}{0pt}%
\pgfsys@defobject{currentmarker}{\pgfqpoint{0.000000in}{-0.048611in}}{\pgfqpoint{0.000000in}{0.000000in}}{%
\pgfpathmoveto{\pgfqpoint{0.000000in}{0.000000in}}%
\pgfpathlineto{\pgfqpoint{0.000000in}{-0.048611in}}%
\pgfusepath{stroke,fill}%
}%
\begin{pgfscope}%
\pgfsys@transformshift{4.633889in}{0.769444in}%
\pgfsys@useobject{currentmarker}{}%
\end{pgfscope}%
\end{pgfscope}%
\begin{pgfscope}%
\definecolor{textcolor}{rgb}{0.000000,0.000000,0.000000}%
\pgfsetstrokecolor{textcolor}%
\pgfsetfillcolor{textcolor}%
\pgftext[x=4.633889in,y=0.672222in,,top]{\color{textcolor}\sffamily\fontsize{7.000000}{8.400000}\selectfont 1300}%
\end{pgfscope}%
\begin{pgfscope}%
\pgfsetbuttcap%
\pgfsetroundjoin%
\definecolor{currentfill}{rgb}{0.000000,0.000000,0.000000}%
\pgfsetfillcolor{currentfill}%
\pgfsetlinewidth{0.803000pt}%
\definecolor{currentstroke}{rgb}{0.000000,0.000000,0.000000}%
\pgfsetstrokecolor{currentstroke}%
\pgfsetdash{}{0pt}%
\pgfsys@defobject{currentmarker}{\pgfqpoint{-0.048611in}{0.000000in}}{\pgfqpoint{0.000000in}{0.000000in}}{%
\pgfpathmoveto{\pgfqpoint{0.000000in}{0.000000in}}%
\pgfpathlineto{\pgfqpoint{-0.048611in}{0.000000in}}%
\pgfusepath{stroke,fill}%
}%
\begin{pgfscope}%
\pgfsys@transformshift{3.069444in}{0.769444in}%
\pgfsys@useobject{currentmarker}{}%
\end{pgfscope}%
\end{pgfscope}%
\begin{pgfscope}%
\definecolor{textcolor}{rgb}{0.000000,0.000000,0.000000}%
\pgfsetstrokecolor{textcolor}%
\pgfsetfillcolor{textcolor}%
\pgftext[x=2.756131in,y=0.732511in,left,base]{\color{textcolor}\sffamily\fontsize{7.000000}{8.400000}\selectfont 42\%}%
\end{pgfscope}%
\begin{pgfscope}%
\pgfsetbuttcap%
\pgfsetroundjoin%
\definecolor{currentfill}{rgb}{0.000000,0.000000,0.000000}%
\pgfsetfillcolor{currentfill}%
\pgfsetlinewidth{0.803000pt}%
\definecolor{currentstroke}{rgb}{0.000000,0.000000,0.000000}%
\pgfsetstrokecolor{currentstroke}%
\pgfsetdash{}{0pt}%
\pgfsys@defobject{currentmarker}{\pgfqpoint{-0.048611in}{0.000000in}}{\pgfqpoint{0.000000in}{0.000000in}}{%
\pgfpathmoveto{\pgfqpoint{0.000000in}{0.000000in}}%
\pgfpathlineto{\pgfqpoint{-0.048611in}{0.000000in}}%
\pgfusepath{stroke,fill}%
}%
\begin{pgfscope}%
\pgfsys@transformshift{3.069444in}{1.154630in}%
\pgfsys@useobject{currentmarker}{}%
\end{pgfscope}%
\end{pgfscope}%
\begin{pgfscope}%
\definecolor{textcolor}{rgb}{0.000000,0.000000,0.000000}%
\pgfsetstrokecolor{textcolor}%
\pgfsetfillcolor{textcolor}%
\pgftext[x=2.756131in,y=1.117697in,left,base]{\color{textcolor}\sffamily\fontsize{7.000000}{8.400000}\selectfont 47\%}%
\end{pgfscope}%
\begin{pgfscope}%
\pgfsetbuttcap%
\pgfsetroundjoin%
\definecolor{currentfill}{rgb}{0.000000,0.000000,0.000000}%
\pgfsetfillcolor{currentfill}%
\pgfsetlinewidth{0.803000pt}%
\definecolor{currentstroke}{rgb}{0.000000,0.000000,0.000000}%
\pgfsetstrokecolor{currentstroke}%
\pgfsetdash{}{0pt}%
\pgfsys@defobject{currentmarker}{\pgfqpoint{-0.048611in}{0.000000in}}{\pgfqpoint{0.000000in}{0.000000in}}{%
\pgfpathmoveto{\pgfqpoint{0.000000in}{0.000000in}}%
\pgfpathlineto{\pgfqpoint{-0.048611in}{0.000000in}}%
\pgfusepath{stroke,fill}%
}%
\begin{pgfscope}%
\pgfsys@transformshift{3.069444in}{1.539815in}%
\pgfsys@useobject{currentmarker}{}%
\end{pgfscope}%
\end{pgfscope}%
\begin{pgfscope}%
\definecolor{textcolor}{rgb}{0.000000,0.000000,0.000000}%
\pgfsetstrokecolor{textcolor}%
\pgfsetfillcolor{textcolor}%
\pgftext[x=2.756131in,y=1.502882in,left,base]{\color{textcolor}\sffamily\fontsize{7.000000}{8.400000}\selectfont 52\%}%
\end{pgfscope}%
\begin{pgfscope}%
\pgfsetbuttcap%
\pgfsetroundjoin%
\definecolor{currentfill}{rgb}{0.000000,0.000000,0.000000}%
\pgfsetfillcolor{currentfill}%
\pgfsetlinewidth{0.803000pt}%
\definecolor{currentstroke}{rgb}{0.000000,0.000000,0.000000}%
\pgfsetstrokecolor{currentstroke}%
\pgfsetdash{}{0pt}%
\pgfsys@defobject{currentmarker}{\pgfqpoint{-0.048611in}{0.000000in}}{\pgfqpoint{0.000000in}{0.000000in}}{%
\pgfpathmoveto{\pgfqpoint{0.000000in}{0.000000in}}%
\pgfpathlineto{\pgfqpoint{-0.048611in}{0.000000in}}%
\pgfusepath{stroke,fill}%
}%
\begin{pgfscope}%
\pgfsys@transformshift{3.069444in}{1.925000in}%
\pgfsys@useobject{currentmarker}{}%
\end{pgfscope}%
\end{pgfscope}%
\begin{pgfscope}%
\definecolor{textcolor}{rgb}{0.000000,0.000000,0.000000}%
\pgfsetstrokecolor{textcolor}%
\pgfsetfillcolor{textcolor}%
\pgftext[x=2.756131in,y=1.888067in,left,base]{\color{textcolor}\sffamily\fontsize{7.000000}{8.400000}\selectfont 57\%}%
\end{pgfscope}%
\begin{pgfscope}%
\pgfsetbuttcap%
\pgfsetroundjoin%
\definecolor{currentfill}{rgb}{0.000000,0.000000,0.000000}%
\pgfsetfillcolor{currentfill}%
\pgfsetlinewidth{0.803000pt}%
\definecolor{currentstroke}{rgb}{0.000000,0.000000,0.000000}%
\pgfsetstrokecolor{currentstroke}%
\pgfsetdash{}{0pt}%
\pgfsys@defobject{currentmarker}{\pgfqpoint{-0.048611in}{0.000000in}}{\pgfqpoint{0.000000in}{0.000000in}}{%
\pgfpathmoveto{\pgfqpoint{0.000000in}{0.000000in}}%
\pgfpathlineto{\pgfqpoint{-0.048611in}{0.000000in}}%
\pgfusepath{stroke,fill}%
}%
\begin{pgfscope}%
\pgfsys@transformshift{3.069444in}{2.310185in}%
\pgfsys@useobject{currentmarker}{}%
\end{pgfscope}%
\end{pgfscope}%
\begin{pgfscope}%
\definecolor{textcolor}{rgb}{0.000000,0.000000,0.000000}%
\pgfsetstrokecolor{textcolor}%
\pgfsetfillcolor{textcolor}%
\pgftext[x=2.756131in,y=2.273252in,left,base]{\color{textcolor}\sffamily\fontsize{7.000000}{8.400000}\selectfont 62\%}%
\end{pgfscope}%
\begin{pgfscope}%
\pgfsetbuttcap%
\pgfsetroundjoin%
\definecolor{currentfill}{rgb}{0.000000,0.000000,0.000000}%
\pgfsetfillcolor{currentfill}%
\pgfsetlinewidth{0.803000pt}%
\definecolor{currentstroke}{rgb}{0.000000,0.000000,0.000000}%
\pgfsetstrokecolor{currentstroke}%
\pgfsetdash{}{0pt}%
\pgfsys@defobject{currentmarker}{\pgfqpoint{-0.048611in}{0.000000in}}{\pgfqpoint{0.000000in}{0.000000in}}{%
\pgfpathmoveto{\pgfqpoint{0.000000in}{0.000000in}}%
\pgfpathlineto{\pgfqpoint{-0.048611in}{0.000000in}}%
\pgfusepath{stroke,fill}%
}%
\begin{pgfscope}%
\pgfsys@transformshift{3.069444in}{2.695370in}%
\pgfsys@useobject{currentmarker}{}%
\end{pgfscope}%
\end{pgfscope}%
\begin{pgfscope}%
\definecolor{textcolor}{rgb}{0.000000,0.000000,0.000000}%
\pgfsetstrokecolor{textcolor}%
\pgfsetfillcolor{textcolor}%
\pgftext[x=2.756131in,y=2.658437in,left,base]{\color{textcolor}\sffamily\fontsize{7.000000}{8.400000}\selectfont 67\%}%
\end{pgfscope}%
\begin{pgfscope}%
\pgfpathrectangle{\pgfqpoint{3.069444in}{0.769444in}}{\pgfqpoint{1.825185in}{2.311111in}}%
\pgfusepath{clip}%
\pgfsetrectcap%
\pgfsetroundjoin%
\pgfsetlinewidth{0.752812pt}%
\definecolor{currentstroke}{rgb}{0.000000,0.000000,1.000000}%
\pgfsetstrokecolor{currentstroke}%
\pgfsetdash{}{0pt}%
\pgfpathmoveto{\pgfqpoint{3.697938in}{0.759444in}}%
\pgfpathlineto{\pgfqpoint{3.752585in}{0.878067in}}%
\pgfpathlineto{\pgfqpoint{3.803289in}{0.759444in}}%
\pgfpathmoveto{\pgfqpoint{3.834727in}{0.759444in}}%
\pgfpathlineto{\pgfqpoint{3.880348in}{0.836467in}}%
\pgfpathlineto{\pgfqpoint{3.944230in}{0.991311in}}%
\pgfpathlineto{\pgfqpoint{4.008111in}{0.935844in}}%
\pgfpathlineto{\pgfqpoint{4.071993in}{1.035222in}}%
\pgfpathlineto{\pgfqpoint{4.135874in}{0.920437in}}%
\pgfpathlineto{\pgfqpoint{4.199756in}{1.027519in}}%
\pgfpathlineto{\pgfqpoint{4.263637in}{1.059104in}}%
\pgfpathlineto{\pgfqpoint{4.327519in}{1.030600in}}%
\pgfusepath{stroke}%
\end{pgfscope}%
\begin{pgfscope}%
\pgfpathrectangle{\pgfqpoint{3.069444in}{0.769444in}}{\pgfqpoint{1.825185in}{2.311111in}}%
\pgfusepath{clip}%
\pgfsetbuttcap%
\pgfsetroundjoin%
\definecolor{currentfill}{rgb}{0.000000,0.000000,1.000000}%
\pgfsetfillcolor{currentfill}%
\pgfsetlinewidth{1.003750pt}%
\definecolor{currentstroke}{rgb}{0.000000,0.000000,1.000000}%
\pgfsetstrokecolor{currentstroke}%
\pgfsetdash{}{0pt}%
\pgfsys@defobject{currentmarker}{\pgfqpoint{-0.010417in}{-0.010417in}}{\pgfqpoint{0.010417in}{0.010417in}}{%
\pgfpathmoveto{\pgfqpoint{0.000000in}{-0.010417in}}%
\pgfpathcurveto{\pgfqpoint{0.002763in}{-0.010417in}}{\pgfqpoint{0.005412in}{-0.009319in}}{\pgfqpoint{0.007366in}{-0.007366in}}%
\pgfpathcurveto{\pgfqpoint{0.009319in}{-0.005412in}}{\pgfqpoint{0.010417in}{-0.002763in}}{\pgfqpoint{0.010417in}{0.000000in}}%
\pgfpathcurveto{\pgfqpoint{0.010417in}{0.002763in}}{\pgfqpoint{0.009319in}{0.005412in}}{\pgfqpoint{0.007366in}{0.007366in}}%
\pgfpathcurveto{\pgfqpoint{0.005412in}{0.009319in}}{\pgfqpoint{0.002763in}{0.010417in}}{\pgfqpoint{0.000000in}{0.010417in}}%
\pgfpathcurveto{\pgfqpoint{-0.002763in}{0.010417in}}{\pgfqpoint{-0.005412in}{0.009319in}}{\pgfqpoint{-0.007366in}{0.007366in}}%
\pgfpathcurveto{\pgfqpoint{-0.009319in}{0.005412in}}{\pgfqpoint{-0.010417in}{0.002763in}}{\pgfqpoint{-0.010417in}{0.000000in}}%
\pgfpathcurveto{\pgfqpoint{-0.010417in}{-0.002763in}}{\pgfqpoint{-0.009319in}{-0.005412in}}{\pgfqpoint{-0.007366in}{-0.007366in}}%
\pgfpathcurveto{\pgfqpoint{-0.005412in}{-0.009319in}}{\pgfqpoint{-0.002763in}{-0.010417in}}{\pgfqpoint{0.000000in}{-0.010417in}}%
\pgfpathclose%
\pgfusepath{stroke,fill}%
}%
\begin{pgfscope}%
\pgfsys@transformshift{3.026422in}{-1.990022in}%
\pgfsys@useobject{currentmarker}{}%
\end{pgfscope}%
\begin{pgfscope}%
\pgfsys@transformshift{3.092911in}{-1.527030in}%
\pgfsys@useobject{currentmarker}{}%
\end{pgfscope}%
\begin{pgfscope}%
\pgfsys@transformshift{3.156793in}{-0.921519in}%
\pgfsys@useobject{currentmarker}{}%
\end{pgfscope}%
\begin{pgfscope}%
\pgfsys@transformshift{3.220674in}{-0.598733in}%
\pgfsys@useobject{currentmarker}{}%
\end{pgfscope}%
\begin{pgfscope}%
\pgfsys@transformshift{3.284556in}{-0.205074in}%
\pgfsys@useobject{currentmarker}{}%
\end{pgfscope}%
\begin{pgfscope}%
\pgfsys@transformshift{3.369296in}{-0.155000in}%
\pgfsys@useobject{currentmarker}{}%
\end{pgfscope}%
\begin{pgfscope}%
\pgfsys@transformshift{3.433178in}{0.177800in}%
\pgfsys@useobject{currentmarker}{}%
\end{pgfscope}%
\begin{pgfscope}%
\pgfsys@transformshift{3.497059in}{0.505978in}%
\pgfsys@useobject{currentmarker}{}%
\end{pgfscope}%
\begin{pgfscope}%
\pgfsys@transformshift{3.560941in}{0.683933in}%
\pgfsys@useobject{currentmarker}{}%
\end{pgfscope}%
\begin{pgfscope}%
\pgfsys@transformshift{3.624822in}{0.571459in}%
\pgfsys@useobject{currentmarker}{}%
\end{pgfscope}%
\begin{pgfscope}%
\pgfsys@transformshift{3.688704in}{0.739400in}%
\pgfsys@useobject{currentmarker}{}%
\end{pgfscope}%
\begin{pgfscope}%
\pgfsys@transformshift{3.752585in}{0.878067in}%
\pgfsys@useobject{currentmarker}{}%
\end{pgfscope}%
\begin{pgfscope}%
\pgfsys@transformshift{3.816467in}{0.728615in}%
\pgfsys@useobject{currentmarker}{}%
\end{pgfscope}%
\begin{pgfscope}%
\pgfsys@transformshift{3.880348in}{0.836467in}%
\pgfsys@useobject{currentmarker}{}%
\end{pgfscope}%
\begin{pgfscope}%
\pgfsys@transformshift{3.944230in}{0.991311in}%
\pgfsys@useobject{currentmarker}{}%
\end{pgfscope}%
\begin{pgfscope}%
\pgfsys@transformshift{4.008111in}{0.935844in}%
\pgfsys@useobject{currentmarker}{}%
\end{pgfscope}%
\begin{pgfscope}%
\pgfsys@transformshift{4.071993in}{1.035222in}%
\pgfsys@useobject{currentmarker}{}%
\end{pgfscope}%
\begin{pgfscope}%
\pgfsys@transformshift{4.135874in}{0.920437in}%
\pgfsys@useobject{currentmarker}{}%
\end{pgfscope}%
\begin{pgfscope}%
\pgfsys@transformshift{4.199756in}{1.027519in}%
\pgfsys@useobject{currentmarker}{}%
\end{pgfscope}%
\begin{pgfscope}%
\pgfsys@transformshift{4.263637in}{1.059104in}%
\pgfsys@useobject{currentmarker}{}%
\end{pgfscope}%
\begin{pgfscope}%
\pgfsys@transformshift{4.327519in}{1.030600in}%
\pgfsys@useobject{currentmarker}{}%
\end{pgfscope}%
\end{pgfscope}%
\begin{pgfscope}%
\pgfpathrectangle{\pgfqpoint{3.069444in}{0.769444in}}{\pgfqpoint{1.825185in}{2.311111in}}%
\pgfusepath{clip}%
\pgfsetrectcap%
\pgfsetroundjoin%
\pgfsetlinewidth{0.752812pt}%
\definecolor{currentstroke}{rgb}{1.000000,0.647059,0.000000}%
\pgfsetstrokecolor{currentstroke}%
\pgfsetdash{}{0pt}%
\pgfpathmoveto{\pgfqpoint{3.252632in}{0.759444in}}%
\pgfpathlineto{\pgfqpoint{3.306719in}{1.077593in}}%
\pgfpathlineto{\pgfqpoint{3.392763in}{1.533652in}}%
\pgfpathlineto{\pgfqpoint{3.478807in}{1.809444in}}%
\pgfpathlineto{\pgfqpoint{3.564852in}{1.806363in}}%
\pgfpathlineto{\pgfqpoint{3.650896in}{1.974304in}}%
\pgfpathlineto{\pgfqpoint{3.736941in}{2.037474in}}%
\pgfpathlineto{\pgfqpoint{3.822985in}{2.171519in}}%
\pgfpathlineto{\pgfqpoint{3.909030in}{2.200022in}}%
\pgfpathlineto{\pgfqpoint{3.995074in}{2.223904in}}%
\pgfpathlineto{\pgfqpoint{4.081119in}{2.176911in}}%
\pgfpathlineto{\pgfqpoint{4.167163in}{2.129148in}}%
\pgfpathlineto{\pgfqpoint{4.253207in}{2.200022in}}%
\pgfpathlineto{\pgfqpoint{4.339252in}{2.216200in}}%
\pgfpathlineto{\pgfqpoint{4.425296in}{2.220822in}}%
\pgfpathlineto{\pgfqpoint{4.511341in}{2.265504in}}%
\pgfpathlineto{\pgfqpoint{4.597385in}{2.257800in}}%
\pgfpathlineto{\pgfqpoint{4.683430in}{2.244704in}}%
\pgfpathlineto{\pgfqpoint{4.769474in}{2.134541in}}%
\pgfusepath{stroke}%
\end{pgfscope}%
\begin{pgfscope}%
\pgfpathrectangle{\pgfqpoint{3.069444in}{0.769444in}}{\pgfqpoint{1.825185in}{2.311111in}}%
\pgfusepath{clip}%
\pgfsetbuttcap%
\pgfsetroundjoin%
\definecolor{currentfill}{rgb}{1.000000,0.647059,0.000000}%
\pgfsetfillcolor{currentfill}%
\pgfsetlinewidth{1.003750pt}%
\definecolor{currentstroke}{rgb}{1.000000,0.647059,0.000000}%
\pgfsetstrokecolor{currentstroke}%
\pgfsetdash{}{0pt}%
\pgfsys@defobject{currentmarker}{\pgfqpoint{-0.010417in}{-0.010417in}}{\pgfqpoint{0.010417in}{0.010417in}}{%
\pgfpathmoveto{\pgfqpoint{0.000000in}{-0.010417in}}%
\pgfpathcurveto{\pgfqpoint{0.002763in}{-0.010417in}}{\pgfqpoint{0.005412in}{-0.009319in}}{\pgfqpoint{0.007366in}{-0.007366in}}%
\pgfpathcurveto{\pgfqpoint{0.009319in}{-0.005412in}}{\pgfqpoint{0.010417in}{-0.002763in}}{\pgfqpoint{0.010417in}{0.000000in}}%
\pgfpathcurveto{\pgfqpoint{0.010417in}{0.002763in}}{\pgfqpoint{0.009319in}{0.005412in}}{\pgfqpoint{0.007366in}{0.007366in}}%
\pgfpathcurveto{\pgfqpoint{0.005412in}{0.009319in}}{\pgfqpoint{0.002763in}{0.010417in}}{\pgfqpoint{0.000000in}{0.010417in}}%
\pgfpathcurveto{\pgfqpoint{-0.002763in}{0.010417in}}{\pgfqpoint{-0.005412in}{0.009319in}}{\pgfqpoint{-0.007366in}{0.007366in}}%
\pgfpathcurveto{\pgfqpoint{-0.009319in}{0.005412in}}{\pgfqpoint{-0.010417in}{0.002763in}}{\pgfqpoint{-0.010417in}{0.000000in}}%
\pgfpathcurveto{\pgfqpoint{-0.010417in}{-0.002763in}}{\pgfqpoint{-0.009319in}{-0.005412in}}{\pgfqpoint{-0.007366in}{-0.007366in}}%
\pgfpathcurveto{\pgfqpoint{-0.005412in}{-0.009319in}}{\pgfqpoint{-0.002763in}{-0.010417in}}{\pgfqpoint{0.000000in}{-0.010417in}}%
\pgfpathclose%
\pgfusepath{stroke,fill}%
}%
\begin{pgfscope}%
\pgfsys@transformshift{3.045978in}{-1.338289in}%
\pgfsys@useobject{currentmarker}{}%
\end{pgfscope}%
\begin{pgfscope}%
\pgfsys@transformshift{3.134630in}{-0.341430in}%
\pgfsys@useobject{currentmarker}{}%
\end{pgfscope}%
\begin{pgfscope}%
\pgfsys@transformshift{3.220674in}{0.571459in}%
\pgfsys@useobject{currentmarker}{}%
\end{pgfscope}%
\begin{pgfscope}%
\pgfsys@transformshift{3.306719in}{1.077593in}%
\pgfsys@useobject{currentmarker}{}%
\end{pgfscope}%
\begin{pgfscope}%
\pgfsys@transformshift{3.392763in}{1.533652in}%
\pgfsys@useobject{currentmarker}{}%
\end{pgfscope}%
\begin{pgfscope}%
\pgfsys@transformshift{3.478807in}{1.809444in}%
\pgfsys@useobject{currentmarker}{}%
\end{pgfscope}%
\begin{pgfscope}%
\pgfsys@transformshift{3.564852in}{1.806363in}%
\pgfsys@useobject{currentmarker}{}%
\end{pgfscope}%
\begin{pgfscope}%
\pgfsys@transformshift{3.650896in}{1.974304in}%
\pgfsys@useobject{currentmarker}{}%
\end{pgfscope}%
\begin{pgfscope}%
\pgfsys@transformshift{3.736941in}{2.037474in}%
\pgfsys@useobject{currentmarker}{}%
\end{pgfscope}%
\begin{pgfscope}%
\pgfsys@transformshift{3.822985in}{2.171519in}%
\pgfsys@useobject{currentmarker}{}%
\end{pgfscope}%
\begin{pgfscope}%
\pgfsys@transformshift{3.909030in}{2.200022in}%
\pgfsys@useobject{currentmarker}{}%
\end{pgfscope}%
\begin{pgfscope}%
\pgfsys@transformshift{3.995074in}{2.223904in}%
\pgfsys@useobject{currentmarker}{}%
\end{pgfscope}%
\begin{pgfscope}%
\pgfsys@transformshift{4.081119in}{2.176911in}%
\pgfsys@useobject{currentmarker}{}%
\end{pgfscope}%
\begin{pgfscope}%
\pgfsys@transformshift{4.167163in}{2.129148in}%
\pgfsys@useobject{currentmarker}{}%
\end{pgfscope}%
\begin{pgfscope}%
\pgfsys@transformshift{4.253207in}{2.200022in}%
\pgfsys@useobject{currentmarker}{}%
\end{pgfscope}%
\begin{pgfscope}%
\pgfsys@transformshift{4.339252in}{2.216200in}%
\pgfsys@useobject{currentmarker}{}%
\end{pgfscope}%
\begin{pgfscope}%
\pgfsys@transformshift{4.425296in}{2.220822in}%
\pgfsys@useobject{currentmarker}{}%
\end{pgfscope}%
\begin{pgfscope}%
\pgfsys@transformshift{4.511341in}{2.265504in}%
\pgfsys@useobject{currentmarker}{}%
\end{pgfscope}%
\begin{pgfscope}%
\pgfsys@transformshift{4.597385in}{2.257800in}%
\pgfsys@useobject{currentmarker}{}%
\end{pgfscope}%
\begin{pgfscope}%
\pgfsys@transformshift{4.683430in}{2.244704in}%
\pgfsys@useobject{currentmarker}{}%
\end{pgfscope}%
\begin{pgfscope}%
\pgfsys@transformshift{4.769474in}{2.134541in}%
\pgfsys@useobject{currentmarker}{}%
\end{pgfscope}%
\end{pgfscope}%
\begin{pgfscope}%
\pgfpathrectangle{\pgfqpoint{3.069444in}{0.769444in}}{\pgfqpoint{1.825185in}{2.311111in}}%
\pgfusepath{clip}%
\pgfsetrectcap%
\pgfsetroundjoin%
\pgfsetlinewidth{0.752812pt}%
\definecolor{currentstroke}{rgb}{0.000000,0.501961,0.000000}%
\pgfsetstrokecolor{currentstroke}%
\pgfsetdash{}{0pt}%
\pgfpathmoveto{\pgfqpoint{3.177467in}{0.759444in}}%
\pgfpathlineto{\pgfqpoint{3.218067in}{1.190067in}}%
\pgfpathlineto{\pgfqpoint{3.347133in}{2.050570in}}%
\pgfpathlineto{\pgfqpoint{3.476200in}{2.323281in}}%
\pgfpathlineto{\pgfqpoint{3.605267in}{2.543607in}}%
\pgfpathlineto{\pgfqpoint{3.734333in}{2.575193in}}%
\pgfpathlineto{\pgfqpoint{3.863400in}{2.706156in}}%
\pgfpathlineto{\pgfqpoint{3.992467in}{2.829415in}}%
\pgfpathlineto{\pgfqpoint{4.121533in}{2.866393in}}%
\pgfpathlineto{\pgfqpoint{4.250600in}{2.819400in}}%
\pgfpathlineto{\pgfqpoint{4.379667in}{2.837889in}}%
\pgfpathlineto{\pgfqpoint{4.508733in}{2.900289in}}%
\pgfpathlineto{\pgfqpoint{4.637800in}{2.894896in}}%
\pgfpathlineto{\pgfqpoint{4.765563in}{2.961148in}}%
\pgfpathlineto{\pgfqpoint{4.894630in}{2.968852in}}%
\pgfpathlineto{\pgfqpoint{4.904630in}{2.965808in}}%
\pgfusepath{stroke}%
\end{pgfscope}%
\begin{pgfscope}%
\pgfpathrectangle{\pgfqpoint{3.069444in}{0.769444in}}{\pgfqpoint{1.825185in}{2.311111in}}%
\pgfusepath{clip}%
\pgfsetbuttcap%
\pgfsetroundjoin%
\definecolor{currentfill}{rgb}{0.000000,0.501961,0.000000}%
\pgfsetfillcolor{currentfill}%
\pgfsetlinewidth{1.003750pt}%
\definecolor{currentstroke}{rgb}{0.000000,0.501961,0.000000}%
\pgfsetstrokecolor{currentstroke}%
\pgfsetdash{}{0pt}%
\pgfsys@defobject{currentmarker}{\pgfqpoint{-0.010417in}{-0.010417in}}{\pgfqpoint{0.010417in}{0.010417in}}{%
\pgfpathmoveto{\pgfqpoint{0.000000in}{-0.010417in}}%
\pgfpathcurveto{\pgfqpoint{0.002763in}{-0.010417in}}{\pgfqpoint{0.005412in}{-0.009319in}}{\pgfqpoint{0.007366in}{-0.007366in}}%
\pgfpathcurveto{\pgfqpoint{0.009319in}{-0.005412in}}{\pgfqpoint{0.010417in}{-0.002763in}}{\pgfqpoint{0.010417in}{0.000000in}}%
\pgfpathcurveto{\pgfqpoint{0.010417in}{0.002763in}}{\pgfqpoint{0.009319in}{0.005412in}}{\pgfqpoint{0.007366in}{0.007366in}}%
\pgfpathcurveto{\pgfqpoint{0.005412in}{0.009319in}}{\pgfqpoint{0.002763in}{0.010417in}}{\pgfqpoint{0.000000in}{0.010417in}}%
\pgfpathcurveto{\pgfqpoint{-0.002763in}{0.010417in}}{\pgfqpoint{-0.005412in}{0.009319in}}{\pgfqpoint{-0.007366in}{0.007366in}}%
\pgfpathcurveto{\pgfqpoint{-0.009319in}{0.005412in}}{\pgfqpoint{-0.010417in}{0.002763in}}{\pgfqpoint{-0.010417in}{0.000000in}}%
\pgfpathcurveto{\pgfqpoint{-0.010417in}{-0.002763in}}{\pgfqpoint{-0.009319in}{-0.005412in}}{\pgfqpoint{-0.007366in}{-0.007366in}}%
\pgfpathcurveto{\pgfqpoint{-0.005412in}{-0.009319in}}{\pgfqpoint{-0.002763in}{-0.010417in}}{\pgfqpoint{0.000000in}{-0.010417in}}%
\pgfpathclose%
\pgfusepath{stroke,fill}%
}%
\begin{pgfscope}%
\pgfsys@transformshift{3.089000in}{-0.178881in}%
\pgfsys@useobject{currentmarker}{}%
\end{pgfscope}%
\begin{pgfscope}%
\pgfsys@transformshift{3.218067in}{1.190067in}%
\pgfsys@useobject{currentmarker}{}%
\end{pgfscope}%
\begin{pgfscope}%
\pgfsys@transformshift{3.347133in}{2.050570in}%
\pgfsys@useobject{currentmarker}{}%
\end{pgfscope}%
\begin{pgfscope}%
\pgfsys@transformshift{3.476200in}{2.323281in}%
\pgfsys@useobject{currentmarker}{}%
\end{pgfscope}%
\begin{pgfscope}%
\pgfsys@transformshift{3.605267in}{2.543607in}%
\pgfsys@useobject{currentmarker}{}%
\end{pgfscope}%
\begin{pgfscope}%
\pgfsys@transformshift{3.734333in}{2.575193in}%
\pgfsys@useobject{currentmarker}{}%
\end{pgfscope}%
\begin{pgfscope}%
\pgfsys@transformshift{3.863400in}{2.706156in}%
\pgfsys@useobject{currentmarker}{}%
\end{pgfscope}%
\begin{pgfscope}%
\pgfsys@transformshift{3.992467in}{2.829415in}%
\pgfsys@useobject{currentmarker}{}%
\end{pgfscope}%
\begin{pgfscope}%
\pgfsys@transformshift{4.121533in}{2.866393in}%
\pgfsys@useobject{currentmarker}{}%
\end{pgfscope}%
\begin{pgfscope}%
\pgfsys@transformshift{4.250600in}{2.819400in}%
\pgfsys@useobject{currentmarker}{}%
\end{pgfscope}%
\begin{pgfscope}%
\pgfsys@transformshift{4.379667in}{2.837889in}%
\pgfsys@useobject{currentmarker}{}%
\end{pgfscope}%
\begin{pgfscope}%
\pgfsys@transformshift{4.508733in}{2.900289in}%
\pgfsys@useobject{currentmarker}{}%
\end{pgfscope}%
\begin{pgfscope}%
\pgfsys@transformshift{4.637800in}{2.894896in}%
\pgfsys@useobject{currentmarker}{}%
\end{pgfscope}%
\begin{pgfscope}%
\pgfsys@transformshift{4.765563in}{2.961148in}%
\pgfsys@useobject{currentmarker}{}%
\end{pgfscope}%
\begin{pgfscope}%
\pgfsys@transformshift{4.894630in}{2.968852in}%
\pgfsys@useobject{currentmarker}{}%
\end{pgfscope}%
\begin{pgfscope}%
\pgfsys@transformshift{5.023696in}{2.929563in}%
\pgfsys@useobject{currentmarker}{}%
\end{pgfscope}%
\begin{pgfscope}%
\pgfsys@transformshift{5.152763in}{2.942659in}%
\pgfsys@useobject{currentmarker}{}%
\end{pgfscope}%
\begin{pgfscope}%
\pgfsys@transformshift{5.281830in}{2.926481in}%
\pgfsys@useobject{currentmarker}{}%
\end{pgfscope}%
\begin{pgfscope}%
\pgfsys@transformshift{5.410896in}{2.981948in}%
\pgfsys@useobject{currentmarker}{}%
\end{pgfscope}%
\begin{pgfscope}%
\pgfsys@transformshift{5.539963in}{2.958067in}%
\pgfsys@useobject{currentmarker}{}%
\end{pgfscope}%
\begin{pgfscope}%
\pgfsys@transformshift{5.669030in}{2.948052in}%
\pgfsys@useobject{currentmarker}{}%
\end{pgfscope}%
\end{pgfscope}%
\begin{pgfscope}%
\pgfpathrectangle{\pgfqpoint{3.069444in}{0.769444in}}{\pgfqpoint{1.825185in}{2.311111in}}%
\pgfusepath{clip}%
\pgfsetrectcap%
\pgfsetroundjoin%
\pgfsetlinewidth{0.752812pt}%
\definecolor{currentstroke}{rgb}{1.000000,0.000000,0.000000}%
\pgfsetstrokecolor{currentstroke}%
\pgfsetdash{}{0pt}%
\pgfpathmoveto{\pgfqpoint{3.176769in}{0.759444in}}%
\pgfpathlineto{\pgfqpoint{3.306719in}{1.927311in}}%
\pgfpathlineto{\pgfqpoint{3.478807in}{2.438837in}}%
\pgfpathlineto{\pgfqpoint{3.650896in}{2.593681in}}%
\pgfpathlineto{\pgfqpoint{3.822985in}{2.837889in}}%
\pgfpathlineto{\pgfqpoint{3.995074in}{2.937267in}}%
\pgfpathlineto{\pgfqpoint{4.167163in}{2.948052in}}%
\pgfpathlineto{\pgfqpoint{4.300916in}{3.090556in}}%
\pgfusepath{stroke}%
\end{pgfscope}%
\begin{pgfscope}%
\pgfpathrectangle{\pgfqpoint{3.069444in}{0.769444in}}{\pgfqpoint{1.825185in}{2.311111in}}%
\pgfusepath{clip}%
\pgfsetbuttcap%
\pgfsetroundjoin%
\definecolor{currentfill}{rgb}{1.000000,0.000000,0.000000}%
\pgfsetfillcolor{currentfill}%
\pgfsetlinewidth{1.003750pt}%
\definecolor{currentstroke}{rgb}{1.000000,0.000000,0.000000}%
\pgfsetstrokecolor{currentstroke}%
\pgfsetdash{}{0pt}%
\pgfsys@defobject{currentmarker}{\pgfqpoint{-0.010417in}{-0.010417in}}{\pgfqpoint{0.010417in}{0.010417in}}{%
\pgfpathmoveto{\pgfqpoint{0.000000in}{-0.010417in}}%
\pgfpathcurveto{\pgfqpoint{0.002763in}{-0.010417in}}{\pgfqpoint{0.005412in}{-0.009319in}}{\pgfqpoint{0.007366in}{-0.007366in}}%
\pgfpathcurveto{\pgfqpoint{0.009319in}{-0.005412in}}{\pgfqpoint{0.010417in}{-0.002763in}}{\pgfqpoint{0.010417in}{0.000000in}}%
\pgfpathcurveto{\pgfqpoint{0.010417in}{0.002763in}}{\pgfqpoint{0.009319in}{0.005412in}}{\pgfqpoint{0.007366in}{0.007366in}}%
\pgfpathcurveto{\pgfqpoint{0.005412in}{0.009319in}}{\pgfqpoint{0.002763in}{0.010417in}}{\pgfqpoint{0.000000in}{0.010417in}}%
\pgfpathcurveto{\pgfqpoint{-0.002763in}{0.010417in}}{\pgfqpoint{-0.005412in}{0.009319in}}{\pgfqpoint{-0.007366in}{0.007366in}}%
\pgfpathcurveto{\pgfqpoint{-0.009319in}{0.005412in}}{\pgfqpoint{-0.010417in}{0.002763in}}{\pgfqpoint{-0.010417in}{0.000000in}}%
\pgfpathcurveto{\pgfqpoint{-0.010417in}{-0.002763in}}{\pgfqpoint{-0.009319in}{-0.005412in}}{\pgfqpoint{-0.007366in}{-0.007366in}}%
\pgfpathcurveto{\pgfqpoint{-0.005412in}{-0.009319in}}{\pgfqpoint{-0.002763in}{-0.010417in}}{\pgfqpoint{0.000000in}{-0.010417in}}%
\pgfpathclose%
\pgfusepath{stroke,fill}%
}%
\begin{pgfscope}%
\pgfsys@transformshift{3.132022in}{0.357296in}%
\pgfsys@useobject{currentmarker}{}%
\end{pgfscope}%
\begin{pgfscope}%
\pgfsys@transformshift{3.306719in}{1.927311in}%
\pgfsys@useobject{currentmarker}{}%
\end{pgfscope}%
\begin{pgfscope}%
\pgfsys@transformshift{3.478807in}{2.438837in}%
\pgfsys@useobject{currentmarker}{}%
\end{pgfscope}%
\begin{pgfscope}%
\pgfsys@transformshift{3.650896in}{2.593681in}%
\pgfsys@useobject{currentmarker}{}%
\end{pgfscope}%
\begin{pgfscope}%
\pgfsys@transformshift{3.822985in}{2.837889in}%
\pgfsys@useobject{currentmarker}{}%
\end{pgfscope}%
\begin{pgfscope}%
\pgfsys@transformshift{3.995074in}{2.937267in}%
\pgfsys@useobject{currentmarker}{}%
\end{pgfscope}%
\begin{pgfscope}%
\pgfsys@transformshift{4.167163in}{2.948052in}%
\pgfsys@useobject{currentmarker}{}%
\end{pgfscope}%
\begin{pgfscope}%
\pgfsys@transformshift{4.339252in}{3.131400in}%
\pgfsys@useobject{currentmarker}{}%
\end{pgfscope}%
\begin{pgfscope}%
\pgfsys@transformshift{4.511341in}{3.157593in}%
\pgfsys@useobject{currentmarker}{}%
\end{pgfscope}%
\begin{pgfscope}%
\pgfsys@transformshift{4.683430in}{3.118304in}%
\pgfsys@useobject{currentmarker}{}%
\end{pgfscope}%
\begin{pgfscope}%
\pgfsys@transformshift{4.855519in}{3.183785in}%
\pgfsys@useobject{currentmarker}{}%
\end{pgfscope}%
\begin{pgfscope}%
\pgfsys@transformshift{5.027607in}{3.168378in}%
\pgfsys@useobject{currentmarker}{}%
\end{pgfscope}%
\begin{pgfscope}%
\pgfsys@transformshift{5.199696in}{3.159904in}%
\pgfsys@useobject{currentmarker}{}%
\end{pgfscope}%
\begin{pgfscope}%
\pgfsys@transformshift{5.371785in}{3.181474in}%
\pgfsys@useobject{currentmarker}{}%
\end{pgfscope}%
\begin{pgfscope}%
\pgfsys@transformshift{5.543874in}{3.170689in}%
\pgfsys@useobject{currentmarker}{}%
\end{pgfscope}%
\begin{pgfscope}%
\pgfsys@transformshift{5.715963in}{3.260052in}%
\pgfsys@useobject{currentmarker}{}%
\end{pgfscope}%
\begin{pgfscope}%
\pgfsys@transformshift{5.888052in}{3.157593in}%
\pgfsys@useobject{currentmarker}{}%
\end{pgfscope}%
\begin{pgfscope}%
\pgfsys@transformshift{6.060141in}{3.225385in}%
\pgfsys@useobject{currentmarker}{}%
\end{pgfscope}%
\begin{pgfscope}%
\pgfsys@transformshift{6.232230in}{3.186096in}%
\pgfsys@useobject{currentmarker}{}%
\end{pgfscope}%
\begin{pgfscope}%
\pgfsys@transformshift{6.404319in}{3.170689in}%
\pgfsys@useobject{currentmarker}{}%
\end{pgfscope}%
\begin{pgfscope}%
\pgfsys@transformshift{6.576407in}{3.186096in}%
\pgfsys@useobject{currentmarker}{}%
\end{pgfscope}%
\end{pgfscope}%
\begin{pgfscope}%
\pgfpathrectangle{\pgfqpoint{3.069444in}{0.769444in}}{\pgfqpoint{1.825185in}{2.311111in}}%
\pgfusepath{clip}%
\pgfsetrectcap%
\pgfsetroundjoin%
\pgfsetlinewidth{0.752812pt}%
\definecolor{currentstroke}{rgb}{0.501961,0.000000,0.501961}%
\pgfsetstrokecolor{currentstroke}%
\pgfsetdash{}{0pt}%
\pgfpathmoveto{\pgfqpoint{3.175044in}{1.202393in}}%
\pgfpathlineto{\pgfqpoint{3.390156in}{2.388763in}}%
\pgfpathlineto{\pgfqpoint{3.606570in}{2.585978in}}%
\pgfpathlineto{\pgfqpoint{3.821681in}{2.837889in}}%
\pgfpathlineto{\pgfqpoint{4.015507in}{3.090556in}}%
\pgfpathmoveto{\pgfqpoint{4.103587in}{3.090556in}}%
\pgfpathlineto{\pgfqpoint{4.251904in}{3.028941in}}%
\pgfpathlineto{\pgfqpoint{4.354926in}{3.090556in}}%
\pgfusepath{stroke}%
\end{pgfscope}%
\begin{pgfscope}%
\pgfpathrectangle{\pgfqpoint{3.069444in}{0.769444in}}{\pgfqpoint{1.825185in}{2.311111in}}%
\pgfusepath{clip}%
\pgfsetbuttcap%
\pgfsetroundjoin%
\definecolor{currentfill}{rgb}{0.501961,0.000000,0.501961}%
\pgfsetfillcolor{currentfill}%
\pgfsetlinewidth{1.003750pt}%
\definecolor{currentstroke}{rgb}{0.501961,0.000000,0.501961}%
\pgfsetstrokecolor{currentstroke}%
\pgfsetdash{}{0pt}%
\pgfsys@defobject{currentmarker}{\pgfqpoint{-0.010417in}{-0.010417in}}{\pgfqpoint{0.010417in}{0.010417in}}{%
\pgfpathmoveto{\pgfqpoint{0.000000in}{-0.010417in}}%
\pgfpathcurveto{\pgfqpoint{0.002763in}{-0.010417in}}{\pgfqpoint{0.005412in}{-0.009319in}}{\pgfqpoint{0.007366in}{-0.007366in}}%
\pgfpathcurveto{\pgfqpoint{0.009319in}{-0.005412in}}{\pgfqpoint{0.010417in}{-0.002763in}}{\pgfqpoint{0.010417in}{0.000000in}}%
\pgfpathcurveto{\pgfqpoint{0.010417in}{0.002763in}}{\pgfqpoint{0.009319in}{0.005412in}}{\pgfqpoint{0.007366in}{0.007366in}}%
\pgfpathcurveto{\pgfqpoint{0.005412in}{0.009319in}}{\pgfqpoint{0.002763in}{0.010417in}}{\pgfqpoint{0.000000in}{0.010417in}}%
\pgfpathcurveto{\pgfqpoint{-0.002763in}{0.010417in}}{\pgfqpoint{-0.005412in}{0.009319in}}{\pgfqpoint{-0.007366in}{0.007366in}}%
\pgfpathcurveto{\pgfqpoint{-0.009319in}{0.005412in}}{\pgfqpoint{-0.010417in}{0.002763in}}{\pgfqpoint{-0.010417in}{0.000000in}}%
\pgfpathcurveto{\pgfqpoint{-0.010417in}{-0.002763in}}{\pgfqpoint{-0.009319in}{-0.005412in}}{\pgfqpoint{-0.007366in}{-0.007366in}}%
\pgfpathcurveto{\pgfqpoint{-0.005412in}{-0.009319in}}{\pgfqpoint{-0.002763in}{-0.010417in}}{\pgfqpoint{0.000000in}{-0.010417in}}%
\pgfpathclose%
\pgfusepath{stroke,fill}%
}%
\begin{pgfscope}%
\pgfsys@transformshift{3.175044in}{1.202393in}%
\pgfsys@useobject{currentmarker}{}%
\end{pgfscope}%
\begin{pgfscope}%
\pgfsys@transformshift{3.390156in}{2.388763in}%
\pgfsys@useobject{currentmarker}{}%
\end{pgfscope}%
\begin{pgfscope}%
\pgfsys@transformshift{3.606570in}{2.585978in}%
\pgfsys@useobject{currentmarker}{}%
\end{pgfscope}%
\begin{pgfscope}%
\pgfsys@transformshift{3.821681in}{2.837889in}%
\pgfsys@useobject{currentmarker}{}%
\end{pgfscope}%
\begin{pgfscope}%
\pgfsys@transformshift{4.036793in}{3.118304in}%
\pgfsys@useobject{currentmarker}{}%
\end{pgfscope}%
\begin{pgfscope}%
\pgfsys@transformshift{4.251904in}{3.028941in}%
\pgfsys@useobject{currentmarker}{}%
\end{pgfscope}%
\begin{pgfscope}%
\pgfsys@transformshift{4.467015in}{3.157593in}%
\pgfsys@useobject{currentmarker}{}%
\end{pgfscope}%
\begin{pgfscope}%
\pgfsys@transformshift{4.682126in}{3.288556in}%
\pgfsys@useobject{currentmarker}{}%
\end{pgfscope}%
\begin{pgfscope}%
\pgfsys@transformshift{4.897237in}{3.299341in}%
\pgfsys@useobject{currentmarker}{}%
\end{pgfscope}%
\begin{pgfscope}%
\pgfsys@transformshift{5.112348in}{3.286244in}%
\pgfsys@useobject{currentmarker}{}%
\end{pgfscope}%
\begin{pgfscope}%
\pgfsys@transformshift{5.327459in}{3.252348in}%
\pgfsys@useobject{currentmarker}{}%
\end{pgfscope}%
\begin{pgfscope}%
\pgfsys@transformshift{5.542570in}{3.327844in}%
\pgfsys@useobject{currentmarker}{}%
\end{pgfscope}%
\begin{pgfscope}%
\pgfsys@transformshift{5.757681in}{3.267756in}%
\pgfsys@useobject{currentmarker}{}%
\end{pgfscope}%
\begin{pgfscope}%
\pgfsys@transformshift{5.972793in}{3.273148in}%
\pgfsys@useobject{currentmarker}{}%
\end{pgfscope}%
\begin{pgfscope}%
\pgfsys@transformshift{6.187904in}{3.346333in}%
\pgfsys@useobject{currentmarker}{}%
\end{pgfscope}%
\begin{pgfscope}%
\pgfsys@transformshift{6.403015in}{3.307044in}%
\pgfsys@useobject{currentmarker}{}%
\end{pgfscope}%
\begin{pgfscope}%
\pgfsys@transformshift{6.618126in}{3.327844in}%
\pgfsys@useobject{currentmarker}{}%
\end{pgfscope}%
\begin{pgfscope}%
\pgfsys@transformshift{6.833237in}{3.349415in}%
\pgfsys@useobject{currentmarker}{}%
\end{pgfscope}%
\begin{pgfscope}%
\pgfsys@transformshift{7.048348in}{3.293948in}%
\pgfsys@useobject{currentmarker}{}%
\end{pgfscope}%
\begin{pgfscope}%
\pgfsys@transformshift{7.264763in}{3.296259in}%
\pgfsys@useobject{currentmarker}{}%
\end{pgfscope}%
\begin{pgfscope}%
\pgfsys@transformshift{7.479874in}{3.401800in}%
\pgfsys@useobject{currentmarker}{}%
\end{pgfscope}%
\end{pgfscope}%
\begin{pgfscope}%
\pgfpathrectangle{\pgfqpoint{3.069444in}{0.769444in}}{\pgfqpoint{1.825185in}{2.311111in}}%
\pgfusepath{clip}%
\pgfsetrectcap%
\pgfsetroundjoin%
\pgfsetlinewidth{0.752812pt}%
\definecolor{currentstroke}{rgb}{0.392157,0.584314,0.929412}%
\pgfsetstrokecolor{currentstroke}%
\pgfsetdash{}{0pt}%
\pgfpathmoveto{\pgfqpoint{3.216763in}{1.587578in}}%
\pgfpathlineto{\pgfqpoint{3.477504in}{2.674570in}}%
\pgfpathlineto{\pgfqpoint{3.735637in}{2.918778in}}%
\pgfpathlineto{\pgfqpoint{3.993770in}{2.978867in}}%
\pgfpathlineto{\pgfqpoint{4.093569in}{3.090556in}}%
\pgfpathmoveto{\pgfqpoint{4.494253in}{3.090556in}}%
\pgfpathlineto{\pgfqpoint{4.510037in}{3.079015in}}%
\pgfpathlineto{\pgfqpoint{4.521278in}{3.090556in}}%
\pgfusepath{stroke}%
\end{pgfscope}%
\begin{pgfscope}%
\pgfpathrectangle{\pgfqpoint{3.069444in}{0.769444in}}{\pgfqpoint{1.825185in}{2.311111in}}%
\pgfusepath{clip}%
\pgfsetbuttcap%
\pgfsetroundjoin%
\definecolor{currentfill}{rgb}{0.392157,0.584314,0.929412}%
\pgfsetfillcolor{currentfill}%
\pgfsetlinewidth{1.003750pt}%
\definecolor{currentstroke}{rgb}{0.392157,0.584314,0.929412}%
\pgfsetstrokecolor{currentstroke}%
\pgfsetdash{}{0pt}%
\pgfsys@defobject{currentmarker}{\pgfqpoint{-0.010417in}{-0.010417in}}{\pgfqpoint{0.010417in}{0.010417in}}{%
\pgfpathmoveto{\pgfqpoint{0.000000in}{-0.010417in}}%
\pgfpathcurveto{\pgfqpoint{0.002763in}{-0.010417in}}{\pgfqpoint{0.005412in}{-0.009319in}}{\pgfqpoint{0.007366in}{-0.007366in}}%
\pgfpathcurveto{\pgfqpoint{0.009319in}{-0.005412in}}{\pgfqpoint{0.010417in}{-0.002763in}}{\pgfqpoint{0.010417in}{0.000000in}}%
\pgfpathcurveto{\pgfqpoint{0.010417in}{0.002763in}}{\pgfqpoint{0.009319in}{0.005412in}}{\pgfqpoint{0.007366in}{0.007366in}}%
\pgfpathcurveto{\pgfqpoint{0.005412in}{0.009319in}}{\pgfqpoint{0.002763in}{0.010417in}}{\pgfqpoint{0.000000in}{0.010417in}}%
\pgfpathcurveto{\pgfqpoint{-0.002763in}{0.010417in}}{\pgfqpoint{-0.005412in}{0.009319in}}{\pgfqpoint{-0.007366in}{0.007366in}}%
\pgfpathcurveto{\pgfqpoint{-0.009319in}{0.005412in}}{\pgfqpoint{-0.010417in}{0.002763in}}{\pgfqpoint{-0.010417in}{0.000000in}}%
\pgfpathcurveto{\pgfqpoint{-0.010417in}{-0.002763in}}{\pgfqpoint{-0.009319in}{-0.005412in}}{\pgfqpoint{-0.007366in}{-0.007366in}}%
\pgfpathcurveto{\pgfqpoint{-0.005412in}{-0.009319in}}{\pgfqpoint{-0.002763in}{-0.010417in}}{\pgfqpoint{0.000000in}{-0.010417in}}%
\pgfpathclose%
\pgfusepath{stroke,fill}%
}%
\begin{pgfscope}%
\pgfsys@transformshift{3.216763in}{1.587578in}%
\pgfsys@useobject{currentmarker}{}%
\end{pgfscope}%
\begin{pgfscope}%
\pgfsys@transformshift{3.477504in}{2.674570in}%
\pgfsys@useobject{currentmarker}{}%
\end{pgfscope}%
\begin{pgfscope}%
\pgfsys@transformshift{3.735637in}{2.918778in}%
\pgfsys@useobject{currentmarker}{}%
\end{pgfscope}%
\begin{pgfscope}%
\pgfsys@transformshift{3.993770in}{2.978867in}%
\pgfsys@useobject{currentmarker}{}%
\end{pgfscope}%
\begin{pgfscope}%
\pgfsys@transformshift{4.251904in}{3.267756in}%
\pgfsys@useobject{currentmarker}{}%
\end{pgfscope}%
\begin{pgfscope}%
\pgfsys@transformshift{4.510037in}{3.079015in}%
\pgfsys@useobject{currentmarker}{}%
\end{pgfscope}%
\begin{pgfscope}%
\pgfsys@transformshift{4.768170in}{3.344022in}%
\pgfsys@useobject{currentmarker}{}%
\end{pgfscope}%
\begin{pgfscope}%
\pgfsys@transformshift{5.026304in}{3.488081in}%
\pgfsys@useobject{currentmarker}{}%
\end{pgfscope}%
\begin{pgfscope}%
\pgfsys@transformshift{5.284437in}{3.535074in}%
\pgfsys@useobject{currentmarker}{}%
\end{pgfscope}%
\begin{pgfscope}%
\pgfsys@transformshift{5.542570in}{3.445711in}%
\pgfsys@useobject{currentmarker}{}%
\end{pgfscope}%
\begin{pgfscope}%
\pgfsys@transformshift{5.800704in}{3.498867in}%
\pgfsys@useobject{currentmarker}{}%
\end{pgfscope}%
\begin{pgfscope}%
\pgfsys@transformshift{6.058837in}{3.501178in}%
\pgfsys@useobject{currentmarker}{}%
\end{pgfscope}%
\begin{pgfscope}%
\pgfsys@transformshift{6.316970in}{3.454185in}%
\pgfsys@useobject{currentmarker}{}%
\end{pgfscope}%
\begin{pgfscope}%
\pgfsys@transformshift{6.575104in}{3.529681in}%
\pgfsys@useobject{currentmarker}{}%
\end{pgfscope}%
\begin{pgfscope}%
\pgfsys@transformshift{6.833237in}{3.529681in}%
\pgfsys@useobject{currentmarker}{}%
\end{pgfscope}%
\begin{pgfscope}%
\pgfsys@transformshift{7.091370in}{3.503489in}%
\pgfsys@useobject{currentmarker}{}%
\end{pgfscope}%
\begin{pgfscope}%
\pgfsys@transformshift{7.349504in}{3.482689in}%
\pgfsys@useobject{currentmarker}{}%
\end{pgfscope}%
\begin{pgfscope}%
\pgfsys@transformshift{7.607637in}{3.574363in}%
\pgfsys@useobject{currentmarker}{}%
\end{pgfscope}%
\begin{pgfscope}%
\pgfsys@transformshift{7.865770in}{3.521978in}%
\pgfsys@useobject{currentmarker}{}%
\end{pgfscope}%
\begin{pgfscope}%
\pgfsys@transformshift{8.123904in}{3.480378in}%
\pgfsys@useobject{currentmarker}{}%
\end{pgfscope}%
\begin{pgfscope}%
\pgfsys@transformshift{8.382037in}{3.548170in}%
\pgfsys@useobject{currentmarker}{}%
\end{pgfscope}%
\end{pgfscope}%
\begin{pgfscope}%
\pgfpathrectangle{\pgfqpoint{3.069444in}{0.769444in}}{\pgfqpoint{1.825185in}{2.311111in}}%
\pgfusepath{clip}%
\pgfsetrectcap%
\pgfsetroundjoin%
\pgfsetlinewidth{0.752812pt}%
\definecolor{currentstroke}{rgb}{1.000000,0.752941,0.796078}%
\pgfsetstrokecolor{currentstroke}%
\pgfsetdash{}{0pt}%
\pgfpathmoveto{\pgfqpoint{3.262393in}{1.835637in}}%
\pgfpathlineto{\pgfqpoint{3.564852in}{2.498926in}}%
\pgfpathlineto{\pgfqpoint{3.850212in}{3.090556in}}%
\pgfusepath{stroke}%
\end{pgfscope}%
\begin{pgfscope}%
\pgfpathrectangle{\pgfqpoint{3.069444in}{0.769444in}}{\pgfqpoint{1.825185in}{2.311111in}}%
\pgfusepath{clip}%
\pgfsetbuttcap%
\pgfsetroundjoin%
\definecolor{currentfill}{rgb}{1.000000,0.752941,0.796078}%
\pgfsetfillcolor{currentfill}%
\pgfsetlinewidth{1.003750pt}%
\definecolor{currentstroke}{rgb}{1.000000,0.752941,0.796078}%
\pgfsetstrokecolor{currentstroke}%
\pgfsetdash{}{0pt}%
\pgfsys@defobject{currentmarker}{\pgfqpoint{-0.010417in}{-0.010417in}}{\pgfqpoint{0.010417in}{0.010417in}}{%
\pgfpathmoveto{\pgfqpoint{0.000000in}{-0.010417in}}%
\pgfpathcurveto{\pgfqpoint{0.002763in}{-0.010417in}}{\pgfqpoint{0.005412in}{-0.009319in}}{\pgfqpoint{0.007366in}{-0.007366in}}%
\pgfpathcurveto{\pgfqpoint{0.009319in}{-0.005412in}}{\pgfqpoint{0.010417in}{-0.002763in}}{\pgfqpoint{0.010417in}{0.000000in}}%
\pgfpathcurveto{\pgfqpoint{0.010417in}{0.002763in}}{\pgfqpoint{0.009319in}{0.005412in}}{\pgfqpoint{0.007366in}{0.007366in}}%
\pgfpathcurveto{\pgfqpoint{0.005412in}{0.009319in}}{\pgfqpoint{0.002763in}{0.010417in}}{\pgfqpoint{0.000000in}{0.010417in}}%
\pgfpathcurveto{\pgfqpoint{-0.002763in}{0.010417in}}{\pgfqpoint{-0.005412in}{0.009319in}}{\pgfqpoint{-0.007366in}{0.007366in}}%
\pgfpathcurveto{\pgfqpoint{-0.009319in}{0.005412in}}{\pgfqpoint{-0.010417in}{0.002763in}}{\pgfqpoint{-0.010417in}{0.000000in}}%
\pgfpathcurveto{\pgfqpoint{-0.010417in}{-0.002763in}}{\pgfqpoint{-0.009319in}{-0.005412in}}{\pgfqpoint{-0.007366in}{-0.007366in}}%
\pgfpathcurveto{\pgfqpoint{-0.005412in}{-0.009319in}}{\pgfqpoint{-0.002763in}{-0.010417in}}{\pgfqpoint{0.000000in}{-0.010417in}}%
\pgfpathclose%
\pgfusepath{stroke,fill}%
}%
\begin{pgfscope}%
\pgfsys@transformshift{3.262393in}{1.835637in}%
\pgfsys@useobject{currentmarker}{}%
\end{pgfscope}%
\begin{pgfscope}%
\pgfsys@transformshift{3.564852in}{2.498926in}%
\pgfsys@useobject{currentmarker}{}%
\end{pgfscope}%
\begin{pgfscope}%
\pgfsys@transformshift{3.867311in}{3.126007in}%
\pgfsys@useobject{currentmarker}{}%
\end{pgfscope}%
\begin{pgfscope}%
\pgfsys@transformshift{4.169770in}{3.249267in}%
\pgfsys@useobject{currentmarker}{}%
\end{pgfscope}%
\begin{pgfscope}%
\pgfsys@transformshift{4.472230in}{3.112911in}%
\pgfsys@useobject{currentmarker}{}%
\end{pgfscope}%
\begin{pgfscope}%
\pgfsys@transformshift{4.774689in}{3.204585in}%
\pgfsys@useobject{currentmarker}{}%
\end{pgfscope}%
\begin{pgfscope}%
\pgfsys@transformshift{5.077148in}{3.252348in}%
\pgfsys@useobject{currentmarker}{}%
\end{pgfscope}%
\begin{pgfscope}%
\pgfsys@transformshift{5.379607in}{3.461889in}%
\pgfsys@useobject{currentmarker}{}%
\end{pgfscope}%
\begin{pgfscope}%
\pgfsys@transformshift{5.682067in}{3.548170in}%
\pgfsys@useobject{currentmarker}{}%
\end{pgfscope}%
\begin{pgfscope}%
\pgfsys@transformshift{5.984526in}{3.542778in}%
\pgfsys@useobject{currentmarker}{}%
\end{pgfscope}%
\begin{pgfscope}%
\pgfsys@transformshift{6.286985in}{3.592852in}%
\pgfsys@useobject{currentmarker}{}%
\end{pgfscope}%
\begin{pgfscope}%
\pgfsys@transformshift{6.589444in}{3.540467in}%
\pgfsys@useobject{currentmarker}{}%
\end{pgfscope}%
\begin{pgfscope}%
\pgfsys@transformshift{6.891904in}{3.558956in}%
\pgfsys@useobject{currentmarker}{}%
\end{pgfscope}%
\begin{pgfscope}%
\pgfsys@transformshift{7.194363in}{3.624437in}%
\pgfsys@useobject{currentmarker}{}%
\end{pgfscope}%
\begin{pgfscope}%
\pgfsys@transformshift{7.496822in}{3.629830in}%
\pgfsys@useobject{currentmarker}{}%
\end{pgfscope}%
\begin{pgfscope}%
\pgfsys@transformshift{7.799281in}{3.642926in}%
\pgfsys@useobject{currentmarker}{}%
\end{pgfscope}%
\begin{pgfscope}%
\pgfsys@transformshift{8.101741in}{3.587459in}%
\pgfsys@useobject{currentmarker}{}%
\end{pgfscope}%
\begin{pgfscope}%
\pgfsys@transformshift{8.404200in}{3.598244in}%
\pgfsys@useobject{currentmarker}{}%
\end{pgfscope}%
\begin{pgfscope}%
\pgfsys@transformshift{8.706659in}{3.661415in}%
\pgfsys@useobject{currentmarker}{}%
\end{pgfscope}%
\begin{pgfscope}%
\pgfsys@transformshift{9.009119in}{3.577444in}%
\pgfsys@useobject{currentmarker}{}%
\end{pgfscope}%
\begin{pgfscope}%
\pgfsys@transformshift{9.311578in}{3.613652in}%
\pgfsys@useobject{currentmarker}{}%
\end{pgfscope}%
\end{pgfscope}%
\begin{pgfscope}%
\pgfpathrectangle{\pgfqpoint{3.069444in}{0.769444in}}{\pgfqpoint{1.825185in}{2.311111in}}%
\pgfusepath{clip}%
\pgfsetrectcap%
\pgfsetroundjoin%
\pgfsetlinewidth{0.752812pt}%
\definecolor{currentstroke}{rgb}{0.501961,0.501961,0.501961}%
\pgfsetstrokecolor{currentstroke}%
\pgfsetdash{}{0pt}%
\pgfpathmoveto{\pgfqpoint{3.309326in}{2.176141in}}%
\pgfpathlineto{\pgfqpoint{3.656111in}{2.827104in}}%
\pgfpathlineto{\pgfqpoint{3.998985in}{3.026630in}}%
\pgfpathlineto{\pgfqpoint{4.122622in}{3.090556in}}%
\pgfusepath{stroke}%
\end{pgfscope}%
\begin{pgfscope}%
\pgfpathrectangle{\pgfqpoint{3.069444in}{0.769444in}}{\pgfqpoint{1.825185in}{2.311111in}}%
\pgfusepath{clip}%
\pgfsetbuttcap%
\pgfsetroundjoin%
\definecolor{currentfill}{rgb}{0.501961,0.501961,0.501961}%
\pgfsetfillcolor{currentfill}%
\pgfsetlinewidth{1.003750pt}%
\definecolor{currentstroke}{rgb}{0.501961,0.501961,0.501961}%
\pgfsetstrokecolor{currentstroke}%
\pgfsetdash{}{0pt}%
\pgfsys@defobject{currentmarker}{\pgfqpoint{-0.010417in}{-0.010417in}}{\pgfqpoint{0.010417in}{0.010417in}}{%
\pgfpathmoveto{\pgfqpoint{0.000000in}{-0.010417in}}%
\pgfpathcurveto{\pgfqpoint{0.002763in}{-0.010417in}}{\pgfqpoint{0.005412in}{-0.009319in}}{\pgfqpoint{0.007366in}{-0.007366in}}%
\pgfpathcurveto{\pgfqpoint{0.009319in}{-0.005412in}}{\pgfqpoint{0.010417in}{-0.002763in}}{\pgfqpoint{0.010417in}{0.000000in}}%
\pgfpathcurveto{\pgfqpoint{0.010417in}{0.002763in}}{\pgfqpoint{0.009319in}{0.005412in}}{\pgfqpoint{0.007366in}{0.007366in}}%
\pgfpathcurveto{\pgfqpoint{0.005412in}{0.009319in}}{\pgfqpoint{0.002763in}{0.010417in}}{\pgfqpoint{0.000000in}{0.010417in}}%
\pgfpathcurveto{\pgfqpoint{-0.002763in}{0.010417in}}{\pgfqpoint{-0.005412in}{0.009319in}}{\pgfqpoint{-0.007366in}{0.007366in}}%
\pgfpathcurveto{\pgfqpoint{-0.009319in}{0.005412in}}{\pgfqpoint{-0.010417in}{0.002763in}}{\pgfqpoint{-0.010417in}{0.000000in}}%
\pgfpathcurveto{\pgfqpoint{-0.010417in}{-0.002763in}}{\pgfqpoint{-0.009319in}{-0.005412in}}{\pgfqpoint{-0.007366in}{-0.007366in}}%
\pgfpathcurveto{\pgfqpoint{-0.005412in}{-0.009319in}}{\pgfqpoint{-0.002763in}{-0.010417in}}{\pgfqpoint{0.000000in}{-0.010417in}}%
\pgfpathclose%
\pgfusepath{stroke,fill}%
}%
\begin{pgfscope}%
\pgfsys@transformshift{3.309326in}{2.176141in}%
\pgfsys@useobject{currentmarker}{}%
\end{pgfscope}%
\begin{pgfscope}%
\pgfsys@transformshift{3.656111in}{2.827104in}%
\pgfsys@useobject{currentmarker}{}%
\end{pgfscope}%
\begin{pgfscope}%
\pgfsys@transformshift{3.998985in}{3.026630in}%
\pgfsys@useobject{currentmarker}{}%
\end{pgfscope}%
\begin{pgfscope}%
\pgfsys@transformshift{4.343163in}{3.204585in}%
\pgfsys@useobject{currentmarker}{}%
\end{pgfscope}%
\begin{pgfscope}%
\pgfsys@transformshift{4.686037in}{3.314748in}%
\pgfsys@useobject{currentmarker}{}%
\end{pgfscope}%
\begin{pgfscope}%
\pgfsys@transformshift{5.030215in}{3.346333in}%
\pgfsys@useobject{currentmarker}{}%
\end{pgfscope}%
\begin{pgfscope}%
\pgfsys@transformshift{5.373089in}{3.280852in}%
\pgfsys@useobject{currentmarker}{}%
\end{pgfscope}%
\begin{pgfscope}%
\pgfsys@transformshift{5.715963in}{3.577444in}%
\pgfsys@useobject{currentmarker}{}%
\end{pgfscope}%
\begin{pgfscope}%
\pgfsys@transformshift{6.060141in}{3.598244in}%
\pgfsys@useobject{currentmarker}{}%
\end{pgfscope}%
\begin{pgfscope}%
\pgfsys@transformshift{6.404319in}{3.632141in}%
\pgfsys@useobject{currentmarker}{}%
\end{pgfscope}%
\begin{pgfscope}%
\pgfsys@transformshift{6.748496in}{3.592852in}%
\pgfsys@useobject{currentmarker}{}%
\end{pgfscope}%
\begin{pgfscope}%
\pgfsys@transformshift{7.092674in}{3.687607in}%
\pgfsys@useobject{currentmarker}{}%
\end{pgfscope}%
\begin{pgfscope}%
\pgfsys@transformshift{7.435548in}{3.622126in}%
\pgfsys@useobject{currentmarker}{}%
\end{pgfscope}%
\begin{pgfscope}%
\pgfsys@transformshift{7.779726in}{3.703015in}%
\pgfsys@useobject{currentmarker}{}%
\end{pgfscope}%
\begin{pgfscope}%
\pgfsys@transformshift{8.123904in}{3.703015in}%
\pgfsys@useobject{currentmarker}{}%
\end{pgfscope}%
\begin{pgfscope}%
\pgfsys@transformshift{8.468081in}{3.697622in}%
\pgfsys@useobject{currentmarker}{}%
\end{pgfscope}%
\begin{pgfscope}%
\pgfsys@transformshift{8.812259in}{3.693000in}%
\pgfsys@useobject{currentmarker}{}%
\end{pgfscope}%
\begin{pgfscope}%
\pgfsys@transformshift{9.156437in}{3.745385in}%
\pgfsys@useobject{currentmarker}{}%
\end{pgfscope}%
\begin{pgfscope}%
\pgfsys@transformshift{9.500615in}{3.666807in}%
\pgfsys@useobject{currentmarker}{}%
\end{pgfscope}%
\begin{pgfscope}%
\pgfsys@transformshift{9.844793in}{3.719193in}%
\pgfsys@useobject{currentmarker}{}%
\end{pgfscope}%
\begin{pgfscope}%
\pgfsys@transformshift{10.188970in}{3.682215in}%
\pgfsys@useobject{currentmarker}{}%
\end{pgfscope}%
\end{pgfscope}%
\begin{pgfscope}%
\pgfpathrectangle{\pgfqpoint{3.069444in}{0.769444in}}{\pgfqpoint{1.825185in}{2.311111in}}%
\pgfusepath{clip}%
\pgfsetrectcap%
\pgfsetroundjoin%
\pgfsetlinewidth{0.752812pt}%
\definecolor{currentstroke}{rgb}{0.501961,0.501961,0.000000}%
\pgfsetstrokecolor{currentstroke}%
\pgfsetdash{}{0pt}%
\pgfpathmoveto{\pgfqpoint{3.347133in}{2.257030in}}%
\pgfpathlineto{\pgfqpoint{3.735637in}{3.084407in}}%
\pgfpathlineto{\pgfqpoint{3.758435in}{3.090556in}}%
\pgfusepath{stroke}%
\end{pgfscope}%
\begin{pgfscope}%
\pgfpathrectangle{\pgfqpoint{3.069444in}{0.769444in}}{\pgfqpoint{1.825185in}{2.311111in}}%
\pgfusepath{clip}%
\pgfsetbuttcap%
\pgfsetroundjoin%
\definecolor{currentfill}{rgb}{0.501961,0.501961,0.000000}%
\pgfsetfillcolor{currentfill}%
\pgfsetlinewidth{1.003750pt}%
\definecolor{currentstroke}{rgb}{0.501961,0.501961,0.000000}%
\pgfsetstrokecolor{currentstroke}%
\pgfsetdash{}{0pt}%
\pgfsys@defobject{currentmarker}{\pgfqpoint{-0.010417in}{-0.010417in}}{\pgfqpoint{0.010417in}{0.010417in}}{%
\pgfpathmoveto{\pgfqpoint{0.000000in}{-0.010417in}}%
\pgfpathcurveto{\pgfqpoint{0.002763in}{-0.010417in}}{\pgfqpoint{0.005412in}{-0.009319in}}{\pgfqpoint{0.007366in}{-0.007366in}}%
\pgfpathcurveto{\pgfqpoint{0.009319in}{-0.005412in}}{\pgfqpoint{0.010417in}{-0.002763in}}{\pgfqpoint{0.010417in}{0.000000in}}%
\pgfpathcurveto{\pgfqpoint{0.010417in}{0.002763in}}{\pgfqpoint{0.009319in}{0.005412in}}{\pgfqpoint{0.007366in}{0.007366in}}%
\pgfpathcurveto{\pgfqpoint{0.005412in}{0.009319in}}{\pgfqpoint{0.002763in}{0.010417in}}{\pgfqpoint{0.000000in}{0.010417in}}%
\pgfpathcurveto{\pgfqpoint{-0.002763in}{0.010417in}}{\pgfqpoint{-0.005412in}{0.009319in}}{\pgfqpoint{-0.007366in}{0.007366in}}%
\pgfpathcurveto{\pgfqpoint{-0.009319in}{0.005412in}}{\pgfqpoint{-0.010417in}{0.002763in}}{\pgfqpoint{-0.010417in}{0.000000in}}%
\pgfpathcurveto{\pgfqpoint{-0.010417in}{-0.002763in}}{\pgfqpoint{-0.009319in}{-0.005412in}}{\pgfqpoint{-0.007366in}{-0.007366in}}%
\pgfpathcurveto{\pgfqpoint{-0.005412in}{-0.009319in}}{\pgfqpoint{-0.002763in}{-0.010417in}}{\pgfqpoint{0.000000in}{-0.010417in}}%
\pgfpathclose%
\pgfusepath{stroke,fill}%
}%
\begin{pgfscope}%
\pgfsys@transformshift{3.347133in}{2.257030in}%
\pgfsys@useobject{currentmarker}{}%
\end{pgfscope}%
\begin{pgfscope}%
\pgfsys@transformshift{3.735637in}{3.084407in}%
\pgfsys@useobject{currentmarker}{}%
\end{pgfscope}%
\begin{pgfscope}%
\pgfsys@transformshift{4.124141in}{3.189178in}%
\pgfsys@useobject{currentmarker}{}%
\end{pgfscope}%
\begin{pgfscope}%
\pgfsys@transformshift{4.512644in}{3.225385in}%
\pgfsys@useobject{currentmarker}{}%
\end{pgfscope}%
\begin{pgfscope}%
\pgfsys@transformshift{4.901148in}{3.375607in}%
\pgfsys@useobject{currentmarker}{}%
\end{pgfscope}%
\begin{pgfscope}%
\pgfsys@transformshift{5.289652in}{3.194570in}%
\pgfsys@useobject{currentmarker}{}%
\end{pgfscope}%
\begin{pgfscope}%
\pgfsys@transformshift{5.678156in}{3.377919in}%
\pgfsys@useobject{currentmarker}{}%
\end{pgfscope}%
\begin{pgfscope}%
\pgfsys@transformshift{6.066659in}{3.700704in}%
\pgfsys@useobject{currentmarker}{}%
\end{pgfscope}%
\begin{pgfscope}%
\pgfsys@transformshift{6.455163in}{3.671430in}%
\pgfsys@useobject{currentmarker}{}%
\end{pgfscope}%
\begin{pgfscope}%
\pgfsys@transformshift{6.843667in}{3.758481in}%
\pgfsys@useobject{currentmarker}{}%
\end{pgfscope}%
\begin{pgfscope}%
\pgfsys@transformshift{7.232170in}{3.719193in}%
\pgfsys@useobject{currentmarker}{}%
\end{pgfscope}%
\begin{pgfscope}%
\pgfsys@transformshift{7.620674in}{3.642926in}%
\pgfsys@useobject{currentmarker}{}%
\end{pgfscope}%
\begin{pgfscope}%
\pgfsys@transformshift{8.009178in}{3.676822in}%
\pgfsys@useobject{currentmarker}{}%
\end{pgfscope}%
\begin{pgfscope}%
\pgfsys@transformshift{8.397681in}{3.736911in}%
\pgfsys@useobject{currentmarker}{}%
\end{pgfscope}%
\begin{pgfscope}%
\pgfsys@transformshift{8.784881in}{3.676822in}%
\pgfsys@useobject{currentmarker}{}%
\end{pgfscope}%
\begin{pgfscope}%
\pgfsys@transformshift{9.173385in}{3.792378in}%
\pgfsys@useobject{currentmarker}{}%
\end{pgfscope}%
\begin{pgfscope}%
\pgfsys@transformshift{9.561889in}{3.732289in}%
\pgfsys@useobject{currentmarker}{}%
\end{pgfscope}%
\begin{pgfscope}%
\pgfsys@transformshift{9.950393in}{3.736911in}%
\pgfsys@useobject{currentmarker}{}%
\end{pgfscope}%
\begin{pgfscope}%
\pgfsys@transformshift{10.338896in}{3.760793in}%
\pgfsys@useobject{currentmarker}{}%
\end{pgfscope}%
\begin{pgfscope}%
\pgfsys@transformshift{10.727400in}{3.723815in}%
\pgfsys@useobject{currentmarker}{}%
\end{pgfscope}%
\begin{pgfscope}%
\pgfsys@transformshift{11.115904in}{3.766185in}%
\pgfsys@useobject{currentmarker}{}%
\end{pgfscope}%
\end{pgfscope}%
\begin{pgfscope}%
\pgfpathrectangle{\pgfqpoint{3.069444in}{0.769444in}}{\pgfqpoint{1.825185in}{2.311111in}}%
\pgfusepath{clip}%
\pgfsetrectcap%
\pgfsetroundjoin%
\pgfsetlinewidth{0.752812pt}%
\definecolor{currentstroke}{rgb}{0.000000,1.000000,1.000000}%
\pgfsetstrokecolor{currentstroke}%
\pgfsetdash{}{0pt}%
\pgfpathmoveto{\pgfqpoint{3.390156in}{2.421889in}}%
\pgfpathlineto{\pgfqpoint{3.824289in}{2.719252in}}%
\pgfpathlineto{\pgfqpoint{4.088567in}{3.090556in}}%
\pgfusepath{stroke}%
\end{pgfscope}%
\begin{pgfscope}%
\pgfpathrectangle{\pgfqpoint{3.069444in}{0.769444in}}{\pgfqpoint{1.825185in}{2.311111in}}%
\pgfusepath{clip}%
\pgfsetbuttcap%
\pgfsetroundjoin%
\definecolor{currentfill}{rgb}{0.000000,1.000000,1.000000}%
\pgfsetfillcolor{currentfill}%
\pgfsetlinewidth{1.003750pt}%
\definecolor{currentstroke}{rgb}{0.000000,1.000000,1.000000}%
\pgfsetstrokecolor{currentstroke}%
\pgfsetdash{}{0pt}%
\pgfsys@defobject{currentmarker}{\pgfqpoint{-0.010417in}{-0.010417in}}{\pgfqpoint{0.010417in}{0.010417in}}{%
\pgfpathmoveto{\pgfqpoint{0.000000in}{-0.010417in}}%
\pgfpathcurveto{\pgfqpoint{0.002763in}{-0.010417in}}{\pgfqpoint{0.005412in}{-0.009319in}}{\pgfqpoint{0.007366in}{-0.007366in}}%
\pgfpathcurveto{\pgfqpoint{0.009319in}{-0.005412in}}{\pgfqpoint{0.010417in}{-0.002763in}}{\pgfqpoint{0.010417in}{0.000000in}}%
\pgfpathcurveto{\pgfqpoint{0.010417in}{0.002763in}}{\pgfqpoint{0.009319in}{0.005412in}}{\pgfqpoint{0.007366in}{0.007366in}}%
\pgfpathcurveto{\pgfqpoint{0.005412in}{0.009319in}}{\pgfqpoint{0.002763in}{0.010417in}}{\pgfqpoint{0.000000in}{0.010417in}}%
\pgfpathcurveto{\pgfqpoint{-0.002763in}{0.010417in}}{\pgfqpoint{-0.005412in}{0.009319in}}{\pgfqpoint{-0.007366in}{0.007366in}}%
\pgfpathcurveto{\pgfqpoint{-0.009319in}{0.005412in}}{\pgfqpoint{-0.010417in}{0.002763in}}{\pgfqpoint{-0.010417in}{0.000000in}}%
\pgfpathcurveto{\pgfqpoint{-0.010417in}{-0.002763in}}{\pgfqpoint{-0.009319in}{-0.005412in}}{\pgfqpoint{-0.007366in}{-0.007366in}}%
\pgfpathcurveto{\pgfqpoint{-0.005412in}{-0.009319in}}{\pgfqpoint{-0.002763in}{-0.010417in}}{\pgfqpoint{0.000000in}{-0.010417in}}%
\pgfpathclose%
\pgfusepath{stroke,fill}%
}%
\begin{pgfscope}%
\pgfsys@transformshift{3.390156in}{2.421889in}%
\pgfsys@useobject{currentmarker}{}%
\end{pgfscope}%
\begin{pgfscope}%
\pgfsys@transformshift{3.824289in}{2.719252in}%
\pgfsys@useobject{currentmarker}{}%
\end{pgfscope}%
\begin{pgfscope}%
\pgfsys@transformshift{4.255815in}{3.325533in}%
\pgfsys@useobject{currentmarker}{}%
\end{pgfscope}%
\begin{pgfscope}%
\pgfsys@transformshift{4.687341in}{3.375607in}%
\pgfsys@useobject{currentmarker}{}%
\end{pgfscope}%
\begin{pgfscope}%
\pgfsys@transformshift{5.118867in}{3.467281in}%
\pgfsys@useobject{currentmarker}{}%
\end{pgfscope}%
\begin{pgfscope}%
\pgfsys@transformshift{5.550393in}{3.535074in}%
\pgfsys@useobject{currentmarker}{}%
\end{pgfscope}%
\begin{pgfscope}%
\pgfsys@transformshift{5.981919in}{3.561267in}%
\pgfsys@useobject{currentmarker}{}%
\end{pgfscope}%
\begin{pgfscope}%
\pgfsys@transformshift{6.413444in}{3.755400in}%
\pgfsys@useobject{currentmarker}{}%
\end{pgfscope}%
\begin{pgfscope}%
\pgfsys@transformshift{6.844970in}{3.786985in}%
\pgfsys@useobject{currentmarker}{}%
\end{pgfscope}%
\begin{pgfscope}%
\pgfsys@transformshift{7.276496in}{3.736911in}%
\pgfsys@useobject{currentmarker}{}%
\end{pgfscope}%
\begin{pgfscope}%
\pgfsys@transformshift{7.708022in}{3.736911in}%
\pgfsys@useobject{currentmarker}{}%
\end{pgfscope}%
\begin{pgfscope}%
\pgfsys@transformshift{8.139548in}{3.663726in}%
\pgfsys@useobject{currentmarker}{}%
\end{pgfscope}%
\begin{pgfscope}%
\pgfsys@transformshift{8.571074in}{3.755400in}%
\pgfsys@useobject{currentmarker}{}%
\end{pgfscope}%
\begin{pgfscope}%
\pgfsys@transformshift{9.002600in}{3.695311in}%
\pgfsys@useobject{currentmarker}{}%
\end{pgfscope}%
\begin{pgfscope}%
\pgfsys@transformshift{9.434126in}{3.776970in}%
\pgfsys@useobject{currentmarker}{}%
\end{pgfscope}%
\begin{pgfscope}%
\pgfsys@transformshift{9.865652in}{3.784674in}%
\pgfsys@useobject{currentmarker}{}%
\end{pgfscope}%
\begin{pgfscope}%
\pgfsys@transformshift{10.297178in}{3.784674in}%
\pgfsys@useobject{currentmarker}{}%
\end{pgfscope}%
\begin{pgfscope}%
\pgfsys@transformshift{10.728704in}{3.747696in}%
\pgfsys@useobject{currentmarker}{}%
\end{pgfscope}%
\begin{pgfscope}%
\pgfsys@transformshift{11.160230in}{3.797770in}%
\pgfsys@useobject{currentmarker}{}%
\end{pgfscope}%
\begin{pgfscope}%
\pgfsys@transformshift{11.591756in}{3.723815in}%
\pgfsys@useobject{currentmarker}{}%
\end{pgfscope}%
\begin{pgfscope}%
\pgfsys@transformshift{12.024585in}{3.790067in}%
\pgfsys@useobject{currentmarker}{}%
\end{pgfscope}%
\end{pgfscope}%
\begin{pgfscope}%
\pgfpathrectangle{\pgfqpoint{3.069444in}{0.769444in}}{\pgfqpoint{1.825185in}{2.311111in}}%
\pgfusepath{clip}%
\pgfsetrectcap%
\pgfsetroundjoin%
\pgfsetlinewidth{0.752812pt}%
\definecolor{currentstroke}{rgb}{0.698039,0.133333,0.133333}%
\pgfsetstrokecolor{currentstroke}%
\pgfsetdash{}{0pt}%
\pgfpathmoveto{\pgfqpoint{3.446215in}{2.780111in}}%
\pgfpathlineto{\pgfqpoint{3.788928in}{3.090556in}}%
\pgfusepath{stroke}%
\end{pgfscope}%
\begin{pgfscope}%
\pgfpathrectangle{\pgfqpoint{3.069444in}{0.769444in}}{\pgfqpoint{1.825185in}{2.311111in}}%
\pgfusepath{clip}%
\pgfsetbuttcap%
\pgfsetroundjoin%
\definecolor{currentfill}{rgb}{0.698039,0.133333,0.133333}%
\pgfsetfillcolor{currentfill}%
\pgfsetlinewidth{1.003750pt}%
\definecolor{currentstroke}{rgb}{0.698039,0.133333,0.133333}%
\pgfsetstrokecolor{currentstroke}%
\pgfsetdash{}{0pt}%
\pgfsys@defobject{currentmarker}{\pgfqpoint{-0.010417in}{-0.010417in}}{\pgfqpoint{0.010417in}{0.010417in}}{%
\pgfpathmoveto{\pgfqpoint{0.000000in}{-0.010417in}}%
\pgfpathcurveto{\pgfqpoint{0.002763in}{-0.010417in}}{\pgfqpoint{0.005412in}{-0.009319in}}{\pgfqpoint{0.007366in}{-0.007366in}}%
\pgfpathcurveto{\pgfqpoint{0.009319in}{-0.005412in}}{\pgfqpoint{0.010417in}{-0.002763in}}{\pgfqpoint{0.010417in}{0.000000in}}%
\pgfpathcurveto{\pgfqpoint{0.010417in}{0.002763in}}{\pgfqpoint{0.009319in}{0.005412in}}{\pgfqpoint{0.007366in}{0.007366in}}%
\pgfpathcurveto{\pgfqpoint{0.005412in}{0.009319in}}{\pgfqpoint{0.002763in}{0.010417in}}{\pgfqpoint{0.000000in}{0.010417in}}%
\pgfpathcurveto{\pgfqpoint{-0.002763in}{0.010417in}}{\pgfqpoint{-0.005412in}{0.009319in}}{\pgfqpoint{-0.007366in}{0.007366in}}%
\pgfpathcurveto{\pgfqpoint{-0.009319in}{0.005412in}}{\pgfqpoint{-0.010417in}{0.002763in}}{\pgfqpoint{-0.010417in}{0.000000in}}%
\pgfpathcurveto{\pgfqpoint{-0.010417in}{-0.002763in}}{\pgfqpoint{-0.009319in}{-0.005412in}}{\pgfqpoint{-0.007366in}{-0.007366in}}%
\pgfpathcurveto{\pgfqpoint{-0.005412in}{-0.009319in}}{\pgfqpoint{-0.002763in}{-0.010417in}}{\pgfqpoint{0.000000in}{-0.010417in}}%
\pgfpathclose%
\pgfusepath{stroke,fill}%
}%
\begin{pgfscope}%
\pgfsys@transformshift{3.446215in}{2.780111in}%
\pgfsys@useobject{currentmarker}{}%
\end{pgfscope}%
\begin{pgfscope}%
\pgfsys@transformshift{3.920763in}{3.209978in}%
\pgfsys@useobject{currentmarker}{}%
\end{pgfscope}%
\begin{pgfscope}%
\pgfsys@transformshift{4.395311in}{3.243874in}%
\pgfsys@useobject{currentmarker}{}%
\end{pgfscope}%
\begin{pgfscope}%
\pgfsys@transformshift{4.869859in}{3.477296in}%
\pgfsys@useobject{currentmarker}{}%
\end{pgfscope}%
\begin{pgfscope}%
\pgfsys@transformshift{5.344407in}{3.448793in}%
\pgfsys@useobject{currentmarker}{}%
\end{pgfscope}%
\begin{pgfscope}%
\pgfsys@transformshift{5.818956in}{3.569741in}%
\pgfsys@useobject{currentmarker}{}%
\end{pgfscope}%
\begin{pgfscope}%
\pgfsys@transformshift{6.383459in}{3.472674in}%
\pgfsys@useobject{currentmarker}{}%
\end{pgfscope}%
\begin{pgfscope}%
\pgfsys@transformshift{6.858007in}{3.786985in}%
\pgfsys@useobject{currentmarker}{}%
\end{pgfscope}%
\begin{pgfscope}%
\pgfsys@transformshift{7.332556in}{3.747696in}%
\pgfsys@useobject{currentmarker}{}%
\end{pgfscope}%
\begin{pgfscope}%
\pgfsys@transformshift{7.807104in}{3.805474in}%
\pgfsys@useobject{currentmarker}{}%
\end{pgfscope}%
\begin{pgfscope}%
\pgfsys@transformshift{8.281652in}{3.865563in}%
\pgfsys@useobject{currentmarker}{}%
\end{pgfscope}%
\begin{pgfscope}%
\pgfsys@transformshift{8.756200in}{3.766185in}%
\pgfsys@useobject{currentmarker}{}%
\end{pgfscope}%
\begin{pgfscope}%
\pgfsys@transformshift{9.230748in}{3.776970in}%
\pgfsys@useobject{currentmarker}{}%
\end{pgfscope}%
\begin{pgfscope}%
\pgfsys@transformshift{9.705296in}{3.813178in}%
\pgfsys@useobject{currentmarker}{}%
\end{pgfscope}%
\begin{pgfscope}%
\pgfsys@transformshift{10.179844in}{3.887133in}%
\pgfsys@useobject{currentmarker}{}%
\end{pgfscope}%
\begin{pgfscope}%
\pgfsys@transformshift{10.654393in}{3.850156in}%
\pgfsys@useobject{currentmarker}{}%
\end{pgfscope}%
\begin{pgfscope}%
\pgfsys@transformshift{11.128941in}{3.865563in}%
\pgfsys@useobject{currentmarker}{}%
\end{pgfscope}%
\begin{pgfscope}%
\pgfsys@transformshift{11.603489in}{3.857859in}%
\pgfsys@useobject{currentmarker}{}%
\end{pgfscope}%
\begin{pgfscope}%
\pgfsys@transformshift{12.078037in}{3.913326in}%
\pgfsys@useobject{currentmarker}{}%
\end{pgfscope}%
\begin{pgfscope}%
\pgfsys@transformshift{12.552585in}{3.823963in}%
\pgfsys@useobject{currentmarker}{}%
\end{pgfscope}%
\begin{pgfscope}%
\pgfsys@transformshift{13.027133in}{3.844763in}%
\pgfsys@useobject{currentmarker}{}%
\end{pgfscope}%
\end{pgfscope}%
\begin{pgfscope}%
\pgfpathrectangle{\pgfqpoint{3.069444in}{0.769444in}}{\pgfqpoint{1.825185in}{2.311111in}}%
\pgfusepath{clip}%
\pgfsetrectcap%
\pgfsetroundjoin%
\pgfsetlinewidth{0.752812pt}%
\definecolor{currentstroke}{rgb}{0.000000,0.750000,0.750000}%
\pgfsetstrokecolor{currentstroke}%
\pgfsetdash{}{0pt}%
\pgfpathmoveto{\pgfqpoint{3.517919in}{2.837889in}}%
\pgfpathlineto{\pgfqpoint{4.075904in}{3.005059in}}%
\pgfpathlineto{\pgfqpoint{4.158720in}{3.090556in}}%
\pgfusepath{stroke}%
\end{pgfscope}%
\begin{pgfscope}%
\pgfpathrectangle{\pgfqpoint{3.069444in}{0.769444in}}{\pgfqpoint{1.825185in}{2.311111in}}%
\pgfusepath{clip}%
\pgfsetbuttcap%
\pgfsetroundjoin%
\definecolor{currentfill}{rgb}{0.000000,0.750000,0.750000}%
\pgfsetfillcolor{currentfill}%
\pgfsetlinewidth{1.003750pt}%
\definecolor{currentstroke}{rgb}{0.000000,0.750000,0.750000}%
\pgfsetstrokecolor{currentstroke}%
\pgfsetdash{}{0pt}%
\pgfsys@defobject{currentmarker}{\pgfqpoint{-0.010417in}{-0.010417in}}{\pgfqpoint{0.010417in}{0.010417in}}{%
\pgfpathmoveto{\pgfqpoint{0.000000in}{-0.010417in}}%
\pgfpathcurveto{\pgfqpoint{0.002763in}{-0.010417in}}{\pgfqpoint{0.005412in}{-0.009319in}}{\pgfqpoint{0.007366in}{-0.007366in}}%
\pgfpathcurveto{\pgfqpoint{0.009319in}{-0.005412in}}{\pgfqpoint{0.010417in}{-0.002763in}}{\pgfqpoint{0.010417in}{0.000000in}}%
\pgfpathcurveto{\pgfqpoint{0.010417in}{0.002763in}}{\pgfqpoint{0.009319in}{0.005412in}}{\pgfqpoint{0.007366in}{0.007366in}}%
\pgfpathcurveto{\pgfqpoint{0.005412in}{0.009319in}}{\pgfqpoint{0.002763in}{0.010417in}}{\pgfqpoint{0.000000in}{0.010417in}}%
\pgfpathcurveto{\pgfqpoint{-0.002763in}{0.010417in}}{\pgfqpoint{-0.005412in}{0.009319in}}{\pgfqpoint{-0.007366in}{0.007366in}}%
\pgfpathcurveto{\pgfqpoint{-0.009319in}{0.005412in}}{\pgfqpoint{-0.010417in}{0.002763in}}{\pgfqpoint{-0.010417in}{0.000000in}}%
\pgfpathcurveto{\pgfqpoint{-0.010417in}{-0.002763in}}{\pgfqpoint{-0.009319in}{-0.005412in}}{\pgfqpoint{-0.007366in}{-0.007366in}}%
\pgfpathcurveto{\pgfqpoint{-0.005412in}{-0.009319in}}{\pgfqpoint{-0.002763in}{-0.010417in}}{\pgfqpoint{0.000000in}{-0.010417in}}%
\pgfpathclose%
\pgfusepath{stroke,fill}%
}%
\begin{pgfscope}%
\pgfsys@transformshift{3.517919in}{2.837889in}%
\pgfsys@useobject{currentmarker}{}%
\end{pgfscope}%
\begin{pgfscope}%
\pgfsys@transformshift{4.075904in}{3.005059in}%
\pgfsys@useobject{currentmarker}{}%
\end{pgfscope}%
\begin{pgfscope}%
\pgfsys@transformshift{4.632585in}{3.579756in}%
\pgfsys@useobject{currentmarker}{}%
\end{pgfscope}%
\begin{pgfscope}%
\pgfsys@transformshift{5.189267in}{3.619044in}%
\pgfsys@useobject{currentmarker}{}%
\end{pgfscope}%
\begin{pgfscope}%
\pgfsys@transformshift{5.747252in}{3.488081in}%
\pgfsys@useobject{currentmarker}{}%
\end{pgfscope}%
\begin{pgfscope}%
\pgfsys@transformshift{6.303933in}{3.650630in}%
\pgfsys@useobject{currentmarker}{}%
\end{pgfscope}%
\begin{pgfscope}%
\pgfsys@transformshift{6.860615in}{3.753089in}%
\pgfsys@useobject{currentmarker}{}%
\end{pgfscope}%
\begin{pgfscope}%
\pgfsys@transformshift{7.418600in}{3.907933in}%
\pgfsys@useobject{currentmarker}{}%
\end{pgfscope}%
\begin{pgfscope}%
\pgfsys@transformshift{7.975281in}{3.863252in}%
\pgfsys@useobject{currentmarker}{}%
\end{pgfscope}%
\begin{pgfscope}%
\pgfsys@transformshift{8.531963in}{3.844763in}%
\pgfsys@useobject{currentmarker}{}%
\end{pgfscope}%
\begin{pgfscope}%
\pgfsys@transformshift{9.088644in}{3.917948in}%
\pgfsys@useobject{currentmarker}{}%
\end{pgfscope}%
\begin{pgfscope}%
\pgfsys@transformshift{9.645326in}{3.952615in}%
\pgfsys@useobject{currentmarker}{}%
\end{pgfscope}%
\begin{pgfscope}%
\pgfsys@transformshift{10.202007in}{3.863252in}%
\pgfsys@useobject{currentmarker}{}%
\end{pgfscope}%
\begin{pgfscope}%
\pgfsys@transformshift{10.758689in}{3.904852in}%
\pgfsys@useobject{currentmarker}{}%
\end{pgfscope}%
\begin{pgfscope}%
\pgfsys@transformshift{11.314067in}{3.934126in}%
\pgfsys@useobject{currentmarker}{}%
\end{pgfscope}%
\begin{pgfscope}%
\pgfsys@transformshift{11.873356in}{3.904852in}%
\pgfsys@useobject{currentmarker}{}%
\end{pgfscope}%
\begin{pgfscope}%
\pgfsys@transformshift{12.430037in}{3.965711in}%
\pgfsys@useobject{currentmarker}{}%
\end{pgfscope}%
\begin{pgfscope}%
\pgfsys@transformshift{13.028437in}{3.934126in}%
\pgfsys@useobject{currentmarker}{}%
\end{pgfscope}%
\begin{pgfscope}%
\pgfsys@transformshift{13.585119in}{3.954926in}%
\pgfsys@useobject{currentmarker}{}%
\end{pgfscope}%
\begin{pgfscope}%
\pgfsys@transformshift{14.141800in}{3.952615in}%
\pgfsys@useobject{currentmarker}{}%
\end{pgfscope}%
\begin{pgfscope}%
\pgfsys@transformshift{14.698481in}{3.973415in}%
\pgfsys@useobject{currentmarker}{}%
\end{pgfscope}%
\end{pgfscope}%
\begin{pgfscope}%
\pgfsetrectcap%
\pgfsetmiterjoin%
\pgfsetlinewidth{0.803000pt}%
\definecolor{currentstroke}{rgb}{0.000000,0.000000,0.000000}%
\pgfsetstrokecolor{currentstroke}%
\pgfsetdash{}{0pt}%
\pgfpathmoveto{\pgfqpoint{3.069444in}{0.769444in}}%
\pgfpathlineto{\pgfqpoint{3.069444in}{3.080556in}}%
\pgfusepath{stroke}%
\end{pgfscope}%
\begin{pgfscope}%
\pgfsetrectcap%
\pgfsetmiterjoin%
\pgfsetlinewidth{0.803000pt}%
\definecolor{currentstroke}{rgb}{0.000000,0.000000,0.000000}%
\pgfsetstrokecolor{currentstroke}%
\pgfsetdash{}{0pt}%
\pgfpathmoveto{\pgfqpoint{4.894630in}{0.769444in}}%
\pgfpathlineto{\pgfqpoint{4.894630in}{3.080556in}}%
\pgfusepath{stroke}%
\end{pgfscope}%
\begin{pgfscope}%
\pgfsetrectcap%
\pgfsetmiterjoin%
\pgfsetlinewidth{0.803000pt}%
\definecolor{currentstroke}{rgb}{0.000000,0.000000,0.000000}%
\pgfsetstrokecolor{currentstroke}%
\pgfsetdash{}{0pt}%
\pgfpathmoveto{\pgfqpoint{3.069444in}{0.769444in}}%
\pgfpathlineto{\pgfqpoint{4.894630in}{0.769444in}}%
\pgfusepath{stroke}%
\end{pgfscope}%
\begin{pgfscope}%
\pgfsetrectcap%
\pgfsetmiterjoin%
\pgfsetlinewidth{0.803000pt}%
\definecolor{currentstroke}{rgb}{0.000000,0.000000,0.000000}%
\pgfsetstrokecolor{currentstroke}%
\pgfsetdash{}{0pt}%
\pgfpathmoveto{\pgfqpoint{3.069444in}{3.080556in}}%
\pgfpathlineto{\pgfqpoint{4.894630in}{3.080556in}}%
\pgfusepath{stroke}%
\end{pgfscope}%
\end{pgfpicture}%
\makeatother%
\endgroup%

					\end{center}
					\caption{Overview of influence of number of trained images on accuracy}
					\label{fig:evaluation_number_train_files}
				\end{figure}

				\noindent ...
		
			\subsubsection{Comparison of different CNN models}

				\noindent ...

				\begin{figure}[H]
					\begin{center}
						%% Creator: Matplotlib, PGF backend
%%
%% To include the figure in your LaTeX document, write
%%   \input{<filename>.pgf}
%%
%% Make sure the required packages are loaded in your preamble
%%   \usepackage{pgf}
%%
%% Figures using additional raster images can only be included by \input if
%% they are in the same directory as the main LaTeX file. For loading figures
%% from other directories you can use the `import` package
%%   \usepackage{import}
%% and then include the figures with
%%   \import{<path to file>}{<filename>.pgf}
%%
%% Matplotlib used the following preamble
%%   \usepackage{fontspec}
%%   \setmainfont{DejaVuSerif.ttf}[Path=C:/Users/bjoern/Anaconda3/envs/keras-gpu/lib/site-packages/matplotlib/mpl-data/fonts/ttf/]
%%   \setsansfont{DejaVuSans.ttf}[Path=C:/Users/bjoern/Anaconda3/envs/keras-gpu/lib/site-packages/matplotlib/mpl-data/fonts/ttf/]
%%   \setmonofont{DejaVuSansMono.ttf}[Path=C:/Users/bjoern/Anaconda3/envs/keras-gpu/lib/site-packages/matplotlib/mpl-data/fonts/ttf/]
%%
\begingroup%
\makeatletter%
\begin{pgfpicture}%
\pgfpathrectangle{\pgfpointorigin}{\pgfqpoint{8.000000in}{4.000000in}}%
\pgfusepath{use as bounding box, clip}%
\begin{pgfscope}%
\pgfsetbuttcap%
\pgfsetmiterjoin%
\definecolor{currentfill}{rgb}{1.000000,1.000000,1.000000}%
\pgfsetfillcolor{currentfill}%
\pgfsetlinewidth{0.000000pt}%
\definecolor{currentstroke}{rgb}{1.000000,1.000000,1.000000}%
\pgfsetstrokecolor{currentstroke}%
\pgfsetdash{}{0pt}%
\pgfpathmoveto{\pgfqpoint{0.000000in}{0.000000in}}%
\pgfpathlineto{\pgfqpoint{8.000000in}{0.000000in}}%
\pgfpathlineto{\pgfqpoint{8.000000in}{4.000000in}}%
\pgfpathlineto{\pgfqpoint{0.000000in}{4.000000in}}%
\pgfpathclose%
\pgfusepath{fill}%
\end{pgfscope}%
\begin{pgfscope}%
\pgfsetbuttcap%
\pgfsetmiterjoin%
\definecolor{currentfill}{rgb}{1.000000,1.000000,1.000000}%
\pgfsetfillcolor{currentfill}%
\pgfsetlinewidth{0.000000pt}%
\definecolor{currentstroke}{rgb}{0.000000,0.000000,0.000000}%
\pgfsetstrokecolor{currentstroke}%
\pgfsetstrokeopacity{0.000000}%
\pgfsetdash{}{0pt}%
\pgfpathmoveto{\pgfqpoint{0.720000in}{0.480000in}}%
\pgfpathlineto{\pgfqpoint{7.760000in}{0.480000in}}%
\pgfpathlineto{\pgfqpoint{7.760000in}{3.600000in}}%
\pgfpathlineto{\pgfqpoint{0.720000in}{3.600000in}}%
\pgfpathclose%
\pgfusepath{fill}%
\end{pgfscope}%
\begin{pgfscope}%
\pgfpathrectangle{\pgfqpoint{0.720000in}{0.480000in}}{\pgfqpoint{7.040000in}{3.120000in}}%
\pgfusepath{clip}%
\pgfsetrectcap%
\pgfsetroundjoin%
\pgfsetlinewidth{0.803000pt}%
\definecolor{currentstroke}{rgb}{0.690196,0.690196,0.690196}%
\pgfsetstrokecolor{currentstroke}%
\pgfsetdash{}{0pt}%
\pgfpathmoveto{\pgfqpoint{0.720000in}{0.480000in}}%
\pgfpathlineto{\pgfqpoint{0.720000in}{3.600000in}}%
\pgfusepath{stroke}%
\end{pgfscope}%
\begin{pgfscope}%
\pgfsetbuttcap%
\pgfsetroundjoin%
\definecolor{currentfill}{rgb}{0.000000,0.000000,0.000000}%
\pgfsetfillcolor{currentfill}%
\pgfsetlinewidth{0.803000pt}%
\definecolor{currentstroke}{rgb}{0.000000,0.000000,0.000000}%
\pgfsetstrokecolor{currentstroke}%
\pgfsetdash{}{0pt}%
\pgfsys@defobject{currentmarker}{\pgfqpoint{0.000000in}{-0.048611in}}{\pgfqpoint{0.000000in}{0.000000in}}{%
\pgfpathmoveto{\pgfqpoint{0.000000in}{0.000000in}}%
\pgfpathlineto{\pgfqpoint{0.000000in}{-0.048611in}}%
\pgfusepath{stroke,fill}%
}%
\begin{pgfscope}%
\pgfsys@transformshift{0.720000in}{0.480000in}%
\pgfsys@useobject{currentmarker}{}%
\end{pgfscope}%
\end{pgfscope}%
\begin{pgfscope}%
\definecolor{textcolor}{rgb}{0.000000,0.000000,0.000000}%
\pgfsetstrokecolor{textcolor}%
\pgfsetfillcolor{textcolor}%
\pgftext[x=0.720000in,y=0.382778in,,top]{\color{textcolor}\sffamily\fontsize{10.000000}{12.000000}\selectfont 0}%
\end{pgfscope}%
\begin{pgfscope}%
\pgfpathrectangle{\pgfqpoint{0.720000in}{0.480000in}}{\pgfqpoint{7.040000in}{3.120000in}}%
\pgfusepath{clip}%
\pgfsetrectcap%
\pgfsetroundjoin%
\pgfsetlinewidth{0.803000pt}%
\definecolor{currentstroke}{rgb}{0.690196,0.690196,0.690196}%
\pgfsetstrokecolor{currentstroke}%
\pgfsetdash{}{0pt}%
\pgfpathmoveto{\pgfqpoint{1.725714in}{0.480000in}}%
\pgfpathlineto{\pgfqpoint{1.725714in}{3.600000in}}%
\pgfusepath{stroke}%
\end{pgfscope}%
\begin{pgfscope}%
\pgfsetbuttcap%
\pgfsetroundjoin%
\definecolor{currentfill}{rgb}{0.000000,0.000000,0.000000}%
\pgfsetfillcolor{currentfill}%
\pgfsetlinewidth{0.803000pt}%
\definecolor{currentstroke}{rgb}{0.000000,0.000000,0.000000}%
\pgfsetstrokecolor{currentstroke}%
\pgfsetdash{}{0pt}%
\pgfsys@defobject{currentmarker}{\pgfqpoint{0.000000in}{-0.048611in}}{\pgfqpoint{0.000000in}{0.000000in}}{%
\pgfpathmoveto{\pgfqpoint{0.000000in}{0.000000in}}%
\pgfpathlineto{\pgfqpoint{0.000000in}{-0.048611in}}%
\pgfusepath{stroke,fill}%
}%
\begin{pgfscope}%
\pgfsys@transformshift{1.725714in}{0.480000in}%
\pgfsys@useobject{currentmarker}{}%
\end{pgfscope}%
\end{pgfscope}%
\begin{pgfscope}%
\definecolor{textcolor}{rgb}{0.000000,0.000000,0.000000}%
\pgfsetstrokecolor{textcolor}%
\pgfsetfillcolor{textcolor}%
\pgftext[x=1.725714in,y=0.382778in,,top]{\color{textcolor}\sffamily\fontsize{10.000000}{12.000000}\selectfont 3600}%
\end{pgfscope}%
\begin{pgfscope}%
\pgfpathrectangle{\pgfqpoint{0.720000in}{0.480000in}}{\pgfqpoint{7.040000in}{3.120000in}}%
\pgfusepath{clip}%
\pgfsetrectcap%
\pgfsetroundjoin%
\pgfsetlinewidth{0.803000pt}%
\definecolor{currentstroke}{rgb}{0.690196,0.690196,0.690196}%
\pgfsetstrokecolor{currentstroke}%
\pgfsetdash{}{0pt}%
\pgfpathmoveto{\pgfqpoint{2.731429in}{0.480000in}}%
\pgfpathlineto{\pgfqpoint{2.731429in}{3.600000in}}%
\pgfusepath{stroke}%
\end{pgfscope}%
\begin{pgfscope}%
\pgfsetbuttcap%
\pgfsetroundjoin%
\definecolor{currentfill}{rgb}{0.000000,0.000000,0.000000}%
\pgfsetfillcolor{currentfill}%
\pgfsetlinewidth{0.803000pt}%
\definecolor{currentstroke}{rgb}{0.000000,0.000000,0.000000}%
\pgfsetstrokecolor{currentstroke}%
\pgfsetdash{}{0pt}%
\pgfsys@defobject{currentmarker}{\pgfqpoint{0.000000in}{-0.048611in}}{\pgfqpoint{0.000000in}{0.000000in}}{%
\pgfpathmoveto{\pgfqpoint{0.000000in}{0.000000in}}%
\pgfpathlineto{\pgfqpoint{0.000000in}{-0.048611in}}%
\pgfusepath{stroke,fill}%
}%
\begin{pgfscope}%
\pgfsys@transformshift{2.731429in}{0.480000in}%
\pgfsys@useobject{currentmarker}{}%
\end{pgfscope}%
\end{pgfscope}%
\begin{pgfscope}%
\definecolor{textcolor}{rgb}{0.000000,0.000000,0.000000}%
\pgfsetstrokecolor{textcolor}%
\pgfsetfillcolor{textcolor}%
\pgftext[x=2.731429in,y=0.382778in,,top]{\color{textcolor}\sffamily\fontsize{10.000000}{12.000000}\selectfont 7200}%
\end{pgfscope}%
\begin{pgfscope}%
\pgfpathrectangle{\pgfqpoint{0.720000in}{0.480000in}}{\pgfqpoint{7.040000in}{3.120000in}}%
\pgfusepath{clip}%
\pgfsetrectcap%
\pgfsetroundjoin%
\pgfsetlinewidth{0.803000pt}%
\definecolor{currentstroke}{rgb}{0.690196,0.690196,0.690196}%
\pgfsetstrokecolor{currentstroke}%
\pgfsetdash{}{0pt}%
\pgfpathmoveto{\pgfqpoint{3.737143in}{0.480000in}}%
\pgfpathlineto{\pgfqpoint{3.737143in}{3.600000in}}%
\pgfusepath{stroke}%
\end{pgfscope}%
\begin{pgfscope}%
\pgfsetbuttcap%
\pgfsetroundjoin%
\definecolor{currentfill}{rgb}{0.000000,0.000000,0.000000}%
\pgfsetfillcolor{currentfill}%
\pgfsetlinewidth{0.803000pt}%
\definecolor{currentstroke}{rgb}{0.000000,0.000000,0.000000}%
\pgfsetstrokecolor{currentstroke}%
\pgfsetdash{}{0pt}%
\pgfsys@defobject{currentmarker}{\pgfqpoint{0.000000in}{-0.048611in}}{\pgfqpoint{0.000000in}{0.000000in}}{%
\pgfpathmoveto{\pgfqpoint{0.000000in}{0.000000in}}%
\pgfpathlineto{\pgfqpoint{0.000000in}{-0.048611in}}%
\pgfusepath{stroke,fill}%
}%
\begin{pgfscope}%
\pgfsys@transformshift{3.737143in}{0.480000in}%
\pgfsys@useobject{currentmarker}{}%
\end{pgfscope}%
\end{pgfscope}%
\begin{pgfscope}%
\definecolor{textcolor}{rgb}{0.000000,0.000000,0.000000}%
\pgfsetstrokecolor{textcolor}%
\pgfsetfillcolor{textcolor}%
\pgftext[x=3.737143in,y=0.382778in,,top]{\color{textcolor}\sffamily\fontsize{10.000000}{12.000000}\selectfont 10800}%
\end{pgfscope}%
\begin{pgfscope}%
\pgfpathrectangle{\pgfqpoint{0.720000in}{0.480000in}}{\pgfqpoint{7.040000in}{3.120000in}}%
\pgfusepath{clip}%
\pgfsetrectcap%
\pgfsetroundjoin%
\pgfsetlinewidth{0.803000pt}%
\definecolor{currentstroke}{rgb}{0.690196,0.690196,0.690196}%
\pgfsetstrokecolor{currentstroke}%
\pgfsetdash{}{0pt}%
\pgfpathmoveto{\pgfqpoint{4.742857in}{0.480000in}}%
\pgfpathlineto{\pgfqpoint{4.742857in}{3.600000in}}%
\pgfusepath{stroke}%
\end{pgfscope}%
\begin{pgfscope}%
\pgfsetbuttcap%
\pgfsetroundjoin%
\definecolor{currentfill}{rgb}{0.000000,0.000000,0.000000}%
\pgfsetfillcolor{currentfill}%
\pgfsetlinewidth{0.803000pt}%
\definecolor{currentstroke}{rgb}{0.000000,0.000000,0.000000}%
\pgfsetstrokecolor{currentstroke}%
\pgfsetdash{}{0pt}%
\pgfsys@defobject{currentmarker}{\pgfqpoint{0.000000in}{-0.048611in}}{\pgfqpoint{0.000000in}{0.000000in}}{%
\pgfpathmoveto{\pgfqpoint{0.000000in}{0.000000in}}%
\pgfpathlineto{\pgfqpoint{0.000000in}{-0.048611in}}%
\pgfusepath{stroke,fill}%
}%
\begin{pgfscope}%
\pgfsys@transformshift{4.742857in}{0.480000in}%
\pgfsys@useobject{currentmarker}{}%
\end{pgfscope}%
\end{pgfscope}%
\begin{pgfscope}%
\definecolor{textcolor}{rgb}{0.000000,0.000000,0.000000}%
\pgfsetstrokecolor{textcolor}%
\pgfsetfillcolor{textcolor}%
\pgftext[x=4.742857in,y=0.382778in,,top]{\color{textcolor}\sffamily\fontsize{10.000000}{12.000000}\selectfont 14400}%
\end{pgfscope}%
\begin{pgfscope}%
\pgfpathrectangle{\pgfqpoint{0.720000in}{0.480000in}}{\pgfqpoint{7.040000in}{3.120000in}}%
\pgfusepath{clip}%
\pgfsetrectcap%
\pgfsetroundjoin%
\pgfsetlinewidth{0.803000pt}%
\definecolor{currentstroke}{rgb}{0.690196,0.690196,0.690196}%
\pgfsetstrokecolor{currentstroke}%
\pgfsetdash{}{0pt}%
\pgfpathmoveto{\pgfqpoint{5.748571in}{0.480000in}}%
\pgfpathlineto{\pgfqpoint{5.748571in}{3.600000in}}%
\pgfusepath{stroke}%
\end{pgfscope}%
\begin{pgfscope}%
\pgfsetbuttcap%
\pgfsetroundjoin%
\definecolor{currentfill}{rgb}{0.000000,0.000000,0.000000}%
\pgfsetfillcolor{currentfill}%
\pgfsetlinewidth{0.803000pt}%
\definecolor{currentstroke}{rgb}{0.000000,0.000000,0.000000}%
\pgfsetstrokecolor{currentstroke}%
\pgfsetdash{}{0pt}%
\pgfsys@defobject{currentmarker}{\pgfqpoint{0.000000in}{-0.048611in}}{\pgfqpoint{0.000000in}{0.000000in}}{%
\pgfpathmoveto{\pgfqpoint{0.000000in}{0.000000in}}%
\pgfpathlineto{\pgfqpoint{0.000000in}{-0.048611in}}%
\pgfusepath{stroke,fill}%
}%
\begin{pgfscope}%
\pgfsys@transformshift{5.748571in}{0.480000in}%
\pgfsys@useobject{currentmarker}{}%
\end{pgfscope}%
\end{pgfscope}%
\begin{pgfscope}%
\definecolor{textcolor}{rgb}{0.000000,0.000000,0.000000}%
\pgfsetstrokecolor{textcolor}%
\pgfsetfillcolor{textcolor}%
\pgftext[x=5.748571in,y=0.382778in,,top]{\color{textcolor}\sffamily\fontsize{10.000000}{12.000000}\selectfont 18000}%
\end{pgfscope}%
\begin{pgfscope}%
\pgfpathrectangle{\pgfqpoint{0.720000in}{0.480000in}}{\pgfqpoint{7.040000in}{3.120000in}}%
\pgfusepath{clip}%
\pgfsetrectcap%
\pgfsetroundjoin%
\pgfsetlinewidth{0.803000pt}%
\definecolor{currentstroke}{rgb}{0.690196,0.690196,0.690196}%
\pgfsetstrokecolor{currentstroke}%
\pgfsetdash{}{0pt}%
\pgfpathmoveto{\pgfqpoint{6.754286in}{0.480000in}}%
\pgfpathlineto{\pgfqpoint{6.754286in}{3.600000in}}%
\pgfusepath{stroke}%
\end{pgfscope}%
\begin{pgfscope}%
\pgfsetbuttcap%
\pgfsetroundjoin%
\definecolor{currentfill}{rgb}{0.000000,0.000000,0.000000}%
\pgfsetfillcolor{currentfill}%
\pgfsetlinewidth{0.803000pt}%
\definecolor{currentstroke}{rgb}{0.000000,0.000000,0.000000}%
\pgfsetstrokecolor{currentstroke}%
\pgfsetdash{}{0pt}%
\pgfsys@defobject{currentmarker}{\pgfqpoint{0.000000in}{-0.048611in}}{\pgfqpoint{0.000000in}{0.000000in}}{%
\pgfpathmoveto{\pgfqpoint{0.000000in}{0.000000in}}%
\pgfpathlineto{\pgfqpoint{0.000000in}{-0.048611in}}%
\pgfusepath{stroke,fill}%
}%
\begin{pgfscope}%
\pgfsys@transformshift{6.754286in}{0.480000in}%
\pgfsys@useobject{currentmarker}{}%
\end{pgfscope}%
\end{pgfscope}%
\begin{pgfscope}%
\definecolor{textcolor}{rgb}{0.000000,0.000000,0.000000}%
\pgfsetstrokecolor{textcolor}%
\pgfsetfillcolor{textcolor}%
\pgftext[x=6.754286in,y=0.382778in,,top]{\color{textcolor}\sffamily\fontsize{10.000000}{12.000000}\selectfont 21600}%
\end{pgfscope}%
\begin{pgfscope}%
\pgfpathrectangle{\pgfqpoint{0.720000in}{0.480000in}}{\pgfqpoint{7.040000in}{3.120000in}}%
\pgfusepath{clip}%
\pgfsetrectcap%
\pgfsetroundjoin%
\pgfsetlinewidth{0.803000pt}%
\definecolor{currentstroke}{rgb}{0.690196,0.690196,0.690196}%
\pgfsetstrokecolor{currentstroke}%
\pgfsetdash{}{0pt}%
\pgfpathmoveto{\pgfqpoint{7.760000in}{0.480000in}}%
\pgfpathlineto{\pgfqpoint{7.760000in}{3.600000in}}%
\pgfusepath{stroke}%
\end{pgfscope}%
\begin{pgfscope}%
\pgfsetbuttcap%
\pgfsetroundjoin%
\definecolor{currentfill}{rgb}{0.000000,0.000000,0.000000}%
\pgfsetfillcolor{currentfill}%
\pgfsetlinewidth{0.803000pt}%
\definecolor{currentstroke}{rgb}{0.000000,0.000000,0.000000}%
\pgfsetstrokecolor{currentstroke}%
\pgfsetdash{}{0pt}%
\pgfsys@defobject{currentmarker}{\pgfqpoint{0.000000in}{-0.048611in}}{\pgfqpoint{0.000000in}{0.000000in}}{%
\pgfpathmoveto{\pgfqpoint{0.000000in}{0.000000in}}%
\pgfpathlineto{\pgfqpoint{0.000000in}{-0.048611in}}%
\pgfusepath{stroke,fill}%
}%
\begin{pgfscope}%
\pgfsys@transformshift{7.760000in}{0.480000in}%
\pgfsys@useobject{currentmarker}{}%
\end{pgfscope}%
\end{pgfscope}%
\begin{pgfscope}%
\definecolor{textcolor}{rgb}{0.000000,0.000000,0.000000}%
\pgfsetstrokecolor{textcolor}%
\pgfsetfillcolor{textcolor}%
\pgftext[x=7.760000in,y=0.382778in,,top]{\color{textcolor}\sffamily\fontsize{10.000000}{12.000000}\selectfont 25200}%
\end{pgfscope}%
\begin{pgfscope}%
\definecolor{textcolor}{rgb}{0.000000,0.000000,0.000000}%
\pgfsetstrokecolor{textcolor}%
\pgfsetfillcolor{textcolor}%
\pgftext[x=4.240000in,y=0.192809in,,top]{\color{textcolor}\sffamily\fontsize{10.000000}{12.000000}\selectfont model training time [s]}%
\end{pgfscope}%
\begin{pgfscope}%
\pgfpathrectangle{\pgfqpoint{0.720000in}{0.480000in}}{\pgfqpoint{7.040000in}{3.120000in}}%
\pgfusepath{clip}%
\pgfsetrectcap%
\pgfsetroundjoin%
\pgfsetlinewidth{0.803000pt}%
\definecolor{currentstroke}{rgb}{0.690196,0.690196,0.690196}%
\pgfsetstrokecolor{currentstroke}%
\pgfsetdash{}{0pt}%
\pgfpathmoveto{\pgfqpoint{0.720000in}{0.672593in}}%
\pgfpathlineto{\pgfqpoint{7.760000in}{0.672593in}}%
\pgfusepath{stroke}%
\end{pgfscope}%
\begin{pgfscope}%
\pgfsetbuttcap%
\pgfsetroundjoin%
\definecolor{currentfill}{rgb}{0.000000,0.000000,0.000000}%
\pgfsetfillcolor{currentfill}%
\pgfsetlinewidth{0.803000pt}%
\definecolor{currentstroke}{rgb}{0.000000,0.000000,0.000000}%
\pgfsetstrokecolor{currentstroke}%
\pgfsetdash{}{0pt}%
\pgfsys@defobject{currentmarker}{\pgfqpoint{-0.048611in}{0.000000in}}{\pgfqpoint{0.000000in}{0.000000in}}{%
\pgfpathmoveto{\pgfqpoint{0.000000in}{0.000000in}}%
\pgfpathlineto{\pgfqpoint{-0.048611in}{0.000000in}}%
\pgfusepath{stroke,fill}%
}%
\begin{pgfscope}%
\pgfsys@transformshift{0.720000in}{0.672593in}%
\pgfsys@useobject{currentmarker}{}%
\end{pgfscope}%
\end{pgfscope}%
\begin{pgfscope}%
\definecolor{textcolor}{rgb}{0.000000,0.000000,0.000000}%
\pgfsetstrokecolor{textcolor}%
\pgfsetfillcolor{textcolor}%
\pgftext[x=0.314076in,y=0.619831in,left,base]{\color{textcolor}\sffamily\fontsize{10.000000}{12.000000}\selectfont 10\%}%
\end{pgfscope}%
\begin{pgfscope}%
\pgfpathrectangle{\pgfqpoint{0.720000in}{0.480000in}}{\pgfqpoint{7.040000in}{3.120000in}}%
\pgfusepath{clip}%
\pgfsetrectcap%
\pgfsetroundjoin%
\pgfsetlinewidth{0.803000pt}%
\definecolor{currentstroke}{rgb}{0.690196,0.690196,0.690196}%
\pgfsetstrokecolor{currentstroke}%
\pgfsetdash{}{0pt}%
\pgfpathmoveto{\pgfqpoint{0.720000in}{1.057778in}}%
\pgfpathlineto{\pgfqpoint{7.760000in}{1.057778in}}%
\pgfusepath{stroke}%
\end{pgfscope}%
\begin{pgfscope}%
\pgfsetbuttcap%
\pgfsetroundjoin%
\definecolor{currentfill}{rgb}{0.000000,0.000000,0.000000}%
\pgfsetfillcolor{currentfill}%
\pgfsetlinewidth{0.803000pt}%
\definecolor{currentstroke}{rgb}{0.000000,0.000000,0.000000}%
\pgfsetstrokecolor{currentstroke}%
\pgfsetdash{}{0pt}%
\pgfsys@defobject{currentmarker}{\pgfqpoint{-0.048611in}{0.000000in}}{\pgfqpoint{0.000000in}{0.000000in}}{%
\pgfpathmoveto{\pgfqpoint{0.000000in}{0.000000in}}%
\pgfpathlineto{\pgfqpoint{-0.048611in}{0.000000in}}%
\pgfusepath{stroke,fill}%
}%
\begin{pgfscope}%
\pgfsys@transformshift{0.720000in}{1.057778in}%
\pgfsys@useobject{currentmarker}{}%
\end{pgfscope}%
\end{pgfscope}%
\begin{pgfscope}%
\definecolor{textcolor}{rgb}{0.000000,0.000000,0.000000}%
\pgfsetstrokecolor{textcolor}%
\pgfsetfillcolor{textcolor}%
\pgftext[x=0.314076in,y=1.005016in,left,base]{\color{textcolor}\sffamily\fontsize{10.000000}{12.000000}\selectfont 20\%}%
\end{pgfscope}%
\begin{pgfscope}%
\pgfpathrectangle{\pgfqpoint{0.720000in}{0.480000in}}{\pgfqpoint{7.040000in}{3.120000in}}%
\pgfusepath{clip}%
\pgfsetrectcap%
\pgfsetroundjoin%
\pgfsetlinewidth{0.803000pt}%
\definecolor{currentstroke}{rgb}{0.690196,0.690196,0.690196}%
\pgfsetstrokecolor{currentstroke}%
\pgfsetdash{}{0pt}%
\pgfpathmoveto{\pgfqpoint{0.720000in}{1.442963in}}%
\pgfpathlineto{\pgfqpoint{7.760000in}{1.442963in}}%
\pgfusepath{stroke}%
\end{pgfscope}%
\begin{pgfscope}%
\pgfsetbuttcap%
\pgfsetroundjoin%
\definecolor{currentfill}{rgb}{0.000000,0.000000,0.000000}%
\pgfsetfillcolor{currentfill}%
\pgfsetlinewidth{0.803000pt}%
\definecolor{currentstroke}{rgb}{0.000000,0.000000,0.000000}%
\pgfsetstrokecolor{currentstroke}%
\pgfsetdash{}{0pt}%
\pgfsys@defobject{currentmarker}{\pgfqpoint{-0.048611in}{0.000000in}}{\pgfqpoint{0.000000in}{0.000000in}}{%
\pgfpathmoveto{\pgfqpoint{0.000000in}{0.000000in}}%
\pgfpathlineto{\pgfqpoint{-0.048611in}{0.000000in}}%
\pgfusepath{stroke,fill}%
}%
\begin{pgfscope}%
\pgfsys@transformshift{0.720000in}{1.442963in}%
\pgfsys@useobject{currentmarker}{}%
\end{pgfscope}%
\end{pgfscope}%
\begin{pgfscope}%
\definecolor{textcolor}{rgb}{0.000000,0.000000,0.000000}%
\pgfsetstrokecolor{textcolor}%
\pgfsetfillcolor{textcolor}%
\pgftext[x=0.314076in,y=1.390201in,left,base]{\color{textcolor}\sffamily\fontsize{10.000000}{12.000000}\selectfont 30\%}%
\end{pgfscope}%
\begin{pgfscope}%
\pgfpathrectangle{\pgfqpoint{0.720000in}{0.480000in}}{\pgfqpoint{7.040000in}{3.120000in}}%
\pgfusepath{clip}%
\pgfsetrectcap%
\pgfsetroundjoin%
\pgfsetlinewidth{0.803000pt}%
\definecolor{currentstroke}{rgb}{0.690196,0.690196,0.690196}%
\pgfsetstrokecolor{currentstroke}%
\pgfsetdash{}{0pt}%
\pgfpathmoveto{\pgfqpoint{0.720000in}{1.828148in}}%
\pgfpathlineto{\pgfqpoint{7.760000in}{1.828148in}}%
\pgfusepath{stroke}%
\end{pgfscope}%
\begin{pgfscope}%
\pgfsetbuttcap%
\pgfsetroundjoin%
\definecolor{currentfill}{rgb}{0.000000,0.000000,0.000000}%
\pgfsetfillcolor{currentfill}%
\pgfsetlinewidth{0.803000pt}%
\definecolor{currentstroke}{rgb}{0.000000,0.000000,0.000000}%
\pgfsetstrokecolor{currentstroke}%
\pgfsetdash{}{0pt}%
\pgfsys@defobject{currentmarker}{\pgfqpoint{-0.048611in}{0.000000in}}{\pgfqpoint{0.000000in}{0.000000in}}{%
\pgfpathmoveto{\pgfqpoint{0.000000in}{0.000000in}}%
\pgfpathlineto{\pgfqpoint{-0.048611in}{0.000000in}}%
\pgfusepath{stroke,fill}%
}%
\begin{pgfscope}%
\pgfsys@transformshift{0.720000in}{1.828148in}%
\pgfsys@useobject{currentmarker}{}%
\end{pgfscope}%
\end{pgfscope}%
\begin{pgfscope}%
\definecolor{textcolor}{rgb}{0.000000,0.000000,0.000000}%
\pgfsetstrokecolor{textcolor}%
\pgfsetfillcolor{textcolor}%
\pgftext[x=0.314076in,y=1.775387in,left,base]{\color{textcolor}\sffamily\fontsize{10.000000}{12.000000}\selectfont 40\%}%
\end{pgfscope}%
\begin{pgfscope}%
\pgfpathrectangle{\pgfqpoint{0.720000in}{0.480000in}}{\pgfqpoint{7.040000in}{3.120000in}}%
\pgfusepath{clip}%
\pgfsetrectcap%
\pgfsetroundjoin%
\pgfsetlinewidth{0.803000pt}%
\definecolor{currentstroke}{rgb}{0.690196,0.690196,0.690196}%
\pgfsetstrokecolor{currentstroke}%
\pgfsetdash{}{0pt}%
\pgfpathmoveto{\pgfqpoint{0.720000in}{2.213333in}}%
\pgfpathlineto{\pgfqpoint{7.760000in}{2.213333in}}%
\pgfusepath{stroke}%
\end{pgfscope}%
\begin{pgfscope}%
\pgfsetbuttcap%
\pgfsetroundjoin%
\definecolor{currentfill}{rgb}{0.000000,0.000000,0.000000}%
\pgfsetfillcolor{currentfill}%
\pgfsetlinewidth{0.803000pt}%
\definecolor{currentstroke}{rgb}{0.000000,0.000000,0.000000}%
\pgfsetstrokecolor{currentstroke}%
\pgfsetdash{}{0pt}%
\pgfsys@defobject{currentmarker}{\pgfqpoint{-0.048611in}{0.000000in}}{\pgfqpoint{0.000000in}{0.000000in}}{%
\pgfpathmoveto{\pgfqpoint{0.000000in}{0.000000in}}%
\pgfpathlineto{\pgfqpoint{-0.048611in}{0.000000in}}%
\pgfusepath{stroke,fill}%
}%
\begin{pgfscope}%
\pgfsys@transformshift{0.720000in}{2.213333in}%
\pgfsys@useobject{currentmarker}{}%
\end{pgfscope}%
\end{pgfscope}%
\begin{pgfscope}%
\definecolor{textcolor}{rgb}{0.000000,0.000000,0.000000}%
\pgfsetstrokecolor{textcolor}%
\pgfsetfillcolor{textcolor}%
\pgftext[x=0.314076in,y=2.160572in,left,base]{\color{textcolor}\sffamily\fontsize{10.000000}{12.000000}\selectfont 50\%}%
\end{pgfscope}%
\begin{pgfscope}%
\pgfpathrectangle{\pgfqpoint{0.720000in}{0.480000in}}{\pgfqpoint{7.040000in}{3.120000in}}%
\pgfusepath{clip}%
\pgfsetrectcap%
\pgfsetroundjoin%
\pgfsetlinewidth{0.803000pt}%
\definecolor{currentstroke}{rgb}{0.690196,0.690196,0.690196}%
\pgfsetstrokecolor{currentstroke}%
\pgfsetdash{}{0pt}%
\pgfpathmoveto{\pgfqpoint{0.720000in}{2.598519in}}%
\pgfpathlineto{\pgfqpoint{7.760000in}{2.598519in}}%
\pgfusepath{stroke}%
\end{pgfscope}%
\begin{pgfscope}%
\pgfsetbuttcap%
\pgfsetroundjoin%
\definecolor{currentfill}{rgb}{0.000000,0.000000,0.000000}%
\pgfsetfillcolor{currentfill}%
\pgfsetlinewidth{0.803000pt}%
\definecolor{currentstroke}{rgb}{0.000000,0.000000,0.000000}%
\pgfsetstrokecolor{currentstroke}%
\pgfsetdash{}{0pt}%
\pgfsys@defobject{currentmarker}{\pgfqpoint{-0.048611in}{0.000000in}}{\pgfqpoint{0.000000in}{0.000000in}}{%
\pgfpathmoveto{\pgfqpoint{0.000000in}{0.000000in}}%
\pgfpathlineto{\pgfqpoint{-0.048611in}{0.000000in}}%
\pgfusepath{stroke,fill}%
}%
\begin{pgfscope}%
\pgfsys@transformshift{0.720000in}{2.598519in}%
\pgfsys@useobject{currentmarker}{}%
\end{pgfscope}%
\end{pgfscope}%
\begin{pgfscope}%
\definecolor{textcolor}{rgb}{0.000000,0.000000,0.000000}%
\pgfsetstrokecolor{textcolor}%
\pgfsetfillcolor{textcolor}%
\pgftext[x=0.314076in,y=2.545757in,left,base]{\color{textcolor}\sffamily\fontsize{10.000000}{12.000000}\selectfont 60\%}%
\end{pgfscope}%
\begin{pgfscope}%
\pgfpathrectangle{\pgfqpoint{0.720000in}{0.480000in}}{\pgfqpoint{7.040000in}{3.120000in}}%
\pgfusepath{clip}%
\pgfsetrectcap%
\pgfsetroundjoin%
\pgfsetlinewidth{0.803000pt}%
\definecolor{currentstroke}{rgb}{0.690196,0.690196,0.690196}%
\pgfsetstrokecolor{currentstroke}%
\pgfsetdash{}{0pt}%
\pgfpathmoveto{\pgfqpoint{0.720000in}{2.983704in}}%
\pgfpathlineto{\pgfqpoint{7.760000in}{2.983704in}}%
\pgfusepath{stroke}%
\end{pgfscope}%
\begin{pgfscope}%
\pgfsetbuttcap%
\pgfsetroundjoin%
\definecolor{currentfill}{rgb}{0.000000,0.000000,0.000000}%
\pgfsetfillcolor{currentfill}%
\pgfsetlinewidth{0.803000pt}%
\definecolor{currentstroke}{rgb}{0.000000,0.000000,0.000000}%
\pgfsetstrokecolor{currentstroke}%
\pgfsetdash{}{0pt}%
\pgfsys@defobject{currentmarker}{\pgfqpoint{-0.048611in}{0.000000in}}{\pgfqpoint{0.000000in}{0.000000in}}{%
\pgfpathmoveto{\pgfqpoint{0.000000in}{0.000000in}}%
\pgfpathlineto{\pgfqpoint{-0.048611in}{0.000000in}}%
\pgfusepath{stroke,fill}%
}%
\begin{pgfscope}%
\pgfsys@transformshift{0.720000in}{2.983704in}%
\pgfsys@useobject{currentmarker}{}%
\end{pgfscope}%
\end{pgfscope}%
\begin{pgfscope}%
\definecolor{textcolor}{rgb}{0.000000,0.000000,0.000000}%
\pgfsetstrokecolor{textcolor}%
\pgfsetfillcolor{textcolor}%
\pgftext[x=0.314076in,y=2.930942in,left,base]{\color{textcolor}\sffamily\fontsize{10.000000}{12.000000}\selectfont 70\%}%
\end{pgfscope}%
\begin{pgfscope}%
\pgfpathrectangle{\pgfqpoint{0.720000in}{0.480000in}}{\pgfqpoint{7.040000in}{3.120000in}}%
\pgfusepath{clip}%
\pgfsetrectcap%
\pgfsetroundjoin%
\pgfsetlinewidth{0.803000pt}%
\definecolor{currentstroke}{rgb}{0.690196,0.690196,0.690196}%
\pgfsetstrokecolor{currentstroke}%
\pgfsetdash{}{0pt}%
\pgfpathmoveto{\pgfqpoint{0.720000in}{3.368889in}}%
\pgfpathlineto{\pgfqpoint{7.760000in}{3.368889in}}%
\pgfusepath{stroke}%
\end{pgfscope}%
\begin{pgfscope}%
\pgfsetbuttcap%
\pgfsetroundjoin%
\definecolor{currentfill}{rgb}{0.000000,0.000000,0.000000}%
\pgfsetfillcolor{currentfill}%
\pgfsetlinewidth{0.803000pt}%
\definecolor{currentstroke}{rgb}{0.000000,0.000000,0.000000}%
\pgfsetstrokecolor{currentstroke}%
\pgfsetdash{}{0pt}%
\pgfsys@defobject{currentmarker}{\pgfqpoint{-0.048611in}{0.000000in}}{\pgfqpoint{0.000000in}{0.000000in}}{%
\pgfpathmoveto{\pgfqpoint{0.000000in}{0.000000in}}%
\pgfpathlineto{\pgfqpoint{-0.048611in}{0.000000in}}%
\pgfusepath{stroke,fill}%
}%
\begin{pgfscope}%
\pgfsys@transformshift{0.720000in}{3.368889in}%
\pgfsys@useobject{currentmarker}{}%
\end{pgfscope}%
\end{pgfscope}%
\begin{pgfscope}%
\definecolor{textcolor}{rgb}{0.000000,0.000000,0.000000}%
\pgfsetstrokecolor{textcolor}%
\pgfsetfillcolor{textcolor}%
\pgftext[x=0.314076in,y=3.316127in,left,base]{\color{textcolor}\sffamily\fontsize{10.000000}{12.000000}\selectfont 80\%}%
\end{pgfscope}%
\begin{pgfscope}%
\definecolor{textcolor}{rgb}{0.000000,0.000000,0.000000}%
\pgfsetstrokecolor{textcolor}%
\pgfsetfillcolor{textcolor}%
\pgftext[x=0.258520in,y=2.040000in,,bottom,rotate=90.000000]{\color{textcolor}\sffamily\fontsize{10.000000}{12.000000}\selectfont accuracy validation top-1 [\%]}%
\end{pgfscope}%
\begin{pgfscope}%
\pgfpathrectangle{\pgfqpoint{0.720000in}{0.480000in}}{\pgfqpoint{7.040000in}{3.120000in}}%
\pgfusepath{clip}%
\pgfsetrectcap%
\pgfsetroundjoin%
\pgfsetlinewidth{0.752812pt}%
\definecolor{currentstroke}{rgb}{0.000000,0.000000,1.000000}%
\pgfsetstrokecolor{currentstroke}%
\pgfsetdash{}{0pt}%
\pgfpathmoveto{\pgfqpoint{0.834540in}{2.867378in}}%
\pgfpathlineto{\pgfqpoint{0.943771in}{3.105807in}}%
\pgfpathlineto{\pgfqpoint{1.052165in}{3.254104in}}%
\pgfpathlineto{\pgfqpoint{1.160838in}{3.197481in}}%
\pgfpathlineto{\pgfqpoint{1.269790in}{3.184385in}}%
\pgfpathlineto{\pgfqpoint{1.378743in}{3.319585in}}%
\pgfpathlineto{\pgfqpoint{1.487695in}{3.244859in}}%
\pgfpathlineto{\pgfqpoint{1.596368in}{3.410104in}}%
\pgfpathlineto{\pgfqpoint{1.705321in}{3.433600in}}%
\pgfpathlineto{\pgfqpoint{1.814552in}{3.397007in}}%
\pgfpathlineto{\pgfqpoint{1.923784in}{3.441689in}}%
\pgfpathlineto{\pgfqpoint{2.032737in}{3.424356in}}%
\pgfpathlineto{\pgfqpoint{2.141968in}{3.438993in}}%
\pgfpathlineto{\pgfqpoint{2.250641in}{3.422044in}}%
\pgfpathlineto{\pgfqpoint{2.359314in}{3.449393in}}%
\pgfpathlineto{\pgfqpoint{2.467987in}{3.477126in}}%
\pgfpathlineto{\pgfqpoint{2.577219in}{3.487526in}}%
\pgfpathlineto{\pgfqpoint{2.685892in}{3.449393in}}%
\pgfpathlineto{\pgfqpoint{2.794286in}{3.465185in}}%
\pgfpathlineto{\pgfqpoint{2.902679in}{3.444000in}}%
\pgfpathlineto{\pgfqpoint{3.011073in}{3.453244in}}%
\pgfusepath{stroke}%
\end{pgfscope}%
\begin{pgfscope}%
\pgfpathrectangle{\pgfqpoint{0.720000in}{0.480000in}}{\pgfqpoint{7.040000in}{3.120000in}}%
\pgfusepath{clip}%
\pgfsetbuttcap%
\pgfsetroundjoin%
\definecolor{currentfill}{rgb}{0.000000,0.000000,1.000000}%
\pgfsetfillcolor{currentfill}%
\pgfsetlinewidth{1.003750pt}%
\definecolor{currentstroke}{rgb}{0.000000,0.000000,1.000000}%
\pgfsetstrokecolor{currentstroke}%
\pgfsetdash{}{0pt}%
\pgfsys@defobject{currentmarker}{\pgfqpoint{-0.010417in}{-0.010417in}}{\pgfqpoint{0.010417in}{0.010417in}}{%
\pgfpathmoveto{\pgfqpoint{0.000000in}{-0.010417in}}%
\pgfpathcurveto{\pgfqpoint{0.002763in}{-0.010417in}}{\pgfqpoint{0.005412in}{-0.009319in}}{\pgfqpoint{0.007366in}{-0.007366in}}%
\pgfpathcurveto{\pgfqpoint{0.009319in}{-0.005412in}}{\pgfqpoint{0.010417in}{-0.002763in}}{\pgfqpoint{0.010417in}{0.000000in}}%
\pgfpathcurveto{\pgfqpoint{0.010417in}{0.002763in}}{\pgfqpoint{0.009319in}{0.005412in}}{\pgfqpoint{0.007366in}{0.007366in}}%
\pgfpathcurveto{\pgfqpoint{0.005412in}{0.009319in}}{\pgfqpoint{0.002763in}{0.010417in}}{\pgfqpoint{0.000000in}{0.010417in}}%
\pgfpathcurveto{\pgfqpoint{-0.002763in}{0.010417in}}{\pgfqpoint{-0.005412in}{0.009319in}}{\pgfqpoint{-0.007366in}{0.007366in}}%
\pgfpathcurveto{\pgfqpoint{-0.009319in}{0.005412in}}{\pgfqpoint{-0.010417in}{0.002763in}}{\pgfqpoint{-0.010417in}{0.000000in}}%
\pgfpathcurveto{\pgfqpoint{-0.010417in}{-0.002763in}}{\pgfqpoint{-0.009319in}{-0.005412in}}{\pgfqpoint{-0.007366in}{-0.007366in}}%
\pgfpathcurveto{\pgfqpoint{-0.005412in}{-0.009319in}}{\pgfqpoint{-0.002763in}{-0.010417in}}{\pgfqpoint{0.000000in}{-0.010417in}}%
\pgfpathclose%
\pgfusepath{stroke,fill}%
}%
\begin{pgfscope}%
\pgfsys@transformshift{0.834540in}{2.867378in}%
\pgfsys@useobject{currentmarker}{}%
\end{pgfscope}%
\begin{pgfscope}%
\pgfsys@transformshift{0.943771in}{3.105807in}%
\pgfsys@useobject{currentmarker}{}%
\end{pgfscope}%
\begin{pgfscope}%
\pgfsys@transformshift{1.052165in}{3.254104in}%
\pgfsys@useobject{currentmarker}{}%
\end{pgfscope}%
\begin{pgfscope}%
\pgfsys@transformshift{1.160838in}{3.197481in}%
\pgfsys@useobject{currentmarker}{}%
\end{pgfscope}%
\begin{pgfscope}%
\pgfsys@transformshift{1.269790in}{3.184385in}%
\pgfsys@useobject{currentmarker}{}%
\end{pgfscope}%
\begin{pgfscope}%
\pgfsys@transformshift{1.378743in}{3.319585in}%
\pgfsys@useobject{currentmarker}{}%
\end{pgfscope}%
\begin{pgfscope}%
\pgfsys@transformshift{1.487695in}{3.244859in}%
\pgfsys@useobject{currentmarker}{}%
\end{pgfscope}%
\begin{pgfscope}%
\pgfsys@transformshift{1.596368in}{3.410104in}%
\pgfsys@useobject{currentmarker}{}%
\end{pgfscope}%
\begin{pgfscope}%
\pgfsys@transformshift{1.705321in}{3.433600in}%
\pgfsys@useobject{currentmarker}{}%
\end{pgfscope}%
\begin{pgfscope}%
\pgfsys@transformshift{1.814552in}{3.397007in}%
\pgfsys@useobject{currentmarker}{}%
\end{pgfscope}%
\begin{pgfscope}%
\pgfsys@transformshift{1.923784in}{3.441689in}%
\pgfsys@useobject{currentmarker}{}%
\end{pgfscope}%
\begin{pgfscope}%
\pgfsys@transformshift{2.032737in}{3.424356in}%
\pgfsys@useobject{currentmarker}{}%
\end{pgfscope}%
\begin{pgfscope}%
\pgfsys@transformshift{2.141968in}{3.438993in}%
\pgfsys@useobject{currentmarker}{}%
\end{pgfscope}%
\begin{pgfscope}%
\pgfsys@transformshift{2.250641in}{3.422044in}%
\pgfsys@useobject{currentmarker}{}%
\end{pgfscope}%
\begin{pgfscope}%
\pgfsys@transformshift{2.359314in}{3.449393in}%
\pgfsys@useobject{currentmarker}{}%
\end{pgfscope}%
\begin{pgfscope}%
\pgfsys@transformshift{2.467987in}{3.477126in}%
\pgfsys@useobject{currentmarker}{}%
\end{pgfscope}%
\begin{pgfscope}%
\pgfsys@transformshift{2.577219in}{3.487526in}%
\pgfsys@useobject{currentmarker}{}%
\end{pgfscope}%
\begin{pgfscope}%
\pgfsys@transformshift{2.685892in}{3.449393in}%
\pgfsys@useobject{currentmarker}{}%
\end{pgfscope}%
\begin{pgfscope}%
\pgfsys@transformshift{2.794286in}{3.465185in}%
\pgfsys@useobject{currentmarker}{}%
\end{pgfscope}%
\begin{pgfscope}%
\pgfsys@transformshift{2.902679in}{3.444000in}%
\pgfsys@useobject{currentmarker}{}%
\end{pgfscope}%
\begin{pgfscope}%
\pgfsys@transformshift{3.011073in}{3.453244in}%
\pgfsys@useobject{currentmarker}{}%
\end{pgfscope}%
\end{pgfscope}%
\begin{pgfscope}%
\pgfpathrectangle{\pgfqpoint{0.720000in}{0.480000in}}{\pgfqpoint{7.040000in}{3.120000in}}%
\pgfusepath{clip}%
\pgfsetrectcap%
\pgfsetroundjoin%
\pgfsetlinewidth{0.752812pt}%
\definecolor{currentstroke}{rgb}{1.000000,0.647059,0.000000}%
\pgfsetstrokecolor{currentstroke}%
\pgfsetdash{}{0pt}%
\pgfpathmoveto{\pgfqpoint{0.976737in}{2.831941in}}%
\pgfpathlineto{\pgfqpoint{1.226489in}{3.096948in}}%
\pgfpathlineto{\pgfqpoint{1.476241in}{3.172830in}}%
\pgfpathlineto{\pgfqpoint{1.725435in}{3.205570in}}%
\pgfpathlineto{\pgfqpoint{1.973790in}{3.166281in}}%
\pgfpathlineto{\pgfqpoint{2.222146in}{3.279911in}}%
\pgfpathlineto{\pgfqpoint{2.470502in}{3.234074in}}%
\pgfpathlineto{\pgfqpoint{2.718857in}{3.421274in}}%
\pgfpathlineto{\pgfqpoint{2.967492in}{3.434370in}}%
\pgfpathlineto{\pgfqpoint{3.216127in}{3.423970in}}%
\pgfpathlineto{\pgfqpoint{3.464762in}{3.459022in}}%
\pgfpathlineto{\pgfqpoint{3.714235in}{3.484059in}}%
\pgfpathlineto{\pgfqpoint{3.964825in}{3.447467in}}%
\pgfpathlineto{\pgfqpoint{4.213460in}{3.456711in}}%
\pgfpathlineto{\pgfqpoint{4.462095in}{3.481363in}}%
\pgfpathlineto{\pgfqpoint{4.710730in}{3.514104in}}%
\pgfpathlineto{\pgfqpoint{4.959365in}{3.508711in}}%
\pgfpathlineto{\pgfqpoint{5.208000in}{3.514104in}}%
\pgfpathlineto{\pgfqpoint{5.456635in}{3.511407in}}%
\pgfpathlineto{\pgfqpoint{5.705270in}{3.487911in}}%
\pgfpathlineto{\pgfqpoint{5.953905in}{3.545304in}}%
\pgfusepath{stroke}%
\end{pgfscope}%
\begin{pgfscope}%
\pgfpathrectangle{\pgfqpoint{0.720000in}{0.480000in}}{\pgfqpoint{7.040000in}{3.120000in}}%
\pgfusepath{clip}%
\pgfsetbuttcap%
\pgfsetroundjoin%
\definecolor{currentfill}{rgb}{1.000000,0.647059,0.000000}%
\pgfsetfillcolor{currentfill}%
\pgfsetlinewidth{1.003750pt}%
\definecolor{currentstroke}{rgb}{1.000000,0.647059,0.000000}%
\pgfsetstrokecolor{currentstroke}%
\pgfsetdash{}{0pt}%
\pgfsys@defobject{currentmarker}{\pgfqpoint{-0.010417in}{-0.010417in}}{\pgfqpoint{0.010417in}{0.010417in}}{%
\pgfpathmoveto{\pgfqpoint{0.000000in}{-0.010417in}}%
\pgfpathcurveto{\pgfqpoint{0.002763in}{-0.010417in}}{\pgfqpoint{0.005412in}{-0.009319in}}{\pgfqpoint{0.007366in}{-0.007366in}}%
\pgfpathcurveto{\pgfqpoint{0.009319in}{-0.005412in}}{\pgfqpoint{0.010417in}{-0.002763in}}{\pgfqpoint{0.010417in}{0.000000in}}%
\pgfpathcurveto{\pgfqpoint{0.010417in}{0.002763in}}{\pgfqpoint{0.009319in}{0.005412in}}{\pgfqpoint{0.007366in}{0.007366in}}%
\pgfpathcurveto{\pgfqpoint{0.005412in}{0.009319in}}{\pgfqpoint{0.002763in}{0.010417in}}{\pgfqpoint{0.000000in}{0.010417in}}%
\pgfpathcurveto{\pgfqpoint{-0.002763in}{0.010417in}}{\pgfqpoint{-0.005412in}{0.009319in}}{\pgfqpoint{-0.007366in}{0.007366in}}%
\pgfpathcurveto{\pgfqpoint{-0.009319in}{0.005412in}}{\pgfqpoint{-0.010417in}{0.002763in}}{\pgfqpoint{-0.010417in}{0.000000in}}%
\pgfpathcurveto{\pgfqpoint{-0.010417in}{-0.002763in}}{\pgfqpoint{-0.009319in}{-0.005412in}}{\pgfqpoint{-0.007366in}{-0.007366in}}%
\pgfpathcurveto{\pgfqpoint{-0.005412in}{-0.009319in}}{\pgfqpoint{-0.002763in}{-0.010417in}}{\pgfqpoint{0.000000in}{-0.010417in}}%
\pgfpathclose%
\pgfusepath{stroke,fill}%
}%
\begin{pgfscope}%
\pgfsys@transformshift{0.976737in}{2.831941in}%
\pgfsys@useobject{currentmarker}{}%
\end{pgfscope}%
\begin{pgfscope}%
\pgfsys@transformshift{1.226489in}{3.096948in}%
\pgfsys@useobject{currentmarker}{}%
\end{pgfscope}%
\begin{pgfscope}%
\pgfsys@transformshift{1.476241in}{3.172830in}%
\pgfsys@useobject{currentmarker}{}%
\end{pgfscope}%
\begin{pgfscope}%
\pgfsys@transformshift{1.725435in}{3.205570in}%
\pgfsys@useobject{currentmarker}{}%
\end{pgfscope}%
\begin{pgfscope}%
\pgfsys@transformshift{1.973790in}{3.166281in}%
\pgfsys@useobject{currentmarker}{}%
\end{pgfscope}%
\begin{pgfscope}%
\pgfsys@transformshift{2.222146in}{3.279911in}%
\pgfsys@useobject{currentmarker}{}%
\end{pgfscope}%
\begin{pgfscope}%
\pgfsys@transformshift{2.470502in}{3.234074in}%
\pgfsys@useobject{currentmarker}{}%
\end{pgfscope}%
\begin{pgfscope}%
\pgfsys@transformshift{2.718857in}{3.421274in}%
\pgfsys@useobject{currentmarker}{}%
\end{pgfscope}%
\begin{pgfscope}%
\pgfsys@transformshift{2.967492in}{3.434370in}%
\pgfsys@useobject{currentmarker}{}%
\end{pgfscope}%
\begin{pgfscope}%
\pgfsys@transformshift{3.216127in}{3.423970in}%
\pgfsys@useobject{currentmarker}{}%
\end{pgfscope}%
\begin{pgfscope}%
\pgfsys@transformshift{3.464762in}{3.459022in}%
\pgfsys@useobject{currentmarker}{}%
\end{pgfscope}%
\begin{pgfscope}%
\pgfsys@transformshift{3.714235in}{3.484059in}%
\pgfsys@useobject{currentmarker}{}%
\end{pgfscope}%
\begin{pgfscope}%
\pgfsys@transformshift{3.964825in}{3.447467in}%
\pgfsys@useobject{currentmarker}{}%
\end{pgfscope}%
\begin{pgfscope}%
\pgfsys@transformshift{4.213460in}{3.456711in}%
\pgfsys@useobject{currentmarker}{}%
\end{pgfscope}%
\begin{pgfscope}%
\pgfsys@transformshift{4.462095in}{3.481363in}%
\pgfsys@useobject{currentmarker}{}%
\end{pgfscope}%
\begin{pgfscope}%
\pgfsys@transformshift{4.710730in}{3.514104in}%
\pgfsys@useobject{currentmarker}{}%
\end{pgfscope}%
\begin{pgfscope}%
\pgfsys@transformshift{4.959365in}{3.508711in}%
\pgfsys@useobject{currentmarker}{}%
\end{pgfscope}%
\begin{pgfscope}%
\pgfsys@transformshift{5.208000in}{3.514104in}%
\pgfsys@useobject{currentmarker}{}%
\end{pgfscope}%
\begin{pgfscope}%
\pgfsys@transformshift{5.456635in}{3.511407in}%
\pgfsys@useobject{currentmarker}{}%
\end{pgfscope}%
\begin{pgfscope}%
\pgfsys@transformshift{5.705270in}{3.487911in}%
\pgfsys@useobject{currentmarker}{}%
\end{pgfscope}%
\begin{pgfscope}%
\pgfsys@transformshift{5.953905in}{3.545304in}%
\pgfsys@useobject{currentmarker}{}%
\end{pgfscope}%
\end{pgfscope}%
\begin{pgfscope}%
\pgfpathrectangle{\pgfqpoint{0.720000in}{0.480000in}}{\pgfqpoint{7.040000in}{3.120000in}}%
\pgfusepath{clip}%
\pgfsetrectcap%
\pgfsetroundjoin%
\pgfsetlinewidth{0.752812pt}%
\definecolor{currentstroke}{rgb}{0.000000,0.501961,0.000000}%
\pgfsetstrokecolor{currentstroke}%
\pgfsetdash{}{0pt}%
\pgfpathmoveto{\pgfqpoint{0.913321in}{2.731407in}}%
\pgfpathlineto{\pgfqpoint{1.073117in}{3.079615in}}%
\pgfpathlineto{\pgfqpoint{1.232356in}{3.135852in}}%
\pgfpathlineto{\pgfqpoint{1.393270in}{3.272207in}}%
\pgfpathlineto{\pgfqpoint{1.551949in}{3.209422in}}%
\pgfpathlineto{\pgfqpoint{1.712584in}{3.269896in}}%
\pgfpathlineto{\pgfqpoint{1.872381in}{3.276444in}}%
\pgfpathlineto{\pgfqpoint{2.032737in}{3.341926in}}%
\pgfpathlineto{\pgfqpoint{2.193371in}{3.309185in}}%
\pgfpathlineto{\pgfqpoint{2.354286in}{3.338074in}}%
\pgfpathlineto{\pgfqpoint{2.513803in}{3.302637in}}%
\pgfpathlineto{\pgfqpoint{2.674159in}{3.341926in}}%
\pgfpathlineto{\pgfqpoint{2.835073in}{3.385067in}}%
\pgfpathlineto{\pgfqpoint{2.994032in}{3.339230in}}%
\pgfpathlineto{\pgfqpoint{3.154387in}{3.360415in}}%
\pgfpathlineto{\pgfqpoint{3.315581in}{3.335378in}}%
\pgfpathlineto{\pgfqpoint{3.474540in}{3.364267in}}%
\pgfpathlineto{\pgfqpoint{3.635175in}{3.353867in}}%
\pgfpathlineto{\pgfqpoint{3.794692in}{3.383911in}}%
\pgfpathlineto{\pgfqpoint{3.954210in}{3.333837in}}%
\pgfpathlineto{\pgfqpoint{4.114286in}{3.356178in}}%
\pgfusepath{stroke}%
\end{pgfscope}%
\begin{pgfscope}%
\pgfpathrectangle{\pgfqpoint{0.720000in}{0.480000in}}{\pgfqpoint{7.040000in}{3.120000in}}%
\pgfusepath{clip}%
\pgfsetbuttcap%
\pgfsetroundjoin%
\definecolor{currentfill}{rgb}{0.000000,0.501961,0.000000}%
\pgfsetfillcolor{currentfill}%
\pgfsetlinewidth{1.003750pt}%
\definecolor{currentstroke}{rgb}{0.000000,0.501961,0.000000}%
\pgfsetstrokecolor{currentstroke}%
\pgfsetdash{}{0pt}%
\pgfsys@defobject{currentmarker}{\pgfqpoint{-0.010417in}{-0.010417in}}{\pgfqpoint{0.010417in}{0.010417in}}{%
\pgfpathmoveto{\pgfqpoint{0.000000in}{-0.010417in}}%
\pgfpathcurveto{\pgfqpoint{0.002763in}{-0.010417in}}{\pgfqpoint{0.005412in}{-0.009319in}}{\pgfqpoint{0.007366in}{-0.007366in}}%
\pgfpathcurveto{\pgfqpoint{0.009319in}{-0.005412in}}{\pgfqpoint{0.010417in}{-0.002763in}}{\pgfqpoint{0.010417in}{0.000000in}}%
\pgfpathcurveto{\pgfqpoint{0.010417in}{0.002763in}}{\pgfqpoint{0.009319in}{0.005412in}}{\pgfqpoint{0.007366in}{0.007366in}}%
\pgfpathcurveto{\pgfqpoint{0.005412in}{0.009319in}}{\pgfqpoint{0.002763in}{0.010417in}}{\pgfqpoint{0.000000in}{0.010417in}}%
\pgfpathcurveto{\pgfqpoint{-0.002763in}{0.010417in}}{\pgfqpoint{-0.005412in}{0.009319in}}{\pgfqpoint{-0.007366in}{0.007366in}}%
\pgfpathcurveto{\pgfqpoint{-0.009319in}{0.005412in}}{\pgfqpoint{-0.010417in}{0.002763in}}{\pgfqpoint{-0.010417in}{0.000000in}}%
\pgfpathcurveto{\pgfqpoint{-0.010417in}{-0.002763in}}{\pgfqpoint{-0.009319in}{-0.005412in}}{\pgfqpoint{-0.007366in}{-0.007366in}}%
\pgfpathcurveto{\pgfqpoint{-0.005412in}{-0.009319in}}{\pgfqpoint{-0.002763in}{-0.010417in}}{\pgfqpoint{0.000000in}{-0.010417in}}%
\pgfpathclose%
\pgfusepath{stroke,fill}%
}%
\begin{pgfscope}%
\pgfsys@transformshift{0.913321in}{2.731407in}%
\pgfsys@useobject{currentmarker}{}%
\end{pgfscope}%
\begin{pgfscope}%
\pgfsys@transformshift{1.073117in}{3.079615in}%
\pgfsys@useobject{currentmarker}{}%
\end{pgfscope}%
\begin{pgfscope}%
\pgfsys@transformshift{1.232356in}{3.135852in}%
\pgfsys@useobject{currentmarker}{}%
\end{pgfscope}%
\begin{pgfscope}%
\pgfsys@transformshift{1.393270in}{3.272207in}%
\pgfsys@useobject{currentmarker}{}%
\end{pgfscope}%
\begin{pgfscope}%
\pgfsys@transformshift{1.551949in}{3.209422in}%
\pgfsys@useobject{currentmarker}{}%
\end{pgfscope}%
\begin{pgfscope}%
\pgfsys@transformshift{1.712584in}{3.269896in}%
\pgfsys@useobject{currentmarker}{}%
\end{pgfscope}%
\begin{pgfscope}%
\pgfsys@transformshift{1.872381in}{3.276444in}%
\pgfsys@useobject{currentmarker}{}%
\end{pgfscope}%
\begin{pgfscope}%
\pgfsys@transformshift{2.032737in}{3.341926in}%
\pgfsys@useobject{currentmarker}{}%
\end{pgfscope}%
\begin{pgfscope}%
\pgfsys@transformshift{2.193371in}{3.309185in}%
\pgfsys@useobject{currentmarker}{}%
\end{pgfscope}%
\begin{pgfscope}%
\pgfsys@transformshift{2.354286in}{3.338074in}%
\pgfsys@useobject{currentmarker}{}%
\end{pgfscope}%
\begin{pgfscope}%
\pgfsys@transformshift{2.513803in}{3.302637in}%
\pgfsys@useobject{currentmarker}{}%
\end{pgfscope}%
\begin{pgfscope}%
\pgfsys@transformshift{2.674159in}{3.341926in}%
\pgfsys@useobject{currentmarker}{}%
\end{pgfscope}%
\begin{pgfscope}%
\pgfsys@transformshift{2.835073in}{3.385067in}%
\pgfsys@useobject{currentmarker}{}%
\end{pgfscope}%
\begin{pgfscope}%
\pgfsys@transformshift{2.994032in}{3.339230in}%
\pgfsys@useobject{currentmarker}{}%
\end{pgfscope}%
\begin{pgfscope}%
\pgfsys@transformshift{3.154387in}{3.360415in}%
\pgfsys@useobject{currentmarker}{}%
\end{pgfscope}%
\begin{pgfscope}%
\pgfsys@transformshift{3.315581in}{3.335378in}%
\pgfsys@useobject{currentmarker}{}%
\end{pgfscope}%
\begin{pgfscope}%
\pgfsys@transformshift{3.474540in}{3.364267in}%
\pgfsys@useobject{currentmarker}{}%
\end{pgfscope}%
\begin{pgfscope}%
\pgfsys@transformshift{3.635175in}{3.353867in}%
\pgfsys@useobject{currentmarker}{}%
\end{pgfscope}%
\begin{pgfscope}%
\pgfsys@transformshift{3.794692in}{3.383911in}%
\pgfsys@useobject{currentmarker}{}%
\end{pgfscope}%
\begin{pgfscope}%
\pgfsys@transformshift{3.954210in}{3.333837in}%
\pgfsys@useobject{currentmarker}{}%
\end{pgfscope}%
\begin{pgfscope}%
\pgfsys@transformshift{4.114286in}{3.356178in}%
\pgfsys@useobject{currentmarker}{}%
\end{pgfscope}%
\end{pgfscope}%
\begin{pgfscope}%
\pgfpathrectangle{\pgfqpoint{0.720000in}{0.480000in}}{\pgfqpoint{7.040000in}{3.120000in}}%
\pgfusepath{clip}%
\pgfsetrectcap%
\pgfsetroundjoin%
\pgfsetlinewidth{0.752812pt}%
\definecolor{currentstroke}{rgb}{1.000000,0.000000,0.000000}%
\pgfsetstrokecolor{currentstroke}%
\pgfsetdash{}{0pt}%
\pgfpathmoveto{\pgfqpoint{0.844038in}{2.939407in}}%
\pgfpathlineto{\pgfqpoint{0.963606in}{3.022993in}}%
\pgfpathlineto{\pgfqpoint{1.082895in}{3.310341in}}%
\pgfpathlineto{\pgfqpoint{1.202184in}{3.329985in}}%
\pgfpathlineto{\pgfqpoint{1.321752in}{3.264504in}}%
\pgfpathlineto{\pgfqpoint{1.441041in}{3.345778in}}%
\pgfpathlineto{\pgfqpoint{1.560330in}{3.397007in}}%
\pgfpathlineto{\pgfqpoint{1.679898in}{3.474430in}}%
\pgfpathlineto{\pgfqpoint{1.799187in}{3.452089in}}%
\pgfpathlineto{\pgfqpoint{1.918476in}{3.442844in}}%
\pgfpathlineto{\pgfqpoint{2.037765in}{3.479437in}}%
\pgfpathlineto{\pgfqpoint{2.157054in}{3.496770in}}%
\pgfpathlineto{\pgfqpoint{2.276343in}{3.452089in}}%
\pgfpathlineto{\pgfqpoint{2.395632in}{3.472889in}}%
\pgfpathlineto{\pgfqpoint{2.514641in}{3.487526in}}%
\pgfpathlineto{\pgfqpoint{2.634489in}{3.472889in}}%
\pgfpathlineto{\pgfqpoint{2.753778in}{3.503319in}}%
\pgfpathlineto{\pgfqpoint{2.882006in}{3.487526in}}%
\pgfpathlineto{\pgfqpoint{3.001295in}{3.497926in}}%
\pgfpathlineto{\pgfqpoint{3.120584in}{3.496770in}}%
\pgfpathlineto{\pgfqpoint{3.239873in}{3.507170in}}%
\pgfusepath{stroke}%
\end{pgfscope}%
\begin{pgfscope}%
\pgfpathrectangle{\pgfqpoint{0.720000in}{0.480000in}}{\pgfqpoint{7.040000in}{3.120000in}}%
\pgfusepath{clip}%
\pgfsetbuttcap%
\pgfsetroundjoin%
\definecolor{currentfill}{rgb}{1.000000,0.000000,0.000000}%
\pgfsetfillcolor{currentfill}%
\pgfsetlinewidth{1.003750pt}%
\definecolor{currentstroke}{rgb}{1.000000,0.000000,0.000000}%
\pgfsetstrokecolor{currentstroke}%
\pgfsetdash{}{0pt}%
\pgfsys@defobject{currentmarker}{\pgfqpoint{-0.010417in}{-0.010417in}}{\pgfqpoint{0.010417in}{0.010417in}}{%
\pgfpathmoveto{\pgfqpoint{0.000000in}{-0.010417in}}%
\pgfpathcurveto{\pgfqpoint{0.002763in}{-0.010417in}}{\pgfqpoint{0.005412in}{-0.009319in}}{\pgfqpoint{0.007366in}{-0.007366in}}%
\pgfpathcurveto{\pgfqpoint{0.009319in}{-0.005412in}}{\pgfqpoint{0.010417in}{-0.002763in}}{\pgfqpoint{0.010417in}{0.000000in}}%
\pgfpathcurveto{\pgfqpoint{0.010417in}{0.002763in}}{\pgfqpoint{0.009319in}{0.005412in}}{\pgfqpoint{0.007366in}{0.007366in}}%
\pgfpathcurveto{\pgfqpoint{0.005412in}{0.009319in}}{\pgfqpoint{0.002763in}{0.010417in}}{\pgfqpoint{0.000000in}{0.010417in}}%
\pgfpathcurveto{\pgfqpoint{-0.002763in}{0.010417in}}{\pgfqpoint{-0.005412in}{0.009319in}}{\pgfqpoint{-0.007366in}{0.007366in}}%
\pgfpathcurveto{\pgfqpoint{-0.009319in}{0.005412in}}{\pgfqpoint{-0.010417in}{0.002763in}}{\pgfqpoint{-0.010417in}{0.000000in}}%
\pgfpathcurveto{\pgfqpoint{-0.010417in}{-0.002763in}}{\pgfqpoint{-0.009319in}{-0.005412in}}{\pgfqpoint{-0.007366in}{-0.007366in}}%
\pgfpathcurveto{\pgfqpoint{-0.005412in}{-0.009319in}}{\pgfqpoint{-0.002763in}{-0.010417in}}{\pgfqpoint{0.000000in}{-0.010417in}}%
\pgfpathclose%
\pgfusepath{stroke,fill}%
}%
\begin{pgfscope}%
\pgfsys@transformshift{0.844038in}{2.939407in}%
\pgfsys@useobject{currentmarker}{}%
\end{pgfscope}%
\begin{pgfscope}%
\pgfsys@transformshift{0.963606in}{3.022993in}%
\pgfsys@useobject{currentmarker}{}%
\end{pgfscope}%
\begin{pgfscope}%
\pgfsys@transformshift{1.082895in}{3.310341in}%
\pgfsys@useobject{currentmarker}{}%
\end{pgfscope}%
\begin{pgfscope}%
\pgfsys@transformshift{1.202184in}{3.329985in}%
\pgfsys@useobject{currentmarker}{}%
\end{pgfscope}%
\begin{pgfscope}%
\pgfsys@transformshift{1.321752in}{3.264504in}%
\pgfsys@useobject{currentmarker}{}%
\end{pgfscope}%
\begin{pgfscope}%
\pgfsys@transformshift{1.441041in}{3.345778in}%
\pgfsys@useobject{currentmarker}{}%
\end{pgfscope}%
\begin{pgfscope}%
\pgfsys@transformshift{1.560330in}{3.397007in}%
\pgfsys@useobject{currentmarker}{}%
\end{pgfscope}%
\begin{pgfscope}%
\pgfsys@transformshift{1.679898in}{3.474430in}%
\pgfsys@useobject{currentmarker}{}%
\end{pgfscope}%
\begin{pgfscope}%
\pgfsys@transformshift{1.799187in}{3.452089in}%
\pgfsys@useobject{currentmarker}{}%
\end{pgfscope}%
\begin{pgfscope}%
\pgfsys@transformshift{1.918476in}{3.442844in}%
\pgfsys@useobject{currentmarker}{}%
\end{pgfscope}%
\begin{pgfscope}%
\pgfsys@transformshift{2.037765in}{3.479437in}%
\pgfsys@useobject{currentmarker}{}%
\end{pgfscope}%
\begin{pgfscope}%
\pgfsys@transformshift{2.157054in}{3.496770in}%
\pgfsys@useobject{currentmarker}{}%
\end{pgfscope}%
\begin{pgfscope}%
\pgfsys@transformshift{2.276343in}{3.452089in}%
\pgfsys@useobject{currentmarker}{}%
\end{pgfscope}%
\begin{pgfscope}%
\pgfsys@transformshift{2.395632in}{3.472889in}%
\pgfsys@useobject{currentmarker}{}%
\end{pgfscope}%
\begin{pgfscope}%
\pgfsys@transformshift{2.514641in}{3.487526in}%
\pgfsys@useobject{currentmarker}{}%
\end{pgfscope}%
\begin{pgfscope}%
\pgfsys@transformshift{2.634489in}{3.472889in}%
\pgfsys@useobject{currentmarker}{}%
\end{pgfscope}%
\begin{pgfscope}%
\pgfsys@transformshift{2.753778in}{3.503319in}%
\pgfsys@useobject{currentmarker}{}%
\end{pgfscope}%
\begin{pgfscope}%
\pgfsys@transformshift{2.882006in}{3.487526in}%
\pgfsys@useobject{currentmarker}{}%
\end{pgfscope}%
\begin{pgfscope}%
\pgfsys@transformshift{3.001295in}{3.497926in}%
\pgfsys@useobject{currentmarker}{}%
\end{pgfscope}%
\begin{pgfscope}%
\pgfsys@transformshift{3.120584in}{3.496770in}%
\pgfsys@useobject{currentmarker}{}%
\end{pgfscope}%
\begin{pgfscope}%
\pgfsys@transformshift{3.239873in}{3.507170in}%
\pgfsys@useobject{currentmarker}{}%
\end{pgfscope}%
\end{pgfscope}%
\begin{pgfscope}%
\pgfpathrectangle{\pgfqpoint{0.720000in}{0.480000in}}{\pgfqpoint{7.040000in}{3.120000in}}%
\pgfusepath{clip}%
\pgfsetrectcap%
\pgfsetroundjoin%
\pgfsetlinewidth{0.752812pt}%
\definecolor{currentstroke}{rgb}{0.501961,0.000000,0.501961}%
\pgfsetstrokecolor{currentstroke}%
\pgfsetdash{}{0pt}%
\pgfpathmoveto{\pgfqpoint{2.063187in}{3.141244in}}%
\pgfpathlineto{\pgfqpoint{3.393803in}{3.180533in}}%
\pgfpathlineto{\pgfqpoint{4.724140in}{3.364652in}}%
\pgfpathlineto{\pgfqpoint{6.055873in}{3.340000in}}%
\pgfpathlineto{\pgfqpoint{7.387886in}{3.321511in}}%
\pgfpathlineto{\pgfqpoint{7.770000in}{3.318192in}}%
\pgfusepath{stroke}%
\end{pgfscope}%
\begin{pgfscope}%
\pgfpathrectangle{\pgfqpoint{0.720000in}{0.480000in}}{\pgfqpoint{7.040000in}{3.120000in}}%
\pgfusepath{clip}%
\pgfsetbuttcap%
\pgfsetroundjoin%
\definecolor{currentfill}{rgb}{0.501961,0.000000,0.501961}%
\pgfsetfillcolor{currentfill}%
\pgfsetlinewidth{1.003750pt}%
\definecolor{currentstroke}{rgb}{0.501961,0.000000,0.501961}%
\pgfsetstrokecolor{currentstroke}%
\pgfsetdash{}{0pt}%
\pgfsys@defobject{currentmarker}{\pgfqpoint{-0.010417in}{-0.010417in}}{\pgfqpoint{0.010417in}{0.010417in}}{%
\pgfpathmoveto{\pgfqpoint{0.000000in}{-0.010417in}}%
\pgfpathcurveto{\pgfqpoint{0.002763in}{-0.010417in}}{\pgfqpoint{0.005412in}{-0.009319in}}{\pgfqpoint{0.007366in}{-0.007366in}}%
\pgfpathcurveto{\pgfqpoint{0.009319in}{-0.005412in}}{\pgfqpoint{0.010417in}{-0.002763in}}{\pgfqpoint{0.010417in}{0.000000in}}%
\pgfpathcurveto{\pgfqpoint{0.010417in}{0.002763in}}{\pgfqpoint{0.009319in}{0.005412in}}{\pgfqpoint{0.007366in}{0.007366in}}%
\pgfpathcurveto{\pgfqpoint{0.005412in}{0.009319in}}{\pgfqpoint{0.002763in}{0.010417in}}{\pgfqpoint{0.000000in}{0.010417in}}%
\pgfpathcurveto{\pgfqpoint{-0.002763in}{0.010417in}}{\pgfqpoint{-0.005412in}{0.009319in}}{\pgfqpoint{-0.007366in}{0.007366in}}%
\pgfpathcurveto{\pgfqpoint{-0.009319in}{0.005412in}}{\pgfqpoint{-0.010417in}{0.002763in}}{\pgfqpoint{-0.010417in}{0.000000in}}%
\pgfpathcurveto{\pgfqpoint{-0.010417in}{-0.002763in}}{\pgfqpoint{-0.009319in}{-0.005412in}}{\pgfqpoint{-0.007366in}{-0.007366in}}%
\pgfpathcurveto{\pgfqpoint{-0.005412in}{-0.009319in}}{\pgfqpoint{-0.002763in}{-0.010417in}}{\pgfqpoint{0.000000in}{-0.010417in}}%
\pgfpathclose%
\pgfusepath{stroke,fill}%
}%
\begin{pgfscope}%
\pgfsys@transformshift{2.063187in}{3.141244in}%
\pgfsys@useobject{currentmarker}{}%
\end{pgfscope}%
\begin{pgfscope}%
\pgfsys@transformshift{3.393803in}{3.180533in}%
\pgfsys@useobject{currentmarker}{}%
\end{pgfscope}%
\begin{pgfscope}%
\pgfsys@transformshift{4.724140in}{3.364652in}%
\pgfsys@useobject{currentmarker}{}%
\end{pgfscope}%
\begin{pgfscope}%
\pgfsys@transformshift{6.055873in}{3.340000in}%
\pgfsys@useobject{currentmarker}{}%
\end{pgfscope}%
\begin{pgfscope}%
\pgfsys@transformshift{7.387886in}{3.321511in}%
\pgfsys@useobject{currentmarker}{}%
\end{pgfscope}%
\begin{pgfscope}%
\pgfsys@transformshift{8.718222in}{3.309956in}%
\pgfsys@useobject{currentmarker}{}%
\end{pgfscope}%
\begin{pgfscope}%
\pgfsys@transformshift{10.049117in}{3.363496in}%
\pgfsys@useobject{currentmarker}{}%
\end{pgfscope}%
\begin{pgfscope}%
\pgfsys@transformshift{11.379733in}{3.500622in}%
\pgfsys@useobject{currentmarker}{}%
\end{pgfscope}%
\begin{pgfscope}%
\pgfsys@transformshift{12.711467in}{3.469037in}%
\pgfsys@useobject{currentmarker}{}%
\end{pgfscope}%
\begin{pgfscope}%
\pgfsys@transformshift{14.047390in}{3.456326in}%
\pgfsys@useobject{currentmarker}{}%
\end{pgfscope}%
\begin{pgfscope}%
\pgfsys@transformshift{15.382756in}{3.475585in}%
\pgfsys@useobject{currentmarker}{}%
\end{pgfscope}%
\begin{pgfscope}%
\pgfsys@transformshift{16.725663in}{3.492533in}%
\pgfsys@useobject{currentmarker}{}%
\end{pgfscope}%
\begin{pgfscope}%
\pgfsys@transformshift{18.072483in}{3.475585in}%
\pgfsys@useobject{currentmarker}{}%
\end{pgfscope}%
\begin{pgfscope}%
\pgfsys@transformshift{19.416787in}{3.513719in}%
\pgfsys@useobject{currentmarker}{}%
\end{pgfscope}%
\begin{pgfscope}%
\pgfsys@transformshift{20.748241in}{3.508326in}%
\pgfsys@useobject{currentmarker}{}%
\end{pgfscope}%
\begin{pgfscope}%
\pgfsys@transformshift{22.078019in}{3.496770in}%
\pgfsys@useobject{currentmarker}{}%
\end{pgfscope}%
\begin{pgfscope}%
\pgfsys@transformshift{23.410032in}{3.507170in}%
\pgfsys@useobject{currentmarker}{}%
\end{pgfscope}%
\begin{pgfscope}%
\pgfsys@transformshift{24.857702in}{3.511022in}%
\pgfsys@useobject{currentmarker}{}%
\end{pgfscope}%
\begin{pgfscope}%
\pgfsys@transformshift{26.204241in}{3.494074in}%
\pgfsys@useobject{currentmarker}{}%
\end{pgfscope}%
\begin{pgfscope}%
\pgfsys@transformshift{27.543238in}{3.497926in}%
\pgfsys@useobject{currentmarker}{}%
\end{pgfscope}%
\begin{pgfscope}%
\pgfsys@transformshift{28.875530in}{3.508326in}%
\pgfsys@useobject{currentmarker}{}%
\end{pgfscope}%
\end{pgfscope}%
\begin{pgfscope}%
\pgfpathrectangle{\pgfqpoint{0.720000in}{0.480000in}}{\pgfqpoint{7.040000in}{3.120000in}}%
\pgfusepath{clip}%
\pgfsetrectcap%
\pgfsetroundjoin%
\pgfsetlinewidth{0.752812pt}%
\definecolor{currentstroke}{rgb}{0.392157,0.584314,0.929412}%
\pgfsetstrokecolor{currentstroke}%
\pgfsetdash{}{0pt}%
\pgfpathmoveto{\pgfqpoint{0.813308in}{2.931704in}}%
\pgfpathlineto{\pgfqpoint{0.912203in}{3.048030in}}%
\pgfpathlineto{\pgfqpoint{1.010819in}{3.190933in}}%
\pgfpathlineto{\pgfqpoint{1.109435in}{3.195170in}}%
\pgfpathlineto{\pgfqpoint{1.208330in}{3.206726in}}%
\pgfpathlineto{\pgfqpoint{1.306946in}{3.210578in}}%
\pgfpathlineto{\pgfqpoint{1.405562in}{3.196326in}}%
\pgfpathlineto{\pgfqpoint{1.504178in}{3.343081in}}%
\pgfpathlineto{\pgfqpoint{1.602794in}{3.358874in}}%
\pgfpathlineto{\pgfqpoint{1.701410in}{3.343081in}}%
\pgfpathlineto{\pgfqpoint{1.800025in}{3.391615in}}%
\pgfpathlineto{\pgfqpoint{1.898921in}{3.403556in}}%
\pgfpathlineto{\pgfqpoint{1.997816in}{3.341926in}}%
\pgfpathlineto{\pgfqpoint{2.096711in}{3.399704in}}%
\pgfpathlineto{\pgfqpoint{2.195327in}{3.400859in}}%
\pgfpathlineto{\pgfqpoint{2.293943in}{3.397007in}}%
\pgfpathlineto{\pgfqpoint{2.392559in}{3.412800in}}%
\pgfpathlineto{\pgfqpoint{2.491454in}{3.406252in}}%
\pgfpathlineto{\pgfqpoint{2.590070in}{3.415496in}}%
\pgfpathlineto{\pgfqpoint{2.688686in}{3.406252in}}%
\pgfpathlineto{\pgfqpoint{2.787302in}{3.391615in}}%
\pgfusepath{stroke}%
\end{pgfscope}%
\begin{pgfscope}%
\pgfpathrectangle{\pgfqpoint{0.720000in}{0.480000in}}{\pgfqpoint{7.040000in}{3.120000in}}%
\pgfusepath{clip}%
\pgfsetbuttcap%
\pgfsetroundjoin%
\definecolor{currentfill}{rgb}{0.392157,0.584314,0.929412}%
\pgfsetfillcolor{currentfill}%
\pgfsetlinewidth{1.003750pt}%
\definecolor{currentstroke}{rgb}{0.392157,0.584314,0.929412}%
\pgfsetstrokecolor{currentstroke}%
\pgfsetdash{}{0pt}%
\pgfsys@defobject{currentmarker}{\pgfqpoint{-0.010417in}{-0.010417in}}{\pgfqpoint{0.010417in}{0.010417in}}{%
\pgfpathmoveto{\pgfqpoint{0.000000in}{-0.010417in}}%
\pgfpathcurveto{\pgfqpoint{0.002763in}{-0.010417in}}{\pgfqpoint{0.005412in}{-0.009319in}}{\pgfqpoint{0.007366in}{-0.007366in}}%
\pgfpathcurveto{\pgfqpoint{0.009319in}{-0.005412in}}{\pgfqpoint{0.010417in}{-0.002763in}}{\pgfqpoint{0.010417in}{0.000000in}}%
\pgfpathcurveto{\pgfqpoint{0.010417in}{0.002763in}}{\pgfqpoint{0.009319in}{0.005412in}}{\pgfqpoint{0.007366in}{0.007366in}}%
\pgfpathcurveto{\pgfqpoint{0.005412in}{0.009319in}}{\pgfqpoint{0.002763in}{0.010417in}}{\pgfqpoint{0.000000in}{0.010417in}}%
\pgfpathcurveto{\pgfqpoint{-0.002763in}{0.010417in}}{\pgfqpoint{-0.005412in}{0.009319in}}{\pgfqpoint{-0.007366in}{0.007366in}}%
\pgfpathcurveto{\pgfqpoint{-0.009319in}{0.005412in}}{\pgfqpoint{-0.010417in}{0.002763in}}{\pgfqpoint{-0.010417in}{0.000000in}}%
\pgfpathcurveto{\pgfqpoint{-0.010417in}{-0.002763in}}{\pgfqpoint{-0.009319in}{-0.005412in}}{\pgfqpoint{-0.007366in}{-0.007366in}}%
\pgfpathcurveto{\pgfqpoint{-0.005412in}{-0.009319in}}{\pgfqpoint{-0.002763in}{-0.010417in}}{\pgfqpoint{0.000000in}{-0.010417in}}%
\pgfpathclose%
\pgfusepath{stroke,fill}%
}%
\begin{pgfscope}%
\pgfsys@transformshift{0.813308in}{2.931704in}%
\pgfsys@useobject{currentmarker}{}%
\end{pgfscope}%
\begin{pgfscope}%
\pgfsys@transformshift{0.912203in}{3.048030in}%
\pgfsys@useobject{currentmarker}{}%
\end{pgfscope}%
\begin{pgfscope}%
\pgfsys@transformshift{1.010819in}{3.190933in}%
\pgfsys@useobject{currentmarker}{}%
\end{pgfscope}%
\begin{pgfscope}%
\pgfsys@transformshift{1.109435in}{3.195170in}%
\pgfsys@useobject{currentmarker}{}%
\end{pgfscope}%
\begin{pgfscope}%
\pgfsys@transformshift{1.208330in}{3.206726in}%
\pgfsys@useobject{currentmarker}{}%
\end{pgfscope}%
\begin{pgfscope}%
\pgfsys@transformshift{1.306946in}{3.210578in}%
\pgfsys@useobject{currentmarker}{}%
\end{pgfscope}%
\begin{pgfscope}%
\pgfsys@transformshift{1.405562in}{3.196326in}%
\pgfsys@useobject{currentmarker}{}%
\end{pgfscope}%
\begin{pgfscope}%
\pgfsys@transformshift{1.504178in}{3.343081in}%
\pgfsys@useobject{currentmarker}{}%
\end{pgfscope}%
\begin{pgfscope}%
\pgfsys@transformshift{1.602794in}{3.358874in}%
\pgfsys@useobject{currentmarker}{}%
\end{pgfscope}%
\begin{pgfscope}%
\pgfsys@transformshift{1.701410in}{3.343081in}%
\pgfsys@useobject{currentmarker}{}%
\end{pgfscope}%
\begin{pgfscope}%
\pgfsys@transformshift{1.800025in}{3.391615in}%
\pgfsys@useobject{currentmarker}{}%
\end{pgfscope}%
\begin{pgfscope}%
\pgfsys@transformshift{1.898921in}{3.403556in}%
\pgfsys@useobject{currentmarker}{}%
\end{pgfscope}%
\begin{pgfscope}%
\pgfsys@transformshift{1.997816in}{3.341926in}%
\pgfsys@useobject{currentmarker}{}%
\end{pgfscope}%
\begin{pgfscope}%
\pgfsys@transformshift{2.096711in}{3.399704in}%
\pgfsys@useobject{currentmarker}{}%
\end{pgfscope}%
\begin{pgfscope}%
\pgfsys@transformshift{2.195327in}{3.400859in}%
\pgfsys@useobject{currentmarker}{}%
\end{pgfscope}%
\begin{pgfscope}%
\pgfsys@transformshift{2.293943in}{3.397007in}%
\pgfsys@useobject{currentmarker}{}%
\end{pgfscope}%
\begin{pgfscope}%
\pgfsys@transformshift{2.392559in}{3.412800in}%
\pgfsys@useobject{currentmarker}{}%
\end{pgfscope}%
\begin{pgfscope}%
\pgfsys@transformshift{2.491454in}{3.406252in}%
\pgfsys@useobject{currentmarker}{}%
\end{pgfscope}%
\begin{pgfscope}%
\pgfsys@transformshift{2.590070in}{3.415496in}%
\pgfsys@useobject{currentmarker}{}%
\end{pgfscope}%
\begin{pgfscope}%
\pgfsys@transformshift{2.688686in}{3.406252in}%
\pgfsys@useobject{currentmarker}{}%
\end{pgfscope}%
\begin{pgfscope}%
\pgfsys@transformshift{2.787302in}{3.391615in}%
\pgfsys@useobject{currentmarker}{}%
\end{pgfscope}%
\end{pgfscope}%
\begin{pgfscope}%
\pgfpathrectangle{\pgfqpoint{0.720000in}{0.480000in}}{\pgfqpoint{7.040000in}{3.120000in}}%
\pgfusepath{clip}%
\pgfsetrectcap%
\pgfsetroundjoin%
\pgfsetlinewidth{0.752812pt}%
\definecolor{currentstroke}{rgb}{1.000000,0.752941,0.796078}%
\pgfsetstrokecolor{currentstroke}%
\pgfsetdash{}{0pt}%
\pgfpathmoveto{\pgfqpoint{0.813587in}{0.509659in}}%
\pgfpathlineto{\pgfqpoint{0.921143in}{0.615200in}}%
\pgfpathlineto{\pgfqpoint{1.028140in}{1.200296in}}%
\pgfpathlineto{\pgfqpoint{1.135137in}{1.865126in}}%
\pgfpathlineto{\pgfqpoint{1.242133in}{2.392444in}}%
\pgfpathlineto{\pgfqpoint{1.349130in}{2.482963in}}%
\pgfpathlineto{\pgfqpoint{1.456127in}{2.619319in}}%
\pgfpathlineto{\pgfqpoint{1.563124in}{3.025689in}}%
\pgfpathlineto{\pgfqpoint{1.670121in}{3.007585in}}%
\pgfpathlineto{\pgfqpoint{1.777117in}{3.022993in}}%
\pgfpathlineto{\pgfqpoint{1.884114in}{2.915526in}}%
\pgfpathlineto{\pgfqpoint{1.991390in}{2.996800in}}%
\pgfpathlineto{\pgfqpoint{2.098387in}{2.961363in}}%
\pgfpathlineto{\pgfqpoint{2.205384in}{2.978696in}}%
\pgfpathlineto{\pgfqpoint{2.312381in}{3.184385in}}%
\pgfpathlineto{\pgfqpoint{2.419378in}{3.159733in}}%
\pgfpathlineto{\pgfqpoint{2.526375in}{3.197481in}}%
\pgfpathlineto{\pgfqpoint{2.633371in}{3.172830in}}%
\pgfpathlineto{\pgfqpoint{2.740368in}{3.188237in}}%
\pgfpathlineto{\pgfqpoint{2.847365in}{3.113511in}}%
\pgfpathlineto{\pgfqpoint{2.954362in}{3.128148in}}%
\pgfusepath{stroke}%
\end{pgfscope}%
\begin{pgfscope}%
\pgfpathrectangle{\pgfqpoint{0.720000in}{0.480000in}}{\pgfqpoint{7.040000in}{3.120000in}}%
\pgfusepath{clip}%
\pgfsetbuttcap%
\pgfsetroundjoin%
\definecolor{currentfill}{rgb}{1.000000,0.752941,0.796078}%
\pgfsetfillcolor{currentfill}%
\pgfsetlinewidth{1.003750pt}%
\definecolor{currentstroke}{rgb}{1.000000,0.752941,0.796078}%
\pgfsetstrokecolor{currentstroke}%
\pgfsetdash{}{0pt}%
\pgfsys@defobject{currentmarker}{\pgfqpoint{-0.010417in}{-0.010417in}}{\pgfqpoint{0.010417in}{0.010417in}}{%
\pgfpathmoveto{\pgfqpoint{0.000000in}{-0.010417in}}%
\pgfpathcurveto{\pgfqpoint{0.002763in}{-0.010417in}}{\pgfqpoint{0.005412in}{-0.009319in}}{\pgfqpoint{0.007366in}{-0.007366in}}%
\pgfpathcurveto{\pgfqpoint{0.009319in}{-0.005412in}}{\pgfqpoint{0.010417in}{-0.002763in}}{\pgfqpoint{0.010417in}{0.000000in}}%
\pgfpathcurveto{\pgfqpoint{0.010417in}{0.002763in}}{\pgfqpoint{0.009319in}{0.005412in}}{\pgfqpoint{0.007366in}{0.007366in}}%
\pgfpathcurveto{\pgfqpoint{0.005412in}{0.009319in}}{\pgfqpoint{0.002763in}{0.010417in}}{\pgfqpoint{0.000000in}{0.010417in}}%
\pgfpathcurveto{\pgfqpoint{-0.002763in}{0.010417in}}{\pgfqpoint{-0.005412in}{0.009319in}}{\pgfqpoint{-0.007366in}{0.007366in}}%
\pgfpathcurveto{\pgfqpoint{-0.009319in}{0.005412in}}{\pgfqpoint{-0.010417in}{0.002763in}}{\pgfqpoint{-0.010417in}{0.000000in}}%
\pgfpathcurveto{\pgfqpoint{-0.010417in}{-0.002763in}}{\pgfqpoint{-0.009319in}{-0.005412in}}{\pgfqpoint{-0.007366in}{-0.007366in}}%
\pgfpathcurveto{\pgfqpoint{-0.005412in}{-0.009319in}}{\pgfqpoint{-0.002763in}{-0.010417in}}{\pgfqpoint{0.000000in}{-0.010417in}}%
\pgfpathclose%
\pgfusepath{stroke,fill}%
}%
\begin{pgfscope}%
\pgfsys@transformshift{0.813587in}{0.509659in}%
\pgfsys@useobject{currentmarker}{}%
\end{pgfscope}%
\begin{pgfscope}%
\pgfsys@transformshift{0.921143in}{0.615200in}%
\pgfsys@useobject{currentmarker}{}%
\end{pgfscope}%
\begin{pgfscope}%
\pgfsys@transformshift{1.028140in}{1.200296in}%
\pgfsys@useobject{currentmarker}{}%
\end{pgfscope}%
\begin{pgfscope}%
\pgfsys@transformshift{1.135137in}{1.865126in}%
\pgfsys@useobject{currentmarker}{}%
\end{pgfscope}%
\begin{pgfscope}%
\pgfsys@transformshift{1.242133in}{2.392444in}%
\pgfsys@useobject{currentmarker}{}%
\end{pgfscope}%
\begin{pgfscope}%
\pgfsys@transformshift{1.349130in}{2.482963in}%
\pgfsys@useobject{currentmarker}{}%
\end{pgfscope}%
\begin{pgfscope}%
\pgfsys@transformshift{1.456127in}{2.619319in}%
\pgfsys@useobject{currentmarker}{}%
\end{pgfscope}%
\begin{pgfscope}%
\pgfsys@transformshift{1.563124in}{3.025689in}%
\pgfsys@useobject{currentmarker}{}%
\end{pgfscope}%
\begin{pgfscope}%
\pgfsys@transformshift{1.670121in}{3.007585in}%
\pgfsys@useobject{currentmarker}{}%
\end{pgfscope}%
\begin{pgfscope}%
\pgfsys@transformshift{1.777117in}{3.022993in}%
\pgfsys@useobject{currentmarker}{}%
\end{pgfscope}%
\begin{pgfscope}%
\pgfsys@transformshift{1.884114in}{2.915526in}%
\pgfsys@useobject{currentmarker}{}%
\end{pgfscope}%
\begin{pgfscope}%
\pgfsys@transformshift{1.991390in}{2.996800in}%
\pgfsys@useobject{currentmarker}{}%
\end{pgfscope}%
\begin{pgfscope}%
\pgfsys@transformshift{2.098387in}{2.961363in}%
\pgfsys@useobject{currentmarker}{}%
\end{pgfscope}%
\begin{pgfscope}%
\pgfsys@transformshift{2.205384in}{2.978696in}%
\pgfsys@useobject{currentmarker}{}%
\end{pgfscope}%
\begin{pgfscope}%
\pgfsys@transformshift{2.312381in}{3.184385in}%
\pgfsys@useobject{currentmarker}{}%
\end{pgfscope}%
\begin{pgfscope}%
\pgfsys@transformshift{2.419378in}{3.159733in}%
\pgfsys@useobject{currentmarker}{}%
\end{pgfscope}%
\begin{pgfscope}%
\pgfsys@transformshift{2.526375in}{3.197481in}%
\pgfsys@useobject{currentmarker}{}%
\end{pgfscope}%
\begin{pgfscope}%
\pgfsys@transformshift{2.633371in}{3.172830in}%
\pgfsys@useobject{currentmarker}{}%
\end{pgfscope}%
\begin{pgfscope}%
\pgfsys@transformshift{2.740368in}{3.188237in}%
\pgfsys@useobject{currentmarker}{}%
\end{pgfscope}%
\begin{pgfscope}%
\pgfsys@transformshift{2.847365in}{3.113511in}%
\pgfsys@useobject{currentmarker}{}%
\end{pgfscope}%
\begin{pgfscope}%
\pgfsys@transformshift{2.954362in}{3.128148in}%
\pgfsys@useobject{currentmarker}{}%
\end{pgfscope}%
\end{pgfscope}%
\begin{pgfscope}%
\pgfpathrectangle{\pgfqpoint{0.720000in}{0.480000in}}{\pgfqpoint{7.040000in}{3.120000in}}%
\pgfusepath{clip}%
\pgfsetrectcap%
\pgfsetroundjoin%
\pgfsetlinewidth{0.752812pt}%
\definecolor{currentstroke}{rgb}{0.501961,0.501961,0.501961}%
\pgfsetstrokecolor{currentstroke}%
\pgfsetdash{}{0pt}%
\pgfpathmoveto{\pgfqpoint{0.848508in}{2.403230in}}%
\pgfpathlineto{\pgfqpoint{0.988470in}{2.941719in}}%
\pgfpathlineto{\pgfqpoint{1.127873in}{3.095407in}}%
\pgfpathlineto{\pgfqpoint{1.267276in}{3.183230in}}%
\pgfpathlineto{\pgfqpoint{1.407238in}{3.218667in}}%
\pgfpathlineto{\pgfqpoint{1.546641in}{3.227911in}}%
\pgfpathlineto{\pgfqpoint{1.685765in}{3.229067in}}%
\pgfpathlineto{\pgfqpoint{1.825448in}{3.260652in}}%
\pgfpathlineto{\pgfqpoint{1.964851in}{3.263348in}}%
\pgfpathlineto{\pgfqpoint{2.123251in}{3.242163in}}%
\pgfpathlineto{\pgfqpoint{2.263771in}{3.256800in}}%
\pgfpathlineto{\pgfqpoint{2.403175in}{3.261807in}}%
\pgfpathlineto{\pgfqpoint{2.543695in}{3.246015in}}%
\pgfpathlineto{\pgfqpoint{2.683937in}{3.260652in}}%
\pgfpathlineto{\pgfqpoint{2.823898in}{3.260652in}}%
\pgfpathlineto{\pgfqpoint{2.963302in}{3.259111in}}%
\pgfpathlineto{\pgfqpoint{3.103543in}{3.263348in}}%
\pgfpathlineto{\pgfqpoint{3.242108in}{3.257956in}}%
\pgfpathlineto{\pgfqpoint{3.380673in}{3.256800in}}%
\pgfpathlineto{\pgfqpoint{3.519797in}{3.264504in}}%
\pgfpathlineto{\pgfqpoint{3.658921in}{3.251407in}}%
\pgfusepath{stroke}%
\end{pgfscope}%
\begin{pgfscope}%
\pgfpathrectangle{\pgfqpoint{0.720000in}{0.480000in}}{\pgfqpoint{7.040000in}{3.120000in}}%
\pgfusepath{clip}%
\pgfsetbuttcap%
\pgfsetroundjoin%
\definecolor{currentfill}{rgb}{0.501961,0.501961,0.501961}%
\pgfsetfillcolor{currentfill}%
\pgfsetlinewidth{1.003750pt}%
\definecolor{currentstroke}{rgb}{0.501961,0.501961,0.501961}%
\pgfsetstrokecolor{currentstroke}%
\pgfsetdash{}{0pt}%
\pgfsys@defobject{currentmarker}{\pgfqpoint{-0.010417in}{-0.010417in}}{\pgfqpoint{0.010417in}{0.010417in}}{%
\pgfpathmoveto{\pgfqpoint{0.000000in}{-0.010417in}}%
\pgfpathcurveto{\pgfqpoint{0.002763in}{-0.010417in}}{\pgfqpoint{0.005412in}{-0.009319in}}{\pgfqpoint{0.007366in}{-0.007366in}}%
\pgfpathcurveto{\pgfqpoint{0.009319in}{-0.005412in}}{\pgfqpoint{0.010417in}{-0.002763in}}{\pgfqpoint{0.010417in}{0.000000in}}%
\pgfpathcurveto{\pgfqpoint{0.010417in}{0.002763in}}{\pgfqpoint{0.009319in}{0.005412in}}{\pgfqpoint{0.007366in}{0.007366in}}%
\pgfpathcurveto{\pgfqpoint{0.005412in}{0.009319in}}{\pgfqpoint{0.002763in}{0.010417in}}{\pgfqpoint{0.000000in}{0.010417in}}%
\pgfpathcurveto{\pgfqpoint{-0.002763in}{0.010417in}}{\pgfqpoint{-0.005412in}{0.009319in}}{\pgfqpoint{-0.007366in}{0.007366in}}%
\pgfpathcurveto{\pgfqpoint{-0.009319in}{0.005412in}}{\pgfqpoint{-0.010417in}{0.002763in}}{\pgfqpoint{-0.010417in}{0.000000in}}%
\pgfpathcurveto{\pgfqpoint{-0.010417in}{-0.002763in}}{\pgfqpoint{-0.009319in}{-0.005412in}}{\pgfqpoint{-0.007366in}{-0.007366in}}%
\pgfpathcurveto{\pgfqpoint{-0.005412in}{-0.009319in}}{\pgfqpoint{-0.002763in}{-0.010417in}}{\pgfqpoint{0.000000in}{-0.010417in}}%
\pgfpathclose%
\pgfusepath{stroke,fill}%
}%
\begin{pgfscope}%
\pgfsys@transformshift{0.848508in}{2.403230in}%
\pgfsys@useobject{currentmarker}{}%
\end{pgfscope}%
\begin{pgfscope}%
\pgfsys@transformshift{0.988470in}{2.941719in}%
\pgfsys@useobject{currentmarker}{}%
\end{pgfscope}%
\begin{pgfscope}%
\pgfsys@transformshift{1.127873in}{3.095407in}%
\pgfsys@useobject{currentmarker}{}%
\end{pgfscope}%
\begin{pgfscope}%
\pgfsys@transformshift{1.267276in}{3.183230in}%
\pgfsys@useobject{currentmarker}{}%
\end{pgfscope}%
\begin{pgfscope}%
\pgfsys@transformshift{1.407238in}{3.218667in}%
\pgfsys@useobject{currentmarker}{}%
\end{pgfscope}%
\begin{pgfscope}%
\pgfsys@transformshift{1.546641in}{3.227911in}%
\pgfsys@useobject{currentmarker}{}%
\end{pgfscope}%
\begin{pgfscope}%
\pgfsys@transformshift{1.685765in}{3.229067in}%
\pgfsys@useobject{currentmarker}{}%
\end{pgfscope}%
\begin{pgfscope}%
\pgfsys@transformshift{1.825448in}{3.260652in}%
\pgfsys@useobject{currentmarker}{}%
\end{pgfscope}%
\begin{pgfscope}%
\pgfsys@transformshift{1.964851in}{3.263348in}%
\pgfsys@useobject{currentmarker}{}%
\end{pgfscope}%
\begin{pgfscope}%
\pgfsys@transformshift{2.123251in}{3.242163in}%
\pgfsys@useobject{currentmarker}{}%
\end{pgfscope}%
\begin{pgfscope}%
\pgfsys@transformshift{2.263771in}{3.256800in}%
\pgfsys@useobject{currentmarker}{}%
\end{pgfscope}%
\begin{pgfscope}%
\pgfsys@transformshift{2.403175in}{3.261807in}%
\pgfsys@useobject{currentmarker}{}%
\end{pgfscope}%
\begin{pgfscope}%
\pgfsys@transformshift{2.543695in}{3.246015in}%
\pgfsys@useobject{currentmarker}{}%
\end{pgfscope}%
\begin{pgfscope}%
\pgfsys@transformshift{2.683937in}{3.260652in}%
\pgfsys@useobject{currentmarker}{}%
\end{pgfscope}%
\begin{pgfscope}%
\pgfsys@transformshift{2.823898in}{3.260652in}%
\pgfsys@useobject{currentmarker}{}%
\end{pgfscope}%
\begin{pgfscope}%
\pgfsys@transformshift{2.963302in}{3.259111in}%
\pgfsys@useobject{currentmarker}{}%
\end{pgfscope}%
\begin{pgfscope}%
\pgfsys@transformshift{3.103543in}{3.263348in}%
\pgfsys@useobject{currentmarker}{}%
\end{pgfscope}%
\begin{pgfscope}%
\pgfsys@transformshift{3.242108in}{3.257956in}%
\pgfsys@useobject{currentmarker}{}%
\end{pgfscope}%
\begin{pgfscope}%
\pgfsys@transformshift{3.380673in}{3.256800in}%
\pgfsys@useobject{currentmarker}{}%
\end{pgfscope}%
\begin{pgfscope}%
\pgfsys@transformshift{3.519797in}{3.264504in}%
\pgfsys@useobject{currentmarker}{}%
\end{pgfscope}%
\begin{pgfscope}%
\pgfsys@transformshift{3.658921in}{3.251407in}%
\pgfsys@useobject{currentmarker}{}%
\end{pgfscope}%
\end{pgfscope}%
\begin{pgfscope}%
\pgfsetrectcap%
\pgfsetmiterjoin%
\pgfsetlinewidth{0.803000pt}%
\definecolor{currentstroke}{rgb}{0.000000,0.000000,0.000000}%
\pgfsetstrokecolor{currentstroke}%
\pgfsetdash{}{0pt}%
\pgfpathmoveto{\pgfqpoint{0.720000in}{0.480000in}}%
\pgfpathlineto{\pgfqpoint{0.720000in}{3.600000in}}%
\pgfusepath{stroke}%
\end{pgfscope}%
\begin{pgfscope}%
\pgfsetrectcap%
\pgfsetmiterjoin%
\pgfsetlinewidth{0.803000pt}%
\definecolor{currentstroke}{rgb}{0.000000,0.000000,0.000000}%
\pgfsetstrokecolor{currentstroke}%
\pgfsetdash{}{0pt}%
\pgfpathmoveto{\pgfqpoint{7.760000in}{0.480000in}}%
\pgfpathlineto{\pgfqpoint{7.760000in}{3.600000in}}%
\pgfusepath{stroke}%
\end{pgfscope}%
\begin{pgfscope}%
\pgfsetrectcap%
\pgfsetmiterjoin%
\pgfsetlinewidth{0.803000pt}%
\definecolor{currentstroke}{rgb}{0.000000,0.000000,0.000000}%
\pgfsetstrokecolor{currentstroke}%
\pgfsetdash{}{0pt}%
\pgfpathmoveto{\pgfqpoint{0.720000in}{0.480000in}}%
\pgfpathlineto{\pgfqpoint{7.760000in}{0.480000in}}%
\pgfusepath{stroke}%
\end{pgfscope}%
\begin{pgfscope}%
\pgfsetrectcap%
\pgfsetmiterjoin%
\pgfsetlinewidth{0.803000pt}%
\definecolor{currentstroke}{rgb}{0.000000,0.000000,0.000000}%
\pgfsetstrokecolor{currentstroke}%
\pgfsetdash{}{0pt}%
\pgfpathmoveto{\pgfqpoint{0.720000in}{3.600000in}}%
\pgfpathlineto{\pgfqpoint{7.760000in}{3.600000in}}%
\pgfusepath{stroke}%
\end{pgfscope}%
\begin{pgfscope}%
\definecolor{textcolor}{rgb}{0.000000,0.000000,0.000000}%
\pgfsetstrokecolor{textcolor}%
\pgfsetfillcolor{textcolor}%
\pgftext[x=1.897572in,y=3.838851in,left,base]{\color{textcolor}\sffamily\fontsize{10.000000}{12.000000}\selectfont model training "food-50"; 21 epochs; min. 02:03:20; max 27:59:44}%
\end{pgfscope}%
\begin{pgfscope}%
\definecolor{textcolor}{rgb}{0.000000,0.000000,0.000000}%
\pgfsetstrokecolor{textcolor}%
\pgfsetfillcolor{textcolor}%
\pgftext[x=3.389611in,y=3.683333in,left,base]{\color{textcolor}\sffamily\fontsize{10.000000}{12.000000}\selectfont (GPU - Nvidia GTX 1060)}%
\end{pgfscope}%
\begin{pgfscope}%
\pgfsetbuttcap%
\pgfsetmiterjoin%
\definecolor{currentfill}{rgb}{1.000000,1.000000,1.000000}%
\pgfsetfillcolor{currentfill}%
\pgfsetfillopacity{0.800000}%
\pgfsetlinewidth{1.003750pt}%
\definecolor{currentstroke}{rgb}{0.800000,0.800000,0.800000}%
\pgfsetstrokecolor{currentstroke}%
\pgfsetstrokeopacity{0.800000}%
\pgfsetdash{}{0pt}%
\pgfpathmoveto{\pgfqpoint{4.171670in}{0.542500in}}%
\pgfpathlineto{\pgfqpoint{7.672500in}{0.542500in}}%
\pgfpathquadraticcurveto{\pgfqpoint{7.697500in}{0.542500in}}{\pgfqpoint{7.697500in}{0.567500in}}%
\pgfpathlineto{\pgfqpoint{7.697500in}{2.022772in}}%
\pgfpathquadraticcurveto{\pgfqpoint{7.697500in}{2.047772in}}{\pgfqpoint{7.672500in}{2.047772in}}%
\pgfpathlineto{\pgfqpoint{4.171670in}{2.047772in}}%
\pgfpathquadraticcurveto{\pgfqpoint{4.146670in}{2.047772in}}{\pgfqpoint{4.146670in}{2.022772in}}%
\pgfpathlineto{\pgfqpoint{4.146670in}{0.567500in}}%
\pgfpathquadraticcurveto{\pgfqpoint{4.146670in}{0.542500in}}{\pgfqpoint{4.171670in}{0.542500in}}%
\pgfpathclose%
\pgfusepath{stroke,fill}%
\end{pgfscope}%
\begin{pgfscope}%
\pgfsetrectcap%
\pgfsetroundjoin%
\pgfsetlinewidth{0.752812pt}%
\definecolor{currentstroke}{rgb}{0.000000,0.000000,1.000000}%
\pgfsetstrokecolor{currentstroke}%
\pgfsetdash{}{0pt}%
\pgfpathmoveto{\pgfqpoint{4.196670in}{1.946552in}}%
\pgfpathlineto{\pgfqpoint{4.446670in}{1.946552in}}%
\pgfusepath{stroke}%
\end{pgfscope}%
\begin{pgfscope}%
\pgfsetbuttcap%
\pgfsetroundjoin%
\definecolor{currentfill}{rgb}{0.000000,0.000000,1.000000}%
\pgfsetfillcolor{currentfill}%
\pgfsetlinewidth{1.003750pt}%
\definecolor{currentstroke}{rgb}{0.000000,0.000000,1.000000}%
\pgfsetstrokecolor{currentstroke}%
\pgfsetdash{}{0pt}%
\pgfsys@defobject{currentmarker}{\pgfqpoint{-0.010417in}{-0.010417in}}{\pgfqpoint{0.010417in}{0.010417in}}{%
\pgfpathmoveto{\pgfqpoint{0.000000in}{-0.010417in}}%
\pgfpathcurveto{\pgfqpoint{0.002763in}{-0.010417in}}{\pgfqpoint{0.005412in}{-0.009319in}}{\pgfqpoint{0.007366in}{-0.007366in}}%
\pgfpathcurveto{\pgfqpoint{0.009319in}{-0.005412in}}{\pgfqpoint{0.010417in}{-0.002763in}}{\pgfqpoint{0.010417in}{0.000000in}}%
\pgfpathcurveto{\pgfqpoint{0.010417in}{0.002763in}}{\pgfqpoint{0.009319in}{0.005412in}}{\pgfqpoint{0.007366in}{0.007366in}}%
\pgfpathcurveto{\pgfqpoint{0.005412in}{0.009319in}}{\pgfqpoint{0.002763in}{0.010417in}}{\pgfqpoint{0.000000in}{0.010417in}}%
\pgfpathcurveto{\pgfqpoint{-0.002763in}{0.010417in}}{\pgfqpoint{-0.005412in}{0.009319in}}{\pgfqpoint{-0.007366in}{0.007366in}}%
\pgfpathcurveto{\pgfqpoint{-0.009319in}{0.005412in}}{\pgfqpoint{-0.010417in}{0.002763in}}{\pgfqpoint{-0.010417in}{0.000000in}}%
\pgfpathcurveto{\pgfqpoint{-0.010417in}{-0.002763in}}{\pgfqpoint{-0.009319in}{-0.005412in}}{\pgfqpoint{-0.007366in}{-0.007366in}}%
\pgfpathcurveto{\pgfqpoint{-0.005412in}{-0.009319in}}{\pgfqpoint{-0.002763in}{-0.010417in}}{\pgfqpoint{0.000000in}{-0.010417in}}%
\pgfpathclose%
\pgfusepath{stroke,fill}%
}%
\begin{pgfscope}%
\pgfsys@transformshift{4.321670in}{1.946552in}%
\pgfsys@useobject{currentmarker}{}%
\end{pgfscope}%
\end{pgfscope}%
\begin{pgfscope}%
\definecolor{textcolor}{rgb}{0.000000,0.000000,0.000000}%
\pgfsetstrokecolor{textcolor}%
\pgfsetfillcolor{textcolor}%
\pgftext[x=4.546670in,y=1.902802in,left,base]{\color{textcolor}\sffamily\fontsize{9.000000}{10.800000}\selectfont DenseNet121 @ bs=16 (83.08\% - 02:16:41)}%
\end{pgfscope}%
\begin{pgfscope}%
\pgfsetrectcap%
\pgfsetroundjoin%
\pgfsetlinewidth{0.752812pt}%
\definecolor{currentstroke}{rgb}{1.000000,0.647059,0.000000}%
\pgfsetstrokecolor{currentstroke}%
\pgfsetdash{}{0pt}%
\pgfpathmoveto{\pgfqpoint{4.196670in}{1.763080in}}%
\pgfpathlineto{\pgfqpoint{4.446670in}{1.763080in}}%
\pgfusepath{stroke}%
\end{pgfscope}%
\begin{pgfscope}%
\pgfsetbuttcap%
\pgfsetroundjoin%
\definecolor{currentfill}{rgb}{1.000000,0.647059,0.000000}%
\pgfsetfillcolor{currentfill}%
\pgfsetlinewidth{1.003750pt}%
\definecolor{currentstroke}{rgb}{1.000000,0.647059,0.000000}%
\pgfsetstrokecolor{currentstroke}%
\pgfsetdash{}{0pt}%
\pgfsys@defobject{currentmarker}{\pgfqpoint{-0.010417in}{-0.010417in}}{\pgfqpoint{0.010417in}{0.010417in}}{%
\pgfpathmoveto{\pgfqpoint{0.000000in}{-0.010417in}}%
\pgfpathcurveto{\pgfqpoint{0.002763in}{-0.010417in}}{\pgfqpoint{0.005412in}{-0.009319in}}{\pgfqpoint{0.007366in}{-0.007366in}}%
\pgfpathcurveto{\pgfqpoint{0.009319in}{-0.005412in}}{\pgfqpoint{0.010417in}{-0.002763in}}{\pgfqpoint{0.010417in}{0.000000in}}%
\pgfpathcurveto{\pgfqpoint{0.010417in}{0.002763in}}{\pgfqpoint{0.009319in}{0.005412in}}{\pgfqpoint{0.007366in}{0.007366in}}%
\pgfpathcurveto{\pgfqpoint{0.005412in}{0.009319in}}{\pgfqpoint{0.002763in}{0.010417in}}{\pgfqpoint{0.000000in}{0.010417in}}%
\pgfpathcurveto{\pgfqpoint{-0.002763in}{0.010417in}}{\pgfqpoint{-0.005412in}{0.009319in}}{\pgfqpoint{-0.007366in}{0.007366in}}%
\pgfpathcurveto{\pgfqpoint{-0.009319in}{0.005412in}}{\pgfqpoint{-0.010417in}{0.002763in}}{\pgfqpoint{-0.010417in}{0.000000in}}%
\pgfpathcurveto{\pgfqpoint{-0.010417in}{-0.002763in}}{\pgfqpoint{-0.009319in}{-0.005412in}}{\pgfqpoint{-0.007366in}{-0.007366in}}%
\pgfpathcurveto{\pgfqpoint{-0.005412in}{-0.009319in}}{\pgfqpoint{-0.002763in}{-0.010417in}}{\pgfqpoint{0.000000in}{-0.010417in}}%
\pgfpathclose%
\pgfusepath{stroke,fill}%
}%
\begin{pgfscope}%
\pgfsys@transformshift{4.321670in}{1.763080in}%
\pgfsys@useobject{currentmarker}{}%
\end{pgfscope}%
\end{pgfscope}%
\begin{pgfscope}%
\definecolor{textcolor}{rgb}{0.000000,0.000000,0.000000}%
\pgfsetstrokecolor{textcolor}%
\pgfsetfillcolor{textcolor}%
\pgftext[x=4.546670in,y=1.719330in,left,base]{\color{textcolor}\sffamily\fontsize{9.000000}{10.800000}\selectfont DenseNet201 @ bs=8 (84.58\% - 05:12:15)}%
\end{pgfscope}%
\begin{pgfscope}%
\pgfsetrectcap%
\pgfsetroundjoin%
\pgfsetlinewidth{0.752812pt}%
\definecolor{currentstroke}{rgb}{0.000000,0.501961,0.000000}%
\pgfsetstrokecolor{currentstroke}%
\pgfsetdash{}{0pt}%
\pgfpathmoveto{\pgfqpoint{4.196670in}{1.579608in}}%
\pgfpathlineto{\pgfqpoint{4.446670in}{1.579608in}}%
\pgfusepath{stroke}%
\end{pgfscope}%
\begin{pgfscope}%
\pgfsetbuttcap%
\pgfsetroundjoin%
\definecolor{currentfill}{rgb}{0.000000,0.501961,0.000000}%
\pgfsetfillcolor{currentfill}%
\pgfsetlinewidth{1.003750pt}%
\definecolor{currentstroke}{rgb}{0.000000,0.501961,0.000000}%
\pgfsetstrokecolor{currentstroke}%
\pgfsetdash{}{0pt}%
\pgfsys@defobject{currentmarker}{\pgfqpoint{-0.010417in}{-0.010417in}}{\pgfqpoint{0.010417in}{0.010417in}}{%
\pgfpathmoveto{\pgfqpoint{0.000000in}{-0.010417in}}%
\pgfpathcurveto{\pgfqpoint{0.002763in}{-0.010417in}}{\pgfqpoint{0.005412in}{-0.009319in}}{\pgfqpoint{0.007366in}{-0.007366in}}%
\pgfpathcurveto{\pgfqpoint{0.009319in}{-0.005412in}}{\pgfqpoint{0.010417in}{-0.002763in}}{\pgfqpoint{0.010417in}{0.000000in}}%
\pgfpathcurveto{\pgfqpoint{0.010417in}{0.002763in}}{\pgfqpoint{0.009319in}{0.005412in}}{\pgfqpoint{0.007366in}{0.007366in}}%
\pgfpathcurveto{\pgfqpoint{0.005412in}{0.009319in}}{\pgfqpoint{0.002763in}{0.010417in}}{\pgfqpoint{0.000000in}{0.010417in}}%
\pgfpathcurveto{\pgfqpoint{-0.002763in}{0.010417in}}{\pgfqpoint{-0.005412in}{0.009319in}}{\pgfqpoint{-0.007366in}{0.007366in}}%
\pgfpathcurveto{\pgfqpoint{-0.009319in}{0.005412in}}{\pgfqpoint{-0.010417in}{0.002763in}}{\pgfqpoint{-0.010417in}{0.000000in}}%
\pgfpathcurveto{\pgfqpoint{-0.010417in}{-0.002763in}}{\pgfqpoint{-0.009319in}{-0.005412in}}{\pgfqpoint{-0.007366in}{-0.007366in}}%
\pgfpathcurveto{\pgfqpoint{-0.005412in}{-0.009319in}}{\pgfqpoint{-0.002763in}{-0.010417in}}{\pgfqpoint{0.000000in}{-0.010417in}}%
\pgfpathclose%
\pgfusepath{stroke,fill}%
}%
\begin{pgfscope}%
\pgfsys@transformshift{4.321670in}{1.579608in}%
\pgfsys@useobject{currentmarker}{}%
\end{pgfscope}%
\end{pgfscope}%
\begin{pgfscope}%
\definecolor{textcolor}{rgb}{0.000000,0.000000,0.000000}%
\pgfsetstrokecolor{textcolor}%
\pgfsetfillcolor{textcolor}%
\pgftext[x=4.546670in,y=1.535858in,left,base]{\color{textcolor}\sffamily\fontsize{9.000000}{10.800000}\selectfont InceptionResNetV2 @ bs=16 (80.42\% - 03:22:30)}%
\end{pgfscope}%
\begin{pgfscope}%
\pgfsetrectcap%
\pgfsetroundjoin%
\pgfsetlinewidth{0.752812pt}%
\definecolor{currentstroke}{rgb}{1.000000,0.000000,0.000000}%
\pgfsetstrokecolor{currentstroke}%
\pgfsetdash{}{0pt}%
\pgfpathmoveto{\pgfqpoint{4.196670in}{1.396137in}}%
\pgfpathlineto{\pgfqpoint{4.446670in}{1.396137in}}%
\pgfusepath{stroke}%
\end{pgfscope}%
\begin{pgfscope}%
\pgfsetbuttcap%
\pgfsetroundjoin%
\definecolor{currentfill}{rgb}{1.000000,0.000000,0.000000}%
\pgfsetfillcolor{currentfill}%
\pgfsetlinewidth{1.003750pt}%
\definecolor{currentstroke}{rgb}{1.000000,0.000000,0.000000}%
\pgfsetstrokecolor{currentstroke}%
\pgfsetdash{}{0pt}%
\pgfsys@defobject{currentmarker}{\pgfqpoint{-0.010417in}{-0.010417in}}{\pgfqpoint{0.010417in}{0.010417in}}{%
\pgfpathmoveto{\pgfqpoint{0.000000in}{-0.010417in}}%
\pgfpathcurveto{\pgfqpoint{0.002763in}{-0.010417in}}{\pgfqpoint{0.005412in}{-0.009319in}}{\pgfqpoint{0.007366in}{-0.007366in}}%
\pgfpathcurveto{\pgfqpoint{0.009319in}{-0.005412in}}{\pgfqpoint{0.010417in}{-0.002763in}}{\pgfqpoint{0.010417in}{0.000000in}}%
\pgfpathcurveto{\pgfqpoint{0.010417in}{0.002763in}}{\pgfqpoint{0.009319in}{0.005412in}}{\pgfqpoint{0.007366in}{0.007366in}}%
\pgfpathcurveto{\pgfqpoint{0.005412in}{0.009319in}}{\pgfqpoint{0.002763in}{0.010417in}}{\pgfqpoint{0.000000in}{0.010417in}}%
\pgfpathcurveto{\pgfqpoint{-0.002763in}{0.010417in}}{\pgfqpoint{-0.005412in}{0.009319in}}{\pgfqpoint{-0.007366in}{0.007366in}}%
\pgfpathcurveto{\pgfqpoint{-0.009319in}{0.005412in}}{\pgfqpoint{-0.010417in}{0.002763in}}{\pgfqpoint{-0.010417in}{0.000000in}}%
\pgfpathcurveto{\pgfqpoint{-0.010417in}{-0.002763in}}{\pgfqpoint{-0.009319in}{-0.005412in}}{\pgfqpoint{-0.007366in}{-0.007366in}}%
\pgfpathcurveto{\pgfqpoint{-0.005412in}{-0.009319in}}{\pgfqpoint{-0.002763in}{-0.010417in}}{\pgfqpoint{0.000000in}{-0.010417in}}%
\pgfpathclose%
\pgfusepath{stroke,fill}%
}%
\begin{pgfscope}%
\pgfsys@transformshift{4.321670in}{1.396137in}%
\pgfsys@useobject{currentmarker}{}%
\end{pgfscope}%
\end{pgfscope}%
\begin{pgfscope}%
\definecolor{textcolor}{rgb}{0.000000,0.000000,0.000000}%
\pgfsetstrokecolor{textcolor}%
\pgfsetfillcolor{textcolor}%
\pgftext[x=4.546670in,y=1.352387in,left,base]{\color{textcolor}\sffamily\fontsize{9.000000}{10.800000}\selectfont InceptionV3 @ bs=16 (83.59\% - 02:30:20)}%
\end{pgfscope}%
\begin{pgfscope}%
\pgfsetrectcap%
\pgfsetroundjoin%
\pgfsetlinewidth{0.752812pt}%
\definecolor{currentstroke}{rgb}{0.501961,0.000000,0.501961}%
\pgfsetstrokecolor{currentstroke}%
\pgfsetdash{}{0pt}%
\pgfpathmoveto{\pgfqpoint{4.196670in}{1.212665in}}%
\pgfpathlineto{\pgfqpoint{4.446670in}{1.212665in}}%
\pgfusepath{stroke}%
\end{pgfscope}%
\begin{pgfscope}%
\pgfsetbuttcap%
\pgfsetroundjoin%
\definecolor{currentfill}{rgb}{0.501961,0.000000,0.501961}%
\pgfsetfillcolor{currentfill}%
\pgfsetlinewidth{1.003750pt}%
\definecolor{currentstroke}{rgb}{0.501961,0.000000,0.501961}%
\pgfsetstrokecolor{currentstroke}%
\pgfsetdash{}{0pt}%
\pgfsys@defobject{currentmarker}{\pgfqpoint{-0.010417in}{-0.010417in}}{\pgfqpoint{0.010417in}{0.010417in}}{%
\pgfpathmoveto{\pgfqpoint{0.000000in}{-0.010417in}}%
\pgfpathcurveto{\pgfqpoint{0.002763in}{-0.010417in}}{\pgfqpoint{0.005412in}{-0.009319in}}{\pgfqpoint{0.007366in}{-0.007366in}}%
\pgfpathcurveto{\pgfqpoint{0.009319in}{-0.005412in}}{\pgfqpoint{0.010417in}{-0.002763in}}{\pgfqpoint{0.010417in}{0.000000in}}%
\pgfpathcurveto{\pgfqpoint{0.010417in}{0.002763in}}{\pgfqpoint{0.009319in}{0.005412in}}{\pgfqpoint{0.007366in}{0.007366in}}%
\pgfpathcurveto{\pgfqpoint{0.005412in}{0.009319in}}{\pgfqpoint{0.002763in}{0.010417in}}{\pgfqpoint{0.000000in}{0.010417in}}%
\pgfpathcurveto{\pgfqpoint{-0.002763in}{0.010417in}}{\pgfqpoint{-0.005412in}{0.009319in}}{\pgfqpoint{-0.007366in}{0.007366in}}%
\pgfpathcurveto{\pgfqpoint{-0.009319in}{0.005412in}}{\pgfqpoint{-0.010417in}{0.002763in}}{\pgfqpoint{-0.010417in}{0.000000in}}%
\pgfpathcurveto{\pgfqpoint{-0.010417in}{-0.002763in}}{\pgfqpoint{-0.009319in}{-0.005412in}}{\pgfqpoint{-0.007366in}{-0.007366in}}%
\pgfpathcurveto{\pgfqpoint{-0.005412in}{-0.009319in}}{\pgfqpoint{-0.002763in}{-0.010417in}}{\pgfqpoint{0.000000in}{-0.010417in}}%
\pgfpathclose%
\pgfusepath{stroke,fill}%
}%
\begin{pgfscope}%
\pgfsys@transformshift{4.321670in}{1.212665in}%
\pgfsys@useobject{currentmarker}{}%
\end{pgfscope}%
\end{pgfscope}%
\begin{pgfscope}%
\definecolor{textcolor}{rgb}{0.000000,0.000000,0.000000}%
\pgfsetstrokecolor{textcolor}%
\pgfsetfillcolor{textcolor}%
\pgftext[x=4.546670in,y=1.168915in,left,base]{\color{textcolor}\sffamily\fontsize{9.000000}{10.800000}\selectfont NASNetLarge @ bs=4 (83.76\% - 27:59:44)}%
\end{pgfscope}%
\begin{pgfscope}%
\pgfsetrectcap%
\pgfsetroundjoin%
\pgfsetlinewidth{0.752812pt}%
\definecolor{currentstroke}{rgb}{0.392157,0.584314,0.929412}%
\pgfsetstrokecolor{currentstroke}%
\pgfsetdash{}{0pt}%
\pgfpathmoveto{\pgfqpoint{4.196670in}{1.029194in}}%
\pgfpathlineto{\pgfqpoint{4.446670in}{1.029194in}}%
\pgfusepath{stroke}%
\end{pgfscope}%
\begin{pgfscope}%
\pgfsetbuttcap%
\pgfsetroundjoin%
\definecolor{currentfill}{rgb}{0.392157,0.584314,0.929412}%
\pgfsetfillcolor{currentfill}%
\pgfsetlinewidth{1.003750pt}%
\definecolor{currentstroke}{rgb}{0.392157,0.584314,0.929412}%
\pgfsetstrokecolor{currentstroke}%
\pgfsetdash{}{0pt}%
\pgfsys@defobject{currentmarker}{\pgfqpoint{-0.010417in}{-0.010417in}}{\pgfqpoint{0.010417in}{0.010417in}}{%
\pgfpathmoveto{\pgfqpoint{0.000000in}{-0.010417in}}%
\pgfpathcurveto{\pgfqpoint{0.002763in}{-0.010417in}}{\pgfqpoint{0.005412in}{-0.009319in}}{\pgfqpoint{0.007366in}{-0.007366in}}%
\pgfpathcurveto{\pgfqpoint{0.009319in}{-0.005412in}}{\pgfqpoint{0.010417in}{-0.002763in}}{\pgfqpoint{0.010417in}{0.000000in}}%
\pgfpathcurveto{\pgfqpoint{0.010417in}{0.002763in}}{\pgfqpoint{0.009319in}{0.005412in}}{\pgfqpoint{0.007366in}{0.007366in}}%
\pgfpathcurveto{\pgfqpoint{0.005412in}{0.009319in}}{\pgfqpoint{0.002763in}{0.010417in}}{\pgfqpoint{0.000000in}{0.010417in}}%
\pgfpathcurveto{\pgfqpoint{-0.002763in}{0.010417in}}{\pgfqpoint{-0.005412in}{0.009319in}}{\pgfqpoint{-0.007366in}{0.007366in}}%
\pgfpathcurveto{\pgfqpoint{-0.009319in}{0.005412in}}{\pgfqpoint{-0.010417in}{0.002763in}}{\pgfqpoint{-0.010417in}{0.000000in}}%
\pgfpathcurveto{\pgfqpoint{-0.010417in}{-0.002763in}}{\pgfqpoint{-0.009319in}{-0.005412in}}{\pgfqpoint{-0.007366in}{-0.007366in}}%
\pgfpathcurveto{\pgfqpoint{-0.005412in}{-0.009319in}}{\pgfqpoint{-0.002763in}{-0.010417in}}{\pgfqpoint{0.000000in}{-0.010417in}}%
\pgfpathclose%
\pgfusepath{stroke,fill}%
}%
\begin{pgfscope}%
\pgfsys@transformshift{4.321670in}{1.029194in}%
\pgfsys@useobject{currentmarker}{}%
\end{pgfscope}%
\end{pgfscope}%
\begin{pgfscope}%
\definecolor{textcolor}{rgb}{0.000000,0.000000,0.000000}%
\pgfsetstrokecolor{textcolor}%
\pgfsetfillcolor{textcolor}%
\pgftext[x=4.546670in,y=0.985444in,left,base]{\color{textcolor}\sffamily\fontsize{9.000000}{10.800000}\selectfont ResNet50 @ bs=16 (81.21\% - 02:03:20)}%
\end{pgfscope}%
\begin{pgfscope}%
\pgfsetrectcap%
\pgfsetroundjoin%
\pgfsetlinewidth{0.752812pt}%
\definecolor{currentstroke}{rgb}{1.000000,0.752941,0.796078}%
\pgfsetstrokecolor{currentstroke}%
\pgfsetdash{}{0pt}%
\pgfpathmoveto{\pgfqpoint{4.196670in}{0.845722in}}%
\pgfpathlineto{\pgfqpoint{4.446670in}{0.845722in}}%
\pgfusepath{stroke}%
\end{pgfscope}%
\begin{pgfscope}%
\pgfsetbuttcap%
\pgfsetroundjoin%
\definecolor{currentfill}{rgb}{1.000000,0.752941,0.796078}%
\pgfsetfillcolor{currentfill}%
\pgfsetlinewidth{1.003750pt}%
\definecolor{currentstroke}{rgb}{1.000000,0.752941,0.796078}%
\pgfsetstrokecolor{currentstroke}%
\pgfsetdash{}{0pt}%
\pgfsys@defobject{currentmarker}{\pgfqpoint{-0.010417in}{-0.010417in}}{\pgfqpoint{0.010417in}{0.010417in}}{%
\pgfpathmoveto{\pgfqpoint{0.000000in}{-0.010417in}}%
\pgfpathcurveto{\pgfqpoint{0.002763in}{-0.010417in}}{\pgfqpoint{0.005412in}{-0.009319in}}{\pgfqpoint{0.007366in}{-0.007366in}}%
\pgfpathcurveto{\pgfqpoint{0.009319in}{-0.005412in}}{\pgfqpoint{0.010417in}{-0.002763in}}{\pgfqpoint{0.010417in}{0.000000in}}%
\pgfpathcurveto{\pgfqpoint{0.010417in}{0.002763in}}{\pgfqpoint{0.009319in}{0.005412in}}{\pgfqpoint{0.007366in}{0.007366in}}%
\pgfpathcurveto{\pgfqpoint{0.005412in}{0.009319in}}{\pgfqpoint{0.002763in}{0.010417in}}{\pgfqpoint{0.000000in}{0.010417in}}%
\pgfpathcurveto{\pgfqpoint{-0.002763in}{0.010417in}}{\pgfqpoint{-0.005412in}{0.009319in}}{\pgfqpoint{-0.007366in}{0.007366in}}%
\pgfpathcurveto{\pgfqpoint{-0.009319in}{0.005412in}}{\pgfqpoint{-0.010417in}{0.002763in}}{\pgfqpoint{-0.010417in}{0.000000in}}%
\pgfpathcurveto{\pgfqpoint{-0.010417in}{-0.002763in}}{\pgfqpoint{-0.009319in}{-0.005412in}}{\pgfqpoint{-0.007366in}{-0.007366in}}%
\pgfpathcurveto{\pgfqpoint{-0.005412in}{-0.009319in}}{\pgfqpoint{-0.002763in}{-0.010417in}}{\pgfqpoint{0.000000in}{-0.010417in}}%
\pgfpathclose%
\pgfusepath{stroke,fill}%
}%
\begin{pgfscope}%
\pgfsys@transformshift{4.321670in}{0.845722in}%
\pgfsys@useobject{currentmarker}{}%
\end{pgfscope}%
\end{pgfscope}%
\begin{pgfscope}%
\definecolor{textcolor}{rgb}{0.000000,0.000000,0.000000}%
\pgfsetstrokecolor{textcolor}%
\pgfsetfillcolor{textcolor}%
\pgftext[x=4.546670in,y=0.801972in,left,base]{\color{textcolor}\sffamily\fontsize{9.000000}{10.800000}\selectfont VGG19 @ bs=16 (75.55\% - 02:13:18)}%
\end{pgfscope}%
\begin{pgfscope}%
\pgfsetrectcap%
\pgfsetroundjoin%
\pgfsetlinewidth{0.752812pt}%
\definecolor{currentstroke}{rgb}{0.501961,0.501961,0.501961}%
\pgfsetstrokecolor{currentstroke}%
\pgfsetdash{}{0pt}%
\pgfpathmoveto{\pgfqpoint{4.196670in}{0.662251in}}%
\pgfpathlineto{\pgfqpoint{4.446670in}{0.662251in}}%
\pgfusepath{stroke}%
\end{pgfscope}%
\begin{pgfscope}%
\pgfsetbuttcap%
\pgfsetroundjoin%
\definecolor{currentfill}{rgb}{0.501961,0.501961,0.501961}%
\pgfsetfillcolor{currentfill}%
\pgfsetlinewidth{1.003750pt}%
\definecolor{currentstroke}{rgb}{0.501961,0.501961,0.501961}%
\pgfsetstrokecolor{currentstroke}%
\pgfsetdash{}{0pt}%
\pgfsys@defobject{currentmarker}{\pgfqpoint{-0.010417in}{-0.010417in}}{\pgfqpoint{0.010417in}{0.010417in}}{%
\pgfpathmoveto{\pgfqpoint{0.000000in}{-0.010417in}}%
\pgfpathcurveto{\pgfqpoint{0.002763in}{-0.010417in}}{\pgfqpoint{0.005412in}{-0.009319in}}{\pgfqpoint{0.007366in}{-0.007366in}}%
\pgfpathcurveto{\pgfqpoint{0.009319in}{-0.005412in}}{\pgfqpoint{0.010417in}{-0.002763in}}{\pgfqpoint{0.010417in}{0.000000in}}%
\pgfpathcurveto{\pgfqpoint{0.010417in}{0.002763in}}{\pgfqpoint{0.009319in}{0.005412in}}{\pgfqpoint{0.007366in}{0.007366in}}%
\pgfpathcurveto{\pgfqpoint{0.005412in}{0.009319in}}{\pgfqpoint{0.002763in}{0.010417in}}{\pgfqpoint{0.000000in}{0.010417in}}%
\pgfpathcurveto{\pgfqpoint{-0.002763in}{0.010417in}}{\pgfqpoint{-0.005412in}{0.009319in}}{\pgfqpoint{-0.007366in}{0.007366in}}%
\pgfpathcurveto{\pgfqpoint{-0.009319in}{0.005412in}}{\pgfqpoint{-0.010417in}{0.002763in}}{\pgfqpoint{-0.010417in}{0.000000in}}%
\pgfpathcurveto{\pgfqpoint{-0.010417in}{-0.002763in}}{\pgfqpoint{-0.009319in}{-0.005412in}}{\pgfqpoint{-0.007366in}{-0.007366in}}%
\pgfpathcurveto{\pgfqpoint{-0.005412in}{-0.009319in}}{\pgfqpoint{-0.002763in}{-0.010417in}}{\pgfqpoint{0.000000in}{-0.010417in}}%
\pgfpathclose%
\pgfusepath{stroke,fill}%
}%
\begin{pgfscope}%
\pgfsys@transformshift{4.321670in}{0.662251in}%
\pgfsys@useobject{currentmarker}{}%
\end{pgfscope}%
\end{pgfscope}%
\begin{pgfscope}%
\definecolor{textcolor}{rgb}{0.000000,0.000000,0.000000}%
\pgfsetstrokecolor{textcolor}%
\pgfsetfillcolor{textcolor}%
\pgftext[x=4.546670in,y=0.618501in,left,base]{\color{textcolor}\sffamily\fontsize{9.000000}{10.800000}\selectfont Xception @ bs=16 (77.29\% - 02:55:20)}%
\end{pgfscope}%
\end{pgfpicture}%
\makeatother%
\endgroup%

					\end{center}
					\caption{Overview of known transfer learning models}
					\label{fig:evaluation_different_models}
				\end{figure}

				\noindent ...
		
			\subsubsection{Use of the transfer learning approach}
			\label{sec:section_use_of_the_transfer_learning_approach}
	
				\noindent ...

				\begin{figure}[H]
					\begin{center}
						%% Creator: Matplotlib, PGF backend
%%
%% To include the figure in your LaTeX document, write
%%   \input{<filename>.pgf}
%%
%% Make sure the required packages are loaded in your preamble
%%   \usepackage{pgf}
%%
%% Figures using additional raster images can only be included by \input if
%% they are in the same directory as the main LaTeX file. For loading figures
%% from other directories you can use the `import` package
%%   \usepackage{import}
%% and then include the figures with
%%   \import{<path to file>}{<filename>.pgf}
%%
%% Matplotlib used the following preamble
%%   \usepackage{fontspec}
%%   \setmainfont{DejaVuSerif.ttf}[Path=C:/Users/bjoern/Anaconda3/envs/keras-gpu/lib/site-packages/matplotlib/mpl-data/fonts/ttf/]
%%   \setsansfont{DejaVuSans.ttf}[Path=C:/Users/bjoern/Anaconda3/envs/keras-gpu/lib/site-packages/matplotlib/mpl-data/fonts/ttf/]
%%   \setmonofont{DejaVuSansMono.ttf}[Path=C:/Users/bjoern/Anaconda3/envs/keras-gpu/lib/site-packages/matplotlib/mpl-data/fonts/ttf/]
%%
\begingroup%
\makeatletter%
\begin{pgfpicture}%
\pgfpathrectangle{\pgfpointorigin}{\pgfqpoint{8.000000in}{4.000000in}}%
\pgfusepath{use as bounding box, clip}%
\begin{pgfscope}%
\pgfsetbuttcap%
\pgfsetmiterjoin%
\definecolor{currentfill}{rgb}{1.000000,1.000000,1.000000}%
\pgfsetfillcolor{currentfill}%
\pgfsetlinewidth{0.000000pt}%
\definecolor{currentstroke}{rgb}{1.000000,1.000000,1.000000}%
\pgfsetstrokecolor{currentstroke}%
\pgfsetdash{}{0pt}%
\pgfpathmoveto{\pgfqpoint{0.000000in}{0.000000in}}%
\pgfpathlineto{\pgfqpoint{8.000000in}{0.000000in}}%
\pgfpathlineto{\pgfqpoint{8.000000in}{4.000000in}}%
\pgfpathlineto{\pgfqpoint{0.000000in}{4.000000in}}%
\pgfpathclose%
\pgfusepath{fill}%
\end{pgfscope}%
\begin{pgfscope}%
\pgfsetbuttcap%
\pgfsetmiterjoin%
\definecolor{currentfill}{rgb}{1.000000,1.000000,1.000000}%
\pgfsetfillcolor{currentfill}%
\pgfsetlinewidth{0.000000pt}%
\definecolor{currentstroke}{rgb}{0.000000,0.000000,0.000000}%
\pgfsetstrokecolor{currentstroke}%
\pgfsetstrokeopacity{0.000000}%
\pgfsetdash{}{0pt}%
\pgfpathmoveto{\pgfqpoint{0.720000in}{0.480000in}}%
\pgfpathlineto{\pgfqpoint{7.760000in}{0.480000in}}%
\pgfpathlineto{\pgfqpoint{7.760000in}{3.600000in}}%
\pgfpathlineto{\pgfqpoint{0.720000in}{3.600000in}}%
\pgfpathclose%
\pgfusepath{fill}%
\end{pgfscope}%
\begin{pgfscope}%
\pgfpathrectangle{\pgfqpoint{0.720000in}{0.480000in}}{\pgfqpoint{7.040000in}{3.120000in}}%
\pgfusepath{clip}%
\pgfsetrectcap%
\pgfsetroundjoin%
\pgfsetlinewidth{0.803000pt}%
\definecolor{currentstroke}{rgb}{0.690196,0.690196,0.690196}%
\pgfsetstrokecolor{currentstroke}%
\pgfsetdash{}{0pt}%
\pgfpathmoveto{\pgfqpoint{0.720000in}{0.480000in}}%
\pgfpathlineto{\pgfqpoint{0.720000in}{3.600000in}}%
\pgfusepath{stroke}%
\end{pgfscope}%
\begin{pgfscope}%
\pgfsetbuttcap%
\pgfsetroundjoin%
\definecolor{currentfill}{rgb}{0.000000,0.000000,0.000000}%
\pgfsetfillcolor{currentfill}%
\pgfsetlinewidth{0.803000pt}%
\definecolor{currentstroke}{rgb}{0.000000,0.000000,0.000000}%
\pgfsetstrokecolor{currentstroke}%
\pgfsetdash{}{0pt}%
\pgfsys@defobject{currentmarker}{\pgfqpoint{0.000000in}{-0.048611in}}{\pgfqpoint{0.000000in}{0.000000in}}{%
\pgfpathmoveto{\pgfqpoint{0.000000in}{0.000000in}}%
\pgfpathlineto{\pgfqpoint{0.000000in}{-0.048611in}}%
\pgfusepath{stroke,fill}%
}%
\begin{pgfscope}%
\pgfsys@transformshift{0.720000in}{0.480000in}%
\pgfsys@useobject{currentmarker}{}%
\end{pgfscope}%
\end{pgfscope}%
\begin{pgfscope}%
\definecolor{textcolor}{rgb}{0.000000,0.000000,0.000000}%
\pgfsetstrokecolor{textcolor}%
\pgfsetfillcolor{textcolor}%
\pgftext[x=0.720000in,y=0.382778in,,top]{\color{textcolor}\sffamily\fontsize{10.000000}{12.000000}\selectfont 0}%
\end{pgfscope}%
\begin{pgfscope}%
\pgfpathrectangle{\pgfqpoint{0.720000in}{0.480000in}}{\pgfqpoint{7.040000in}{3.120000in}}%
\pgfusepath{clip}%
\pgfsetrectcap%
\pgfsetroundjoin%
\pgfsetlinewidth{0.803000pt}%
\definecolor{currentstroke}{rgb}{0.690196,0.690196,0.690196}%
\pgfsetstrokecolor{currentstroke}%
\pgfsetdash{}{0pt}%
\pgfpathmoveto{\pgfqpoint{3.066667in}{0.480000in}}%
\pgfpathlineto{\pgfqpoint{3.066667in}{3.600000in}}%
\pgfusepath{stroke}%
\end{pgfscope}%
\begin{pgfscope}%
\pgfsetbuttcap%
\pgfsetroundjoin%
\definecolor{currentfill}{rgb}{0.000000,0.000000,0.000000}%
\pgfsetfillcolor{currentfill}%
\pgfsetlinewidth{0.803000pt}%
\definecolor{currentstroke}{rgb}{0.000000,0.000000,0.000000}%
\pgfsetstrokecolor{currentstroke}%
\pgfsetdash{}{0pt}%
\pgfsys@defobject{currentmarker}{\pgfqpoint{0.000000in}{-0.048611in}}{\pgfqpoint{0.000000in}{0.000000in}}{%
\pgfpathmoveto{\pgfqpoint{0.000000in}{0.000000in}}%
\pgfpathlineto{\pgfqpoint{0.000000in}{-0.048611in}}%
\pgfusepath{stroke,fill}%
}%
\begin{pgfscope}%
\pgfsys@transformshift{3.066667in}{0.480000in}%
\pgfsys@useobject{currentmarker}{}%
\end{pgfscope}%
\end{pgfscope}%
\begin{pgfscope}%
\definecolor{textcolor}{rgb}{0.000000,0.000000,0.000000}%
\pgfsetstrokecolor{textcolor}%
\pgfsetfillcolor{textcolor}%
\pgftext[x=3.066667in,y=0.382778in,,top]{\color{textcolor}\sffamily\fontsize{10.000000}{12.000000}\selectfont 3600}%
\end{pgfscope}%
\begin{pgfscope}%
\pgfpathrectangle{\pgfqpoint{0.720000in}{0.480000in}}{\pgfqpoint{7.040000in}{3.120000in}}%
\pgfusepath{clip}%
\pgfsetrectcap%
\pgfsetroundjoin%
\pgfsetlinewidth{0.803000pt}%
\definecolor{currentstroke}{rgb}{0.690196,0.690196,0.690196}%
\pgfsetstrokecolor{currentstroke}%
\pgfsetdash{}{0pt}%
\pgfpathmoveto{\pgfqpoint{5.413333in}{0.480000in}}%
\pgfpathlineto{\pgfqpoint{5.413333in}{3.600000in}}%
\pgfusepath{stroke}%
\end{pgfscope}%
\begin{pgfscope}%
\pgfsetbuttcap%
\pgfsetroundjoin%
\definecolor{currentfill}{rgb}{0.000000,0.000000,0.000000}%
\pgfsetfillcolor{currentfill}%
\pgfsetlinewidth{0.803000pt}%
\definecolor{currentstroke}{rgb}{0.000000,0.000000,0.000000}%
\pgfsetstrokecolor{currentstroke}%
\pgfsetdash{}{0pt}%
\pgfsys@defobject{currentmarker}{\pgfqpoint{0.000000in}{-0.048611in}}{\pgfqpoint{0.000000in}{0.000000in}}{%
\pgfpathmoveto{\pgfqpoint{0.000000in}{0.000000in}}%
\pgfpathlineto{\pgfqpoint{0.000000in}{-0.048611in}}%
\pgfusepath{stroke,fill}%
}%
\begin{pgfscope}%
\pgfsys@transformshift{5.413333in}{0.480000in}%
\pgfsys@useobject{currentmarker}{}%
\end{pgfscope}%
\end{pgfscope}%
\begin{pgfscope}%
\definecolor{textcolor}{rgb}{0.000000,0.000000,0.000000}%
\pgfsetstrokecolor{textcolor}%
\pgfsetfillcolor{textcolor}%
\pgftext[x=5.413333in,y=0.382778in,,top]{\color{textcolor}\sffamily\fontsize{10.000000}{12.000000}\selectfont 7200}%
\end{pgfscope}%
\begin{pgfscope}%
\pgfpathrectangle{\pgfqpoint{0.720000in}{0.480000in}}{\pgfqpoint{7.040000in}{3.120000in}}%
\pgfusepath{clip}%
\pgfsetrectcap%
\pgfsetroundjoin%
\pgfsetlinewidth{0.803000pt}%
\definecolor{currentstroke}{rgb}{0.690196,0.690196,0.690196}%
\pgfsetstrokecolor{currentstroke}%
\pgfsetdash{}{0pt}%
\pgfpathmoveto{\pgfqpoint{7.760000in}{0.480000in}}%
\pgfpathlineto{\pgfqpoint{7.760000in}{3.600000in}}%
\pgfusepath{stroke}%
\end{pgfscope}%
\begin{pgfscope}%
\pgfsetbuttcap%
\pgfsetroundjoin%
\definecolor{currentfill}{rgb}{0.000000,0.000000,0.000000}%
\pgfsetfillcolor{currentfill}%
\pgfsetlinewidth{0.803000pt}%
\definecolor{currentstroke}{rgb}{0.000000,0.000000,0.000000}%
\pgfsetstrokecolor{currentstroke}%
\pgfsetdash{}{0pt}%
\pgfsys@defobject{currentmarker}{\pgfqpoint{0.000000in}{-0.048611in}}{\pgfqpoint{0.000000in}{0.000000in}}{%
\pgfpathmoveto{\pgfqpoint{0.000000in}{0.000000in}}%
\pgfpathlineto{\pgfqpoint{0.000000in}{-0.048611in}}%
\pgfusepath{stroke,fill}%
}%
\begin{pgfscope}%
\pgfsys@transformshift{7.760000in}{0.480000in}%
\pgfsys@useobject{currentmarker}{}%
\end{pgfscope}%
\end{pgfscope}%
\begin{pgfscope}%
\definecolor{textcolor}{rgb}{0.000000,0.000000,0.000000}%
\pgfsetstrokecolor{textcolor}%
\pgfsetfillcolor{textcolor}%
\pgftext[x=7.760000in,y=0.382778in,,top]{\color{textcolor}\sffamily\fontsize{10.000000}{12.000000}\selectfont 10800}%
\end{pgfscope}%
\begin{pgfscope}%
\definecolor{textcolor}{rgb}{0.000000,0.000000,0.000000}%
\pgfsetstrokecolor{textcolor}%
\pgfsetfillcolor{textcolor}%
\pgftext[x=4.240000in,y=0.192809in,,top]{\color{textcolor}\sffamily\fontsize{10.000000}{12.000000}\selectfont model training time [s]}%
\end{pgfscope}%
\begin{pgfscope}%
\pgfpathrectangle{\pgfqpoint{0.720000in}{0.480000in}}{\pgfqpoint{7.040000in}{3.120000in}}%
\pgfusepath{clip}%
\pgfsetrectcap%
\pgfsetroundjoin%
\pgfsetlinewidth{0.803000pt}%
\definecolor{currentstroke}{rgb}{0.690196,0.690196,0.690196}%
\pgfsetstrokecolor{currentstroke}%
\pgfsetdash{}{0pt}%
\pgfpathmoveto{\pgfqpoint{0.720000in}{0.962474in}}%
\pgfpathlineto{\pgfqpoint{7.760000in}{0.962474in}}%
\pgfusepath{stroke}%
\end{pgfscope}%
\begin{pgfscope}%
\pgfsetbuttcap%
\pgfsetroundjoin%
\definecolor{currentfill}{rgb}{0.000000,0.000000,0.000000}%
\pgfsetfillcolor{currentfill}%
\pgfsetlinewidth{0.803000pt}%
\definecolor{currentstroke}{rgb}{0.000000,0.000000,0.000000}%
\pgfsetstrokecolor{currentstroke}%
\pgfsetdash{}{0pt}%
\pgfsys@defobject{currentmarker}{\pgfqpoint{-0.048611in}{0.000000in}}{\pgfqpoint{0.000000in}{0.000000in}}{%
\pgfpathmoveto{\pgfqpoint{0.000000in}{0.000000in}}%
\pgfpathlineto{\pgfqpoint{-0.048611in}{0.000000in}}%
\pgfusepath{stroke,fill}%
}%
\begin{pgfscope}%
\pgfsys@transformshift{0.720000in}{0.962474in}%
\pgfsys@useobject{currentmarker}{}%
\end{pgfscope}%
\end{pgfscope}%
\begin{pgfscope}%
\definecolor{textcolor}{rgb}{0.000000,0.000000,0.000000}%
\pgfsetstrokecolor{textcolor}%
\pgfsetfillcolor{textcolor}%
\pgftext[x=0.314076in,y=0.909713in,left,base]{\color{textcolor}\sffamily\fontsize{10.000000}{12.000000}\selectfont 20\%}%
\end{pgfscope}%
\begin{pgfscope}%
\pgfpathrectangle{\pgfqpoint{0.720000in}{0.480000in}}{\pgfqpoint{7.040000in}{3.120000in}}%
\pgfusepath{clip}%
\pgfsetrectcap%
\pgfsetroundjoin%
\pgfsetlinewidth{0.803000pt}%
\definecolor{currentstroke}{rgb}{0.690196,0.690196,0.690196}%
\pgfsetstrokecolor{currentstroke}%
\pgfsetdash{}{0pt}%
\pgfpathmoveto{\pgfqpoint{0.720000in}{1.605773in}}%
\pgfpathlineto{\pgfqpoint{7.760000in}{1.605773in}}%
\pgfusepath{stroke}%
\end{pgfscope}%
\begin{pgfscope}%
\pgfsetbuttcap%
\pgfsetroundjoin%
\definecolor{currentfill}{rgb}{0.000000,0.000000,0.000000}%
\pgfsetfillcolor{currentfill}%
\pgfsetlinewidth{0.803000pt}%
\definecolor{currentstroke}{rgb}{0.000000,0.000000,0.000000}%
\pgfsetstrokecolor{currentstroke}%
\pgfsetdash{}{0pt}%
\pgfsys@defobject{currentmarker}{\pgfqpoint{-0.048611in}{0.000000in}}{\pgfqpoint{0.000000in}{0.000000in}}{%
\pgfpathmoveto{\pgfqpoint{0.000000in}{0.000000in}}%
\pgfpathlineto{\pgfqpoint{-0.048611in}{0.000000in}}%
\pgfusepath{stroke,fill}%
}%
\begin{pgfscope}%
\pgfsys@transformshift{0.720000in}{1.605773in}%
\pgfsys@useobject{currentmarker}{}%
\end{pgfscope}%
\end{pgfscope}%
\begin{pgfscope}%
\definecolor{textcolor}{rgb}{0.000000,0.000000,0.000000}%
\pgfsetstrokecolor{textcolor}%
\pgfsetfillcolor{textcolor}%
\pgftext[x=0.314076in,y=1.553012in,left,base]{\color{textcolor}\sffamily\fontsize{10.000000}{12.000000}\selectfont 40\%}%
\end{pgfscope}%
\begin{pgfscope}%
\pgfpathrectangle{\pgfqpoint{0.720000in}{0.480000in}}{\pgfqpoint{7.040000in}{3.120000in}}%
\pgfusepath{clip}%
\pgfsetrectcap%
\pgfsetroundjoin%
\pgfsetlinewidth{0.803000pt}%
\definecolor{currentstroke}{rgb}{0.690196,0.690196,0.690196}%
\pgfsetstrokecolor{currentstroke}%
\pgfsetdash{}{0pt}%
\pgfpathmoveto{\pgfqpoint{0.720000in}{2.249072in}}%
\pgfpathlineto{\pgfqpoint{7.760000in}{2.249072in}}%
\pgfusepath{stroke}%
\end{pgfscope}%
\begin{pgfscope}%
\pgfsetbuttcap%
\pgfsetroundjoin%
\definecolor{currentfill}{rgb}{0.000000,0.000000,0.000000}%
\pgfsetfillcolor{currentfill}%
\pgfsetlinewidth{0.803000pt}%
\definecolor{currentstroke}{rgb}{0.000000,0.000000,0.000000}%
\pgfsetstrokecolor{currentstroke}%
\pgfsetdash{}{0pt}%
\pgfsys@defobject{currentmarker}{\pgfqpoint{-0.048611in}{0.000000in}}{\pgfqpoint{0.000000in}{0.000000in}}{%
\pgfpathmoveto{\pgfqpoint{0.000000in}{0.000000in}}%
\pgfpathlineto{\pgfqpoint{-0.048611in}{0.000000in}}%
\pgfusepath{stroke,fill}%
}%
\begin{pgfscope}%
\pgfsys@transformshift{0.720000in}{2.249072in}%
\pgfsys@useobject{currentmarker}{}%
\end{pgfscope}%
\end{pgfscope}%
\begin{pgfscope}%
\definecolor{textcolor}{rgb}{0.000000,0.000000,0.000000}%
\pgfsetstrokecolor{textcolor}%
\pgfsetfillcolor{textcolor}%
\pgftext[x=0.314076in,y=2.196311in,left,base]{\color{textcolor}\sffamily\fontsize{10.000000}{12.000000}\selectfont 60\%}%
\end{pgfscope}%
\begin{pgfscope}%
\pgfpathrectangle{\pgfqpoint{0.720000in}{0.480000in}}{\pgfqpoint{7.040000in}{3.120000in}}%
\pgfusepath{clip}%
\pgfsetrectcap%
\pgfsetroundjoin%
\pgfsetlinewidth{0.803000pt}%
\definecolor{currentstroke}{rgb}{0.690196,0.690196,0.690196}%
\pgfsetstrokecolor{currentstroke}%
\pgfsetdash{}{0pt}%
\pgfpathmoveto{\pgfqpoint{0.720000in}{2.892371in}}%
\pgfpathlineto{\pgfqpoint{7.760000in}{2.892371in}}%
\pgfusepath{stroke}%
\end{pgfscope}%
\begin{pgfscope}%
\pgfsetbuttcap%
\pgfsetroundjoin%
\definecolor{currentfill}{rgb}{0.000000,0.000000,0.000000}%
\pgfsetfillcolor{currentfill}%
\pgfsetlinewidth{0.803000pt}%
\definecolor{currentstroke}{rgb}{0.000000,0.000000,0.000000}%
\pgfsetstrokecolor{currentstroke}%
\pgfsetdash{}{0pt}%
\pgfsys@defobject{currentmarker}{\pgfqpoint{-0.048611in}{0.000000in}}{\pgfqpoint{0.000000in}{0.000000in}}{%
\pgfpathmoveto{\pgfqpoint{0.000000in}{0.000000in}}%
\pgfpathlineto{\pgfqpoint{-0.048611in}{0.000000in}}%
\pgfusepath{stroke,fill}%
}%
\begin{pgfscope}%
\pgfsys@transformshift{0.720000in}{2.892371in}%
\pgfsys@useobject{currentmarker}{}%
\end{pgfscope}%
\end{pgfscope}%
\begin{pgfscope}%
\definecolor{textcolor}{rgb}{0.000000,0.000000,0.000000}%
\pgfsetstrokecolor{textcolor}%
\pgfsetfillcolor{textcolor}%
\pgftext[x=0.314076in,y=2.839610in,left,base]{\color{textcolor}\sffamily\fontsize{10.000000}{12.000000}\selectfont 80\%}%
\end{pgfscope}%
\begin{pgfscope}%
\pgfpathrectangle{\pgfqpoint{0.720000in}{0.480000in}}{\pgfqpoint{7.040000in}{3.120000in}}%
\pgfusepath{clip}%
\pgfsetrectcap%
\pgfsetroundjoin%
\pgfsetlinewidth{0.803000pt}%
\definecolor{currentstroke}{rgb}{0.690196,0.690196,0.690196}%
\pgfsetstrokecolor{currentstroke}%
\pgfsetdash{}{0pt}%
\pgfpathmoveto{\pgfqpoint{0.720000in}{3.535670in}}%
\pgfpathlineto{\pgfqpoint{7.760000in}{3.535670in}}%
\pgfusepath{stroke}%
\end{pgfscope}%
\begin{pgfscope}%
\pgfsetbuttcap%
\pgfsetroundjoin%
\definecolor{currentfill}{rgb}{0.000000,0.000000,0.000000}%
\pgfsetfillcolor{currentfill}%
\pgfsetlinewidth{0.803000pt}%
\definecolor{currentstroke}{rgb}{0.000000,0.000000,0.000000}%
\pgfsetstrokecolor{currentstroke}%
\pgfsetdash{}{0pt}%
\pgfsys@defobject{currentmarker}{\pgfqpoint{-0.048611in}{0.000000in}}{\pgfqpoint{0.000000in}{0.000000in}}{%
\pgfpathmoveto{\pgfqpoint{0.000000in}{0.000000in}}%
\pgfpathlineto{\pgfqpoint{-0.048611in}{0.000000in}}%
\pgfusepath{stroke,fill}%
}%
\begin{pgfscope}%
\pgfsys@transformshift{0.720000in}{3.535670in}%
\pgfsys@useobject{currentmarker}{}%
\end{pgfscope}%
\end{pgfscope}%
\begin{pgfscope}%
\definecolor{textcolor}{rgb}{0.000000,0.000000,0.000000}%
\pgfsetstrokecolor{textcolor}%
\pgfsetfillcolor{textcolor}%
\pgftext[x=0.225710in,y=3.482909in,left,base]{\color{textcolor}\sffamily\fontsize{10.000000}{12.000000}\selectfont 100\%}%
\end{pgfscope}%
\begin{pgfscope}%
\definecolor{textcolor}{rgb}{0.000000,0.000000,0.000000}%
\pgfsetstrokecolor{textcolor}%
\pgfsetfillcolor{textcolor}%
\pgftext[x=0.170155in,y=2.040000in,,bottom,rotate=90.000000]{\color{textcolor}\sffamily\fontsize{10.000000}{12.000000}\selectfont accuracy validation and training top-1 [\%]}%
\end{pgfscope}%
\begin{pgfscope}%
\pgfpathrectangle{\pgfqpoint{0.720000in}{0.480000in}}{\pgfqpoint{7.040000in}{3.120000in}}%
\pgfusepath{clip}%
\pgfsetrectcap%
\pgfsetroundjoin%
\pgfsetlinewidth{0.752812pt}%
\definecolor{currentstroke}{rgb}{0.000000,0.000000,1.000000}%
\pgfsetstrokecolor{currentstroke}%
\pgfsetdash{}{0pt}%
\pgfpathmoveto{\pgfqpoint{1.009422in}{2.533732in}}%
\pgfpathlineto{\pgfqpoint{1.288415in}{2.603530in}}%
\pgfpathlineto{\pgfqpoint{1.566756in}{2.843480in}}%
\pgfpathlineto{\pgfqpoint{1.845096in}{2.859885in}}%
\pgfpathlineto{\pgfqpoint{2.124089in}{2.805204in}}%
\pgfpathlineto{\pgfqpoint{2.402430in}{2.873072in}}%
\pgfpathlineto{\pgfqpoint{2.680770in}{2.915852in}}%
\pgfpathlineto{\pgfqpoint{2.959763in}{2.980503in}}%
\pgfpathlineto{\pgfqpoint{3.238104in}{2.961847in}}%
\pgfpathlineto{\pgfqpoint{3.516444in}{2.954128in}}%
\pgfpathlineto{\pgfqpoint{3.794785in}{2.984685in}}%
\pgfpathlineto{\pgfqpoint{4.073126in}{2.999159in}}%
\pgfpathlineto{\pgfqpoint{4.351467in}{2.961847in}}%
\pgfpathlineto{\pgfqpoint{4.629807in}{2.979216in}}%
\pgfpathlineto{\pgfqpoint{4.907496in}{2.991439in}}%
\pgfpathlineto{\pgfqpoint{5.187141in}{2.979216in}}%
\pgfpathlineto{\pgfqpoint{5.465481in}{3.004627in}}%
\pgfpathlineto{\pgfqpoint{5.764681in}{2.991439in}}%
\pgfpathlineto{\pgfqpoint{6.043022in}{3.000124in}}%
\pgfpathlineto{\pgfqpoint{6.321363in}{2.999159in}}%
\pgfpathlineto{\pgfqpoint{6.599704in}{3.007843in}}%
\pgfusepath{stroke}%
\end{pgfscope}%
\begin{pgfscope}%
\pgfpathrectangle{\pgfqpoint{0.720000in}{0.480000in}}{\pgfqpoint{7.040000in}{3.120000in}}%
\pgfusepath{clip}%
\pgfsetbuttcap%
\pgfsetroundjoin%
\definecolor{currentfill}{rgb}{0.000000,0.000000,1.000000}%
\pgfsetfillcolor{currentfill}%
\pgfsetlinewidth{1.003750pt}%
\definecolor{currentstroke}{rgb}{0.000000,0.000000,1.000000}%
\pgfsetstrokecolor{currentstroke}%
\pgfsetdash{}{0pt}%
\pgfsys@defobject{currentmarker}{\pgfqpoint{-0.010417in}{-0.010417in}}{\pgfqpoint{0.010417in}{0.010417in}}{%
\pgfpathmoveto{\pgfqpoint{0.000000in}{-0.010417in}}%
\pgfpathcurveto{\pgfqpoint{0.002763in}{-0.010417in}}{\pgfqpoint{0.005412in}{-0.009319in}}{\pgfqpoint{0.007366in}{-0.007366in}}%
\pgfpathcurveto{\pgfqpoint{0.009319in}{-0.005412in}}{\pgfqpoint{0.010417in}{-0.002763in}}{\pgfqpoint{0.010417in}{0.000000in}}%
\pgfpathcurveto{\pgfqpoint{0.010417in}{0.002763in}}{\pgfqpoint{0.009319in}{0.005412in}}{\pgfqpoint{0.007366in}{0.007366in}}%
\pgfpathcurveto{\pgfqpoint{0.005412in}{0.009319in}}{\pgfqpoint{0.002763in}{0.010417in}}{\pgfqpoint{0.000000in}{0.010417in}}%
\pgfpathcurveto{\pgfqpoint{-0.002763in}{0.010417in}}{\pgfqpoint{-0.005412in}{0.009319in}}{\pgfqpoint{-0.007366in}{0.007366in}}%
\pgfpathcurveto{\pgfqpoint{-0.009319in}{0.005412in}}{\pgfqpoint{-0.010417in}{0.002763in}}{\pgfqpoint{-0.010417in}{0.000000in}}%
\pgfpathcurveto{\pgfqpoint{-0.010417in}{-0.002763in}}{\pgfqpoint{-0.009319in}{-0.005412in}}{\pgfqpoint{-0.007366in}{-0.007366in}}%
\pgfpathcurveto{\pgfqpoint{-0.005412in}{-0.009319in}}{\pgfqpoint{-0.002763in}{-0.010417in}}{\pgfqpoint{0.000000in}{-0.010417in}}%
\pgfpathclose%
\pgfusepath{stroke,fill}%
}%
\begin{pgfscope}%
\pgfsys@transformshift{1.009422in}{2.533732in}%
\pgfsys@useobject{currentmarker}{}%
\end{pgfscope}%
\begin{pgfscope}%
\pgfsys@transformshift{1.288415in}{2.603530in}%
\pgfsys@useobject{currentmarker}{}%
\end{pgfscope}%
\begin{pgfscope}%
\pgfsys@transformshift{1.566756in}{2.843480in}%
\pgfsys@useobject{currentmarker}{}%
\end{pgfscope}%
\begin{pgfscope}%
\pgfsys@transformshift{1.845096in}{2.859885in}%
\pgfsys@useobject{currentmarker}{}%
\end{pgfscope}%
\begin{pgfscope}%
\pgfsys@transformshift{2.124089in}{2.805204in}%
\pgfsys@useobject{currentmarker}{}%
\end{pgfscope}%
\begin{pgfscope}%
\pgfsys@transformshift{2.402430in}{2.873072in}%
\pgfsys@useobject{currentmarker}{}%
\end{pgfscope}%
\begin{pgfscope}%
\pgfsys@transformshift{2.680770in}{2.915852in}%
\pgfsys@useobject{currentmarker}{}%
\end{pgfscope}%
\begin{pgfscope}%
\pgfsys@transformshift{2.959763in}{2.980503in}%
\pgfsys@useobject{currentmarker}{}%
\end{pgfscope}%
\begin{pgfscope}%
\pgfsys@transformshift{3.238104in}{2.961847in}%
\pgfsys@useobject{currentmarker}{}%
\end{pgfscope}%
\begin{pgfscope}%
\pgfsys@transformshift{3.516444in}{2.954128in}%
\pgfsys@useobject{currentmarker}{}%
\end{pgfscope}%
\begin{pgfscope}%
\pgfsys@transformshift{3.794785in}{2.984685in}%
\pgfsys@useobject{currentmarker}{}%
\end{pgfscope}%
\begin{pgfscope}%
\pgfsys@transformshift{4.073126in}{2.999159in}%
\pgfsys@useobject{currentmarker}{}%
\end{pgfscope}%
\begin{pgfscope}%
\pgfsys@transformshift{4.351467in}{2.961847in}%
\pgfsys@useobject{currentmarker}{}%
\end{pgfscope}%
\begin{pgfscope}%
\pgfsys@transformshift{4.629807in}{2.979216in}%
\pgfsys@useobject{currentmarker}{}%
\end{pgfscope}%
\begin{pgfscope}%
\pgfsys@transformshift{4.907496in}{2.991439in}%
\pgfsys@useobject{currentmarker}{}%
\end{pgfscope}%
\begin{pgfscope}%
\pgfsys@transformshift{5.187141in}{2.979216in}%
\pgfsys@useobject{currentmarker}{}%
\end{pgfscope}%
\begin{pgfscope}%
\pgfsys@transformshift{5.465481in}{3.004627in}%
\pgfsys@useobject{currentmarker}{}%
\end{pgfscope}%
\begin{pgfscope}%
\pgfsys@transformshift{5.764681in}{2.991439in}%
\pgfsys@useobject{currentmarker}{}%
\end{pgfscope}%
\begin{pgfscope}%
\pgfsys@transformshift{6.043022in}{3.000124in}%
\pgfsys@useobject{currentmarker}{}%
\end{pgfscope}%
\begin{pgfscope}%
\pgfsys@transformshift{6.321363in}{2.999159in}%
\pgfsys@useobject{currentmarker}{}%
\end{pgfscope}%
\begin{pgfscope}%
\pgfsys@transformshift{6.599704in}{3.007843in}%
\pgfsys@useobject{currentmarker}{}%
\end{pgfscope}%
\end{pgfscope}%
\begin{pgfscope}%
\pgfpathrectangle{\pgfqpoint{0.720000in}{0.480000in}}{\pgfqpoint{7.040000in}{3.120000in}}%
\pgfusepath{clip}%
\pgfsetbuttcap%
\pgfsetroundjoin%
\pgfsetlinewidth{0.501875pt}%
\definecolor{currentstroke}{rgb}{0.000000,0.000000,1.000000}%
\pgfsetstrokecolor{currentstroke}%
\pgfsetdash{{1.850000pt}{0.800000pt}}{0.000000pt}%
\pgfpathmoveto{\pgfqpoint{1.009422in}{1.655307in}}%
\pgfpathlineto{\pgfqpoint{1.288415in}{2.599348in}}%
\pgfpathlineto{\pgfqpoint{1.566756in}{2.936437in}}%
\pgfpathlineto{\pgfqpoint{1.845096in}{3.146796in}}%
\pgfpathlineto{\pgfqpoint{2.124089in}{3.253584in}}%
\pgfpathlineto{\pgfqpoint{2.402430in}{3.335926in}}%
\pgfpathlineto{\pgfqpoint{2.680770in}{3.400577in}}%
\pgfpathlineto{\pgfqpoint{2.959763in}{3.459761in}}%
\pgfpathlineto{\pgfqpoint{3.238104in}{3.477773in}}%
\pgfpathlineto{\pgfqpoint{3.516444in}{3.489353in}}%
\pgfpathlineto{\pgfqpoint{3.794785in}{3.503505in}}%
\pgfpathlineto{\pgfqpoint{4.073126in}{3.507365in}}%
\pgfpathlineto{\pgfqpoint{4.351467in}{3.503184in}}%
\pgfpathlineto{\pgfqpoint{4.629807in}{3.509295in}}%
\pgfpathlineto{\pgfqpoint{4.907496in}{3.513798in}}%
\pgfpathlineto{\pgfqpoint{5.187141in}{3.516049in}}%
\pgfpathlineto{\pgfqpoint{5.465481in}{3.517658in}}%
\pgfpathlineto{\pgfqpoint{5.764681in}{3.516371in}}%
\pgfpathlineto{\pgfqpoint{6.043022in}{3.517979in}}%
\pgfpathlineto{\pgfqpoint{6.321363in}{3.519909in}}%
\pgfpathlineto{\pgfqpoint{6.599704in}{3.521839in}}%
\pgfusepath{stroke}%
\end{pgfscope}%
\begin{pgfscope}%
\pgfpathrectangle{\pgfqpoint{0.720000in}{0.480000in}}{\pgfqpoint{7.040000in}{3.120000in}}%
\pgfusepath{clip}%
\pgfsetbuttcap%
\pgfsetroundjoin%
\definecolor{currentfill}{rgb}{0.000000,0.000000,1.000000}%
\pgfsetfillcolor{currentfill}%
\pgfsetlinewidth{1.003750pt}%
\definecolor{currentstroke}{rgb}{0.000000,0.000000,1.000000}%
\pgfsetstrokecolor{currentstroke}%
\pgfsetdash{}{0pt}%
\pgfsys@defobject{currentmarker}{\pgfqpoint{-0.006944in}{-0.006944in}}{\pgfqpoint{0.006944in}{0.006944in}}{%
\pgfpathmoveto{\pgfqpoint{0.000000in}{-0.006944in}}%
\pgfpathcurveto{\pgfqpoint{0.001842in}{-0.006944in}}{\pgfqpoint{0.003608in}{-0.006213in}}{\pgfqpoint{0.004910in}{-0.004910in}}%
\pgfpathcurveto{\pgfqpoint{0.006213in}{-0.003608in}}{\pgfqpoint{0.006944in}{-0.001842in}}{\pgfqpoint{0.006944in}{0.000000in}}%
\pgfpathcurveto{\pgfqpoint{0.006944in}{0.001842in}}{\pgfqpoint{0.006213in}{0.003608in}}{\pgfqpoint{0.004910in}{0.004910in}}%
\pgfpathcurveto{\pgfqpoint{0.003608in}{0.006213in}}{\pgfqpoint{0.001842in}{0.006944in}}{\pgfqpoint{0.000000in}{0.006944in}}%
\pgfpathcurveto{\pgfqpoint{-0.001842in}{0.006944in}}{\pgfqpoint{-0.003608in}{0.006213in}}{\pgfqpoint{-0.004910in}{0.004910in}}%
\pgfpathcurveto{\pgfqpoint{-0.006213in}{0.003608in}}{\pgfqpoint{-0.006944in}{0.001842in}}{\pgfqpoint{-0.006944in}{0.000000in}}%
\pgfpathcurveto{\pgfqpoint{-0.006944in}{-0.001842in}}{\pgfqpoint{-0.006213in}{-0.003608in}}{\pgfqpoint{-0.004910in}{-0.004910in}}%
\pgfpathcurveto{\pgfqpoint{-0.003608in}{-0.006213in}}{\pgfqpoint{-0.001842in}{-0.006944in}}{\pgfqpoint{0.000000in}{-0.006944in}}%
\pgfpathclose%
\pgfusepath{stroke,fill}%
}%
\begin{pgfscope}%
\pgfsys@transformshift{1.009422in}{1.655307in}%
\pgfsys@useobject{currentmarker}{}%
\end{pgfscope}%
\begin{pgfscope}%
\pgfsys@transformshift{1.288415in}{2.599348in}%
\pgfsys@useobject{currentmarker}{}%
\end{pgfscope}%
\begin{pgfscope}%
\pgfsys@transformshift{1.566756in}{2.936437in}%
\pgfsys@useobject{currentmarker}{}%
\end{pgfscope}%
\begin{pgfscope}%
\pgfsys@transformshift{1.845096in}{3.146796in}%
\pgfsys@useobject{currentmarker}{}%
\end{pgfscope}%
\begin{pgfscope}%
\pgfsys@transformshift{2.124089in}{3.253584in}%
\pgfsys@useobject{currentmarker}{}%
\end{pgfscope}%
\begin{pgfscope}%
\pgfsys@transformshift{2.402430in}{3.335926in}%
\pgfsys@useobject{currentmarker}{}%
\end{pgfscope}%
\begin{pgfscope}%
\pgfsys@transformshift{2.680770in}{3.400577in}%
\pgfsys@useobject{currentmarker}{}%
\end{pgfscope}%
\begin{pgfscope}%
\pgfsys@transformshift{2.959763in}{3.459761in}%
\pgfsys@useobject{currentmarker}{}%
\end{pgfscope}%
\begin{pgfscope}%
\pgfsys@transformshift{3.238104in}{3.477773in}%
\pgfsys@useobject{currentmarker}{}%
\end{pgfscope}%
\begin{pgfscope}%
\pgfsys@transformshift{3.516444in}{3.489353in}%
\pgfsys@useobject{currentmarker}{}%
\end{pgfscope}%
\begin{pgfscope}%
\pgfsys@transformshift{3.794785in}{3.503505in}%
\pgfsys@useobject{currentmarker}{}%
\end{pgfscope}%
\begin{pgfscope}%
\pgfsys@transformshift{4.073126in}{3.507365in}%
\pgfsys@useobject{currentmarker}{}%
\end{pgfscope}%
\begin{pgfscope}%
\pgfsys@transformshift{4.351467in}{3.503184in}%
\pgfsys@useobject{currentmarker}{}%
\end{pgfscope}%
\begin{pgfscope}%
\pgfsys@transformshift{4.629807in}{3.509295in}%
\pgfsys@useobject{currentmarker}{}%
\end{pgfscope}%
\begin{pgfscope}%
\pgfsys@transformshift{4.907496in}{3.513798in}%
\pgfsys@useobject{currentmarker}{}%
\end{pgfscope}%
\begin{pgfscope}%
\pgfsys@transformshift{5.187141in}{3.516049in}%
\pgfsys@useobject{currentmarker}{}%
\end{pgfscope}%
\begin{pgfscope}%
\pgfsys@transformshift{5.465481in}{3.517658in}%
\pgfsys@useobject{currentmarker}{}%
\end{pgfscope}%
\begin{pgfscope}%
\pgfsys@transformshift{5.764681in}{3.516371in}%
\pgfsys@useobject{currentmarker}{}%
\end{pgfscope}%
\begin{pgfscope}%
\pgfsys@transformshift{6.043022in}{3.517979in}%
\pgfsys@useobject{currentmarker}{}%
\end{pgfscope}%
\begin{pgfscope}%
\pgfsys@transformshift{6.321363in}{3.519909in}%
\pgfsys@useobject{currentmarker}{}%
\end{pgfscope}%
\begin{pgfscope}%
\pgfsys@transformshift{6.599704in}{3.521839in}%
\pgfsys@useobject{currentmarker}{}%
\end{pgfscope}%
\end{pgfscope}%
\begin{pgfscope}%
\pgfpathrectangle{\pgfqpoint{0.720000in}{0.480000in}}{\pgfqpoint{7.040000in}{3.120000in}}%
\pgfusepath{clip}%
\pgfsetrectcap%
\pgfsetroundjoin%
\pgfsetlinewidth{0.752812pt}%
\definecolor{currentstroke}{rgb}{1.000000,0.647059,0.000000}%
\pgfsetstrokecolor{currentstroke}%
\pgfsetdash{}{0pt}%
\pgfpathmoveto{\pgfqpoint{1.038104in}{0.591291in}}%
\pgfpathlineto{\pgfqpoint{1.317096in}{0.679423in}}%
\pgfpathlineto{\pgfqpoint{1.594785in}{0.681674in}}%
\pgfpathlineto{\pgfqpoint{1.872474in}{0.813229in}}%
\pgfpathlineto{\pgfqpoint{2.151467in}{0.960866in}}%
\pgfpathlineto{\pgfqpoint{2.429807in}{1.002680in}}%
\pgfpathlineto{\pgfqpoint{2.707496in}{0.952181in}}%
\pgfpathlineto{\pgfqpoint{2.985185in}{1.256784in}}%
\pgfpathlineto{\pgfqpoint{3.262874in}{1.205320in}}%
\pgfpathlineto{\pgfqpoint{3.540563in}{1.302779in}}%
\pgfpathlineto{\pgfqpoint{3.818252in}{1.292808in}}%
\pgfpathlineto{\pgfqpoint{4.095941in}{1.335588in}}%
\pgfpathlineto{\pgfqpoint{4.373630in}{1.335588in}}%
\pgfpathlineto{\pgfqpoint{4.651319in}{1.406672in}}%
\pgfpathlineto{\pgfqpoint{4.929659in}{1.520536in}}%
\pgfpathlineto{\pgfqpoint{5.208000in}{1.540157in}}%
\pgfpathlineto{\pgfqpoint{5.486341in}{1.512816in}}%
\pgfpathlineto{\pgfqpoint{5.764030in}{1.493196in}}%
\pgfpathlineto{\pgfqpoint{6.042370in}{1.494161in}}%
\pgfpathlineto{\pgfqpoint{6.320711in}{1.521823in}}%
\pgfpathlineto{\pgfqpoint{6.599052in}{1.549163in}}%
\pgfusepath{stroke}%
\end{pgfscope}%
\begin{pgfscope}%
\pgfpathrectangle{\pgfqpoint{0.720000in}{0.480000in}}{\pgfqpoint{7.040000in}{3.120000in}}%
\pgfusepath{clip}%
\pgfsetbuttcap%
\pgfsetroundjoin%
\definecolor{currentfill}{rgb}{1.000000,0.647059,0.000000}%
\pgfsetfillcolor{currentfill}%
\pgfsetlinewidth{1.003750pt}%
\definecolor{currentstroke}{rgb}{1.000000,0.647059,0.000000}%
\pgfsetstrokecolor{currentstroke}%
\pgfsetdash{}{0pt}%
\pgfsys@defobject{currentmarker}{\pgfqpoint{-0.010417in}{-0.010417in}}{\pgfqpoint{0.010417in}{0.010417in}}{%
\pgfpathmoveto{\pgfqpoint{0.000000in}{-0.010417in}}%
\pgfpathcurveto{\pgfqpoint{0.002763in}{-0.010417in}}{\pgfqpoint{0.005412in}{-0.009319in}}{\pgfqpoint{0.007366in}{-0.007366in}}%
\pgfpathcurveto{\pgfqpoint{0.009319in}{-0.005412in}}{\pgfqpoint{0.010417in}{-0.002763in}}{\pgfqpoint{0.010417in}{0.000000in}}%
\pgfpathcurveto{\pgfqpoint{0.010417in}{0.002763in}}{\pgfqpoint{0.009319in}{0.005412in}}{\pgfqpoint{0.007366in}{0.007366in}}%
\pgfpathcurveto{\pgfqpoint{0.005412in}{0.009319in}}{\pgfqpoint{0.002763in}{0.010417in}}{\pgfqpoint{0.000000in}{0.010417in}}%
\pgfpathcurveto{\pgfqpoint{-0.002763in}{0.010417in}}{\pgfqpoint{-0.005412in}{0.009319in}}{\pgfqpoint{-0.007366in}{0.007366in}}%
\pgfpathcurveto{\pgfqpoint{-0.009319in}{0.005412in}}{\pgfqpoint{-0.010417in}{0.002763in}}{\pgfqpoint{-0.010417in}{0.000000in}}%
\pgfpathcurveto{\pgfqpoint{-0.010417in}{-0.002763in}}{\pgfqpoint{-0.009319in}{-0.005412in}}{\pgfqpoint{-0.007366in}{-0.007366in}}%
\pgfpathcurveto{\pgfqpoint{-0.005412in}{-0.009319in}}{\pgfqpoint{-0.002763in}{-0.010417in}}{\pgfqpoint{0.000000in}{-0.010417in}}%
\pgfpathclose%
\pgfusepath{stroke,fill}%
}%
\begin{pgfscope}%
\pgfsys@transformshift{1.038104in}{0.591291in}%
\pgfsys@useobject{currentmarker}{}%
\end{pgfscope}%
\begin{pgfscope}%
\pgfsys@transformshift{1.317096in}{0.679423in}%
\pgfsys@useobject{currentmarker}{}%
\end{pgfscope}%
\begin{pgfscope}%
\pgfsys@transformshift{1.594785in}{0.681674in}%
\pgfsys@useobject{currentmarker}{}%
\end{pgfscope}%
\begin{pgfscope}%
\pgfsys@transformshift{1.872474in}{0.813229in}%
\pgfsys@useobject{currentmarker}{}%
\end{pgfscope}%
\begin{pgfscope}%
\pgfsys@transformshift{2.151467in}{0.960866in}%
\pgfsys@useobject{currentmarker}{}%
\end{pgfscope}%
\begin{pgfscope}%
\pgfsys@transformshift{2.429807in}{1.002680in}%
\pgfsys@useobject{currentmarker}{}%
\end{pgfscope}%
\begin{pgfscope}%
\pgfsys@transformshift{2.707496in}{0.952181in}%
\pgfsys@useobject{currentmarker}{}%
\end{pgfscope}%
\begin{pgfscope}%
\pgfsys@transformshift{2.985185in}{1.256784in}%
\pgfsys@useobject{currentmarker}{}%
\end{pgfscope}%
\begin{pgfscope}%
\pgfsys@transformshift{3.262874in}{1.205320in}%
\pgfsys@useobject{currentmarker}{}%
\end{pgfscope}%
\begin{pgfscope}%
\pgfsys@transformshift{3.540563in}{1.302779in}%
\pgfsys@useobject{currentmarker}{}%
\end{pgfscope}%
\begin{pgfscope}%
\pgfsys@transformshift{3.818252in}{1.292808in}%
\pgfsys@useobject{currentmarker}{}%
\end{pgfscope}%
\begin{pgfscope}%
\pgfsys@transformshift{4.095941in}{1.335588in}%
\pgfsys@useobject{currentmarker}{}%
\end{pgfscope}%
\begin{pgfscope}%
\pgfsys@transformshift{4.373630in}{1.335588in}%
\pgfsys@useobject{currentmarker}{}%
\end{pgfscope}%
\begin{pgfscope}%
\pgfsys@transformshift{4.651319in}{1.406672in}%
\pgfsys@useobject{currentmarker}{}%
\end{pgfscope}%
\begin{pgfscope}%
\pgfsys@transformshift{4.929659in}{1.520536in}%
\pgfsys@useobject{currentmarker}{}%
\end{pgfscope}%
\begin{pgfscope}%
\pgfsys@transformshift{5.208000in}{1.540157in}%
\pgfsys@useobject{currentmarker}{}%
\end{pgfscope}%
\begin{pgfscope}%
\pgfsys@transformshift{5.486341in}{1.512816in}%
\pgfsys@useobject{currentmarker}{}%
\end{pgfscope}%
\begin{pgfscope}%
\pgfsys@transformshift{5.764030in}{1.493196in}%
\pgfsys@useobject{currentmarker}{}%
\end{pgfscope}%
\begin{pgfscope}%
\pgfsys@transformshift{6.042370in}{1.494161in}%
\pgfsys@useobject{currentmarker}{}%
\end{pgfscope}%
\begin{pgfscope}%
\pgfsys@transformshift{6.320711in}{1.521823in}%
\pgfsys@useobject{currentmarker}{}%
\end{pgfscope}%
\begin{pgfscope}%
\pgfsys@transformshift{6.599052in}{1.549163in}%
\pgfsys@useobject{currentmarker}{}%
\end{pgfscope}%
\end{pgfscope}%
\begin{pgfscope}%
\pgfpathrectangle{\pgfqpoint{0.720000in}{0.480000in}}{\pgfqpoint{7.040000in}{3.120000in}}%
\pgfusepath{clip}%
\pgfsetbuttcap%
\pgfsetroundjoin%
\pgfsetlinewidth{0.501875pt}%
\definecolor{currentstroke}{rgb}{1.000000,0.647059,0.000000}%
\pgfsetstrokecolor{currentstroke}%
\pgfsetdash{{1.850000pt}{0.800000pt}}{0.000000pt}%
\pgfpathmoveto{\pgfqpoint{1.038104in}{0.494153in}}%
\pgfpathlineto{\pgfqpoint{1.317096in}{0.597724in}}%
\pgfpathlineto{\pgfqpoint{1.594785in}{0.677814in}}%
\pgfpathlineto{\pgfqpoint{1.872474in}{0.729922in}}%
\pgfpathlineto{\pgfqpoint{2.151467in}{0.815802in}}%
\pgfpathlineto{\pgfqpoint{2.429807in}{0.913584in}}%
\pgfpathlineto{\pgfqpoint{2.707496in}{1.003967in}}%
\pgfpathlineto{\pgfqpoint{2.985185in}{1.126515in}}%
\pgfpathlineto{\pgfqpoint{3.262874in}{1.212396in}}%
\pgfpathlineto{\pgfqpoint{3.540563in}{1.275439in}}%
\pgfpathlineto{\pgfqpoint{3.818252in}{1.364214in}}%
\pgfpathlineto{\pgfqpoint{4.095941in}{1.448487in}}%
\pgfpathlineto{\pgfqpoint{4.373630in}{1.518928in}}%
\pgfpathlineto{\pgfqpoint{4.651319in}{1.598375in}}%
\pgfpathlineto{\pgfqpoint{4.929659in}{1.767885in}}%
\pgfpathlineto{\pgfqpoint{5.208000in}{1.847010in}}%
\pgfpathlineto{\pgfqpoint{5.486341in}{1.941254in}}%
\pgfpathlineto{\pgfqpoint{5.764030in}{2.002045in}}%
\pgfpathlineto{\pgfqpoint{6.042370in}{2.058656in}}%
\pgfpathlineto{\pgfqpoint{6.320711in}{2.135530in}}%
\pgfpathlineto{\pgfqpoint{6.599052in}{2.210153in}}%
\pgfusepath{stroke}%
\end{pgfscope}%
\begin{pgfscope}%
\pgfpathrectangle{\pgfqpoint{0.720000in}{0.480000in}}{\pgfqpoint{7.040000in}{3.120000in}}%
\pgfusepath{clip}%
\pgfsetbuttcap%
\pgfsetroundjoin%
\definecolor{currentfill}{rgb}{1.000000,0.647059,0.000000}%
\pgfsetfillcolor{currentfill}%
\pgfsetlinewidth{1.003750pt}%
\definecolor{currentstroke}{rgb}{1.000000,0.647059,0.000000}%
\pgfsetstrokecolor{currentstroke}%
\pgfsetdash{}{0pt}%
\pgfsys@defobject{currentmarker}{\pgfqpoint{-0.006944in}{-0.006944in}}{\pgfqpoint{0.006944in}{0.006944in}}{%
\pgfpathmoveto{\pgfqpoint{0.000000in}{-0.006944in}}%
\pgfpathcurveto{\pgfqpoint{0.001842in}{-0.006944in}}{\pgfqpoint{0.003608in}{-0.006213in}}{\pgfqpoint{0.004910in}{-0.004910in}}%
\pgfpathcurveto{\pgfqpoint{0.006213in}{-0.003608in}}{\pgfqpoint{0.006944in}{-0.001842in}}{\pgfqpoint{0.006944in}{0.000000in}}%
\pgfpathcurveto{\pgfqpoint{0.006944in}{0.001842in}}{\pgfqpoint{0.006213in}{0.003608in}}{\pgfqpoint{0.004910in}{0.004910in}}%
\pgfpathcurveto{\pgfqpoint{0.003608in}{0.006213in}}{\pgfqpoint{0.001842in}{0.006944in}}{\pgfqpoint{0.000000in}{0.006944in}}%
\pgfpathcurveto{\pgfqpoint{-0.001842in}{0.006944in}}{\pgfqpoint{-0.003608in}{0.006213in}}{\pgfqpoint{-0.004910in}{0.004910in}}%
\pgfpathcurveto{\pgfqpoint{-0.006213in}{0.003608in}}{\pgfqpoint{-0.006944in}{0.001842in}}{\pgfqpoint{-0.006944in}{0.000000in}}%
\pgfpathcurveto{\pgfqpoint{-0.006944in}{-0.001842in}}{\pgfqpoint{-0.006213in}{-0.003608in}}{\pgfqpoint{-0.004910in}{-0.004910in}}%
\pgfpathcurveto{\pgfqpoint{-0.003608in}{-0.006213in}}{\pgfqpoint{-0.001842in}{-0.006944in}}{\pgfqpoint{0.000000in}{-0.006944in}}%
\pgfpathclose%
\pgfusepath{stroke,fill}%
}%
\begin{pgfscope}%
\pgfsys@transformshift{1.038104in}{0.494153in}%
\pgfsys@useobject{currentmarker}{}%
\end{pgfscope}%
\begin{pgfscope}%
\pgfsys@transformshift{1.317096in}{0.597724in}%
\pgfsys@useobject{currentmarker}{}%
\end{pgfscope}%
\begin{pgfscope}%
\pgfsys@transformshift{1.594785in}{0.677814in}%
\pgfsys@useobject{currentmarker}{}%
\end{pgfscope}%
\begin{pgfscope}%
\pgfsys@transformshift{1.872474in}{0.729922in}%
\pgfsys@useobject{currentmarker}{}%
\end{pgfscope}%
\begin{pgfscope}%
\pgfsys@transformshift{2.151467in}{0.815802in}%
\pgfsys@useobject{currentmarker}{}%
\end{pgfscope}%
\begin{pgfscope}%
\pgfsys@transformshift{2.429807in}{0.913584in}%
\pgfsys@useobject{currentmarker}{}%
\end{pgfscope}%
\begin{pgfscope}%
\pgfsys@transformshift{2.707496in}{1.003967in}%
\pgfsys@useobject{currentmarker}{}%
\end{pgfscope}%
\begin{pgfscope}%
\pgfsys@transformshift{2.985185in}{1.126515in}%
\pgfsys@useobject{currentmarker}{}%
\end{pgfscope}%
\begin{pgfscope}%
\pgfsys@transformshift{3.262874in}{1.212396in}%
\pgfsys@useobject{currentmarker}{}%
\end{pgfscope}%
\begin{pgfscope}%
\pgfsys@transformshift{3.540563in}{1.275439in}%
\pgfsys@useobject{currentmarker}{}%
\end{pgfscope}%
\begin{pgfscope}%
\pgfsys@transformshift{3.818252in}{1.364214in}%
\pgfsys@useobject{currentmarker}{}%
\end{pgfscope}%
\begin{pgfscope}%
\pgfsys@transformshift{4.095941in}{1.448487in}%
\pgfsys@useobject{currentmarker}{}%
\end{pgfscope}%
\begin{pgfscope}%
\pgfsys@transformshift{4.373630in}{1.518928in}%
\pgfsys@useobject{currentmarker}{}%
\end{pgfscope}%
\begin{pgfscope}%
\pgfsys@transformshift{4.651319in}{1.598375in}%
\pgfsys@useobject{currentmarker}{}%
\end{pgfscope}%
\begin{pgfscope}%
\pgfsys@transformshift{4.929659in}{1.767885in}%
\pgfsys@useobject{currentmarker}{}%
\end{pgfscope}%
\begin{pgfscope}%
\pgfsys@transformshift{5.208000in}{1.847010in}%
\pgfsys@useobject{currentmarker}{}%
\end{pgfscope}%
\begin{pgfscope}%
\pgfsys@transformshift{5.486341in}{1.941254in}%
\pgfsys@useobject{currentmarker}{}%
\end{pgfscope}%
\begin{pgfscope}%
\pgfsys@transformshift{5.764030in}{2.002045in}%
\pgfsys@useobject{currentmarker}{}%
\end{pgfscope}%
\begin{pgfscope}%
\pgfsys@transformshift{6.042370in}{2.058656in}%
\pgfsys@useobject{currentmarker}{}%
\end{pgfscope}%
\begin{pgfscope}%
\pgfsys@transformshift{6.320711in}{2.135530in}%
\pgfsys@useobject{currentmarker}{}%
\end{pgfscope}%
\begin{pgfscope}%
\pgfsys@transformshift{6.599052in}{2.210153in}%
\pgfsys@useobject{currentmarker}{}%
\end{pgfscope}%
\end{pgfscope}%
\begin{pgfscope}%
\pgfsetrectcap%
\pgfsetmiterjoin%
\pgfsetlinewidth{0.803000pt}%
\definecolor{currentstroke}{rgb}{0.000000,0.000000,0.000000}%
\pgfsetstrokecolor{currentstroke}%
\pgfsetdash{}{0pt}%
\pgfpathmoveto{\pgfqpoint{0.720000in}{0.480000in}}%
\pgfpathlineto{\pgfqpoint{0.720000in}{3.600000in}}%
\pgfusepath{stroke}%
\end{pgfscope}%
\begin{pgfscope}%
\pgfsetrectcap%
\pgfsetmiterjoin%
\pgfsetlinewidth{0.803000pt}%
\definecolor{currentstroke}{rgb}{0.000000,0.000000,0.000000}%
\pgfsetstrokecolor{currentstroke}%
\pgfsetdash{}{0pt}%
\pgfpathmoveto{\pgfqpoint{7.760000in}{0.480000in}}%
\pgfpathlineto{\pgfqpoint{7.760000in}{3.600000in}}%
\pgfusepath{stroke}%
\end{pgfscope}%
\begin{pgfscope}%
\pgfsetrectcap%
\pgfsetmiterjoin%
\pgfsetlinewidth{0.803000pt}%
\definecolor{currentstroke}{rgb}{0.000000,0.000000,0.000000}%
\pgfsetstrokecolor{currentstroke}%
\pgfsetdash{}{0pt}%
\pgfpathmoveto{\pgfqpoint{0.720000in}{0.480000in}}%
\pgfpathlineto{\pgfqpoint{7.760000in}{0.480000in}}%
\pgfusepath{stroke}%
\end{pgfscope}%
\begin{pgfscope}%
\pgfsetrectcap%
\pgfsetmiterjoin%
\pgfsetlinewidth{0.803000pt}%
\definecolor{currentstroke}{rgb}{0.000000,0.000000,0.000000}%
\pgfsetstrokecolor{currentstroke}%
\pgfsetdash{}{0pt}%
\pgfpathmoveto{\pgfqpoint{0.720000in}{3.600000in}}%
\pgfpathlineto{\pgfqpoint{7.760000in}{3.600000in}}%
\pgfusepath{stroke}%
\end{pgfscope}%
\begin{pgfscope}%
\definecolor{textcolor}{rgb}{0.000000,0.000000,0.000000}%
\pgfsetstrokecolor{textcolor}%
\pgfsetfillcolor{textcolor}%
\pgftext[x=1.437910in,y=3.838851in,left,base]{\color{textcolor}\sffamily\fontsize{10.000000}{12.000000}\selectfont model training "food-50"; InceptionV3; 21 epochs; min. 02:30:19; max 02:30:20}%
\end{pgfscope}%
\begin{pgfscope}%
\definecolor{textcolor}{rgb}{0.000000,0.000000,0.000000}%
\pgfsetstrokecolor{textcolor}%
\pgfsetfillcolor{textcolor}%
\pgftext[x=3.389611in,y=3.683333in,left,base]{\color{textcolor}\sffamily\fontsize{10.000000}{12.000000}\selectfont (GPU - Nvidia GTX 1060)}%
\end{pgfscope}%
\begin{pgfscope}%
\pgfsetbuttcap%
\pgfsetmiterjoin%
\definecolor{currentfill}{rgb}{1.000000,1.000000,1.000000}%
\pgfsetfillcolor{currentfill}%
\pgfsetfillopacity{0.800000}%
\pgfsetlinewidth{1.003750pt}%
\definecolor{currentstroke}{rgb}{0.800000,0.800000,0.800000}%
\pgfsetstrokecolor{currentstroke}%
\pgfsetstrokeopacity{0.800000}%
\pgfsetdash{}{0pt}%
\pgfpathmoveto{\pgfqpoint{3.665139in}{0.542500in}}%
\pgfpathlineto{\pgfqpoint{7.672500in}{0.542500in}}%
\pgfpathquadraticcurveto{\pgfqpoint{7.697500in}{0.542500in}}{\pgfqpoint{7.697500in}{0.567500in}}%
\pgfpathlineto{\pgfqpoint{7.697500in}{1.288886in}}%
\pgfpathquadraticcurveto{\pgfqpoint{7.697500in}{1.313886in}}{\pgfqpoint{7.672500in}{1.313886in}}%
\pgfpathlineto{\pgfqpoint{3.665139in}{1.313886in}}%
\pgfpathquadraticcurveto{\pgfqpoint{3.640139in}{1.313886in}}{\pgfqpoint{3.640139in}{1.288886in}}%
\pgfpathlineto{\pgfqpoint{3.640139in}{0.567500in}}%
\pgfpathquadraticcurveto{\pgfqpoint{3.640139in}{0.542500in}}{\pgfqpoint{3.665139in}{0.542500in}}%
\pgfpathclose%
\pgfusepath{stroke,fill}%
\end{pgfscope}%
\begin{pgfscope}%
\pgfsetrectcap%
\pgfsetroundjoin%
\pgfsetlinewidth{0.752812pt}%
\definecolor{currentstroke}{rgb}{0.000000,0.000000,1.000000}%
\pgfsetstrokecolor{currentstroke}%
\pgfsetdash{}{0pt}%
\pgfpathmoveto{\pgfqpoint{3.690139in}{1.212665in}}%
\pgfpathlineto{\pgfqpoint{3.940139in}{1.212665in}}%
\pgfusepath{stroke}%
\end{pgfscope}%
\begin{pgfscope}%
\pgfsetbuttcap%
\pgfsetroundjoin%
\definecolor{currentfill}{rgb}{0.000000,0.000000,1.000000}%
\pgfsetfillcolor{currentfill}%
\pgfsetlinewidth{1.003750pt}%
\definecolor{currentstroke}{rgb}{0.000000,0.000000,1.000000}%
\pgfsetstrokecolor{currentstroke}%
\pgfsetdash{}{0pt}%
\pgfsys@defobject{currentmarker}{\pgfqpoint{-0.010417in}{-0.010417in}}{\pgfqpoint{0.010417in}{0.010417in}}{%
\pgfpathmoveto{\pgfqpoint{0.000000in}{-0.010417in}}%
\pgfpathcurveto{\pgfqpoint{0.002763in}{-0.010417in}}{\pgfqpoint{0.005412in}{-0.009319in}}{\pgfqpoint{0.007366in}{-0.007366in}}%
\pgfpathcurveto{\pgfqpoint{0.009319in}{-0.005412in}}{\pgfqpoint{0.010417in}{-0.002763in}}{\pgfqpoint{0.010417in}{0.000000in}}%
\pgfpathcurveto{\pgfqpoint{0.010417in}{0.002763in}}{\pgfqpoint{0.009319in}{0.005412in}}{\pgfqpoint{0.007366in}{0.007366in}}%
\pgfpathcurveto{\pgfqpoint{0.005412in}{0.009319in}}{\pgfqpoint{0.002763in}{0.010417in}}{\pgfqpoint{0.000000in}{0.010417in}}%
\pgfpathcurveto{\pgfqpoint{-0.002763in}{0.010417in}}{\pgfqpoint{-0.005412in}{0.009319in}}{\pgfqpoint{-0.007366in}{0.007366in}}%
\pgfpathcurveto{\pgfqpoint{-0.009319in}{0.005412in}}{\pgfqpoint{-0.010417in}{0.002763in}}{\pgfqpoint{-0.010417in}{0.000000in}}%
\pgfpathcurveto{\pgfqpoint{-0.010417in}{-0.002763in}}{\pgfqpoint{-0.009319in}{-0.005412in}}{\pgfqpoint{-0.007366in}{-0.007366in}}%
\pgfpathcurveto{\pgfqpoint{-0.005412in}{-0.009319in}}{\pgfqpoint{-0.002763in}{-0.010417in}}{\pgfqpoint{0.000000in}{-0.010417in}}%
\pgfpathclose%
\pgfusepath{stroke,fill}%
}%
\begin{pgfscope}%
\pgfsys@transformshift{3.815139in}{1.212665in}%
\pgfsys@useobject{currentmarker}{}%
\end{pgfscope}%
\end{pgfscope}%
\begin{pgfscope}%
\definecolor{textcolor}{rgb}{0.000000,0.000000,0.000000}%
\pgfsetstrokecolor{textcolor}%
\pgfsetfillcolor{textcolor}%
\pgftext[x=4.040139in,y=1.168915in,left,base]{\color{textcolor}\sffamily\fontsize{9.000000}{10.800000}\selectfont imagenet (83.59\% - 02:30:20) (Validation)}%
\end{pgfscope}%
\begin{pgfscope}%
\pgfsetbuttcap%
\pgfsetroundjoin%
\pgfsetlinewidth{0.501875pt}%
\definecolor{currentstroke}{rgb}{0.000000,0.000000,1.000000}%
\pgfsetstrokecolor{currentstroke}%
\pgfsetdash{{1.850000pt}{0.800000pt}}{0.000000pt}%
\pgfpathmoveto{\pgfqpoint{3.690139in}{1.029194in}}%
\pgfpathlineto{\pgfqpoint{3.940139in}{1.029194in}}%
\pgfusepath{stroke}%
\end{pgfscope}%
\begin{pgfscope}%
\pgfsetbuttcap%
\pgfsetroundjoin%
\definecolor{currentfill}{rgb}{0.000000,0.000000,1.000000}%
\pgfsetfillcolor{currentfill}%
\pgfsetlinewidth{1.003750pt}%
\definecolor{currentstroke}{rgb}{0.000000,0.000000,1.000000}%
\pgfsetstrokecolor{currentstroke}%
\pgfsetdash{}{0pt}%
\pgfsys@defobject{currentmarker}{\pgfqpoint{-0.006944in}{-0.006944in}}{\pgfqpoint{0.006944in}{0.006944in}}{%
\pgfpathmoveto{\pgfqpoint{0.000000in}{-0.006944in}}%
\pgfpathcurveto{\pgfqpoint{0.001842in}{-0.006944in}}{\pgfqpoint{0.003608in}{-0.006213in}}{\pgfqpoint{0.004910in}{-0.004910in}}%
\pgfpathcurveto{\pgfqpoint{0.006213in}{-0.003608in}}{\pgfqpoint{0.006944in}{-0.001842in}}{\pgfqpoint{0.006944in}{0.000000in}}%
\pgfpathcurveto{\pgfqpoint{0.006944in}{0.001842in}}{\pgfqpoint{0.006213in}{0.003608in}}{\pgfqpoint{0.004910in}{0.004910in}}%
\pgfpathcurveto{\pgfqpoint{0.003608in}{0.006213in}}{\pgfqpoint{0.001842in}{0.006944in}}{\pgfqpoint{0.000000in}{0.006944in}}%
\pgfpathcurveto{\pgfqpoint{-0.001842in}{0.006944in}}{\pgfqpoint{-0.003608in}{0.006213in}}{\pgfqpoint{-0.004910in}{0.004910in}}%
\pgfpathcurveto{\pgfqpoint{-0.006213in}{0.003608in}}{\pgfqpoint{-0.006944in}{0.001842in}}{\pgfqpoint{-0.006944in}{0.000000in}}%
\pgfpathcurveto{\pgfqpoint{-0.006944in}{-0.001842in}}{\pgfqpoint{-0.006213in}{-0.003608in}}{\pgfqpoint{-0.004910in}{-0.004910in}}%
\pgfpathcurveto{\pgfqpoint{-0.003608in}{-0.006213in}}{\pgfqpoint{-0.001842in}{-0.006944in}}{\pgfqpoint{0.000000in}{-0.006944in}}%
\pgfpathclose%
\pgfusepath{stroke,fill}%
}%
\begin{pgfscope}%
\pgfsys@transformshift{3.815139in}{1.029194in}%
\pgfsys@useobject{currentmarker}{}%
\end{pgfscope}%
\end{pgfscope}%
\begin{pgfscope}%
\definecolor{textcolor}{rgb}{0.000000,0.000000,0.000000}%
\pgfsetstrokecolor{textcolor}%
\pgfsetfillcolor{textcolor}%
\pgftext[x=4.040139in,y=0.985444in,left,base]{\color{textcolor}\sffamily\fontsize{9.000000}{10.800000}\selectfont imagenet (99.57\% - 02:30:20) (Training)}%
\end{pgfscope}%
\begin{pgfscope}%
\pgfsetrectcap%
\pgfsetroundjoin%
\pgfsetlinewidth{0.752812pt}%
\definecolor{currentstroke}{rgb}{1.000000,0.647059,0.000000}%
\pgfsetstrokecolor{currentstroke}%
\pgfsetdash{}{0pt}%
\pgfpathmoveto{\pgfqpoint{3.690139in}{0.845722in}}%
\pgfpathlineto{\pgfqpoint{3.940139in}{0.845722in}}%
\pgfusepath{stroke}%
\end{pgfscope}%
\begin{pgfscope}%
\pgfsetbuttcap%
\pgfsetroundjoin%
\definecolor{currentfill}{rgb}{1.000000,0.647059,0.000000}%
\pgfsetfillcolor{currentfill}%
\pgfsetlinewidth{1.003750pt}%
\definecolor{currentstroke}{rgb}{1.000000,0.647059,0.000000}%
\pgfsetstrokecolor{currentstroke}%
\pgfsetdash{}{0pt}%
\pgfsys@defobject{currentmarker}{\pgfqpoint{-0.010417in}{-0.010417in}}{\pgfqpoint{0.010417in}{0.010417in}}{%
\pgfpathmoveto{\pgfqpoint{0.000000in}{-0.010417in}}%
\pgfpathcurveto{\pgfqpoint{0.002763in}{-0.010417in}}{\pgfqpoint{0.005412in}{-0.009319in}}{\pgfqpoint{0.007366in}{-0.007366in}}%
\pgfpathcurveto{\pgfqpoint{0.009319in}{-0.005412in}}{\pgfqpoint{0.010417in}{-0.002763in}}{\pgfqpoint{0.010417in}{0.000000in}}%
\pgfpathcurveto{\pgfqpoint{0.010417in}{0.002763in}}{\pgfqpoint{0.009319in}{0.005412in}}{\pgfqpoint{0.007366in}{0.007366in}}%
\pgfpathcurveto{\pgfqpoint{0.005412in}{0.009319in}}{\pgfqpoint{0.002763in}{0.010417in}}{\pgfqpoint{0.000000in}{0.010417in}}%
\pgfpathcurveto{\pgfqpoint{-0.002763in}{0.010417in}}{\pgfqpoint{-0.005412in}{0.009319in}}{\pgfqpoint{-0.007366in}{0.007366in}}%
\pgfpathcurveto{\pgfqpoint{-0.009319in}{0.005412in}}{\pgfqpoint{-0.010417in}{0.002763in}}{\pgfqpoint{-0.010417in}{0.000000in}}%
\pgfpathcurveto{\pgfqpoint{-0.010417in}{-0.002763in}}{\pgfqpoint{-0.009319in}{-0.005412in}}{\pgfqpoint{-0.007366in}{-0.007366in}}%
\pgfpathcurveto{\pgfqpoint{-0.005412in}{-0.009319in}}{\pgfqpoint{-0.002763in}{-0.010417in}}{\pgfqpoint{0.000000in}{-0.010417in}}%
\pgfpathclose%
\pgfusepath{stroke,fill}%
}%
\begin{pgfscope}%
\pgfsys@transformshift{3.815139in}{0.845722in}%
\pgfsys@useobject{currentmarker}{}%
\end{pgfscope}%
\end{pgfscope}%
\begin{pgfscope}%
\definecolor{textcolor}{rgb}{0.000000,0.000000,0.000000}%
\pgfsetstrokecolor{textcolor}%
\pgfsetfillcolor{textcolor}%
\pgftext[x=4.040139in,y=0.801972in,left,base]{\color{textcolor}\sffamily\fontsize{9.000000}{10.800000}\selectfont without transfer learning (38.24\% - 02:30:19) (Validation)}%
\end{pgfscope}%
\begin{pgfscope}%
\pgfsetbuttcap%
\pgfsetroundjoin%
\pgfsetlinewidth{0.501875pt}%
\definecolor{currentstroke}{rgb}{1.000000,0.647059,0.000000}%
\pgfsetstrokecolor{currentstroke}%
\pgfsetdash{{1.850000pt}{0.800000pt}}{0.000000pt}%
\pgfpathmoveto{\pgfqpoint{3.690139in}{0.662251in}}%
\pgfpathlineto{\pgfqpoint{3.940139in}{0.662251in}}%
\pgfusepath{stroke}%
\end{pgfscope}%
\begin{pgfscope}%
\pgfsetbuttcap%
\pgfsetroundjoin%
\definecolor{currentfill}{rgb}{1.000000,0.647059,0.000000}%
\pgfsetfillcolor{currentfill}%
\pgfsetlinewidth{1.003750pt}%
\definecolor{currentstroke}{rgb}{1.000000,0.647059,0.000000}%
\pgfsetstrokecolor{currentstroke}%
\pgfsetdash{}{0pt}%
\pgfsys@defobject{currentmarker}{\pgfqpoint{-0.006944in}{-0.006944in}}{\pgfqpoint{0.006944in}{0.006944in}}{%
\pgfpathmoveto{\pgfqpoint{0.000000in}{-0.006944in}}%
\pgfpathcurveto{\pgfqpoint{0.001842in}{-0.006944in}}{\pgfqpoint{0.003608in}{-0.006213in}}{\pgfqpoint{0.004910in}{-0.004910in}}%
\pgfpathcurveto{\pgfqpoint{0.006213in}{-0.003608in}}{\pgfqpoint{0.006944in}{-0.001842in}}{\pgfqpoint{0.006944in}{0.000000in}}%
\pgfpathcurveto{\pgfqpoint{0.006944in}{0.001842in}}{\pgfqpoint{0.006213in}{0.003608in}}{\pgfqpoint{0.004910in}{0.004910in}}%
\pgfpathcurveto{\pgfqpoint{0.003608in}{0.006213in}}{\pgfqpoint{0.001842in}{0.006944in}}{\pgfqpoint{0.000000in}{0.006944in}}%
\pgfpathcurveto{\pgfqpoint{-0.001842in}{0.006944in}}{\pgfqpoint{-0.003608in}{0.006213in}}{\pgfqpoint{-0.004910in}{0.004910in}}%
\pgfpathcurveto{\pgfqpoint{-0.006213in}{0.003608in}}{\pgfqpoint{-0.006944in}{0.001842in}}{\pgfqpoint{-0.006944in}{0.000000in}}%
\pgfpathcurveto{\pgfqpoint{-0.006944in}{-0.001842in}}{\pgfqpoint{-0.006213in}{-0.003608in}}{\pgfqpoint{-0.004910in}{-0.004910in}}%
\pgfpathcurveto{\pgfqpoint{-0.003608in}{-0.006213in}}{\pgfqpoint{-0.001842in}{-0.006944in}}{\pgfqpoint{0.000000in}{-0.006944in}}%
\pgfpathclose%
\pgfusepath{stroke,fill}%
}%
\begin{pgfscope}%
\pgfsys@transformshift{3.815139in}{0.662251in}%
\pgfsys@useobject{currentmarker}{}%
\end{pgfscope}%
\end{pgfscope}%
\begin{pgfscope}%
\definecolor{textcolor}{rgb}{0.000000,0.000000,0.000000}%
\pgfsetstrokecolor{textcolor}%
\pgfsetfillcolor{textcolor}%
\pgftext[x=4.040139in,y=0.618501in,left,base]{\color{textcolor}\sffamily\fontsize{9.000000}{10.800000}\selectfont without transfer learning (58.79\% - 02:30:19) (Training)}%
\end{pgfscope}%
\end{pgfpicture}%
\makeatother%
\endgroup%

					\end{center}
					\caption{Overview of use of the transfer learning approach}
					\label{fig:evaluation_transfer_learning}
				\end{figure}

				\noindent ...

			\subsubsection{Influence of different error optimizers}

				\paragraph{Comparison Optimizer}

				\noindent ...

				\begin{figure}[H]
					\begin{center}
						%% Creator: Matplotlib, PGF backend
%%
%% To include the figure in your LaTeX document, write
%%   \input{<filename>.pgf}
%%
%% Make sure the required packages are loaded in your preamble
%%   \usepackage{pgf}
%%
%% Figures using additional raster images can only be included by \input if
%% they are in the same directory as the main LaTeX file. For loading figures
%% from other directories you can use the `import` package
%%   \usepackage{import}
%% and then include the figures with
%%   \import{<path to file>}{<filename>.pgf}
%%
%% Matplotlib used the following preamble
%%   \usepackage{fontspec}
%%   \setmainfont{DejaVuSerif.ttf}[Path=C:/Users/bjoern/Anaconda3/envs/keras-gpu/lib/site-packages/matplotlib/mpl-data/fonts/ttf/]
%%   \setsansfont{DejaVuSans.ttf}[Path=C:/Users/bjoern/Anaconda3/envs/keras-gpu/lib/site-packages/matplotlib/mpl-data/fonts/ttf/]
%%   \setmonofont{DejaVuSansMono.ttf}[Path=C:/Users/bjoern/Anaconda3/envs/keras-gpu/lib/site-packages/matplotlib/mpl-data/fonts/ttf/]
%%
\begingroup%
\makeatletter%
\begin{pgfpicture}%
\pgfpathrectangle{\pgfpointorigin}{\pgfqpoint{6.400000in}{4.800000in}}%
\pgfusepath{use as bounding box, clip}%
\begin{pgfscope}%
\pgfsetbuttcap%
\pgfsetmiterjoin%
\definecolor{currentfill}{rgb}{1.000000,1.000000,1.000000}%
\pgfsetfillcolor{currentfill}%
\pgfsetlinewidth{0.000000pt}%
\definecolor{currentstroke}{rgb}{1.000000,1.000000,1.000000}%
\pgfsetstrokecolor{currentstroke}%
\pgfsetdash{}{0pt}%
\pgfpathmoveto{\pgfqpoint{0.000000in}{0.000000in}}%
\pgfpathlineto{\pgfqpoint{6.400000in}{0.000000in}}%
\pgfpathlineto{\pgfqpoint{6.400000in}{4.800000in}}%
\pgfpathlineto{\pgfqpoint{0.000000in}{4.800000in}}%
\pgfpathclose%
\pgfusepath{fill}%
\end{pgfscope}%
\begin{pgfscope}%
\pgfsetbuttcap%
\pgfsetmiterjoin%
\definecolor{currentfill}{rgb}{1.000000,1.000000,1.000000}%
\pgfsetfillcolor{currentfill}%
\pgfsetlinewidth{0.000000pt}%
\definecolor{currentstroke}{rgb}{0.000000,0.000000,0.000000}%
\pgfsetstrokecolor{currentstroke}%
\pgfsetstrokeopacity{0.000000}%
\pgfsetdash{}{0pt}%
\pgfpathmoveto{\pgfqpoint{0.800000in}{0.528000in}}%
\pgfpathlineto{\pgfqpoint{5.760000in}{0.528000in}}%
\pgfpathlineto{\pgfqpoint{5.760000in}{4.224000in}}%
\pgfpathlineto{\pgfqpoint{0.800000in}{4.224000in}}%
\pgfpathclose%
\pgfusepath{fill}%
\end{pgfscope}%
\begin{pgfscope}%
\pgfpathrectangle{\pgfqpoint{0.800000in}{0.528000in}}{\pgfqpoint{4.960000in}{3.696000in}}%
\pgfusepath{clip}%
\pgfsetrectcap%
\pgfsetroundjoin%
\pgfsetlinewidth{0.803000pt}%
\definecolor{currentstroke}{rgb}{0.690196,0.690196,0.690196}%
\pgfsetstrokecolor{currentstroke}%
\pgfsetdash{}{0pt}%
\pgfpathmoveto{\pgfqpoint{0.800000in}{0.528000in}}%
\pgfpathlineto{\pgfqpoint{0.800000in}{4.224000in}}%
\pgfusepath{stroke}%
\end{pgfscope}%
\begin{pgfscope}%
\pgfsetbuttcap%
\pgfsetroundjoin%
\definecolor{currentfill}{rgb}{0.000000,0.000000,0.000000}%
\pgfsetfillcolor{currentfill}%
\pgfsetlinewidth{0.803000pt}%
\definecolor{currentstroke}{rgb}{0.000000,0.000000,0.000000}%
\pgfsetstrokecolor{currentstroke}%
\pgfsetdash{}{0pt}%
\pgfsys@defobject{currentmarker}{\pgfqpoint{0.000000in}{-0.048611in}}{\pgfqpoint{0.000000in}{0.000000in}}{%
\pgfpathmoveto{\pgfqpoint{0.000000in}{0.000000in}}%
\pgfpathlineto{\pgfqpoint{0.000000in}{-0.048611in}}%
\pgfusepath{stroke,fill}%
}%
\begin{pgfscope}%
\pgfsys@transformshift{0.800000in}{0.528000in}%
\pgfsys@useobject{currentmarker}{}%
\end{pgfscope}%
\end{pgfscope}%
\begin{pgfscope}%
\definecolor{textcolor}{rgb}{0.000000,0.000000,0.000000}%
\pgfsetstrokecolor{textcolor}%
\pgfsetfillcolor{textcolor}%
\pgftext[x=0.800000in,y=0.430778in,,top]{\color{textcolor}\sffamily\fontsize{10.000000}{12.000000}\selectfont 0}%
\end{pgfscope}%
\begin{pgfscope}%
\pgfpathrectangle{\pgfqpoint{0.800000in}{0.528000in}}{\pgfqpoint{4.960000in}{3.696000in}}%
\pgfusepath{clip}%
\pgfsetrectcap%
\pgfsetroundjoin%
\pgfsetlinewidth{0.803000pt}%
\definecolor{currentstroke}{rgb}{0.690196,0.690196,0.690196}%
\pgfsetstrokecolor{currentstroke}%
\pgfsetdash{}{0pt}%
\pgfpathmoveto{\pgfqpoint{2.453333in}{0.528000in}}%
\pgfpathlineto{\pgfqpoint{2.453333in}{4.224000in}}%
\pgfusepath{stroke}%
\end{pgfscope}%
\begin{pgfscope}%
\pgfsetbuttcap%
\pgfsetroundjoin%
\definecolor{currentfill}{rgb}{0.000000,0.000000,0.000000}%
\pgfsetfillcolor{currentfill}%
\pgfsetlinewidth{0.803000pt}%
\definecolor{currentstroke}{rgb}{0.000000,0.000000,0.000000}%
\pgfsetstrokecolor{currentstroke}%
\pgfsetdash{}{0pt}%
\pgfsys@defobject{currentmarker}{\pgfqpoint{0.000000in}{-0.048611in}}{\pgfqpoint{0.000000in}{0.000000in}}{%
\pgfpathmoveto{\pgfqpoint{0.000000in}{0.000000in}}%
\pgfpathlineto{\pgfqpoint{0.000000in}{-0.048611in}}%
\pgfusepath{stroke,fill}%
}%
\begin{pgfscope}%
\pgfsys@transformshift{2.453333in}{0.528000in}%
\pgfsys@useobject{currentmarker}{}%
\end{pgfscope}%
\end{pgfscope}%
\begin{pgfscope}%
\definecolor{textcolor}{rgb}{0.000000,0.000000,0.000000}%
\pgfsetstrokecolor{textcolor}%
\pgfsetfillcolor{textcolor}%
\pgftext[x=2.453333in,y=0.430778in,,top]{\color{textcolor}\sffamily\fontsize{10.000000}{12.000000}\selectfont 3600}%
\end{pgfscope}%
\begin{pgfscope}%
\pgfpathrectangle{\pgfqpoint{0.800000in}{0.528000in}}{\pgfqpoint{4.960000in}{3.696000in}}%
\pgfusepath{clip}%
\pgfsetrectcap%
\pgfsetroundjoin%
\pgfsetlinewidth{0.803000pt}%
\definecolor{currentstroke}{rgb}{0.690196,0.690196,0.690196}%
\pgfsetstrokecolor{currentstroke}%
\pgfsetdash{}{0pt}%
\pgfpathmoveto{\pgfqpoint{4.106667in}{0.528000in}}%
\pgfpathlineto{\pgfqpoint{4.106667in}{4.224000in}}%
\pgfusepath{stroke}%
\end{pgfscope}%
\begin{pgfscope}%
\pgfsetbuttcap%
\pgfsetroundjoin%
\definecolor{currentfill}{rgb}{0.000000,0.000000,0.000000}%
\pgfsetfillcolor{currentfill}%
\pgfsetlinewidth{0.803000pt}%
\definecolor{currentstroke}{rgb}{0.000000,0.000000,0.000000}%
\pgfsetstrokecolor{currentstroke}%
\pgfsetdash{}{0pt}%
\pgfsys@defobject{currentmarker}{\pgfqpoint{0.000000in}{-0.048611in}}{\pgfqpoint{0.000000in}{0.000000in}}{%
\pgfpathmoveto{\pgfqpoint{0.000000in}{0.000000in}}%
\pgfpathlineto{\pgfqpoint{0.000000in}{-0.048611in}}%
\pgfusepath{stroke,fill}%
}%
\begin{pgfscope}%
\pgfsys@transformshift{4.106667in}{0.528000in}%
\pgfsys@useobject{currentmarker}{}%
\end{pgfscope}%
\end{pgfscope}%
\begin{pgfscope}%
\definecolor{textcolor}{rgb}{0.000000,0.000000,0.000000}%
\pgfsetstrokecolor{textcolor}%
\pgfsetfillcolor{textcolor}%
\pgftext[x=4.106667in,y=0.430778in,,top]{\color{textcolor}\sffamily\fontsize{10.000000}{12.000000}\selectfont 7200}%
\end{pgfscope}%
\begin{pgfscope}%
\pgfpathrectangle{\pgfqpoint{0.800000in}{0.528000in}}{\pgfqpoint{4.960000in}{3.696000in}}%
\pgfusepath{clip}%
\pgfsetrectcap%
\pgfsetroundjoin%
\pgfsetlinewidth{0.803000pt}%
\definecolor{currentstroke}{rgb}{0.690196,0.690196,0.690196}%
\pgfsetstrokecolor{currentstroke}%
\pgfsetdash{}{0pt}%
\pgfpathmoveto{\pgfqpoint{5.760000in}{0.528000in}}%
\pgfpathlineto{\pgfqpoint{5.760000in}{4.224000in}}%
\pgfusepath{stroke}%
\end{pgfscope}%
\begin{pgfscope}%
\pgfsetbuttcap%
\pgfsetroundjoin%
\definecolor{currentfill}{rgb}{0.000000,0.000000,0.000000}%
\pgfsetfillcolor{currentfill}%
\pgfsetlinewidth{0.803000pt}%
\definecolor{currentstroke}{rgb}{0.000000,0.000000,0.000000}%
\pgfsetstrokecolor{currentstroke}%
\pgfsetdash{}{0pt}%
\pgfsys@defobject{currentmarker}{\pgfqpoint{0.000000in}{-0.048611in}}{\pgfqpoint{0.000000in}{0.000000in}}{%
\pgfpathmoveto{\pgfqpoint{0.000000in}{0.000000in}}%
\pgfpathlineto{\pgfqpoint{0.000000in}{-0.048611in}}%
\pgfusepath{stroke,fill}%
}%
\begin{pgfscope}%
\pgfsys@transformshift{5.760000in}{0.528000in}%
\pgfsys@useobject{currentmarker}{}%
\end{pgfscope}%
\end{pgfscope}%
\begin{pgfscope}%
\definecolor{textcolor}{rgb}{0.000000,0.000000,0.000000}%
\pgfsetstrokecolor{textcolor}%
\pgfsetfillcolor{textcolor}%
\pgftext[x=5.760000in,y=0.430778in,,top]{\color{textcolor}\sffamily\fontsize{10.000000}{12.000000}\selectfont 10800}%
\end{pgfscope}%
\begin{pgfscope}%
\definecolor{textcolor}{rgb}{0.000000,0.000000,0.000000}%
\pgfsetstrokecolor{textcolor}%
\pgfsetfillcolor{textcolor}%
\pgftext[x=3.280000in,y=0.240809in,,top]{\color{textcolor}\sffamily\fontsize{10.000000}{12.000000}\selectfont duration [s]}%
\end{pgfscope}%
\begin{pgfscope}%
\pgfpathrectangle{\pgfqpoint{0.800000in}{0.528000in}}{\pgfqpoint{4.960000in}{3.696000in}}%
\pgfusepath{clip}%
\pgfsetrectcap%
\pgfsetroundjoin%
\pgfsetlinewidth{0.803000pt}%
\definecolor{currentstroke}{rgb}{0.690196,0.690196,0.690196}%
\pgfsetstrokecolor{currentstroke}%
\pgfsetdash{}{0pt}%
\pgfpathmoveto{\pgfqpoint{0.800000in}{0.528000in}}%
\pgfpathlineto{\pgfqpoint{5.760000in}{0.528000in}}%
\pgfusepath{stroke}%
\end{pgfscope}%
\begin{pgfscope}%
\pgfsetbuttcap%
\pgfsetroundjoin%
\definecolor{currentfill}{rgb}{0.000000,0.000000,0.000000}%
\pgfsetfillcolor{currentfill}%
\pgfsetlinewidth{0.803000pt}%
\definecolor{currentstroke}{rgb}{0.000000,0.000000,0.000000}%
\pgfsetstrokecolor{currentstroke}%
\pgfsetdash{}{0pt}%
\pgfsys@defobject{currentmarker}{\pgfqpoint{-0.048611in}{0.000000in}}{\pgfqpoint{0.000000in}{0.000000in}}{%
\pgfpathmoveto{\pgfqpoint{0.000000in}{0.000000in}}%
\pgfpathlineto{\pgfqpoint{-0.048611in}{0.000000in}}%
\pgfusepath{stroke,fill}%
}%
\begin{pgfscope}%
\pgfsys@transformshift{0.800000in}{0.528000in}%
\pgfsys@useobject{currentmarker}{}%
\end{pgfscope}%
\end{pgfscope}%
\begin{pgfscope}%
\definecolor{textcolor}{rgb}{0.000000,0.000000,0.000000}%
\pgfsetstrokecolor{textcolor}%
\pgfsetfillcolor{textcolor}%
\pgftext[x=0.482441in,y=0.475238in,left,base]{\color{textcolor}\sffamily\fontsize{10.000000}{12.000000}\selectfont 0\%}%
\end{pgfscope}%
\begin{pgfscope}%
\pgfpathrectangle{\pgfqpoint{0.800000in}{0.528000in}}{\pgfqpoint{4.960000in}{3.696000in}}%
\pgfusepath{clip}%
\pgfsetrectcap%
\pgfsetroundjoin%
\pgfsetlinewidth{0.803000pt}%
\definecolor{currentstroke}{rgb}{0.690196,0.690196,0.690196}%
\pgfsetstrokecolor{currentstroke}%
\pgfsetdash{}{0pt}%
\pgfpathmoveto{\pgfqpoint{0.800000in}{0.962824in}}%
\pgfpathlineto{\pgfqpoint{5.760000in}{0.962824in}}%
\pgfusepath{stroke}%
\end{pgfscope}%
\begin{pgfscope}%
\pgfsetbuttcap%
\pgfsetroundjoin%
\definecolor{currentfill}{rgb}{0.000000,0.000000,0.000000}%
\pgfsetfillcolor{currentfill}%
\pgfsetlinewidth{0.803000pt}%
\definecolor{currentstroke}{rgb}{0.000000,0.000000,0.000000}%
\pgfsetstrokecolor{currentstroke}%
\pgfsetdash{}{0pt}%
\pgfsys@defobject{currentmarker}{\pgfqpoint{-0.048611in}{0.000000in}}{\pgfqpoint{0.000000in}{0.000000in}}{%
\pgfpathmoveto{\pgfqpoint{0.000000in}{0.000000in}}%
\pgfpathlineto{\pgfqpoint{-0.048611in}{0.000000in}}%
\pgfusepath{stroke,fill}%
}%
\begin{pgfscope}%
\pgfsys@transformshift{0.800000in}{0.962824in}%
\pgfsys@useobject{currentmarker}{}%
\end{pgfscope}%
\end{pgfscope}%
\begin{pgfscope}%
\definecolor{textcolor}{rgb}{0.000000,0.000000,0.000000}%
\pgfsetstrokecolor{textcolor}%
\pgfsetfillcolor{textcolor}%
\pgftext[x=0.394076in,y=0.910062in,left,base]{\color{textcolor}\sffamily\fontsize{10.000000}{12.000000}\selectfont 10\%}%
\end{pgfscope}%
\begin{pgfscope}%
\pgfpathrectangle{\pgfqpoint{0.800000in}{0.528000in}}{\pgfqpoint{4.960000in}{3.696000in}}%
\pgfusepath{clip}%
\pgfsetrectcap%
\pgfsetroundjoin%
\pgfsetlinewidth{0.803000pt}%
\definecolor{currentstroke}{rgb}{0.690196,0.690196,0.690196}%
\pgfsetstrokecolor{currentstroke}%
\pgfsetdash{}{0pt}%
\pgfpathmoveto{\pgfqpoint{0.800000in}{1.397647in}}%
\pgfpathlineto{\pgfqpoint{5.760000in}{1.397647in}}%
\pgfusepath{stroke}%
\end{pgfscope}%
\begin{pgfscope}%
\pgfsetbuttcap%
\pgfsetroundjoin%
\definecolor{currentfill}{rgb}{0.000000,0.000000,0.000000}%
\pgfsetfillcolor{currentfill}%
\pgfsetlinewidth{0.803000pt}%
\definecolor{currentstroke}{rgb}{0.000000,0.000000,0.000000}%
\pgfsetstrokecolor{currentstroke}%
\pgfsetdash{}{0pt}%
\pgfsys@defobject{currentmarker}{\pgfqpoint{-0.048611in}{0.000000in}}{\pgfqpoint{0.000000in}{0.000000in}}{%
\pgfpathmoveto{\pgfqpoint{0.000000in}{0.000000in}}%
\pgfpathlineto{\pgfqpoint{-0.048611in}{0.000000in}}%
\pgfusepath{stroke,fill}%
}%
\begin{pgfscope}%
\pgfsys@transformshift{0.800000in}{1.397647in}%
\pgfsys@useobject{currentmarker}{}%
\end{pgfscope}%
\end{pgfscope}%
\begin{pgfscope}%
\definecolor{textcolor}{rgb}{0.000000,0.000000,0.000000}%
\pgfsetstrokecolor{textcolor}%
\pgfsetfillcolor{textcolor}%
\pgftext[x=0.394076in,y=1.344886in,left,base]{\color{textcolor}\sffamily\fontsize{10.000000}{12.000000}\selectfont 20\%}%
\end{pgfscope}%
\begin{pgfscope}%
\pgfpathrectangle{\pgfqpoint{0.800000in}{0.528000in}}{\pgfqpoint{4.960000in}{3.696000in}}%
\pgfusepath{clip}%
\pgfsetrectcap%
\pgfsetroundjoin%
\pgfsetlinewidth{0.803000pt}%
\definecolor{currentstroke}{rgb}{0.690196,0.690196,0.690196}%
\pgfsetstrokecolor{currentstroke}%
\pgfsetdash{}{0pt}%
\pgfpathmoveto{\pgfqpoint{0.800000in}{1.832471in}}%
\pgfpathlineto{\pgfqpoint{5.760000in}{1.832471in}}%
\pgfusepath{stroke}%
\end{pgfscope}%
\begin{pgfscope}%
\pgfsetbuttcap%
\pgfsetroundjoin%
\definecolor{currentfill}{rgb}{0.000000,0.000000,0.000000}%
\pgfsetfillcolor{currentfill}%
\pgfsetlinewidth{0.803000pt}%
\definecolor{currentstroke}{rgb}{0.000000,0.000000,0.000000}%
\pgfsetstrokecolor{currentstroke}%
\pgfsetdash{}{0pt}%
\pgfsys@defobject{currentmarker}{\pgfqpoint{-0.048611in}{0.000000in}}{\pgfqpoint{0.000000in}{0.000000in}}{%
\pgfpathmoveto{\pgfqpoint{0.000000in}{0.000000in}}%
\pgfpathlineto{\pgfqpoint{-0.048611in}{0.000000in}}%
\pgfusepath{stroke,fill}%
}%
\begin{pgfscope}%
\pgfsys@transformshift{0.800000in}{1.832471in}%
\pgfsys@useobject{currentmarker}{}%
\end{pgfscope}%
\end{pgfscope}%
\begin{pgfscope}%
\definecolor{textcolor}{rgb}{0.000000,0.000000,0.000000}%
\pgfsetstrokecolor{textcolor}%
\pgfsetfillcolor{textcolor}%
\pgftext[x=0.394076in,y=1.779709in,left,base]{\color{textcolor}\sffamily\fontsize{10.000000}{12.000000}\selectfont 30\%}%
\end{pgfscope}%
\begin{pgfscope}%
\pgfpathrectangle{\pgfqpoint{0.800000in}{0.528000in}}{\pgfqpoint{4.960000in}{3.696000in}}%
\pgfusepath{clip}%
\pgfsetrectcap%
\pgfsetroundjoin%
\pgfsetlinewidth{0.803000pt}%
\definecolor{currentstroke}{rgb}{0.690196,0.690196,0.690196}%
\pgfsetstrokecolor{currentstroke}%
\pgfsetdash{}{0pt}%
\pgfpathmoveto{\pgfqpoint{0.800000in}{2.267294in}}%
\pgfpathlineto{\pgfqpoint{5.760000in}{2.267294in}}%
\pgfusepath{stroke}%
\end{pgfscope}%
\begin{pgfscope}%
\pgfsetbuttcap%
\pgfsetroundjoin%
\definecolor{currentfill}{rgb}{0.000000,0.000000,0.000000}%
\pgfsetfillcolor{currentfill}%
\pgfsetlinewidth{0.803000pt}%
\definecolor{currentstroke}{rgb}{0.000000,0.000000,0.000000}%
\pgfsetstrokecolor{currentstroke}%
\pgfsetdash{}{0pt}%
\pgfsys@defobject{currentmarker}{\pgfqpoint{-0.048611in}{0.000000in}}{\pgfqpoint{0.000000in}{0.000000in}}{%
\pgfpathmoveto{\pgfqpoint{0.000000in}{0.000000in}}%
\pgfpathlineto{\pgfqpoint{-0.048611in}{0.000000in}}%
\pgfusepath{stroke,fill}%
}%
\begin{pgfscope}%
\pgfsys@transformshift{0.800000in}{2.267294in}%
\pgfsys@useobject{currentmarker}{}%
\end{pgfscope}%
\end{pgfscope}%
\begin{pgfscope}%
\definecolor{textcolor}{rgb}{0.000000,0.000000,0.000000}%
\pgfsetstrokecolor{textcolor}%
\pgfsetfillcolor{textcolor}%
\pgftext[x=0.394076in,y=2.214533in,left,base]{\color{textcolor}\sffamily\fontsize{10.000000}{12.000000}\selectfont 40\%}%
\end{pgfscope}%
\begin{pgfscope}%
\pgfpathrectangle{\pgfqpoint{0.800000in}{0.528000in}}{\pgfqpoint{4.960000in}{3.696000in}}%
\pgfusepath{clip}%
\pgfsetrectcap%
\pgfsetroundjoin%
\pgfsetlinewidth{0.803000pt}%
\definecolor{currentstroke}{rgb}{0.690196,0.690196,0.690196}%
\pgfsetstrokecolor{currentstroke}%
\pgfsetdash{}{0pt}%
\pgfpathmoveto{\pgfqpoint{0.800000in}{2.702118in}}%
\pgfpathlineto{\pgfqpoint{5.760000in}{2.702118in}}%
\pgfusepath{stroke}%
\end{pgfscope}%
\begin{pgfscope}%
\pgfsetbuttcap%
\pgfsetroundjoin%
\definecolor{currentfill}{rgb}{0.000000,0.000000,0.000000}%
\pgfsetfillcolor{currentfill}%
\pgfsetlinewidth{0.803000pt}%
\definecolor{currentstroke}{rgb}{0.000000,0.000000,0.000000}%
\pgfsetstrokecolor{currentstroke}%
\pgfsetdash{}{0pt}%
\pgfsys@defobject{currentmarker}{\pgfqpoint{-0.048611in}{0.000000in}}{\pgfqpoint{0.000000in}{0.000000in}}{%
\pgfpathmoveto{\pgfqpoint{0.000000in}{0.000000in}}%
\pgfpathlineto{\pgfqpoint{-0.048611in}{0.000000in}}%
\pgfusepath{stroke,fill}%
}%
\begin{pgfscope}%
\pgfsys@transformshift{0.800000in}{2.702118in}%
\pgfsys@useobject{currentmarker}{}%
\end{pgfscope}%
\end{pgfscope}%
\begin{pgfscope}%
\definecolor{textcolor}{rgb}{0.000000,0.000000,0.000000}%
\pgfsetstrokecolor{textcolor}%
\pgfsetfillcolor{textcolor}%
\pgftext[x=0.394076in,y=2.649356in,left,base]{\color{textcolor}\sffamily\fontsize{10.000000}{12.000000}\selectfont 50\%}%
\end{pgfscope}%
\begin{pgfscope}%
\pgfpathrectangle{\pgfqpoint{0.800000in}{0.528000in}}{\pgfqpoint{4.960000in}{3.696000in}}%
\pgfusepath{clip}%
\pgfsetrectcap%
\pgfsetroundjoin%
\pgfsetlinewidth{0.803000pt}%
\definecolor{currentstroke}{rgb}{0.690196,0.690196,0.690196}%
\pgfsetstrokecolor{currentstroke}%
\pgfsetdash{}{0pt}%
\pgfpathmoveto{\pgfqpoint{0.800000in}{3.136941in}}%
\pgfpathlineto{\pgfqpoint{5.760000in}{3.136941in}}%
\pgfusepath{stroke}%
\end{pgfscope}%
\begin{pgfscope}%
\pgfsetbuttcap%
\pgfsetroundjoin%
\definecolor{currentfill}{rgb}{0.000000,0.000000,0.000000}%
\pgfsetfillcolor{currentfill}%
\pgfsetlinewidth{0.803000pt}%
\definecolor{currentstroke}{rgb}{0.000000,0.000000,0.000000}%
\pgfsetstrokecolor{currentstroke}%
\pgfsetdash{}{0pt}%
\pgfsys@defobject{currentmarker}{\pgfqpoint{-0.048611in}{0.000000in}}{\pgfqpoint{0.000000in}{0.000000in}}{%
\pgfpathmoveto{\pgfqpoint{0.000000in}{0.000000in}}%
\pgfpathlineto{\pgfqpoint{-0.048611in}{0.000000in}}%
\pgfusepath{stroke,fill}%
}%
\begin{pgfscope}%
\pgfsys@transformshift{0.800000in}{3.136941in}%
\pgfsys@useobject{currentmarker}{}%
\end{pgfscope}%
\end{pgfscope}%
\begin{pgfscope}%
\definecolor{textcolor}{rgb}{0.000000,0.000000,0.000000}%
\pgfsetstrokecolor{textcolor}%
\pgfsetfillcolor{textcolor}%
\pgftext[x=0.394076in,y=3.084180in,left,base]{\color{textcolor}\sffamily\fontsize{10.000000}{12.000000}\selectfont 60\%}%
\end{pgfscope}%
\begin{pgfscope}%
\pgfpathrectangle{\pgfqpoint{0.800000in}{0.528000in}}{\pgfqpoint{4.960000in}{3.696000in}}%
\pgfusepath{clip}%
\pgfsetrectcap%
\pgfsetroundjoin%
\pgfsetlinewidth{0.803000pt}%
\definecolor{currentstroke}{rgb}{0.690196,0.690196,0.690196}%
\pgfsetstrokecolor{currentstroke}%
\pgfsetdash{}{0pt}%
\pgfpathmoveto{\pgfqpoint{0.800000in}{3.571765in}}%
\pgfpathlineto{\pgfqpoint{5.760000in}{3.571765in}}%
\pgfusepath{stroke}%
\end{pgfscope}%
\begin{pgfscope}%
\pgfsetbuttcap%
\pgfsetroundjoin%
\definecolor{currentfill}{rgb}{0.000000,0.000000,0.000000}%
\pgfsetfillcolor{currentfill}%
\pgfsetlinewidth{0.803000pt}%
\definecolor{currentstroke}{rgb}{0.000000,0.000000,0.000000}%
\pgfsetstrokecolor{currentstroke}%
\pgfsetdash{}{0pt}%
\pgfsys@defobject{currentmarker}{\pgfqpoint{-0.048611in}{0.000000in}}{\pgfqpoint{0.000000in}{0.000000in}}{%
\pgfpathmoveto{\pgfqpoint{0.000000in}{0.000000in}}%
\pgfpathlineto{\pgfqpoint{-0.048611in}{0.000000in}}%
\pgfusepath{stroke,fill}%
}%
\begin{pgfscope}%
\pgfsys@transformshift{0.800000in}{3.571765in}%
\pgfsys@useobject{currentmarker}{}%
\end{pgfscope}%
\end{pgfscope}%
\begin{pgfscope}%
\definecolor{textcolor}{rgb}{0.000000,0.000000,0.000000}%
\pgfsetstrokecolor{textcolor}%
\pgfsetfillcolor{textcolor}%
\pgftext[x=0.394076in,y=3.519003in,left,base]{\color{textcolor}\sffamily\fontsize{10.000000}{12.000000}\selectfont 70\%}%
\end{pgfscope}%
\begin{pgfscope}%
\pgfpathrectangle{\pgfqpoint{0.800000in}{0.528000in}}{\pgfqpoint{4.960000in}{3.696000in}}%
\pgfusepath{clip}%
\pgfsetrectcap%
\pgfsetroundjoin%
\pgfsetlinewidth{0.803000pt}%
\definecolor{currentstroke}{rgb}{0.690196,0.690196,0.690196}%
\pgfsetstrokecolor{currentstroke}%
\pgfsetdash{}{0pt}%
\pgfpathmoveto{\pgfqpoint{0.800000in}{4.006588in}}%
\pgfpathlineto{\pgfqpoint{5.760000in}{4.006588in}}%
\pgfusepath{stroke}%
\end{pgfscope}%
\begin{pgfscope}%
\pgfsetbuttcap%
\pgfsetroundjoin%
\definecolor{currentfill}{rgb}{0.000000,0.000000,0.000000}%
\pgfsetfillcolor{currentfill}%
\pgfsetlinewidth{0.803000pt}%
\definecolor{currentstroke}{rgb}{0.000000,0.000000,0.000000}%
\pgfsetstrokecolor{currentstroke}%
\pgfsetdash{}{0pt}%
\pgfsys@defobject{currentmarker}{\pgfqpoint{-0.048611in}{0.000000in}}{\pgfqpoint{0.000000in}{0.000000in}}{%
\pgfpathmoveto{\pgfqpoint{0.000000in}{0.000000in}}%
\pgfpathlineto{\pgfqpoint{-0.048611in}{0.000000in}}%
\pgfusepath{stroke,fill}%
}%
\begin{pgfscope}%
\pgfsys@transformshift{0.800000in}{4.006588in}%
\pgfsys@useobject{currentmarker}{}%
\end{pgfscope}%
\end{pgfscope}%
\begin{pgfscope}%
\definecolor{textcolor}{rgb}{0.000000,0.000000,0.000000}%
\pgfsetstrokecolor{textcolor}%
\pgfsetfillcolor{textcolor}%
\pgftext[x=0.394076in,y=3.953827in,left,base]{\color{textcolor}\sffamily\fontsize{10.000000}{12.000000}\selectfont 80\%}%
\end{pgfscope}%
\begin{pgfscope}%
\definecolor{textcolor}{rgb}{0.000000,0.000000,0.000000}%
\pgfsetstrokecolor{textcolor}%
\pgfsetfillcolor{textcolor}%
\pgftext[x=0.338520in,y=2.376000in,,bottom,rotate=90.000000]{\color{textcolor}\sffamily\fontsize{10.000000}{12.000000}\selectfont accuracy validation top-1 [\%]}%
\end{pgfscope}%
\begin{pgfscope}%
\pgfpathrectangle{\pgfqpoint{0.800000in}{0.528000in}}{\pgfqpoint{4.960000in}{3.696000in}}%
\pgfusepath{clip}%
\pgfsetrectcap%
\pgfsetroundjoin%
\pgfsetlinewidth{0.501875pt}%
\definecolor{currentstroke}{rgb}{0.000000,0.000000,1.000000}%
\pgfsetstrokecolor{currentstroke}%
\pgfsetdash{}{0pt}%
\pgfpathmoveto{\pgfqpoint{1.033304in}{0.761500in}}%
\pgfpathlineto{\pgfqpoint{1.258341in}{0.929342in}}%
\pgfpathlineto{\pgfqpoint{1.484296in}{1.133709in}}%
\pgfpathlineto{\pgfqpoint{1.708415in}{1.346773in}}%
\pgfpathlineto{\pgfqpoint{1.932533in}{1.586360in}}%
\pgfpathlineto{\pgfqpoint{2.158489in}{1.793771in}}%
\pgfpathlineto{\pgfqpoint{2.382148in}{1.986398in}}%
\pgfpathlineto{\pgfqpoint{2.604889in}{2.063362in}}%
\pgfpathlineto{\pgfqpoint{2.827630in}{2.143369in}}%
\pgfpathlineto{\pgfqpoint{3.050370in}{2.221638in}}%
\pgfpathlineto{\pgfqpoint{3.273111in}{2.326865in}}%
\pgfpathlineto{\pgfqpoint{3.496311in}{2.396437in}}%
\pgfpathlineto{\pgfqpoint{3.719052in}{2.472096in}}%
\pgfpathlineto{\pgfqpoint{3.941793in}{2.519492in}}%
\pgfpathlineto{\pgfqpoint{4.165452in}{2.554712in}}%
\pgfpathlineto{\pgfqpoint{4.389111in}{2.613848in}}%
\pgfpathlineto{\pgfqpoint{4.612770in}{2.643851in}}%
\pgfpathlineto{\pgfqpoint{4.835970in}{2.690812in}}%
\pgfpathlineto{\pgfqpoint{5.059630in}{2.699944in}}%
\pgfpathlineto{\pgfqpoint{5.283289in}{2.722120in}}%
\pgfpathlineto{\pgfqpoint{5.506948in}{2.757775in}}%
\pgfusepath{stroke}%
\end{pgfscope}%
\begin{pgfscope}%
\pgfpathrectangle{\pgfqpoint{0.800000in}{0.528000in}}{\pgfqpoint{4.960000in}{3.696000in}}%
\pgfusepath{clip}%
\pgfsetbuttcap%
\pgfsetroundjoin%
\definecolor{currentfill}{rgb}{0.000000,0.000000,1.000000}%
\pgfsetfillcolor{currentfill}%
\pgfsetlinewidth{1.003750pt}%
\definecolor{currentstroke}{rgb}{0.000000,0.000000,1.000000}%
\pgfsetstrokecolor{currentstroke}%
\pgfsetdash{}{0pt}%
\pgfsys@defobject{currentmarker}{\pgfqpoint{-0.006944in}{-0.006944in}}{\pgfqpoint{0.006944in}{0.006944in}}{%
\pgfpathmoveto{\pgfqpoint{0.000000in}{-0.006944in}}%
\pgfpathcurveto{\pgfqpoint{0.001842in}{-0.006944in}}{\pgfqpoint{0.003608in}{-0.006213in}}{\pgfqpoint{0.004910in}{-0.004910in}}%
\pgfpathcurveto{\pgfqpoint{0.006213in}{-0.003608in}}{\pgfqpoint{0.006944in}{-0.001842in}}{\pgfqpoint{0.006944in}{0.000000in}}%
\pgfpathcurveto{\pgfqpoint{0.006944in}{0.001842in}}{\pgfqpoint{0.006213in}{0.003608in}}{\pgfqpoint{0.004910in}{0.004910in}}%
\pgfpathcurveto{\pgfqpoint{0.003608in}{0.006213in}}{\pgfqpoint{0.001842in}{0.006944in}}{\pgfqpoint{0.000000in}{0.006944in}}%
\pgfpathcurveto{\pgfqpoint{-0.001842in}{0.006944in}}{\pgfqpoint{-0.003608in}{0.006213in}}{\pgfqpoint{-0.004910in}{0.004910in}}%
\pgfpathcurveto{\pgfqpoint{-0.006213in}{0.003608in}}{\pgfqpoint{-0.006944in}{0.001842in}}{\pgfqpoint{-0.006944in}{0.000000in}}%
\pgfpathcurveto{\pgfqpoint{-0.006944in}{-0.001842in}}{\pgfqpoint{-0.006213in}{-0.003608in}}{\pgfqpoint{-0.004910in}{-0.004910in}}%
\pgfpathcurveto{\pgfqpoint{-0.003608in}{-0.006213in}}{\pgfqpoint{-0.001842in}{-0.006944in}}{\pgfqpoint{0.000000in}{-0.006944in}}%
\pgfpathclose%
\pgfusepath{stroke,fill}%
}%
\begin{pgfscope}%
\pgfsys@transformshift{1.033304in}{0.761500in}%
\pgfsys@useobject{currentmarker}{}%
\end{pgfscope}%
\begin{pgfscope}%
\pgfsys@transformshift{1.258341in}{0.929342in}%
\pgfsys@useobject{currentmarker}{}%
\end{pgfscope}%
\begin{pgfscope}%
\pgfsys@transformshift{1.484296in}{1.133709in}%
\pgfsys@useobject{currentmarker}{}%
\end{pgfscope}%
\begin{pgfscope}%
\pgfsys@transformshift{1.708415in}{1.346773in}%
\pgfsys@useobject{currentmarker}{}%
\end{pgfscope}%
\begin{pgfscope}%
\pgfsys@transformshift{1.932533in}{1.586360in}%
\pgfsys@useobject{currentmarker}{}%
\end{pgfscope}%
\begin{pgfscope}%
\pgfsys@transformshift{2.158489in}{1.793771in}%
\pgfsys@useobject{currentmarker}{}%
\end{pgfscope}%
\begin{pgfscope}%
\pgfsys@transformshift{2.382148in}{1.986398in}%
\pgfsys@useobject{currentmarker}{}%
\end{pgfscope}%
\begin{pgfscope}%
\pgfsys@transformshift{2.604889in}{2.063362in}%
\pgfsys@useobject{currentmarker}{}%
\end{pgfscope}%
\begin{pgfscope}%
\pgfsys@transformshift{2.827630in}{2.143369in}%
\pgfsys@useobject{currentmarker}{}%
\end{pgfscope}%
\begin{pgfscope}%
\pgfsys@transformshift{3.050370in}{2.221638in}%
\pgfsys@useobject{currentmarker}{}%
\end{pgfscope}%
\begin{pgfscope}%
\pgfsys@transformshift{3.273111in}{2.326865in}%
\pgfsys@useobject{currentmarker}{}%
\end{pgfscope}%
\begin{pgfscope}%
\pgfsys@transformshift{3.496311in}{2.396437in}%
\pgfsys@useobject{currentmarker}{}%
\end{pgfscope}%
\begin{pgfscope}%
\pgfsys@transformshift{3.719052in}{2.472096in}%
\pgfsys@useobject{currentmarker}{}%
\end{pgfscope}%
\begin{pgfscope}%
\pgfsys@transformshift{3.941793in}{2.519492in}%
\pgfsys@useobject{currentmarker}{}%
\end{pgfscope}%
\begin{pgfscope}%
\pgfsys@transformshift{4.165452in}{2.554712in}%
\pgfsys@useobject{currentmarker}{}%
\end{pgfscope}%
\begin{pgfscope}%
\pgfsys@transformshift{4.389111in}{2.613848in}%
\pgfsys@useobject{currentmarker}{}%
\end{pgfscope}%
\begin{pgfscope}%
\pgfsys@transformshift{4.612770in}{2.643851in}%
\pgfsys@useobject{currentmarker}{}%
\end{pgfscope}%
\begin{pgfscope}%
\pgfsys@transformshift{4.835970in}{2.690812in}%
\pgfsys@useobject{currentmarker}{}%
\end{pgfscope}%
\begin{pgfscope}%
\pgfsys@transformshift{5.059630in}{2.699944in}%
\pgfsys@useobject{currentmarker}{}%
\end{pgfscope}%
\begin{pgfscope}%
\pgfsys@transformshift{5.283289in}{2.722120in}%
\pgfsys@useobject{currentmarker}{}%
\end{pgfscope}%
\begin{pgfscope}%
\pgfsys@transformshift{5.506948in}{2.757775in}%
\pgfsys@useobject{currentmarker}{}%
\end{pgfscope}%
\end{pgfscope}%
\begin{pgfscope}%
\pgfpathrectangle{\pgfqpoint{0.800000in}{0.528000in}}{\pgfqpoint{4.960000in}{3.696000in}}%
\pgfusepath{clip}%
\pgfsetbuttcap%
\pgfsetroundjoin%
\pgfsetlinewidth{0.250937pt}%
\definecolor{currentstroke}{rgb}{0.000000,0.000000,1.000000}%
\pgfsetstrokecolor{currentstroke}%
\pgfsetdash{{0.925000pt}{0.400000pt}}{0.000000pt}%
\pgfpathmoveto{\pgfqpoint{1.033304in}{1.362861in}}%
\pgfpathlineto{\pgfqpoint{1.258341in}{1.775944in}}%
\pgfpathlineto{\pgfqpoint{1.484296in}{2.167285in}}%
\pgfpathlineto{\pgfqpoint{1.708415in}{2.510795in}}%
\pgfpathlineto{\pgfqpoint{1.932533in}{2.806475in}}%
\pgfpathlineto{\pgfqpoint{2.158489in}{3.102155in}}%
\pgfpathlineto{\pgfqpoint{2.382148in}{3.332612in}}%
\pgfpathlineto{\pgfqpoint{2.604889in}{3.423925in}}%
\pgfpathlineto{\pgfqpoint{2.827630in}{3.510889in}}%
\pgfpathlineto{\pgfqpoint{3.050370in}{3.589158in}}%
\pgfpathlineto{\pgfqpoint{3.273111in}{3.671774in}}%
\pgfpathlineto{\pgfqpoint{3.496311in}{3.745694in}}%
\pgfpathlineto{\pgfqpoint{3.719052in}{3.789176in}}%
\pgfpathlineto{\pgfqpoint{3.941793in}{3.854400in}}%
\pgfpathlineto{\pgfqpoint{4.165452in}{3.876141in}}%
\pgfpathlineto{\pgfqpoint{4.389111in}{3.910927in}}%
\pgfpathlineto{\pgfqpoint{4.612770in}{3.932668in}}%
\pgfpathlineto{\pgfqpoint{4.835970in}{3.958758in}}%
\pgfpathlineto{\pgfqpoint{5.059630in}{3.984847in}}%
\pgfpathlineto{\pgfqpoint{5.283289in}{3.997892in}}%
\pgfpathlineto{\pgfqpoint{5.506948in}{4.041374in}}%
\pgfusepath{stroke}%
\end{pgfscope}%
\begin{pgfscope}%
\pgfpathrectangle{\pgfqpoint{0.800000in}{0.528000in}}{\pgfqpoint{4.960000in}{3.696000in}}%
\pgfusepath{clip}%
\pgfsetbuttcap%
\pgfsetroundjoin%
\definecolor{currentfill}{rgb}{0.000000,0.000000,1.000000}%
\pgfsetfillcolor{currentfill}%
\pgfsetlinewidth{1.003750pt}%
\definecolor{currentstroke}{rgb}{0.000000,0.000000,1.000000}%
\pgfsetstrokecolor{currentstroke}%
\pgfsetdash{}{0pt}%
\pgfsys@defobject{currentmarker}{\pgfqpoint{-0.003472in}{-0.003472in}}{\pgfqpoint{0.003472in}{0.003472in}}{%
\pgfpathmoveto{\pgfqpoint{0.000000in}{-0.003472in}}%
\pgfpathcurveto{\pgfqpoint{0.000921in}{-0.003472in}}{\pgfqpoint{0.001804in}{-0.003106in}}{\pgfqpoint{0.002455in}{-0.002455in}}%
\pgfpathcurveto{\pgfqpoint{0.003106in}{-0.001804in}}{\pgfqpoint{0.003472in}{-0.000921in}}{\pgfqpoint{0.003472in}{0.000000in}}%
\pgfpathcurveto{\pgfqpoint{0.003472in}{0.000921in}}{\pgfqpoint{0.003106in}{0.001804in}}{\pgfqpoint{0.002455in}{0.002455in}}%
\pgfpathcurveto{\pgfqpoint{0.001804in}{0.003106in}}{\pgfqpoint{0.000921in}{0.003472in}}{\pgfqpoint{0.000000in}{0.003472in}}%
\pgfpathcurveto{\pgfqpoint{-0.000921in}{0.003472in}}{\pgfqpoint{-0.001804in}{0.003106in}}{\pgfqpoint{-0.002455in}{0.002455in}}%
\pgfpathcurveto{\pgfqpoint{-0.003106in}{0.001804in}}{\pgfqpoint{-0.003472in}{0.000921in}}{\pgfqpoint{-0.003472in}{0.000000in}}%
\pgfpathcurveto{\pgfqpoint{-0.003472in}{-0.000921in}}{\pgfqpoint{-0.003106in}{-0.001804in}}{\pgfqpoint{-0.002455in}{-0.002455in}}%
\pgfpathcurveto{\pgfqpoint{-0.001804in}{-0.003106in}}{\pgfqpoint{-0.000921in}{-0.003472in}}{\pgfqpoint{0.000000in}{-0.003472in}}%
\pgfpathclose%
\pgfusepath{stroke,fill}%
}%
\begin{pgfscope}%
\pgfsys@transformshift{1.033304in}{1.362861in}%
\pgfsys@useobject{currentmarker}{}%
\end{pgfscope}%
\begin{pgfscope}%
\pgfsys@transformshift{1.258341in}{1.775944in}%
\pgfsys@useobject{currentmarker}{}%
\end{pgfscope}%
\begin{pgfscope}%
\pgfsys@transformshift{1.484296in}{2.167285in}%
\pgfsys@useobject{currentmarker}{}%
\end{pgfscope}%
\begin{pgfscope}%
\pgfsys@transformshift{1.708415in}{2.510795in}%
\pgfsys@useobject{currentmarker}{}%
\end{pgfscope}%
\begin{pgfscope}%
\pgfsys@transformshift{1.932533in}{2.806475in}%
\pgfsys@useobject{currentmarker}{}%
\end{pgfscope}%
\begin{pgfscope}%
\pgfsys@transformshift{2.158489in}{3.102155in}%
\pgfsys@useobject{currentmarker}{}%
\end{pgfscope}%
\begin{pgfscope}%
\pgfsys@transformshift{2.382148in}{3.332612in}%
\pgfsys@useobject{currentmarker}{}%
\end{pgfscope}%
\begin{pgfscope}%
\pgfsys@transformshift{2.604889in}{3.423925in}%
\pgfsys@useobject{currentmarker}{}%
\end{pgfscope}%
\begin{pgfscope}%
\pgfsys@transformshift{2.827630in}{3.510889in}%
\pgfsys@useobject{currentmarker}{}%
\end{pgfscope}%
\begin{pgfscope}%
\pgfsys@transformshift{3.050370in}{3.589158in}%
\pgfsys@useobject{currentmarker}{}%
\end{pgfscope}%
\begin{pgfscope}%
\pgfsys@transformshift{3.273111in}{3.671774in}%
\pgfsys@useobject{currentmarker}{}%
\end{pgfscope}%
\begin{pgfscope}%
\pgfsys@transformshift{3.496311in}{3.745694in}%
\pgfsys@useobject{currentmarker}{}%
\end{pgfscope}%
\begin{pgfscope}%
\pgfsys@transformshift{3.719052in}{3.789176in}%
\pgfsys@useobject{currentmarker}{}%
\end{pgfscope}%
\begin{pgfscope}%
\pgfsys@transformshift{3.941793in}{3.854400in}%
\pgfsys@useobject{currentmarker}{}%
\end{pgfscope}%
\begin{pgfscope}%
\pgfsys@transformshift{4.165452in}{3.876141in}%
\pgfsys@useobject{currentmarker}{}%
\end{pgfscope}%
\begin{pgfscope}%
\pgfsys@transformshift{4.389111in}{3.910927in}%
\pgfsys@useobject{currentmarker}{}%
\end{pgfscope}%
\begin{pgfscope}%
\pgfsys@transformshift{4.612770in}{3.932668in}%
\pgfsys@useobject{currentmarker}{}%
\end{pgfscope}%
\begin{pgfscope}%
\pgfsys@transformshift{4.835970in}{3.958758in}%
\pgfsys@useobject{currentmarker}{}%
\end{pgfscope}%
\begin{pgfscope}%
\pgfsys@transformshift{5.059630in}{3.984847in}%
\pgfsys@useobject{currentmarker}{}%
\end{pgfscope}%
\begin{pgfscope}%
\pgfsys@transformshift{5.283289in}{3.997892in}%
\pgfsys@useobject{currentmarker}{}%
\end{pgfscope}%
\begin{pgfscope}%
\pgfsys@transformshift{5.506948in}{4.041374in}%
\pgfsys@useobject{currentmarker}{}%
\end{pgfscope}%
\end{pgfscope}%
\begin{pgfscope}%
\pgfpathrectangle{\pgfqpoint{0.800000in}{0.528000in}}{\pgfqpoint{4.960000in}{3.696000in}}%
\pgfusepath{clip}%
\pgfsetrectcap%
\pgfsetroundjoin%
\pgfsetlinewidth{0.501875pt}%
\definecolor{currentstroke}{rgb}{1.000000,0.647059,0.000000}%
\pgfsetstrokecolor{currentstroke}%
\pgfsetdash{}{0pt}%
\pgfpathmoveto{\pgfqpoint{1.011259in}{3.775697in}}%
\pgfpathlineto{\pgfqpoint{1.213793in}{3.947887in}}%
\pgfpathlineto{\pgfqpoint{1.415407in}{4.024851in}}%
\pgfpathlineto{\pgfqpoint{1.617481in}{4.043983in}}%
\pgfpathlineto{\pgfqpoint{1.819556in}{4.056158in}}%
\pgfpathlineto{\pgfqpoint{2.021170in}{4.056158in}}%
\pgfpathlineto{\pgfqpoint{2.222785in}{4.043983in}}%
\pgfpathlineto{\pgfqpoint{2.424400in}{4.078334in}}%
\pgfpathlineto{\pgfqpoint{2.626474in}{4.083987in}}%
\pgfpathlineto{\pgfqpoint{2.828089in}{4.094423in}}%
\pgfpathlineto{\pgfqpoint{3.029704in}{4.088770in}}%
\pgfpathlineto{\pgfqpoint{3.231778in}{4.073986in}}%
\pgfpathlineto{\pgfqpoint{3.434770in}{4.085726in}}%
\pgfpathlineto{\pgfqpoint{3.636844in}{4.093118in}}%
\pgfpathlineto{\pgfqpoint{3.838459in}{4.090074in}}%
\pgfpathlineto{\pgfqpoint{4.040533in}{4.091379in}}%
\pgfpathlineto{\pgfqpoint{4.242607in}{4.090074in}}%
\pgfpathlineto{\pgfqpoint{4.444222in}{4.091379in}}%
\pgfpathlineto{\pgfqpoint{4.645837in}{4.088770in}}%
\pgfpathlineto{\pgfqpoint{4.847911in}{4.085726in}}%
\pgfpathlineto{\pgfqpoint{5.050444in}{4.090074in}}%
\pgfusepath{stroke}%
\end{pgfscope}%
\begin{pgfscope}%
\pgfpathrectangle{\pgfqpoint{0.800000in}{0.528000in}}{\pgfqpoint{4.960000in}{3.696000in}}%
\pgfusepath{clip}%
\pgfsetbuttcap%
\pgfsetroundjoin%
\definecolor{currentfill}{rgb}{1.000000,0.647059,0.000000}%
\pgfsetfillcolor{currentfill}%
\pgfsetlinewidth{1.003750pt}%
\definecolor{currentstroke}{rgb}{1.000000,0.647059,0.000000}%
\pgfsetstrokecolor{currentstroke}%
\pgfsetdash{}{0pt}%
\pgfsys@defobject{currentmarker}{\pgfqpoint{-0.006944in}{-0.006944in}}{\pgfqpoint{0.006944in}{0.006944in}}{%
\pgfpathmoveto{\pgfqpoint{0.000000in}{-0.006944in}}%
\pgfpathcurveto{\pgfqpoint{0.001842in}{-0.006944in}}{\pgfqpoint{0.003608in}{-0.006213in}}{\pgfqpoint{0.004910in}{-0.004910in}}%
\pgfpathcurveto{\pgfqpoint{0.006213in}{-0.003608in}}{\pgfqpoint{0.006944in}{-0.001842in}}{\pgfqpoint{0.006944in}{0.000000in}}%
\pgfpathcurveto{\pgfqpoint{0.006944in}{0.001842in}}{\pgfqpoint{0.006213in}{0.003608in}}{\pgfqpoint{0.004910in}{0.004910in}}%
\pgfpathcurveto{\pgfqpoint{0.003608in}{0.006213in}}{\pgfqpoint{0.001842in}{0.006944in}}{\pgfqpoint{0.000000in}{0.006944in}}%
\pgfpathcurveto{\pgfqpoint{-0.001842in}{0.006944in}}{\pgfqpoint{-0.003608in}{0.006213in}}{\pgfqpoint{-0.004910in}{0.004910in}}%
\pgfpathcurveto{\pgfqpoint{-0.006213in}{0.003608in}}{\pgfqpoint{-0.006944in}{0.001842in}}{\pgfqpoint{-0.006944in}{0.000000in}}%
\pgfpathcurveto{\pgfqpoint{-0.006944in}{-0.001842in}}{\pgfqpoint{-0.006213in}{-0.003608in}}{\pgfqpoint{-0.004910in}{-0.004910in}}%
\pgfpathcurveto{\pgfqpoint{-0.003608in}{-0.006213in}}{\pgfqpoint{-0.001842in}{-0.006944in}}{\pgfqpoint{0.000000in}{-0.006944in}}%
\pgfpathclose%
\pgfusepath{stroke,fill}%
}%
\begin{pgfscope}%
\pgfsys@transformshift{1.011259in}{3.775697in}%
\pgfsys@useobject{currentmarker}{}%
\end{pgfscope}%
\begin{pgfscope}%
\pgfsys@transformshift{1.213793in}{3.947887in}%
\pgfsys@useobject{currentmarker}{}%
\end{pgfscope}%
\begin{pgfscope}%
\pgfsys@transformshift{1.415407in}{4.024851in}%
\pgfsys@useobject{currentmarker}{}%
\end{pgfscope}%
\begin{pgfscope}%
\pgfsys@transformshift{1.617481in}{4.043983in}%
\pgfsys@useobject{currentmarker}{}%
\end{pgfscope}%
\begin{pgfscope}%
\pgfsys@transformshift{1.819556in}{4.056158in}%
\pgfsys@useobject{currentmarker}{}%
\end{pgfscope}%
\begin{pgfscope}%
\pgfsys@transformshift{2.021170in}{4.056158in}%
\pgfsys@useobject{currentmarker}{}%
\end{pgfscope}%
\begin{pgfscope}%
\pgfsys@transformshift{2.222785in}{4.043983in}%
\pgfsys@useobject{currentmarker}{}%
\end{pgfscope}%
\begin{pgfscope}%
\pgfsys@transformshift{2.424400in}{4.078334in}%
\pgfsys@useobject{currentmarker}{}%
\end{pgfscope}%
\begin{pgfscope}%
\pgfsys@transformshift{2.626474in}{4.083987in}%
\pgfsys@useobject{currentmarker}{}%
\end{pgfscope}%
\begin{pgfscope}%
\pgfsys@transformshift{2.828089in}{4.094423in}%
\pgfsys@useobject{currentmarker}{}%
\end{pgfscope}%
\begin{pgfscope}%
\pgfsys@transformshift{3.029704in}{4.088770in}%
\pgfsys@useobject{currentmarker}{}%
\end{pgfscope}%
\begin{pgfscope}%
\pgfsys@transformshift{3.231778in}{4.073986in}%
\pgfsys@useobject{currentmarker}{}%
\end{pgfscope}%
\begin{pgfscope}%
\pgfsys@transformshift{3.434770in}{4.085726in}%
\pgfsys@useobject{currentmarker}{}%
\end{pgfscope}%
\begin{pgfscope}%
\pgfsys@transformshift{3.636844in}{4.093118in}%
\pgfsys@useobject{currentmarker}{}%
\end{pgfscope}%
\begin{pgfscope}%
\pgfsys@transformshift{3.838459in}{4.090074in}%
\pgfsys@useobject{currentmarker}{}%
\end{pgfscope}%
\begin{pgfscope}%
\pgfsys@transformshift{4.040533in}{4.091379in}%
\pgfsys@useobject{currentmarker}{}%
\end{pgfscope}%
\begin{pgfscope}%
\pgfsys@transformshift{4.242607in}{4.090074in}%
\pgfsys@useobject{currentmarker}{}%
\end{pgfscope}%
\begin{pgfscope}%
\pgfsys@transformshift{4.444222in}{4.091379in}%
\pgfsys@useobject{currentmarker}{}%
\end{pgfscope}%
\begin{pgfscope}%
\pgfsys@transformshift{4.645837in}{4.088770in}%
\pgfsys@useobject{currentmarker}{}%
\end{pgfscope}%
\begin{pgfscope}%
\pgfsys@transformshift{4.847911in}{4.085726in}%
\pgfsys@useobject{currentmarker}{}%
\end{pgfscope}%
\begin{pgfscope}%
\pgfsys@transformshift{5.050444in}{4.090074in}%
\pgfsys@useobject{currentmarker}{}%
\end{pgfscope}%
\end{pgfscope}%
\begin{pgfscope}%
\pgfpathrectangle{\pgfqpoint{0.800000in}{0.528000in}}{\pgfqpoint{4.960000in}{3.696000in}}%
\pgfusepath{clip}%
\pgfsetbuttcap%
\pgfsetroundjoin%
\pgfsetlinewidth{0.250937pt}%
\definecolor{currentstroke}{rgb}{1.000000,0.647059,0.000000}%
\pgfsetstrokecolor{currentstroke}%
\pgfsetdash{{0.925000pt}{0.400000pt}}{0.000000pt}%
\pgfusepath{stroke}%
\end{pgfscope}%
\begin{pgfscope}%
\pgfpathrectangle{\pgfqpoint{0.800000in}{0.528000in}}{\pgfqpoint{4.960000in}{3.696000in}}%
\pgfusepath{clip}%
\pgfsetbuttcap%
\pgfsetroundjoin%
\definecolor{currentfill}{rgb}{1.000000,0.647059,0.000000}%
\pgfsetfillcolor{currentfill}%
\pgfsetlinewidth{1.003750pt}%
\definecolor{currentstroke}{rgb}{1.000000,0.647059,0.000000}%
\pgfsetstrokecolor{currentstroke}%
\pgfsetdash{}{0pt}%
\pgfsys@defobject{currentmarker}{\pgfqpoint{-0.003472in}{-0.003472in}}{\pgfqpoint{0.003472in}{0.003472in}}{%
\pgfpathmoveto{\pgfqpoint{0.000000in}{-0.003472in}}%
\pgfpathcurveto{\pgfqpoint{0.000921in}{-0.003472in}}{\pgfqpoint{0.001804in}{-0.003106in}}{\pgfqpoint{0.002455in}{-0.002455in}}%
\pgfpathcurveto{\pgfqpoint{0.003106in}{-0.001804in}}{\pgfqpoint{0.003472in}{-0.000921in}}{\pgfqpoint{0.003472in}{0.000000in}}%
\pgfpathcurveto{\pgfqpoint{0.003472in}{0.000921in}}{\pgfqpoint{0.003106in}{0.001804in}}{\pgfqpoint{0.002455in}{0.002455in}}%
\pgfpathcurveto{\pgfqpoint{0.001804in}{0.003106in}}{\pgfqpoint{0.000921in}{0.003472in}}{\pgfqpoint{0.000000in}{0.003472in}}%
\pgfpathcurveto{\pgfqpoint{-0.000921in}{0.003472in}}{\pgfqpoint{-0.001804in}{0.003106in}}{\pgfqpoint{-0.002455in}{0.002455in}}%
\pgfpathcurveto{\pgfqpoint{-0.003106in}{0.001804in}}{\pgfqpoint{-0.003472in}{0.000921in}}{\pgfqpoint{-0.003472in}{0.000000in}}%
\pgfpathcurveto{\pgfqpoint{-0.003472in}{-0.000921in}}{\pgfqpoint{-0.003106in}{-0.001804in}}{\pgfqpoint{-0.002455in}{-0.002455in}}%
\pgfpathcurveto{\pgfqpoint{-0.001804in}{-0.003106in}}{\pgfqpoint{-0.000921in}{-0.003472in}}{\pgfqpoint{0.000000in}{-0.003472in}}%
\pgfpathclose%
\pgfusepath{stroke,fill}%
}%
\begin{pgfscope}%
\pgfsys@transformshift{1.011259in}{4.641431in}%
\pgfsys@useobject{currentmarker}{}%
\end{pgfscope}%
\begin{pgfscope}%
\pgfsys@transformshift{1.213793in}{4.689261in}%
\pgfsys@useobject{currentmarker}{}%
\end{pgfscope}%
\begin{pgfscope}%
\pgfsys@transformshift{1.415407in}{4.715351in}%
\pgfsys@useobject{currentmarker}{}%
\end{pgfscope}%
\begin{pgfscope}%
\pgfsys@transformshift{1.617481in}{4.719699in}%
\pgfsys@useobject{currentmarker}{}%
\end{pgfscope}%
\begin{pgfscope}%
\pgfsys@transformshift{1.819556in}{4.719699in}%
\pgfsys@useobject{currentmarker}{}%
\end{pgfscope}%
\begin{pgfscope}%
\pgfsys@transformshift{2.021170in}{4.719699in}%
\pgfsys@useobject{currentmarker}{}%
\end{pgfscope}%
\begin{pgfscope}%
\pgfsys@transformshift{2.222785in}{4.711002in}%
\pgfsys@useobject{currentmarker}{}%
\end{pgfscope}%
\begin{pgfscope}%
\pgfsys@transformshift{2.424400in}{4.715351in}%
\pgfsys@useobject{currentmarker}{}%
\end{pgfscope}%
\begin{pgfscope}%
\pgfsys@transformshift{2.626474in}{4.715351in}%
\pgfsys@useobject{currentmarker}{}%
\end{pgfscope}%
\begin{pgfscope}%
\pgfsys@transformshift{2.828089in}{4.715351in}%
\pgfsys@useobject{currentmarker}{}%
\end{pgfscope}%
\begin{pgfscope}%
\pgfsys@transformshift{3.029704in}{4.719699in}%
\pgfsys@useobject{currentmarker}{}%
\end{pgfscope}%
\begin{pgfscope}%
\pgfsys@transformshift{3.231778in}{4.719699in}%
\pgfsys@useobject{currentmarker}{}%
\end{pgfscope}%
\begin{pgfscope}%
\pgfsys@transformshift{3.434770in}{4.711002in}%
\pgfsys@useobject{currentmarker}{}%
\end{pgfscope}%
\begin{pgfscope}%
\pgfsys@transformshift{3.636844in}{4.724047in}%
\pgfsys@useobject{currentmarker}{}%
\end{pgfscope}%
\begin{pgfscope}%
\pgfsys@transformshift{3.838459in}{4.724047in}%
\pgfsys@useobject{currentmarker}{}%
\end{pgfscope}%
\begin{pgfscope}%
\pgfsys@transformshift{4.040533in}{4.728395in}%
\pgfsys@useobject{currentmarker}{}%
\end{pgfscope}%
\begin{pgfscope}%
\pgfsys@transformshift{4.242607in}{4.724047in}%
\pgfsys@useobject{currentmarker}{}%
\end{pgfscope}%
\begin{pgfscope}%
\pgfsys@transformshift{4.444222in}{4.719699in}%
\pgfsys@useobject{currentmarker}{}%
\end{pgfscope}%
\begin{pgfscope}%
\pgfsys@transformshift{4.645837in}{4.719699in}%
\pgfsys@useobject{currentmarker}{}%
\end{pgfscope}%
\begin{pgfscope}%
\pgfsys@transformshift{4.847911in}{4.724047in}%
\pgfsys@useobject{currentmarker}{}%
\end{pgfscope}%
\begin{pgfscope}%
\pgfsys@transformshift{5.050444in}{4.724047in}%
\pgfsys@useobject{currentmarker}{}%
\end{pgfscope}%
\end{pgfscope}%
\begin{pgfscope}%
\pgfpathrectangle{\pgfqpoint{0.800000in}{0.528000in}}{\pgfqpoint{4.960000in}{3.696000in}}%
\pgfusepath{clip}%
\pgfsetrectcap%
\pgfsetroundjoin%
\pgfsetlinewidth{0.501875pt}%
\definecolor{currentstroke}{rgb}{0.000000,0.501961,0.000000}%
\pgfsetstrokecolor{currentstroke}%
\pgfsetdash{}{0pt}%
\pgfpathmoveto{\pgfqpoint{1.021363in}{0.677144in}}%
\pgfpathlineto{\pgfqpoint{1.234000in}{0.649316in}}%
\pgfpathlineto{\pgfqpoint{1.445719in}{0.718888in}}%
\pgfpathlineto{\pgfqpoint{1.657437in}{0.742803in}}%
\pgfpathlineto{\pgfqpoint{1.869615in}{0.939778in}}%
\pgfpathlineto{\pgfqpoint{2.081333in}{1.183714in}}%
\pgfpathlineto{\pgfqpoint{2.293511in}{1.314161in}}%
\pgfpathlineto{\pgfqpoint{2.505689in}{1.425041in}}%
\pgfpathlineto{\pgfqpoint{2.717867in}{1.463740in}}%
\pgfpathlineto{\pgfqpoint{2.930044in}{1.632452in}}%
\pgfpathlineto{\pgfqpoint{3.142222in}{1.454609in}}%
\pgfpathlineto{\pgfqpoint{3.354400in}{1.500700in}}%
\pgfpathlineto{\pgfqpoint{3.573007in}{1.681152in}}%
\pgfpathlineto{\pgfqpoint{3.785644in}{1.487221in}}%
\pgfpathlineto{\pgfqpoint{3.997822in}{2.070754in}}%
\pgfpathlineto{\pgfqpoint{4.210459in}{2.229030in}}%
\pgfpathlineto{\pgfqpoint{4.423096in}{1.986398in}}%
\pgfpathlineto{\pgfqpoint{4.635274in}{2.166850in}}%
\pgfpathlineto{\pgfqpoint{4.848370in}{2.060318in}}%
\pgfpathlineto{\pgfqpoint{5.060548in}{2.092930in}}%
\pgfpathlineto{\pgfqpoint{5.273185in}{2.109018in}}%
\pgfusepath{stroke}%
\end{pgfscope}%
\begin{pgfscope}%
\pgfpathrectangle{\pgfqpoint{0.800000in}{0.528000in}}{\pgfqpoint{4.960000in}{3.696000in}}%
\pgfusepath{clip}%
\pgfsetbuttcap%
\pgfsetroundjoin%
\definecolor{currentfill}{rgb}{0.000000,0.501961,0.000000}%
\pgfsetfillcolor{currentfill}%
\pgfsetlinewidth{1.003750pt}%
\definecolor{currentstroke}{rgb}{0.000000,0.501961,0.000000}%
\pgfsetstrokecolor{currentstroke}%
\pgfsetdash{}{0pt}%
\pgfsys@defobject{currentmarker}{\pgfqpoint{-0.006944in}{-0.006944in}}{\pgfqpoint{0.006944in}{0.006944in}}{%
\pgfpathmoveto{\pgfqpoint{0.000000in}{-0.006944in}}%
\pgfpathcurveto{\pgfqpoint{0.001842in}{-0.006944in}}{\pgfqpoint{0.003608in}{-0.006213in}}{\pgfqpoint{0.004910in}{-0.004910in}}%
\pgfpathcurveto{\pgfqpoint{0.006213in}{-0.003608in}}{\pgfqpoint{0.006944in}{-0.001842in}}{\pgfqpoint{0.006944in}{0.000000in}}%
\pgfpathcurveto{\pgfqpoint{0.006944in}{0.001842in}}{\pgfqpoint{0.006213in}{0.003608in}}{\pgfqpoint{0.004910in}{0.004910in}}%
\pgfpathcurveto{\pgfqpoint{0.003608in}{0.006213in}}{\pgfqpoint{0.001842in}{0.006944in}}{\pgfqpoint{0.000000in}{0.006944in}}%
\pgfpathcurveto{\pgfqpoint{-0.001842in}{0.006944in}}{\pgfqpoint{-0.003608in}{0.006213in}}{\pgfqpoint{-0.004910in}{0.004910in}}%
\pgfpathcurveto{\pgfqpoint{-0.006213in}{0.003608in}}{\pgfqpoint{-0.006944in}{0.001842in}}{\pgfqpoint{-0.006944in}{0.000000in}}%
\pgfpathcurveto{\pgfqpoint{-0.006944in}{-0.001842in}}{\pgfqpoint{-0.006213in}{-0.003608in}}{\pgfqpoint{-0.004910in}{-0.004910in}}%
\pgfpathcurveto{\pgfqpoint{-0.003608in}{-0.006213in}}{\pgfqpoint{-0.001842in}{-0.006944in}}{\pgfqpoint{0.000000in}{-0.006944in}}%
\pgfpathclose%
\pgfusepath{stroke,fill}%
}%
\begin{pgfscope}%
\pgfsys@transformshift{1.021363in}{0.677144in}%
\pgfsys@useobject{currentmarker}{}%
\end{pgfscope}%
\begin{pgfscope}%
\pgfsys@transformshift{1.234000in}{0.649316in}%
\pgfsys@useobject{currentmarker}{}%
\end{pgfscope}%
\begin{pgfscope}%
\pgfsys@transformshift{1.445719in}{0.718888in}%
\pgfsys@useobject{currentmarker}{}%
\end{pgfscope}%
\begin{pgfscope}%
\pgfsys@transformshift{1.657437in}{0.742803in}%
\pgfsys@useobject{currentmarker}{}%
\end{pgfscope}%
\begin{pgfscope}%
\pgfsys@transformshift{1.869615in}{0.939778in}%
\pgfsys@useobject{currentmarker}{}%
\end{pgfscope}%
\begin{pgfscope}%
\pgfsys@transformshift{2.081333in}{1.183714in}%
\pgfsys@useobject{currentmarker}{}%
\end{pgfscope}%
\begin{pgfscope}%
\pgfsys@transformshift{2.293511in}{1.314161in}%
\pgfsys@useobject{currentmarker}{}%
\end{pgfscope}%
\begin{pgfscope}%
\pgfsys@transformshift{2.505689in}{1.425041in}%
\pgfsys@useobject{currentmarker}{}%
\end{pgfscope}%
\begin{pgfscope}%
\pgfsys@transformshift{2.717867in}{1.463740in}%
\pgfsys@useobject{currentmarker}{}%
\end{pgfscope}%
\begin{pgfscope}%
\pgfsys@transformshift{2.930044in}{1.632452in}%
\pgfsys@useobject{currentmarker}{}%
\end{pgfscope}%
\begin{pgfscope}%
\pgfsys@transformshift{3.142222in}{1.454609in}%
\pgfsys@useobject{currentmarker}{}%
\end{pgfscope}%
\begin{pgfscope}%
\pgfsys@transformshift{3.354400in}{1.500700in}%
\pgfsys@useobject{currentmarker}{}%
\end{pgfscope}%
\begin{pgfscope}%
\pgfsys@transformshift{3.573007in}{1.681152in}%
\pgfsys@useobject{currentmarker}{}%
\end{pgfscope}%
\begin{pgfscope}%
\pgfsys@transformshift{3.785644in}{1.487221in}%
\pgfsys@useobject{currentmarker}{}%
\end{pgfscope}%
\begin{pgfscope}%
\pgfsys@transformshift{3.997822in}{2.070754in}%
\pgfsys@useobject{currentmarker}{}%
\end{pgfscope}%
\begin{pgfscope}%
\pgfsys@transformshift{4.210459in}{2.229030in}%
\pgfsys@useobject{currentmarker}{}%
\end{pgfscope}%
\begin{pgfscope}%
\pgfsys@transformshift{4.423096in}{1.986398in}%
\pgfsys@useobject{currentmarker}{}%
\end{pgfscope}%
\begin{pgfscope}%
\pgfsys@transformshift{4.635274in}{2.166850in}%
\pgfsys@useobject{currentmarker}{}%
\end{pgfscope}%
\begin{pgfscope}%
\pgfsys@transformshift{4.848370in}{2.060318in}%
\pgfsys@useobject{currentmarker}{}%
\end{pgfscope}%
\begin{pgfscope}%
\pgfsys@transformshift{5.060548in}{2.092930in}%
\pgfsys@useobject{currentmarker}{}%
\end{pgfscope}%
\begin{pgfscope}%
\pgfsys@transformshift{5.273185in}{2.109018in}%
\pgfsys@useobject{currentmarker}{}%
\end{pgfscope}%
\end{pgfscope}%
\begin{pgfscope}%
\pgfpathrectangle{\pgfqpoint{0.800000in}{0.528000in}}{\pgfqpoint{4.960000in}{3.696000in}}%
\pgfusepath{clip}%
\pgfsetbuttcap%
\pgfsetroundjoin%
\pgfsetlinewidth{0.250937pt}%
\definecolor{currentstroke}{rgb}{0.000000,0.501961,0.000000}%
\pgfsetstrokecolor{currentstroke}%
\pgfsetdash{{0.925000pt}{0.400000pt}}{0.000000pt}%
\pgfpathmoveto{\pgfqpoint{1.021363in}{1.167191in}}%
\pgfpathlineto{\pgfqpoint{1.234000in}{1.062833in}}%
\pgfpathlineto{\pgfqpoint{1.445719in}{1.397647in}}%
\pgfpathlineto{\pgfqpoint{1.657437in}{1.401995in}}%
\pgfpathlineto{\pgfqpoint{1.869615in}{1.975962in}}%
\pgfpathlineto{\pgfqpoint{2.081333in}{2.767341in}}%
\pgfpathlineto{\pgfqpoint{2.293511in}{3.028235in}}%
\pgfpathlineto{\pgfqpoint{2.505689in}{3.028235in}}%
\pgfpathlineto{\pgfqpoint{2.717867in}{3.145638in}}%
\pgfpathlineto{\pgfqpoint{2.930044in}{3.410880in}}%
\pgfpathlineto{\pgfqpoint{3.142222in}{2.989101in}}%
\pgfpathlineto{\pgfqpoint{3.354400in}{3.045628in}}%
\pgfpathlineto{\pgfqpoint{3.573007in}{3.310871in}}%
\pgfpathlineto{\pgfqpoint{3.785644in}{2.867351in}}%
\pgfpathlineto{\pgfqpoint{3.997822in}{3.736998in}}%
\pgfpathlineto{\pgfqpoint{4.210459in}{3.793525in}}%
\pgfpathlineto{\pgfqpoint{4.423096in}{3.602202in}}%
\pgfpathlineto{\pgfqpoint{4.635274in}{3.771784in}}%
\pgfpathlineto{\pgfqpoint{4.848370in}{3.658729in}}%
\pgfpathlineto{\pgfqpoint{5.060548in}{3.671774in}}%
\pgfpathlineto{\pgfqpoint{5.273185in}{3.676122in}}%
\pgfusepath{stroke}%
\end{pgfscope}%
\begin{pgfscope}%
\pgfpathrectangle{\pgfqpoint{0.800000in}{0.528000in}}{\pgfqpoint{4.960000in}{3.696000in}}%
\pgfusepath{clip}%
\pgfsetbuttcap%
\pgfsetroundjoin%
\definecolor{currentfill}{rgb}{0.000000,0.501961,0.000000}%
\pgfsetfillcolor{currentfill}%
\pgfsetlinewidth{1.003750pt}%
\definecolor{currentstroke}{rgb}{0.000000,0.501961,0.000000}%
\pgfsetstrokecolor{currentstroke}%
\pgfsetdash{}{0pt}%
\pgfsys@defobject{currentmarker}{\pgfqpoint{-0.003472in}{-0.003472in}}{\pgfqpoint{0.003472in}{0.003472in}}{%
\pgfpathmoveto{\pgfqpoint{0.000000in}{-0.003472in}}%
\pgfpathcurveto{\pgfqpoint{0.000921in}{-0.003472in}}{\pgfqpoint{0.001804in}{-0.003106in}}{\pgfqpoint{0.002455in}{-0.002455in}}%
\pgfpathcurveto{\pgfqpoint{0.003106in}{-0.001804in}}{\pgfqpoint{0.003472in}{-0.000921in}}{\pgfqpoint{0.003472in}{0.000000in}}%
\pgfpathcurveto{\pgfqpoint{0.003472in}{0.000921in}}{\pgfqpoint{0.003106in}{0.001804in}}{\pgfqpoint{0.002455in}{0.002455in}}%
\pgfpathcurveto{\pgfqpoint{0.001804in}{0.003106in}}{\pgfqpoint{0.000921in}{0.003472in}}{\pgfqpoint{0.000000in}{0.003472in}}%
\pgfpathcurveto{\pgfqpoint{-0.000921in}{0.003472in}}{\pgfqpoint{-0.001804in}{0.003106in}}{\pgfqpoint{-0.002455in}{0.002455in}}%
\pgfpathcurveto{\pgfqpoint{-0.003106in}{0.001804in}}{\pgfqpoint{-0.003472in}{0.000921in}}{\pgfqpoint{-0.003472in}{0.000000in}}%
\pgfpathcurveto{\pgfqpoint{-0.003472in}{-0.000921in}}{\pgfqpoint{-0.003106in}{-0.001804in}}{\pgfqpoint{-0.002455in}{-0.002455in}}%
\pgfpathcurveto{\pgfqpoint{-0.001804in}{-0.003106in}}{\pgfqpoint{-0.000921in}{-0.003472in}}{\pgfqpoint{0.000000in}{-0.003472in}}%
\pgfpathclose%
\pgfusepath{stroke,fill}%
}%
\begin{pgfscope}%
\pgfsys@transformshift{1.021363in}{1.167191in}%
\pgfsys@useobject{currentmarker}{}%
\end{pgfscope}%
\begin{pgfscope}%
\pgfsys@transformshift{1.234000in}{1.062833in}%
\pgfsys@useobject{currentmarker}{}%
\end{pgfscope}%
\begin{pgfscope}%
\pgfsys@transformshift{1.445719in}{1.397647in}%
\pgfsys@useobject{currentmarker}{}%
\end{pgfscope}%
\begin{pgfscope}%
\pgfsys@transformshift{1.657437in}{1.401995in}%
\pgfsys@useobject{currentmarker}{}%
\end{pgfscope}%
\begin{pgfscope}%
\pgfsys@transformshift{1.869615in}{1.975962in}%
\pgfsys@useobject{currentmarker}{}%
\end{pgfscope}%
\begin{pgfscope}%
\pgfsys@transformshift{2.081333in}{2.767341in}%
\pgfsys@useobject{currentmarker}{}%
\end{pgfscope}%
\begin{pgfscope}%
\pgfsys@transformshift{2.293511in}{3.028235in}%
\pgfsys@useobject{currentmarker}{}%
\end{pgfscope}%
\begin{pgfscope}%
\pgfsys@transformshift{2.505689in}{3.028235in}%
\pgfsys@useobject{currentmarker}{}%
\end{pgfscope}%
\begin{pgfscope}%
\pgfsys@transformshift{2.717867in}{3.145638in}%
\pgfsys@useobject{currentmarker}{}%
\end{pgfscope}%
\begin{pgfscope}%
\pgfsys@transformshift{2.930044in}{3.410880in}%
\pgfsys@useobject{currentmarker}{}%
\end{pgfscope}%
\begin{pgfscope}%
\pgfsys@transformshift{3.142222in}{2.989101in}%
\pgfsys@useobject{currentmarker}{}%
\end{pgfscope}%
\begin{pgfscope}%
\pgfsys@transformshift{3.354400in}{3.045628in}%
\pgfsys@useobject{currentmarker}{}%
\end{pgfscope}%
\begin{pgfscope}%
\pgfsys@transformshift{3.573007in}{3.310871in}%
\pgfsys@useobject{currentmarker}{}%
\end{pgfscope}%
\begin{pgfscope}%
\pgfsys@transformshift{3.785644in}{2.867351in}%
\pgfsys@useobject{currentmarker}{}%
\end{pgfscope}%
\begin{pgfscope}%
\pgfsys@transformshift{3.997822in}{3.736998in}%
\pgfsys@useobject{currentmarker}{}%
\end{pgfscope}%
\begin{pgfscope}%
\pgfsys@transformshift{4.210459in}{3.793525in}%
\pgfsys@useobject{currentmarker}{}%
\end{pgfscope}%
\begin{pgfscope}%
\pgfsys@transformshift{4.423096in}{3.602202in}%
\pgfsys@useobject{currentmarker}{}%
\end{pgfscope}%
\begin{pgfscope}%
\pgfsys@transformshift{4.635274in}{3.771784in}%
\pgfsys@useobject{currentmarker}{}%
\end{pgfscope}%
\begin{pgfscope}%
\pgfsys@transformshift{4.848370in}{3.658729in}%
\pgfsys@useobject{currentmarker}{}%
\end{pgfscope}%
\begin{pgfscope}%
\pgfsys@transformshift{5.060548in}{3.671774in}%
\pgfsys@useobject{currentmarker}{}%
\end{pgfscope}%
\begin{pgfscope}%
\pgfsys@transformshift{5.273185in}{3.676122in}%
\pgfsys@useobject{currentmarker}{}%
\end{pgfscope}%
\end{pgfscope}%
\begin{pgfscope}%
\pgfpathrectangle{\pgfqpoint{0.800000in}{0.528000in}}{\pgfqpoint{4.960000in}{3.696000in}}%
\pgfusepath{clip}%
\pgfsetrectcap%
\pgfsetroundjoin%
\pgfsetlinewidth{0.501875pt}%
\definecolor{currentstroke}{rgb}{1.000000,0.000000,0.000000}%
\pgfsetstrokecolor{currentstroke}%
\pgfsetdash{}{0pt}%
\pgfpathmoveto{\pgfqpoint{1.013556in}{0.662360in}}%
\pgfpathlineto{\pgfqpoint{1.219763in}{0.658447in}}%
\pgfpathlineto{\pgfqpoint{1.425052in}{0.724975in}}%
\pgfpathlineto{\pgfqpoint{1.630341in}{0.788459in}}%
\pgfpathlineto{\pgfqpoint{1.835630in}{0.775415in}}%
\pgfpathlineto{\pgfqpoint{2.040919in}{0.887599in}}%
\pgfpathlineto{\pgfqpoint{2.246207in}{0.867162in}}%
\pgfpathlineto{\pgfqpoint{2.451496in}{1.081530in}}%
\pgfpathlineto{\pgfqpoint{2.656785in}{1.047614in}}%
\pgfpathlineto{\pgfqpoint{2.862533in}{1.056745in}}%
\pgfpathlineto{\pgfqpoint{3.068281in}{1.163277in}}%
\pgfpathlineto{\pgfqpoint{3.274030in}{1.165886in}}%
\pgfpathlineto{\pgfqpoint{3.479319in}{1.185453in}}%
\pgfpathlineto{\pgfqpoint{3.684607in}{1.148493in}}%
\pgfpathlineto{\pgfqpoint{3.890356in}{1.306769in}}%
\pgfpathlineto{\pgfqpoint{4.096563in}{1.346773in}}%
\pgfpathlineto{\pgfqpoint{4.302770in}{1.377645in}}%
\pgfpathlineto{\pgfqpoint{4.508519in}{1.362861in}}%
\pgfpathlineto{\pgfqpoint{4.714267in}{1.360252in}}%
\pgfpathlineto{\pgfqpoint{4.920015in}{1.274157in}}%
\pgfpathlineto{\pgfqpoint{5.125763in}{1.367644in}}%
\pgfusepath{stroke}%
\end{pgfscope}%
\begin{pgfscope}%
\pgfpathrectangle{\pgfqpoint{0.800000in}{0.528000in}}{\pgfqpoint{4.960000in}{3.696000in}}%
\pgfusepath{clip}%
\pgfsetbuttcap%
\pgfsetroundjoin%
\definecolor{currentfill}{rgb}{1.000000,0.000000,0.000000}%
\pgfsetfillcolor{currentfill}%
\pgfsetlinewidth{1.003750pt}%
\definecolor{currentstroke}{rgb}{1.000000,0.000000,0.000000}%
\pgfsetstrokecolor{currentstroke}%
\pgfsetdash{}{0pt}%
\pgfsys@defobject{currentmarker}{\pgfqpoint{-0.006944in}{-0.006944in}}{\pgfqpoint{0.006944in}{0.006944in}}{%
\pgfpathmoveto{\pgfqpoint{0.000000in}{-0.006944in}}%
\pgfpathcurveto{\pgfqpoint{0.001842in}{-0.006944in}}{\pgfqpoint{0.003608in}{-0.006213in}}{\pgfqpoint{0.004910in}{-0.004910in}}%
\pgfpathcurveto{\pgfqpoint{0.006213in}{-0.003608in}}{\pgfqpoint{0.006944in}{-0.001842in}}{\pgfqpoint{0.006944in}{0.000000in}}%
\pgfpathcurveto{\pgfqpoint{0.006944in}{0.001842in}}{\pgfqpoint{0.006213in}{0.003608in}}{\pgfqpoint{0.004910in}{0.004910in}}%
\pgfpathcurveto{\pgfqpoint{0.003608in}{0.006213in}}{\pgfqpoint{0.001842in}{0.006944in}}{\pgfqpoint{0.000000in}{0.006944in}}%
\pgfpathcurveto{\pgfqpoint{-0.001842in}{0.006944in}}{\pgfqpoint{-0.003608in}{0.006213in}}{\pgfqpoint{-0.004910in}{0.004910in}}%
\pgfpathcurveto{\pgfqpoint{-0.006213in}{0.003608in}}{\pgfqpoint{-0.006944in}{0.001842in}}{\pgfqpoint{-0.006944in}{0.000000in}}%
\pgfpathcurveto{\pgfqpoint{-0.006944in}{-0.001842in}}{\pgfqpoint{-0.006213in}{-0.003608in}}{\pgfqpoint{-0.004910in}{-0.004910in}}%
\pgfpathcurveto{\pgfqpoint{-0.003608in}{-0.006213in}}{\pgfqpoint{-0.001842in}{-0.006944in}}{\pgfqpoint{0.000000in}{-0.006944in}}%
\pgfpathclose%
\pgfusepath{stroke,fill}%
}%
\begin{pgfscope}%
\pgfsys@transformshift{1.013556in}{0.662360in}%
\pgfsys@useobject{currentmarker}{}%
\end{pgfscope}%
\begin{pgfscope}%
\pgfsys@transformshift{1.219763in}{0.658447in}%
\pgfsys@useobject{currentmarker}{}%
\end{pgfscope}%
\begin{pgfscope}%
\pgfsys@transformshift{1.425052in}{0.724975in}%
\pgfsys@useobject{currentmarker}{}%
\end{pgfscope}%
\begin{pgfscope}%
\pgfsys@transformshift{1.630341in}{0.788459in}%
\pgfsys@useobject{currentmarker}{}%
\end{pgfscope}%
\begin{pgfscope}%
\pgfsys@transformshift{1.835630in}{0.775415in}%
\pgfsys@useobject{currentmarker}{}%
\end{pgfscope}%
\begin{pgfscope}%
\pgfsys@transformshift{2.040919in}{0.887599in}%
\pgfsys@useobject{currentmarker}{}%
\end{pgfscope}%
\begin{pgfscope}%
\pgfsys@transformshift{2.246207in}{0.867162in}%
\pgfsys@useobject{currentmarker}{}%
\end{pgfscope}%
\begin{pgfscope}%
\pgfsys@transformshift{2.451496in}{1.081530in}%
\pgfsys@useobject{currentmarker}{}%
\end{pgfscope}%
\begin{pgfscope}%
\pgfsys@transformshift{2.656785in}{1.047614in}%
\pgfsys@useobject{currentmarker}{}%
\end{pgfscope}%
\begin{pgfscope}%
\pgfsys@transformshift{2.862533in}{1.056745in}%
\pgfsys@useobject{currentmarker}{}%
\end{pgfscope}%
\begin{pgfscope}%
\pgfsys@transformshift{3.068281in}{1.163277in}%
\pgfsys@useobject{currentmarker}{}%
\end{pgfscope}%
\begin{pgfscope}%
\pgfsys@transformshift{3.274030in}{1.165886in}%
\pgfsys@useobject{currentmarker}{}%
\end{pgfscope}%
\begin{pgfscope}%
\pgfsys@transformshift{3.479319in}{1.185453in}%
\pgfsys@useobject{currentmarker}{}%
\end{pgfscope}%
\begin{pgfscope}%
\pgfsys@transformshift{3.684607in}{1.148493in}%
\pgfsys@useobject{currentmarker}{}%
\end{pgfscope}%
\begin{pgfscope}%
\pgfsys@transformshift{3.890356in}{1.306769in}%
\pgfsys@useobject{currentmarker}{}%
\end{pgfscope}%
\begin{pgfscope}%
\pgfsys@transformshift{4.096563in}{1.346773in}%
\pgfsys@useobject{currentmarker}{}%
\end{pgfscope}%
\begin{pgfscope}%
\pgfsys@transformshift{4.302770in}{1.377645in}%
\pgfsys@useobject{currentmarker}{}%
\end{pgfscope}%
\begin{pgfscope}%
\pgfsys@transformshift{4.508519in}{1.362861in}%
\pgfsys@useobject{currentmarker}{}%
\end{pgfscope}%
\begin{pgfscope}%
\pgfsys@transformshift{4.714267in}{1.360252in}%
\pgfsys@useobject{currentmarker}{}%
\end{pgfscope}%
\begin{pgfscope}%
\pgfsys@transformshift{4.920015in}{1.274157in}%
\pgfsys@useobject{currentmarker}{}%
\end{pgfscope}%
\begin{pgfscope}%
\pgfsys@transformshift{5.125763in}{1.367644in}%
\pgfsys@useobject{currentmarker}{}%
\end{pgfscope}%
\end{pgfscope}%
\begin{pgfscope}%
\pgfpathrectangle{\pgfqpoint{0.800000in}{0.528000in}}{\pgfqpoint{4.960000in}{3.696000in}}%
\pgfusepath{clip}%
\pgfsetbuttcap%
\pgfsetroundjoin%
\pgfsetlinewidth{0.250937pt}%
\definecolor{currentstroke}{rgb}{1.000000,0.000000,0.000000}%
\pgfsetstrokecolor{currentstroke}%
\pgfsetdash{{0.925000pt}{0.400000pt}}{0.000000pt}%
\pgfpathmoveto{\pgfqpoint{1.013556in}{1.154146in}}%
\pgfpathlineto{\pgfqpoint{1.219763in}{1.193280in}}%
\pgfpathlineto{\pgfqpoint{1.425052in}{1.410692in}}%
\pgfpathlineto{\pgfqpoint{1.630341in}{1.575925in}}%
\pgfpathlineto{\pgfqpoint{1.835630in}{1.549835in}}%
\pgfpathlineto{\pgfqpoint{2.040919in}{1.902042in}}%
\pgfpathlineto{\pgfqpoint{2.246207in}{1.923784in}}%
\pgfpathlineto{\pgfqpoint{2.451496in}{2.410786in}}%
\pgfpathlineto{\pgfqpoint{2.656785in}{2.332518in}}%
\pgfpathlineto{\pgfqpoint{2.862533in}{2.254249in}}%
\pgfpathlineto{\pgfqpoint{3.068281in}{2.549929in}}%
\pgfpathlineto{\pgfqpoint{3.274030in}{2.558626in}}%
\pgfpathlineto{\pgfqpoint{3.479319in}{2.584715in}}%
\pgfpathlineto{\pgfqpoint{3.684607in}{2.536885in}}%
\pgfpathlineto{\pgfqpoint{3.890356in}{2.815172in}}%
\pgfpathlineto{\pgfqpoint{4.096563in}{2.923878in}}%
\pgfpathlineto{\pgfqpoint{4.302770in}{2.958664in}}%
\pgfpathlineto{\pgfqpoint{4.508519in}{2.832565in}}%
\pgfpathlineto{\pgfqpoint{4.714267in}{2.958664in}}%
\pgfpathlineto{\pgfqpoint{4.920015in}{2.836913in}}%
\pgfpathlineto{\pgfqpoint{5.125763in}{2.902136in}}%
\pgfusepath{stroke}%
\end{pgfscope}%
\begin{pgfscope}%
\pgfpathrectangle{\pgfqpoint{0.800000in}{0.528000in}}{\pgfqpoint{4.960000in}{3.696000in}}%
\pgfusepath{clip}%
\pgfsetbuttcap%
\pgfsetroundjoin%
\definecolor{currentfill}{rgb}{1.000000,0.000000,0.000000}%
\pgfsetfillcolor{currentfill}%
\pgfsetlinewidth{1.003750pt}%
\definecolor{currentstroke}{rgb}{1.000000,0.000000,0.000000}%
\pgfsetstrokecolor{currentstroke}%
\pgfsetdash{}{0pt}%
\pgfsys@defobject{currentmarker}{\pgfqpoint{-0.003472in}{-0.003472in}}{\pgfqpoint{0.003472in}{0.003472in}}{%
\pgfpathmoveto{\pgfqpoint{0.000000in}{-0.003472in}}%
\pgfpathcurveto{\pgfqpoint{0.000921in}{-0.003472in}}{\pgfqpoint{0.001804in}{-0.003106in}}{\pgfqpoint{0.002455in}{-0.002455in}}%
\pgfpathcurveto{\pgfqpoint{0.003106in}{-0.001804in}}{\pgfqpoint{0.003472in}{-0.000921in}}{\pgfqpoint{0.003472in}{0.000000in}}%
\pgfpathcurveto{\pgfqpoint{0.003472in}{0.000921in}}{\pgfqpoint{0.003106in}{0.001804in}}{\pgfqpoint{0.002455in}{0.002455in}}%
\pgfpathcurveto{\pgfqpoint{0.001804in}{0.003106in}}{\pgfqpoint{0.000921in}{0.003472in}}{\pgfqpoint{0.000000in}{0.003472in}}%
\pgfpathcurveto{\pgfqpoint{-0.000921in}{0.003472in}}{\pgfqpoint{-0.001804in}{0.003106in}}{\pgfqpoint{-0.002455in}{0.002455in}}%
\pgfpathcurveto{\pgfqpoint{-0.003106in}{0.001804in}}{\pgfqpoint{-0.003472in}{0.000921in}}{\pgfqpoint{-0.003472in}{0.000000in}}%
\pgfpathcurveto{\pgfqpoint{-0.003472in}{-0.000921in}}{\pgfqpoint{-0.003106in}{-0.001804in}}{\pgfqpoint{-0.002455in}{-0.002455in}}%
\pgfpathcurveto{\pgfqpoint{-0.001804in}{-0.003106in}}{\pgfqpoint{-0.000921in}{-0.003472in}}{\pgfqpoint{0.000000in}{-0.003472in}}%
\pgfpathclose%
\pgfusepath{stroke,fill}%
}%
\begin{pgfscope}%
\pgfsys@transformshift{1.013556in}{1.154146in}%
\pgfsys@useobject{currentmarker}{}%
\end{pgfscope}%
\begin{pgfscope}%
\pgfsys@transformshift{1.219763in}{1.193280in}%
\pgfsys@useobject{currentmarker}{}%
\end{pgfscope}%
\begin{pgfscope}%
\pgfsys@transformshift{1.425052in}{1.410692in}%
\pgfsys@useobject{currentmarker}{}%
\end{pgfscope}%
\begin{pgfscope}%
\pgfsys@transformshift{1.630341in}{1.575925in}%
\pgfsys@useobject{currentmarker}{}%
\end{pgfscope}%
\begin{pgfscope}%
\pgfsys@transformshift{1.835630in}{1.549835in}%
\pgfsys@useobject{currentmarker}{}%
\end{pgfscope}%
\begin{pgfscope}%
\pgfsys@transformshift{2.040919in}{1.902042in}%
\pgfsys@useobject{currentmarker}{}%
\end{pgfscope}%
\begin{pgfscope}%
\pgfsys@transformshift{2.246207in}{1.923784in}%
\pgfsys@useobject{currentmarker}{}%
\end{pgfscope}%
\begin{pgfscope}%
\pgfsys@transformshift{2.451496in}{2.410786in}%
\pgfsys@useobject{currentmarker}{}%
\end{pgfscope}%
\begin{pgfscope}%
\pgfsys@transformshift{2.656785in}{2.332518in}%
\pgfsys@useobject{currentmarker}{}%
\end{pgfscope}%
\begin{pgfscope}%
\pgfsys@transformshift{2.862533in}{2.254249in}%
\pgfsys@useobject{currentmarker}{}%
\end{pgfscope}%
\begin{pgfscope}%
\pgfsys@transformshift{3.068281in}{2.549929in}%
\pgfsys@useobject{currentmarker}{}%
\end{pgfscope}%
\begin{pgfscope}%
\pgfsys@transformshift{3.274030in}{2.558626in}%
\pgfsys@useobject{currentmarker}{}%
\end{pgfscope}%
\begin{pgfscope}%
\pgfsys@transformshift{3.479319in}{2.584715in}%
\pgfsys@useobject{currentmarker}{}%
\end{pgfscope}%
\begin{pgfscope}%
\pgfsys@transformshift{3.684607in}{2.536885in}%
\pgfsys@useobject{currentmarker}{}%
\end{pgfscope}%
\begin{pgfscope}%
\pgfsys@transformshift{3.890356in}{2.815172in}%
\pgfsys@useobject{currentmarker}{}%
\end{pgfscope}%
\begin{pgfscope}%
\pgfsys@transformshift{4.096563in}{2.923878in}%
\pgfsys@useobject{currentmarker}{}%
\end{pgfscope}%
\begin{pgfscope}%
\pgfsys@transformshift{4.302770in}{2.958664in}%
\pgfsys@useobject{currentmarker}{}%
\end{pgfscope}%
\begin{pgfscope}%
\pgfsys@transformshift{4.508519in}{2.832565in}%
\pgfsys@useobject{currentmarker}{}%
\end{pgfscope}%
\begin{pgfscope}%
\pgfsys@transformshift{4.714267in}{2.958664in}%
\pgfsys@useobject{currentmarker}{}%
\end{pgfscope}%
\begin{pgfscope}%
\pgfsys@transformshift{4.920015in}{2.836913in}%
\pgfsys@useobject{currentmarker}{}%
\end{pgfscope}%
\begin{pgfscope}%
\pgfsys@transformshift{5.125763in}{2.902136in}%
\pgfsys@useobject{currentmarker}{}%
\end{pgfscope}%
\end{pgfscope}%
\begin{pgfscope}%
\pgfpathrectangle{\pgfqpoint{0.800000in}{0.528000in}}{\pgfqpoint{4.960000in}{3.696000in}}%
\pgfusepath{clip}%
\pgfsetrectcap%
\pgfsetroundjoin%
\pgfsetlinewidth{0.501875pt}%
\definecolor{currentstroke}{rgb}{0.501961,0.000000,0.501961}%
\pgfsetstrokecolor{currentstroke}%
\pgfsetdash{}{0pt}%
\pgfpathmoveto{\pgfqpoint{1.014933in}{0.662360in}}%
\pgfpathlineto{\pgfqpoint{1.222059in}{0.671492in}}%
\pgfpathlineto{\pgfqpoint{1.428726in}{0.671492in}}%
\pgfpathlineto{\pgfqpoint{1.634933in}{0.671492in}}%
\pgfpathlineto{\pgfqpoint{1.842519in}{0.634532in}}%
\pgfpathlineto{\pgfqpoint{2.049644in}{0.671492in}}%
\pgfpathlineto{\pgfqpoint{2.256770in}{0.671492in}}%
\pgfpathlineto{\pgfqpoint{2.464356in}{0.661056in}}%
\pgfpathlineto{\pgfqpoint{2.676993in}{0.671492in}}%
\pgfpathlineto{\pgfqpoint{2.886874in}{0.661056in}}%
\pgfpathlineto{\pgfqpoint{3.094000in}{0.671492in}}%
\pgfpathlineto{\pgfqpoint{3.301126in}{0.671492in}}%
\pgfpathlineto{\pgfqpoint{3.508252in}{0.670187in}}%
\pgfpathlineto{\pgfqpoint{3.715378in}{0.671492in}}%
\pgfpathlineto{\pgfqpoint{3.922044in}{0.671492in}}%
\pgfpathlineto{\pgfqpoint{4.128711in}{0.671492in}}%
\pgfpathlineto{\pgfqpoint{4.334919in}{0.671492in}}%
\pgfpathlineto{\pgfqpoint{4.541126in}{0.671492in}}%
\pgfpathlineto{\pgfqpoint{4.747793in}{0.671492in}}%
\pgfpathlineto{\pgfqpoint{4.954000in}{0.671492in}}%
\pgfpathlineto{\pgfqpoint{5.160207in}{0.671492in}}%
\pgfusepath{stroke}%
\end{pgfscope}%
\begin{pgfscope}%
\pgfpathrectangle{\pgfqpoint{0.800000in}{0.528000in}}{\pgfqpoint{4.960000in}{3.696000in}}%
\pgfusepath{clip}%
\pgfsetbuttcap%
\pgfsetroundjoin%
\definecolor{currentfill}{rgb}{0.501961,0.000000,0.501961}%
\pgfsetfillcolor{currentfill}%
\pgfsetlinewidth{1.003750pt}%
\definecolor{currentstroke}{rgb}{0.501961,0.000000,0.501961}%
\pgfsetstrokecolor{currentstroke}%
\pgfsetdash{}{0pt}%
\pgfsys@defobject{currentmarker}{\pgfqpoint{-0.006944in}{-0.006944in}}{\pgfqpoint{0.006944in}{0.006944in}}{%
\pgfpathmoveto{\pgfqpoint{0.000000in}{-0.006944in}}%
\pgfpathcurveto{\pgfqpoint{0.001842in}{-0.006944in}}{\pgfqpoint{0.003608in}{-0.006213in}}{\pgfqpoint{0.004910in}{-0.004910in}}%
\pgfpathcurveto{\pgfqpoint{0.006213in}{-0.003608in}}{\pgfqpoint{0.006944in}{-0.001842in}}{\pgfqpoint{0.006944in}{0.000000in}}%
\pgfpathcurveto{\pgfqpoint{0.006944in}{0.001842in}}{\pgfqpoint{0.006213in}{0.003608in}}{\pgfqpoint{0.004910in}{0.004910in}}%
\pgfpathcurveto{\pgfqpoint{0.003608in}{0.006213in}}{\pgfqpoint{0.001842in}{0.006944in}}{\pgfqpoint{0.000000in}{0.006944in}}%
\pgfpathcurveto{\pgfqpoint{-0.001842in}{0.006944in}}{\pgfqpoint{-0.003608in}{0.006213in}}{\pgfqpoint{-0.004910in}{0.004910in}}%
\pgfpathcurveto{\pgfqpoint{-0.006213in}{0.003608in}}{\pgfqpoint{-0.006944in}{0.001842in}}{\pgfqpoint{-0.006944in}{0.000000in}}%
\pgfpathcurveto{\pgfqpoint{-0.006944in}{-0.001842in}}{\pgfqpoint{-0.006213in}{-0.003608in}}{\pgfqpoint{-0.004910in}{-0.004910in}}%
\pgfpathcurveto{\pgfqpoint{-0.003608in}{-0.006213in}}{\pgfqpoint{-0.001842in}{-0.006944in}}{\pgfqpoint{0.000000in}{-0.006944in}}%
\pgfpathclose%
\pgfusepath{stroke,fill}%
}%
\begin{pgfscope}%
\pgfsys@transformshift{1.014933in}{0.662360in}%
\pgfsys@useobject{currentmarker}{}%
\end{pgfscope}%
\begin{pgfscope}%
\pgfsys@transformshift{1.222059in}{0.671492in}%
\pgfsys@useobject{currentmarker}{}%
\end{pgfscope}%
\begin{pgfscope}%
\pgfsys@transformshift{1.428726in}{0.671492in}%
\pgfsys@useobject{currentmarker}{}%
\end{pgfscope}%
\begin{pgfscope}%
\pgfsys@transformshift{1.634933in}{0.671492in}%
\pgfsys@useobject{currentmarker}{}%
\end{pgfscope}%
\begin{pgfscope}%
\pgfsys@transformshift{1.842519in}{0.634532in}%
\pgfsys@useobject{currentmarker}{}%
\end{pgfscope}%
\begin{pgfscope}%
\pgfsys@transformshift{2.049644in}{0.671492in}%
\pgfsys@useobject{currentmarker}{}%
\end{pgfscope}%
\begin{pgfscope}%
\pgfsys@transformshift{2.256770in}{0.671492in}%
\pgfsys@useobject{currentmarker}{}%
\end{pgfscope}%
\begin{pgfscope}%
\pgfsys@transformshift{2.464356in}{0.661056in}%
\pgfsys@useobject{currentmarker}{}%
\end{pgfscope}%
\begin{pgfscope}%
\pgfsys@transformshift{2.676993in}{0.671492in}%
\pgfsys@useobject{currentmarker}{}%
\end{pgfscope}%
\begin{pgfscope}%
\pgfsys@transformshift{2.886874in}{0.661056in}%
\pgfsys@useobject{currentmarker}{}%
\end{pgfscope}%
\begin{pgfscope}%
\pgfsys@transformshift{3.094000in}{0.671492in}%
\pgfsys@useobject{currentmarker}{}%
\end{pgfscope}%
\begin{pgfscope}%
\pgfsys@transformshift{3.301126in}{0.671492in}%
\pgfsys@useobject{currentmarker}{}%
\end{pgfscope}%
\begin{pgfscope}%
\pgfsys@transformshift{3.508252in}{0.670187in}%
\pgfsys@useobject{currentmarker}{}%
\end{pgfscope}%
\begin{pgfscope}%
\pgfsys@transformshift{3.715378in}{0.671492in}%
\pgfsys@useobject{currentmarker}{}%
\end{pgfscope}%
\begin{pgfscope}%
\pgfsys@transformshift{3.922044in}{0.671492in}%
\pgfsys@useobject{currentmarker}{}%
\end{pgfscope}%
\begin{pgfscope}%
\pgfsys@transformshift{4.128711in}{0.671492in}%
\pgfsys@useobject{currentmarker}{}%
\end{pgfscope}%
\begin{pgfscope}%
\pgfsys@transformshift{4.334919in}{0.671492in}%
\pgfsys@useobject{currentmarker}{}%
\end{pgfscope}%
\begin{pgfscope}%
\pgfsys@transformshift{4.541126in}{0.671492in}%
\pgfsys@useobject{currentmarker}{}%
\end{pgfscope}%
\begin{pgfscope}%
\pgfsys@transformshift{4.747793in}{0.671492in}%
\pgfsys@useobject{currentmarker}{}%
\end{pgfscope}%
\begin{pgfscope}%
\pgfsys@transformshift{4.954000in}{0.671492in}%
\pgfsys@useobject{currentmarker}{}%
\end{pgfscope}%
\begin{pgfscope}%
\pgfsys@transformshift{5.160207in}{0.671492in}%
\pgfsys@useobject{currentmarker}{}%
\end{pgfscope}%
\end{pgfscope}%
\begin{pgfscope}%
\pgfpathrectangle{\pgfqpoint{0.800000in}{0.528000in}}{\pgfqpoint{4.960000in}{3.696000in}}%
\pgfusepath{clip}%
\pgfsetbuttcap%
\pgfsetroundjoin%
\pgfsetlinewidth{0.250937pt}%
\definecolor{currentstroke}{rgb}{0.501961,0.000000,0.501961}%
\pgfsetstrokecolor{currentstroke}%
\pgfsetdash{{0.925000pt}{0.400000pt}}{0.000000pt}%
\pgfpathmoveto{\pgfqpoint{1.014933in}{1.171539in}}%
\pgfpathlineto{\pgfqpoint{1.222059in}{1.188932in}}%
\pgfpathlineto{\pgfqpoint{1.428726in}{1.171539in}}%
\pgfpathlineto{\pgfqpoint{1.634933in}{1.171539in}}%
\pgfpathlineto{\pgfqpoint{1.842519in}{1.167191in}}%
\pgfpathlineto{\pgfqpoint{2.049644in}{1.162842in}}%
\pgfpathlineto{\pgfqpoint{2.256770in}{1.175887in}}%
\pgfpathlineto{\pgfqpoint{2.464356in}{1.197628in}}%
\pgfpathlineto{\pgfqpoint{2.676993in}{1.184584in}}%
\pgfpathlineto{\pgfqpoint{2.886874in}{1.197628in}}%
\pgfpathlineto{\pgfqpoint{3.094000in}{1.184584in}}%
\pgfpathlineto{\pgfqpoint{3.301126in}{1.188932in}}%
\pgfpathlineto{\pgfqpoint{3.508252in}{1.175887in}}%
\pgfpathlineto{\pgfqpoint{3.715378in}{1.158494in}}%
\pgfpathlineto{\pgfqpoint{3.922044in}{1.167191in}}%
\pgfpathlineto{\pgfqpoint{4.128711in}{1.167191in}}%
\pgfpathlineto{\pgfqpoint{4.334919in}{1.171539in}}%
\pgfpathlineto{\pgfqpoint{4.541126in}{1.175887in}}%
\pgfpathlineto{\pgfqpoint{4.747793in}{1.197628in}}%
\pgfpathlineto{\pgfqpoint{4.954000in}{1.197628in}}%
\pgfpathlineto{\pgfqpoint{5.160207in}{1.197628in}}%
\pgfusepath{stroke}%
\end{pgfscope}%
\begin{pgfscope}%
\pgfpathrectangle{\pgfqpoint{0.800000in}{0.528000in}}{\pgfqpoint{4.960000in}{3.696000in}}%
\pgfusepath{clip}%
\pgfsetbuttcap%
\pgfsetroundjoin%
\definecolor{currentfill}{rgb}{0.501961,0.000000,0.501961}%
\pgfsetfillcolor{currentfill}%
\pgfsetlinewidth{1.003750pt}%
\definecolor{currentstroke}{rgb}{0.501961,0.000000,0.501961}%
\pgfsetstrokecolor{currentstroke}%
\pgfsetdash{}{0pt}%
\pgfsys@defobject{currentmarker}{\pgfqpoint{-0.003472in}{-0.003472in}}{\pgfqpoint{0.003472in}{0.003472in}}{%
\pgfpathmoveto{\pgfqpoint{0.000000in}{-0.003472in}}%
\pgfpathcurveto{\pgfqpoint{0.000921in}{-0.003472in}}{\pgfqpoint{0.001804in}{-0.003106in}}{\pgfqpoint{0.002455in}{-0.002455in}}%
\pgfpathcurveto{\pgfqpoint{0.003106in}{-0.001804in}}{\pgfqpoint{0.003472in}{-0.000921in}}{\pgfqpoint{0.003472in}{0.000000in}}%
\pgfpathcurveto{\pgfqpoint{0.003472in}{0.000921in}}{\pgfqpoint{0.003106in}{0.001804in}}{\pgfqpoint{0.002455in}{0.002455in}}%
\pgfpathcurveto{\pgfqpoint{0.001804in}{0.003106in}}{\pgfqpoint{0.000921in}{0.003472in}}{\pgfqpoint{0.000000in}{0.003472in}}%
\pgfpathcurveto{\pgfqpoint{-0.000921in}{0.003472in}}{\pgfqpoint{-0.001804in}{0.003106in}}{\pgfqpoint{-0.002455in}{0.002455in}}%
\pgfpathcurveto{\pgfqpoint{-0.003106in}{0.001804in}}{\pgfqpoint{-0.003472in}{0.000921in}}{\pgfqpoint{-0.003472in}{0.000000in}}%
\pgfpathcurveto{\pgfqpoint{-0.003472in}{-0.000921in}}{\pgfqpoint{-0.003106in}{-0.001804in}}{\pgfqpoint{-0.002455in}{-0.002455in}}%
\pgfpathcurveto{\pgfqpoint{-0.001804in}{-0.003106in}}{\pgfqpoint{-0.000921in}{-0.003472in}}{\pgfqpoint{0.000000in}{-0.003472in}}%
\pgfpathclose%
\pgfusepath{stroke,fill}%
}%
\begin{pgfscope}%
\pgfsys@transformshift{1.014933in}{1.171539in}%
\pgfsys@useobject{currentmarker}{}%
\end{pgfscope}%
\begin{pgfscope}%
\pgfsys@transformshift{1.222059in}{1.188932in}%
\pgfsys@useobject{currentmarker}{}%
\end{pgfscope}%
\begin{pgfscope}%
\pgfsys@transformshift{1.428726in}{1.171539in}%
\pgfsys@useobject{currentmarker}{}%
\end{pgfscope}%
\begin{pgfscope}%
\pgfsys@transformshift{1.634933in}{1.171539in}%
\pgfsys@useobject{currentmarker}{}%
\end{pgfscope}%
\begin{pgfscope}%
\pgfsys@transformshift{1.842519in}{1.167191in}%
\pgfsys@useobject{currentmarker}{}%
\end{pgfscope}%
\begin{pgfscope}%
\pgfsys@transformshift{2.049644in}{1.162842in}%
\pgfsys@useobject{currentmarker}{}%
\end{pgfscope}%
\begin{pgfscope}%
\pgfsys@transformshift{2.256770in}{1.175887in}%
\pgfsys@useobject{currentmarker}{}%
\end{pgfscope}%
\begin{pgfscope}%
\pgfsys@transformshift{2.464356in}{1.197628in}%
\pgfsys@useobject{currentmarker}{}%
\end{pgfscope}%
\begin{pgfscope}%
\pgfsys@transformshift{2.676993in}{1.184584in}%
\pgfsys@useobject{currentmarker}{}%
\end{pgfscope}%
\begin{pgfscope}%
\pgfsys@transformshift{2.886874in}{1.197628in}%
\pgfsys@useobject{currentmarker}{}%
\end{pgfscope}%
\begin{pgfscope}%
\pgfsys@transformshift{3.094000in}{1.184584in}%
\pgfsys@useobject{currentmarker}{}%
\end{pgfscope}%
\begin{pgfscope}%
\pgfsys@transformshift{3.301126in}{1.188932in}%
\pgfsys@useobject{currentmarker}{}%
\end{pgfscope}%
\begin{pgfscope}%
\pgfsys@transformshift{3.508252in}{1.175887in}%
\pgfsys@useobject{currentmarker}{}%
\end{pgfscope}%
\begin{pgfscope}%
\pgfsys@transformshift{3.715378in}{1.158494in}%
\pgfsys@useobject{currentmarker}{}%
\end{pgfscope}%
\begin{pgfscope}%
\pgfsys@transformshift{3.922044in}{1.167191in}%
\pgfsys@useobject{currentmarker}{}%
\end{pgfscope}%
\begin{pgfscope}%
\pgfsys@transformshift{4.128711in}{1.167191in}%
\pgfsys@useobject{currentmarker}{}%
\end{pgfscope}%
\begin{pgfscope}%
\pgfsys@transformshift{4.334919in}{1.171539in}%
\pgfsys@useobject{currentmarker}{}%
\end{pgfscope}%
\begin{pgfscope}%
\pgfsys@transformshift{4.541126in}{1.175887in}%
\pgfsys@useobject{currentmarker}{}%
\end{pgfscope}%
\begin{pgfscope}%
\pgfsys@transformshift{4.747793in}{1.197628in}%
\pgfsys@useobject{currentmarker}{}%
\end{pgfscope}%
\begin{pgfscope}%
\pgfsys@transformshift{4.954000in}{1.197628in}%
\pgfsys@useobject{currentmarker}{}%
\end{pgfscope}%
\begin{pgfscope}%
\pgfsys@transformshift{5.160207in}{1.197628in}%
\pgfsys@useobject{currentmarker}{}%
\end{pgfscope}%
\end{pgfscope}%
\begin{pgfscope}%
\pgfpathrectangle{\pgfqpoint{0.800000in}{0.528000in}}{\pgfqpoint{4.960000in}{3.696000in}}%
\pgfusepath{clip}%
\pgfsetrectcap%
\pgfsetroundjoin%
\pgfsetlinewidth{0.501875pt}%
\definecolor{currentstroke}{rgb}{0.647059,0.164706,0.164706}%
\pgfsetstrokecolor{currentstroke}%
\pgfsetdash{}{0pt}%
\pgfpathmoveto{\pgfqpoint{1.003911in}{3.521760in}}%
\pgfpathlineto{\pgfqpoint{1.200474in}{3.616117in}}%
\pgfpathlineto{\pgfqpoint{1.396578in}{3.940495in}}%
\pgfpathlineto{\pgfqpoint{1.592681in}{3.962671in}}%
\pgfpathlineto{\pgfqpoint{1.789244in}{3.888751in}}%
\pgfpathlineto{\pgfqpoint{1.985348in}{3.980499in}}%
\pgfpathlineto{\pgfqpoint{2.181452in}{4.038330in}}%
\pgfpathlineto{\pgfqpoint{2.378015in}{4.125730in}}%
\pgfpathlineto{\pgfqpoint{2.574119in}{4.100510in}}%
\pgfpathlineto{\pgfqpoint{2.770222in}{4.090074in}}%
\pgfpathlineto{\pgfqpoint{2.966326in}{4.131383in}}%
\pgfpathlineto{\pgfqpoint{3.162430in}{4.150950in}}%
\pgfpathlineto{\pgfqpoint{3.358533in}{4.100510in}}%
\pgfpathlineto{\pgfqpoint{3.554637in}{4.123991in}}%
\pgfpathlineto{\pgfqpoint{3.750281in}{4.140514in}}%
\pgfpathlineto{\pgfqpoint{3.947304in}{4.123991in}}%
\pgfpathlineto{\pgfqpoint{4.143407in}{4.158342in}}%
\pgfpathlineto{\pgfqpoint{4.354207in}{4.140514in}}%
\pgfpathlineto{\pgfqpoint{4.550311in}{4.152254in}}%
\pgfpathlineto{\pgfqpoint{4.746415in}{4.150950in}}%
\pgfpathlineto{\pgfqpoint{4.942519in}{4.162690in}}%
\pgfusepath{stroke}%
\end{pgfscope}%
\begin{pgfscope}%
\pgfpathrectangle{\pgfqpoint{0.800000in}{0.528000in}}{\pgfqpoint{4.960000in}{3.696000in}}%
\pgfusepath{clip}%
\pgfsetbuttcap%
\pgfsetroundjoin%
\definecolor{currentfill}{rgb}{0.647059,0.164706,0.164706}%
\pgfsetfillcolor{currentfill}%
\pgfsetlinewidth{1.003750pt}%
\definecolor{currentstroke}{rgb}{0.647059,0.164706,0.164706}%
\pgfsetstrokecolor{currentstroke}%
\pgfsetdash{}{0pt}%
\pgfsys@defobject{currentmarker}{\pgfqpoint{-0.006944in}{-0.006944in}}{\pgfqpoint{0.006944in}{0.006944in}}{%
\pgfpathmoveto{\pgfqpoint{0.000000in}{-0.006944in}}%
\pgfpathcurveto{\pgfqpoint{0.001842in}{-0.006944in}}{\pgfqpoint{0.003608in}{-0.006213in}}{\pgfqpoint{0.004910in}{-0.004910in}}%
\pgfpathcurveto{\pgfqpoint{0.006213in}{-0.003608in}}{\pgfqpoint{0.006944in}{-0.001842in}}{\pgfqpoint{0.006944in}{0.000000in}}%
\pgfpathcurveto{\pgfqpoint{0.006944in}{0.001842in}}{\pgfqpoint{0.006213in}{0.003608in}}{\pgfqpoint{0.004910in}{0.004910in}}%
\pgfpathcurveto{\pgfqpoint{0.003608in}{0.006213in}}{\pgfqpoint{0.001842in}{0.006944in}}{\pgfqpoint{0.000000in}{0.006944in}}%
\pgfpathcurveto{\pgfqpoint{-0.001842in}{0.006944in}}{\pgfqpoint{-0.003608in}{0.006213in}}{\pgfqpoint{-0.004910in}{0.004910in}}%
\pgfpathcurveto{\pgfqpoint{-0.006213in}{0.003608in}}{\pgfqpoint{-0.006944in}{0.001842in}}{\pgfqpoint{-0.006944in}{0.000000in}}%
\pgfpathcurveto{\pgfqpoint{-0.006944in}{-0.001842in}}{\pgfqpoint{-0.006213in}{-0.003608in}}{\pgfqpoint{-0.004910in}{-0.004910in}}%
\pgfpathcurveto{\pgfqpoint{-0.003608in}{-0.006213in}}{\pgfqpoint{-0.001842in}{-0.006944in}}{\pgfqpoint{0.000000in}{-0.006944in}}%
\pgfpathclose%
\pgfusepath{stroke,fill}%
}%
\begin{pgfscope}%
\pgfsys@transformshift{1.003911in}{3.521760in}%
\pgfsys@useobject{currentmarker}{}%
\end{pgfscope}%
\begin{pgfscope}%
\pgfsys@transformshift{1.200474in}{3.616117in}%
\pgfsys@useobject{currentmarker}{}%
\end{pgfscope}%
\begin{pgfscope}%
\pgfsys@transformshift{1.396578in}{3.940495in}%
\pgfsys@useobject{currentmarker}{}%
\end{pgfscope}%
\begin{pgfscope}%
\pgfsys@transformshift{1.592681in}{3.962671in}%
\pgfsys@useobject{currentmarker}{}%
\end{pgfscope}%
\begin{pgfscope}%
\pgfsys@transformshift{1.789244in}{3.888751in}%
\pgfsys@useobject{currentmarker}{}%
\end{pgfscope}%
\begin{pgfscope}%
\pgfsys@transformshift{1.985348in}{3.980499in}%
\pgfsys@useobject{currentmarker}{}%
\end{pgfscope}%
\begin{pgfscope}%
\pgfsys@transformshift{2.181452in}{4.038330in}%
\pgfsys@useobject{currentmarker}{}%
\end{pgfscope}%
\begin{pgfscope}%
\pgfsys@transformshift{2.378015in}{4.125730in}%
\pgfsys@useobject{currentmarker}{}%
\end{pgfscope}%
\begin{pgfscope}%
\pgfsys@transformshift{2.574119in}{4.100510in}%
\pgfsys@useobject{currentmarker}{}%
\end{pgfscope}%
\begin{pgfscope}%
\pgfsys@transformshift{2.770222in}{4.090074in}%
\pgfsys@useobject{currentmarker}{}%
\end{pgfscope}%
\begin{pgfscope}%
\pgfsys@transformshift{2.966326in}{4.131383in}%
\pgfsys@useobject{currentmarker}{}%
\end{pgfscope}%
\begin{pgfscope}%
\pgfsys@transformshift{3.162430in}{4.150950in}%
\pgfsys@useobject{currentmarker}{}%
\end{pgfscope}%
\begin{pgfscope}%
\pgfsys@transformshift{3.358533in}{4.100510in}%
\pgfsys@useobject{currentmarker}{}%
\end{pgfscope}%
\begin{pgfscope}%
\pgfsys@transformshift{3.554637in}{4.123991in}%
\pgfsys@useobject{currentmarker}{}%
\end{pgfscope}%
\begin{pgfscope}%
\pgfsys@transformshift{3.750281in}{4.140514in}%
\pgfsys@useobject{currentmarker}{}%
\end{pgfscope}%
\begin{pgfscope}%
\pgfsys@transformshift{3.947304in}{4.123991in}%
\pgfsys@useobject{currentmarker}{}%
\end{pgfscope}%
\begin{pgfscope}%
\pgfsys@transformshift{4.143407in}{4.158342in}%
\pgfsys@useobject{currentmarker}{}%
\end{pgfscope}%
\begin{pgfscope}%
\pgfsys@transformshift{4.354207in}{4.140514in}%
\pgfsys@useobject{currentmarker}{}%
\end{pgfscope}%
\begin{pgfscope}%
\pgfsys@transformshift{4.550311in}{4.152254in}%
\pgfsys@useobject{currentmarker}{}%
\end{pgfscope}%
\begin{pgfscope}%
\pgfsys@transformshift{4.746415in}{4.150950in}%
\pgfsys@useobject{currentmarker}{}%
\end{pgfscope}%
\begin{pgfscope}%
\pgfsys@transformshift{4.942519in}{4.162690in}%
\pgfsys@useobject{currentmarker}{}%
\end{pgfscope}%
\end{pgfscope}%
\begin{pgfscope}%
\pgfpathrectangle{\pgfqpoint{0.800000in}{0.528000in}}{\pgfqpoint{4.960000in}{3.696000in}}%
\pgfusepath{clip}%
\pgfsetbuttcap%
\pgfsetroundjoin%
\pgfsetlinewidth{0.250937pt}%
\definecolor{currentstroke}{rgb}{0.647059,0.164706,0.164706}%
\pgfsetstrokecolor{currentstroke}%
\pgfsetdash{{0.925000pt}{0.400000pt}}{0.000000pt}%
\pgfusepath{stroke}%
\end{pgfscope}%
\begin{pgfscope}%
\pgfpathrectangle{\pgfqpoint{0.800000in}{0.528000in}}{\pgfqpoint{4.960000in}{3.696000in}}%
\pgfusepath{clip}%
\pgfsetbuttcap%
\pgfsetroundjoin%
\definecolor{currentfill}{rgb}{0.647059,0.164706,0.164706}%
\pgfsetfillcolor{currentfill}%
\pgfsetlinewidth{1.003750pt}%
\definecolor{currentstroke}{rgb}{0.647059,0.164706,0.164706}%
\pgfsetstrokecolor{currentstroke}%
\pgfsetdash{}{0pt}%
\pgfsys@defobject{currentmarker}{\pgfqpoint{-0.003472in}{-0.003472in}}{\pgfqpoint{0.003472in}{0.003472in}}{%
\pgfpathmoveto{\pgfqpoint{0.000000in}{-0.003472in}}%
\pgfpathcurveto{\pgfqpoint{0.000921in}{-0.003472in}}{\pgfqpoint{0.001804in}{-0.003106in}}{\pgfqpoint{0.002455in}{-0.002455in}}%
\pgfpathcurveto{\pgfqpoint{0.003106in}{-0.001804in}}{\pgfqpoint{0.003472in}{-0.000921in}}{\pgfqpoint{0.003472in}{0.000000in}}%
\pgfpathcurveto{\pgfqpoint{0.003472in}{0.000921in}}{\pgfqpoint{0.003106in}{0.001804in}}{\pgfqpoint{0.002455in}{0.002455in}}%
\pgfpathcurveto{\pgfqpoint{0.001804in}{0.003106in}}{\pgfqpoint{0.000921in}{0.003472in}}{\pgfqpoint{0.000000in}{0.003472in}}%
\pgfpathcurveto{\pgfqpoint{-0.000921in}{0.003472in}}{\pgfqpoint{-0.001804in}{0.003106in}}{\pgfqpoint{-0.002455in}{0.002455in}}%
\pgfpathcurveto{\pgfqpoint{-0.003106in}{0.001804in}}{\pgfqpoint{-0.003472in}{0.000921in}}{\pgfqpoint{-0.003472in}{0.000000in}}%
\pgfpathcurveto{\pgfqpoint{-0.003472in}{-0.000921in}}{\pgfqpoint{-0.003106in}{-0.001804in}}{\pgfqpoint{-0.002455in}{-0.002455in}}%
\pgfpathcurveto{\pgfqpoint{-0.001804in}{-0.003106in}}{\pgfqpoint{-0.000921in}{-0.003472in}}{\pgfqpoint{0.000000in}{-0.003472in}}%
\pgfpathclose%
\pgfusepath{stroke,fill}%
}%
\begin{pgfscope}%
\pgfsys@transformshift{1.003911in}{4.480546in}%
\pgfsys@useobject{currentmarker}{}%
\end{pgfscope}%
\begin{pgfscope}%
\pgfsys@transformshift{1.200474in}{4.558814in}%
\pgfsys@useobject{currentmarker}{}%
\end{pgfscope}%
\begin{pgfscope}%
\pgfsys@transformshift{1.396578in}{4.684913in}%
\pgfsys@useobject{currentmarker}{}%
\end{pgfscope}%
\begin{pgfscope}%
\pgfsys@transformshift{1.592681in}{4.663172in}%
\pgfsys@useobject{currentmarker}{}%
\end{pgfscope}%
\begin{pgfscope}%
\pgfsys@transformshift{1.789244in}{4.645779in}%
\pgfsys@useobject{currentmarker}{}%
\end{pgfscope}%
\begin{pgfscope}%
\pgfsys@transformshift{1.985348in}{4.706654in}%
\pgfsys@useobject{currentmarker}{}%
\end{pgfscope}%
\begin{pgfscope}%
\pgfsys@transformshift{2.181452in}{4.711002in}%
\pgfsys@useobject{currentmarker}{}%
\end{pgfscope}%
\begin{pgfscope}%
\pgfsys@transformshift{2.378015in}{4.750136in}%
\pgfsys@useobject{currentmarker}{}%
\end{pgfscope}%
\begin{pgfscope}%
\pgfsys@transformshift{2.574119in}{4.750136in}%
\pgfsys@useobject{currentmarker}{}%
\end{pgfscope}%
\begin{pgfscope}%
\pgfsys@transformshift{2.770222in}{4.737092in}%
\pgfsys@useobject{currentmarker}{}%
\end{pgfscope}%
\begin{pgfscope}%
\pgfsys@transformshift{2.966326in}{4.750136in}%
\pgfsys@useobject{currentmarker}{}%
\end{pgfscope}%
\begin{pgfscope}%
\pgfsys@transformshift{3.162430in}{4.732744in}%
\pgfsys@useobject{currentmarker}{}%
\end{pgfscope}%
\begin{pgfscope}%
\pgfsys@transformshift{3.358533in}{4.732744in}%
\pgfsys@useobject{currentmarker}{}%
\end{pgfscope}%
\begin{pgfscope}%
\pgfsys@transformshift{3.554637in}{4.719699in}%
\pgfsys@useobject{currentmarker}{}%
\end{pgfscope}%
\begin{pgfscope}%
\pgfsys@transformshift{3.750281in}{4.745788in}%
\pgfsys@useobject{currentmarker}{}%
\end{pgfscope}%
\begin{pgfscope}%
\pgfsys@transformshift{3.947304in}{4.737092in}%
\pgfsys@useobject{currentmarker}{}%
\end{pgfscope}%
\begin{pgfscope}%
\pgfsys@transformshift{4.143407in}{4.745788in}%
\pgfsys@useobject{currentmarker}{}%
\end{pgfscope}%
\begin{pgfscope}%
\pgfsys@transformshift{4.354207in}{4.741440in}%
\pgfsys@useobject{currentmarker}{}%
\end{pgfscope}%
\begin{pgfscope}%
\pgfsys@transformshift{4.550311in}{4.741440in}%
\pgfsys@useobject{currentmarker}{}%
\end{pgfscope}%
\begin{pgfscope}%
\pgfsys@transformshift{4.746415in}{4.741440in}%
\pgfsys@useobject{currentmarker}{}%
\end{pgfscope}%
\begin{pgfscope}%
\pgfsys@transformshift{4.942519in}{4.745788in}%
\pgfsys@useobject{currentmarker}{}%
\end{pgfscope}%
\end{pgfscope}%
\begin{pgfscope}%
\pgfpathrectangle{\pgfqpoint{0.800000in}{0.528000in}}{\pgfqpoint{4.960000in}{3.696000in}}%
\pgfusepath{clip}%
\pgfsetrectcap%
\pgfsetroundjoin%
\pgfsetlinewidth{0.501875pt}%
\definecolor{currentstroke}{rgb}{1.000000,0.752941,0.796078}%
\pgfsetstrokecolor{currentstroke}%
\pgfsetdash{}{0pt}%
\pgfpathmoveto{\pgfqpoint{1.001615in}{3.526108in}}%
\pgfpathlineto{\pgfqpoint{1.194963in}{3.730475in}}%
\pgfpathlineto{\pgfqpoint{1.387393in}{3.863531in}}%
\pgfpathlineto{\pgfqpoint{1.579822in}{3.848747in}}%
\pgfpathlineto{\pgfqpoint{1.772252in}{3.939191in}}%
\pgfpathlineto{\pgfqpoint{1.964681in}{3.767435in}}%
\pgfpathlineto{\pgfqpoint{2.157570in}{3.986586in}}%
\pgfpathlineto{\pgfqpoint{2.350000in}{4.061811in}}%
\pgfpathlineto{\pgfqpoint{2.542889in}{4.109207in}}%
\pgfpathlineto{\pgfqpoint{2.735778in}{4.088770in}}%
\pgfpathlineto{\pgfqpoint{2.928207in}{4.098771in}}%
\pgfpathlineto{\pgfqpoint{3.120637in}{4.104858in}}%
\pgfpathlineto{\pgfqpoint{3.313067in}{4.067898in}}%
\pgfpathlineto{\pgfqpoint{3.505496in}{4.116599in}}%
\pgfpathlineto{\pgfqpoint{3.697926in}{4.131383in}}%
\pgfpathlineto{\pgfqpoint{3.890356in}{4.146167in}}%
\pgfpathlineto{\pgfqpoint{4.082785in}{4.138775in}}%
\pgfpathlineto{\pgfqpoint{4.275215in}{4.116599in}}%
\pgfpathlineto{\pgfqpoint{4.467644in}{4.152254in}}%
\pgfpathlineto{\pgfqpoint{4.660074in}{4.116599in}}%
\pgfpathlineto{\pgfqpoint{4.852504in}{4.118338in}}%
\pgfusepath{stroke}%
\end{pgfscope}%
\begin{pgfscope}%
\pgfpathrectangle{\pgfqpoint{0.800000in}{0.528000in}}{\pgfqpoint{4.960000in}{3.696000in}}%
\pgfusepath{clip}%
\pgfsetbuttcap%
\pgfsetroundjoin%
\definecolor{currentfill}{rgb}{1.000000,0.752941,0.796078}%
\pgfsetfillcolor{currentfill}%
\pgfsetlinewidth{1.003750pt}%
\definecolor{currentstroke}{rgb}{1.000000,0.752941,0.796078}%
\pgfsetstrokecolor{currentstroke}%
\pgfsetdash{}{0pt}%
\pgfsys@defobject{currentmarker}{\pgfqpoint{-0.006944in}{-0.006944in}}{\pgfqpoint{0.006944in}{0.006944in}}{%
\pgfpathmoveto{\pgfqpoint{0.000000in}{-0.006944in}}%
\pgfpathcurveto{\pgfqpoint{0.001842in}{-0.006944in}}{\pgfqpoint{0.003608in}{-0.006213in}}{\pgfqpoint{0.004910in}{-0.004910in}}%
\pgfpathcurveto{\pgfqpoint{0.006213in}{-0.003608in}}{\pgfqpoint{0.006944in}{-0.001842in}}{\pgfqpoint{0.006944in}{0.000000in}}%
\pgfpathcurveto{\pgfqpoint{0.006944in}{0.001842in}}{\pgfqpoint{0.006213in}{0.003608in}}{\pgfqpoint{0.004910in}{0.004910in}}%
\pgfpathcurveto{\pgfqpoint{0.003608in}{0.006213in}}{\pgfqpoint{0.001842in}{0.006944in}}{\pgfqpoint{0.000000in}{0.006944in}}%
\pgfpathcurveto{\pgfqpoint{-0.001842in}{0.006944in}}{\pgfqpoint{-0.003608in}{0.006213in}}{\pgfqpoint{-0.004910in}{0.004910in}}%
\pgfpathcurveto{\pgfqpoint{-0.006213in}{0.003608in}}{\pgfqpoint{-0.006944in}{0.001842in}}{\pgfqpoint{-0.006944in}{0.000000in}}%
\pgfpathcurveto{\pgfqpoint{-0.006944in}{-0.001842in}}{\pgfqpoint{-0.006213in}{-0.003608in}}{\pgfqpoint{-0.004910in}{-0.004910in}}%
\pgfpathcurveto{\pgfqpoint{-0.003608in}{-0.006213in}}{\pgfqpoint{-0.001842in}{-0.006944in}}{\pgfqpoint{0.000000in}{-0.006944in}}%
\pgfpathclose%
\pgfusepath{stroke,fill}%
}%
\begin{pgfscope}%
\pgfsys@transformshift{1.001615in}{3.526108in}%
\pgfsys@useobject{currentmarker}{}%
\end{pgfscope}%
\begin{pgfscope}%
\pgfsys@transformshift{1.194963in}{3.730475in}%
\pgfsys@useobject{currentmarker}{}%
\end{pgfscope}%
\begin{pgfscope}%
\pgfsys@transformshift{1.387393in}{3.863531in}%
\pgfsys@useobject{currentmarker}{}%
\end{pgfscope}%
\begin{pgfscope}%
\pgfsys@transformshift{1.579822in}{3.848747in}%
\pgfsys@useobject{currentmarker}{}%
\end{pgfscope}%
\begin{pgfscope}%
\pgfsys@transformshift{1.772252in}{3.939191in}%
\pgfsys@useobject{currentmarker}{}%
\end{pgfscope}%
\begin{pgfscope}%
\pgfsys@transformshift{1.964681in}{3.767435in}%
\pgfsys@useobject{currentmarker}{}%
\end{pgfscope}%
\begin{pgfscope}%
\pgfsys@transformshift{2.157570in}{3.986586in}%
\pgfsys@useobject{currentmarker}{}%
\end{pgfscope}%
\begin{pgfscope}%
\pgfsys@transformshift{2.350000in}{4.061811in}%
\pgfsys@useobject{currentmarker}{}%
\end{pgfscope}%
\begin{pgfscope}%
\pgfsys@transformshift{2.542889in}{4.109207in}%
\pgfsys@useobject{currentmarker}{}%
\end{pgfscope}%
\begin{pgfscope}%
\pgfsys@transformshift{2.735778in}{4.088770in}%
\pgfsys@useobject{currentmarker}{}%
\end{pgfscope}%
\begin{pgfscope}%
\pgfsys@transformshift{2.928207in}{4.098771in}%
\pgfsys@useobject{currentmarker}{}%
\end{pgfscope}%
\begin{pgfscope}%
\pgfsys@transformshift{3.120637in}{4.104858in}%
\pgfsys@useobject{currentmarker}{}%
\end{pgfscope}%
\begin{pgfscope}%
\pgfsys@transformshift{3.313067in}{4.067898in}%
\pgfsys@useobject{currentmarker}{}%
\end{pgfscope}%
\begin{pgfscope}%
\pgfsys@transformshift{3.505496in}{4.116599in}%
\pgfsys@useobject{currentmarker}{}%
\end{pgfscope}%
\begin{pgfscope}%
\pgfsys@transformshift{3.697926in}{4.131383in}%
\pgfsys@useobject{currentmarker}{}%
\end{pgfscope}%
\begin{pgfscope}%
\pgfsys@transformshift{3.890356in}{4.146167in}%
\pgfsys@useobject{currentmarker}{}%
\end{pgfscope}%
\begin{pgfscope}%
\pgfsys@transformshift{4.082785in}{4.138775in}%
\pgfsys@useobject{currentmarker}{}%
\end{pgfscope}%
\begin{pgfscope}%
\pgfsys@transformshift{4.275215in}{4.116599in}%
\pgfsys@useobject{currentmarker}{}%
\end{pgfscope}%
\begin{pgfscope}%
\pgfsys@transformshift{4.467644in}{4.152254in}%
\pgfsys@useobject{currentmarker}{}%
\end{pgfscope}%
\begin{pgfscope}%
\pgfsys@transformshift{4.660074in}{4.116599in}%
\pgfsys@useobject{currentmarker}{}%
\end{pgfscope}%
\begin{pgfscope}%
\pgfsys@transformshift{4.852504in}{4.118338in}%
\pgfsys@useobject{currentmarker}{}%
\end{pgfscope}%
\end{pgfscope}%
\begin{pgfscope}%
\pgfpathrectangle{\pgfqpoint{0.800000in}{0.528000in}}{\pgfqpoint{4.960000in}{3.696000in}}%
\pgfusepath{clip}%
\pgfsetbuttcap%
\pgfsetroundjoin%
\pgfsetlinewidth{0.250937pt}%
\definecolor{currentstroke}{rgb}{1.000000,0.752941,0.796078}%
\pgfsetstrokecolor{currentstroke}%
\pgfsetdash{{0.925000pt}{0.400000pt}}{0.000000pt}%
\pgfusepath{stroke}%
\end{pgfscope}%
\begin{pgfscope}%
\pgfpathrectangle{\pgfqpoint{0.800000in}{0.528000in}}{\pgfqpoint{4.960000in}{3.696000in}}%
\pgfusepath{clip}%
\pgfsetbuttcap%
\pgfsetroundjoin%
\definecolor{currentfill}{rgb}{1.000000,0.752941,0.796078}%
\pgfsetfillcolor{currentfill}%
\pgfsetlinewidth{1.003750pt}%
\definecolor{currentstroke}{rgb}{1.000000,0.752941,0.796078}%
\pgfsetstrokecolor{currentstroke}%
\pgfsetdash{}{0pt}%
\pgfsys@defobject{currentmarker}{\pgfqpoint{-0.003472in}{-0.003472in}}{\pgfqpoint{0.003472in}{0.003472in}}{%
\pgfpathmoveto{\pgfqpoint{0.000000in}{-0.003472in}}%
\pgfpathcurveto{\pgfqpoint{0.000921in}{-0.003472in}}{\pgfqpoint{0.001804in}{-0.003106in}}{\pgfqpoint{0.002455in}{-0.002455in}}%
\pgfpathcurveto{\pgfqpoint{0.003106in}{-0.001804in}}{\pgfqpoint{0.003472in}{-0.000921in}}{\pgfqpoint{0.003472in}{0.000000in}}%
\pgfpathcurveto{\pgfqpoint{0.003472in}{0.000921in}}{\pgfqpoint{0.003106in}{0.001804in}}{\pgfqpoint{0.002455in}{0.002455in}}%
\pgfpathcurveto{\pgfqpoint{0.001804in}{0.003106in}}{\pgfqpoint{0.000921in}{0.003472in}}{\pgfqpoint{0.000000in}{0.003472in}}%
\pgfpathcurveto{\pgfqpoint{-0.000921in}{0.003472in}}{\pgfqpoint{-0.001804in}{0.003106in}}{\pgfqpoint{-0.002455in}{0.002455in}}%
\pgfpathcurveto{\pgfqpoint{-0.003106in}{0.001804in}}{\pgfqpoint{-0.003472in}{0.000921in}}{\pgfqpoint{-0.003472in}{0.000000in}}%
\pgfpathcurveto{\pgfqpoint{-0.003472in}{-0.000921in}}{\pgfqpoint{-0.003106in}{-0.001804in}}{\pgfqpoint{-0.002455in}{-0.002455in}}%
\pgfpathcurveto{\pgfqpoint{-0.001804in}{-0.003106in}}{\pgfqpoint{-0.000921in}{-0.003472in}}{\pgfqpoint{0.000000in}{-0.003472in}}%
\pgfpathclose%
\pgfusepath{stroke,fill}%
}%
\begin{pgfscope}%
\pgfsys@transformshift{1.001615in}{4.532725in}%
\pgfsys@useobject{currentmarker}{}%
\end{pgfscope}%
\begin{pgfscope}%
\pgfsys@transformshift{1.194963in}{4.624038in}%
\pgfsys@useobject{currentmarker}{}%
\end{pgfscope}%
\begin{pgfscope}%
\pgfsys@transformshift{1.387393in}{4.697958in}%
\pgfsys@useobject{currentmarker}{}%
\end{pgfscope}%
\begin{pgfscope}%
\pgfsys@transformshift{1.579822in}{4.650127in}%
\pgfsys@useobject{currentmarker}{}%
\end{pgfscope}%
\begin{pgfscope}%
\pgfsys@transformshift{1.772252in}{4.711002in}%
\pgfsys@useobject{currentmarker}{}%
\end{pgfscope}%
\begin{pgfscope}%
\pgfsys@transformshift{1.964681in}{4.645779in}%
\pgfsys@useobject{currentmarker}{}%
\end{pgfscope}%
\begin{pgfscope}%
\pgfsys@transformshift{2.157570in}{4.706654in}%
\pgfsys@useobject{currentmarker}{}%
\end{pgfscope}%
\begin{pgfscope}%
\pgfsys@transformshift{2.350000in}{4.732744in}%
\pgfsys@useobject{currentmarker}{}%
\end{pgfscope}%
\begin{pgfscope}%
\pgfsys@transformshift{2.542889in}{4.737092in}%
\pgfsys@useobject{currentmarker}{}%
\end{pgfscope}%
\begin{pgfscope}%
\pgfsys@transformshift{2.735778in}{4.728395in}%
\pgfsys@useobject{currentmarker}{}%
\end{pgfscope}%
\begin{pgfscope}%
\pgfsys@transformshift{2.928207in}{4.741440in}%
\pgfsys@useobject{currentmarker}{}%
\end{pgfscope}%
\begin{pgfscope}%
\pgfsys@transformshift{3.120637in}{4.750136in}%
\pgfsys@useobject{currentmarker}{}%
\end{pgfscope}%
\begin{pgfscope}%
\pgfsys@transformshift{3.313067in}{4.732744in}%
\pgfsys@useobject{currentmarker}{}%
\end{pgfscope}%
\begin{pgfscope}%
\pgfsys@transformshift{3.505496in}{4.737092in}%
\pgfsys@useobject{currentmarker}{}%
\end{pgfscope}%
\begin{pgfscope}%
\pgfsys@transformshift{3.697926in}{4.741440in}%
\pgfsys@useobject{currentmarker}{}%
\end{pgfscope}%
\begin{pgfscope}%
\pgfsys@transformshift{3.890356in}{4.741440in}%
\pgfsys@useobject{currentmarker}{}%
\end{pgfscope}%
\begin{pgfscope}%
\pgfsys@transformshift{4.082785in}{4.750136in}%
\pgfsys@useobject{currentmarker}{}%
\end{pgfscope}%
\begin{pgfscope}%
\pgfsys@transformshift{4.275215in}{4.741440in}%
\pgfsys@useobject{currentmarker}{}%
\end{pgfscope}%
\begin{pgfscope}%
\pgfsys@transformshift{4.467644in}{4.741440in}%
\pgfsys@useobject{currentmarker}{}%
\end{pgfscope}%
\begin{pgfscope}%
\pgfsys@transformshift{4.660074in}{4.732744in}%
\pgfsys@useobject{currentmarker}{}%
\end{pgfscope}%
\begin{pgfscope}%
\pgfsys@transformshift{4.852504in}{4.745788in}%
\pgfsys@useobject{currentmarker}{}%
\end{pgfscope}%
\end{pgfscope}%
\begin{pgfscope}%
\pgfsetrectcap%
\pgfsetmiterjoin%
\pgfsetlinewidth{0.803000pt}%
\definecolor{currentstroke}{rgb}{0.000000,0.000000,0.000000}%
\pgfsetstrokecolor{currentstroke}%
\pgfsetdash{}{0pt}%
\pgfpathmoveto{\pgfqpoint{0.800000in}{0.528000in}}%
\pgfpathlineto{\pgfqpoint{0.800000in}{4.224000in}}%
\pgfusepath{stroke}%
\end{pgfscope}%
\begin{pgfscope}%
\pgfsetrectcap%
\pgfsetmiterjoin%
\pgfsetlinewidth{0.803000pt}%
\definecolor{currentstroke}{rgb}{0.000000,0.000000,0.000000}%
\pgfsetstrokecolor{currentstroke}%
\pgfsetdash{}{0pt}%
\pgfpathmoveto{\pgfqpoint{5.760000in}{0.528000in}}%
\pgfpathlineto{\pgfqpoint{5.760000in}{4.224000in}}%
\pgfusepath{stroke}%
\end{pgfscope}%
\begin{pgfscope}%
\pgfsetrectcap%
\pgfsetmiterjoin%
\pgfsetlinewidth{0.803000pt}%
\definecolor{currentstroke}{rgb}{0.000000,0.000000,0.000000}%
\pgfsetstrokecolor{currentstroke}%
\pgfsetdash{}{0pt}%
\pgfpathmoveto{\pgfqpoint{0.800000in}{0.528000in}}%
\pgfpathlineto{\pgfqpoint{5.760000in}{0.528000in}}%
\pgfusepath{stroke}%
\end{pgfscope}%
\begin{pgfscope}%
\pgfsetrectcap%
\pgfsetmiterjoin%
\pgfsetlinewidth{0.803000pt}%
\definecolor{currentstroke}{rgb}{0.000000,0.000000,0.000000}%
\pgfsetstrokecolor{currentstroke}%
\pgfsetdash{}{0pt}%
\pgfpathmoveto{\pgfqpoint{0.800000in}{4.224000in}}%
\pgfpathlineto{\pgfqpoint{5.760000in}{4.224000in}}%
\pgfusepath{stroke}%
\end{pgfscope}%
\begin{pgfscope}%
\definecolor{textcolor}{rgb}{0.000000,0.000000,0.000000}%
\pgfsetstrokecolor{textcolor}%
\pgfsetfillcolor{textcolor}%
\pgftext[x=0.477910in,y=4.462851in,left,base]{\color{textcolor}\sffamily\fontsize{10.000000}{12.000000}\selectfont model training "food-50"; InceptionV3; 21 epochs; min. 02:27:04; max 02:50:49}%
\end{pgfscope}%
\begin{pgfscope}%
\definecolor{textcolor}{rgb}{0.000000,0.000000,0.000000}%
\pgfsetstrokecolor{textcolor}%
\pgfsetfillcolor{textcolor}%
\pgftext[x=2.429611in,y=4.307333in,left,base]{\color{textcolor}\sffamily\fontsize{10.000000}{12.000000}\selectfont (GPU - Nvidia GTX 1060)}%
\end{pgfscope}%
\begin{pgfscope}%
\pgfsetbuttcap%
\pgfsetmiterjoin%
\definecolor{currentfill}{rgb}{1.000000,1.000000,1.000000}%
\pgfsetfillcolor{currentfill}%
\pgfsetfillopacity{0.800000}%
\pgfsetlinewidth{1.003750pt}%
\definecolor{currentstroke}{rgb}{0.800000,0.800000,0.800000}%
\pgfsetstrokecolor{currentstroke}%
\pgfsetstrokeopacity{0.800000}%
\pgfsetdash{}{0pt}%
\pgfpathmoveto{\pgfqpoint{0.868056in}{1.362517in}}%
\pgfpathlineto{\pgfqpoint{4.537986in}{1.362517in}}%
\pgfpathquadraticcurveto{\pgfqpoint{4.557430in}{1.362517in}}{\pgfqpoint{4.557430in}{1.381961in}}%
\pgfpathlineto{\pgfqpoint{4.557430in}{3.370039in}}%
\pgfpathquadraticcurveto{\pgfqpoint{4.557430in}{3.389483in}}{\pgfqpoint{4.537986in}{3.389483in}}%
\pgfpathlineto{\pgfqpoint{0.868056in}{3.389483in}}%
\pgfpathquadraticcurveto{\pgfqpoint{0.848611in}{3.389483in}}{\pgfqpoint{0.848611in}{3.370039in}}%
\pgfpathlineto{\pgfqpoint{0.848611in}{1.381961in}}%
\pgfpathquadraticcurveto{\pgfqpoint{0.848611in}{1.362517in}}{\pgfqpoint{0.868056in}{1.362517in}}%
\pgfpathclose%
\pgfusepath{stroke,fill}%
\end{pgfscope}%
\begin{pgfscope}%
\pgfsetrectcap%
\pgfsetroundjoin%
\pgfsetlinewidth{0.501875pt}%
\definecolor{currentstroke}{rgb}{0.000000,0.000000,1.000000}%
\pgfsetstrokecolor{currentstroke}%
\pgfsetdash{}{0pt}%
\pgfpathmoveto{\pgfqpoint{0.887500in}{3.310756in}}%
\pgfpathlineto{\pgfqpoint{1.081944in}{3.310756in}}%
\pgfusepath{stroke}%
\end{pgfscope}%
\begin{pgfscope}%
\pgfsetbuttcap%
\pgfsetroundjoin%
\definecolor{currentfill}{rgb}{0.000000,0.000000,1.000000}%
\pgfsetfillcolor{currentfill}%
\pgfsetlinewidth{1.003750pt}%
\definecolor{currentstroke}{rgb}{0.000000,0.000000,1.000000}%
\pgfsetstrokecolor{currentstroke}%
\pgfsetdash{}{0pt}%
\pgfsys@defobject{currentmarker}{\pgfqpoint{-0.006944in}{-0.006944in}}{\pgfqpoint{0.006944in}{0.006944in}}{%
\pgfpathmoveto{\pgfqpoint{0.000000in}{-0.006944in}}%
\pgfpathcurveto{\pgfqpoint{0.001842in}{-0.006944in}}{\pgfqpoint{0.003608in}{-0.006213in}}{\pgfqpoint{0.004910in}{-0.004910in}}%
\pgfpathcurveto{\pgfqpoint{0.006213in}{-0.003608in}}{\pgfqpoint{0.006944in}{-0.001842in}}{\pgfqpoint{0.006944in}{0.000000in}}%
\pgfpathcurveto{\pgfqpoint{0.006944in}{0.001842in}}{\pgfqpoint{0.006213in}{0.003608in}}{\pgfqpoint{0.004910in}{0.004910in}}%
\pgfpathcurveto{\pgfqpoint{0.003608in}{0.006213in}}{\pgfqpoint{0.001842in}{0.006944in}}{\pgfqpoint{0.000000in}{0.006944in}}%
\pgfpathcurveto{\pgfqpoint{-0.001842in}{0.006944in}}{\pgfqpoint{-0.003608in}{0.006213in}}{\pgfqpoint{-0.004910in}{0.004910in}}%
\pgfpathcurveto{\pgfqpoint{-0.006213in}{0.003608in}}{\pgfqpoint{-0.006944in}{0.001842in}}{\pgfqpoint{-0.006944in}{0.000000in}}%
\pgfpathcurveto{\pgfqpoint{-0.006944in}{-0.001842in}}{\pgfqpoint{-0.006213in}{-0.003608in}}{\pgfqpoint{-0.004910in}{-0.004910in}}%
\pgfpathcurveto{\pgfqpoint{-0.003608in}{-0.006213in}}{\pgfqpoint{-0.001842in}{-0.006944in}}{\pgfqpoint{0.000000in}{-0.006944in}}%
\pgfpathclose%
\pgfusepath{stroke,fill}%
}%
\begin{pgfscope}%
\pgfsys@transformshift{0.984722in}{3.310756in}%
\pgfsys@useobject{currentmarker}{}%
\end{pgfscope}%
\end{pgfscope}%
\begin{pgfscope}%
\definecolor{textcolor}{rgb}{0.000000,0.000000,0.000000}%
\pgfsetstrokecolor{textcolor}%
\pgfsetfillcolor{textcolor}%
\pgftext[x=1.159722in,y=3.276728in,left,base]{\color{textcolor}\sffamily\fontsize{7.000000}{8.400000}\selectfont adadelta 0.001 0.5 7 0.0 0.0 (51.28\% - 02:50:49) (top-1)}%
\end{pgfscope}%
\begin{pgfscope}%
\pgfsetbuttcap%
\pgfsetroundjoin%
\pgfsetlinewidth{0.250937pt}%
\definecolor{currentstroke}{rgb}{0.000000,0.000000,1.000000}%
\pgfsetstrokecolor{currentstroke}%
\pgfsetdash{{0.925000pt}{0.400000pt}}{0.000000pt}%
\pgfpathmoveto{\pgfqpoint{0.887500in}{3.168056in}}%
\pgfpathlineto{\pgfqpoint{1.081944in}{3.168056in}}%
\pgfusepath{stroke}%
\end{pgfscope}%
\begin{pgfscope}%
\pgfsetbuttcap%
\pgfsetroundjoin%
\definecolor{currentfill}{rgb}{0.000000,0.000000,1.000000}%
\pgfsetfillcolor{currentfill}%
\pgfsetlinewidth{1.003750pt}%
\definecolor{currentstroke}{rgb}{0.000000,0.000000,1.000000}%
\pgfsetstrokecolor{currentstroke}%
\pgfsetdash{}{0pt}%
\pgfsys@defobject{currentmarker}{\pgfqpoint{-0.003472in}{-0.003472in}}{\pgfqpoint{0.003472in}{0.003472in}}{%
\pgfpathmoveto{\pgfqpoint{0.000000in}{-0.003472in}}%
\pgfpathcurveto{\pgfqpoint{0.000921in}{-0.003472in}}{\pgfqpoint{0.001804in}{-0.003106in}}{\pgfqpoint{0.002455in}{-0.002455in}}%
\pgfpathcurveto{\pgfqpoint{0.003106in}{-0.001804in}}{\pgfqpoint{0.003472in}{-0.000921in}}{\pgfqpoint{0.003472in}{0.000000in}}%
\pgfpathcurveto{\pgfqpoint{0.003472in}{0.000921in}}{\pgfqpoint{0.003106in}{0.001804in}}{\pgfqpoint{0.002455in}{0.002455in}}%
\pgfpathcurveto{\pgfqpoint{0.001804in}{0.003106in}}{\pgfqpoint{0.000921in}{0.003472in}}{\pgfqpoint{0.000000in}{0.003472in}}%
\pgfpathcurveto{\pgfqpoint{-0.000921in}{0.003472in}}{\pgfqpoint{-0.001804in}{0.003106in}}{\pgfqpoint{-0.002455in}{0.002455in}}%
\pgfpathcurveto{\pgfqpoint{-0.003106in}{0.001804in}}{\pgfqpoint{-0.003472in}{0.000921in}}{\pgfqpoint{-0.003472in}{0.000000in}}%
\pgfpathcurveto{\pgfqpoint{-0.003472in}{-0.000921in}}{\pgfqpoint{-0.003106in}{-0.001804in}}{\pgfqpoint{-0.002455in}{-0.002455in}}%
\pgfpathcurveto{\pgfqpoint{-0.001804in}{-0.003106in}}{\pgfqpoint{-0.000921in}{-0.003472in}}{\pgfqpoint{0.000000in}{-0.003472in}}%
\pgfpathclose%
\pgfusepath{stroke,fill}%
}%
\begin{pgfscope}%
\pgfsys@transformshift{0.984722in}{3.168056in}%
\pgfsys@useobject{currentmarker}{}%
\end{pgfscope}%
\end{pgfscope}%
\begin{pgfscope}%
\definecolor{textcolor}{rgb}{0.000000,0.000000,0.000000}%
\pgfsetstrokecolor{textcolor}%
\pgfsetfillcolor{textcolor}%
\pgftext[x=1.159722in,y=3.134028in,left,base]{\color{textcolor}\sffamily\fontsize{7.000000}{8.400000}\selectfont adadelta 0.001 0.5 7 0.0 0.0 (80.80\% - 02:50:49) (top-5)}%
\end{pgfscope}%
\begin{pgfscope}%
\pgfsetrectcap%
\pgfsetroundjoin%
\pgfsetlinewidth{0.501875pt}%
\definecolor{currentstroke}{rgb}{1.000000,0.647059,0.000000}%
\pgfsetstrokecolor{currentstroke}%
\pgfsetdash{}{0pt}%
\pgfpathmoveto{\pgfqpoint{0.887500in}{3.025356in}}%
\pgfpathlineto{\pgfqpoint{1.081944in}{3.025356in}}%
\pgfusepath{stroke}%
\end{pgfscope}%
\begin{pgfscope}%
\pgfsetbuttcap%
\pgfsetroundjoin%
\definecolor{currentfill}{rgb}{1.000000,0.647059,0.000000}%
\pgfsetfillcolor{currentfill}%
\pgfsetlinewidth{1.003750pt}%
\definecolor{currentstroke}{rgb}{1.000000,0.647059,0.000000}%
\pgfsetstrokecolor{currentstroke}%
\pgfsetdash{}{0pt}%
\pgfsys@defobject{currentmarker}{\pgfqpoint{-0.006944in}{-0.006944in}}{\pgfqpoint{0.006944in}{0.006944in}}{%
\pgfpathmoveto{\pgfqpoint{0.000000in}{-0.006944in}}%
\pgfpathcurveto{\pgfqpoint{0.001842in}{-0.006944in}}{\pgfqpoint{0.003608in}{-0.006213in}}{\pgfqpoint{0.004910in}{-0.004910in}}%
\pgfpathcurveto{\pgfqpoint{0.006213in}{-0.003608in}}{\pgfqpoint{0.006944in}{-0.001842in}}{\pgfqpoint{0.006944in}{0.000000in}}%
\pgfpathcurveto{\pgfqpoint{0.006944in}{0.001842in}}{\pgfqpoint{0.006213in}{0.003608in}}{\pgfqpoint{0.004910in}{0.004910in}}%
\pgfpathcurveto{\pgfqpoint{0.003608in}{0.006213in}}{\pgfqpoint{0.001842in}{0.006944in}}{\pgfqpoint{0.000000in}{0.006944in}}%
\pgfpathcurveto{\pgfqpoint{-0.001842in}{0.006944in}}{\pgfqpoint{-0.003608in}{0.006213in}}{\pgfqpoint{-0.004910in}{0.004910in}}%
\pgfpathcurveto{\pgfqpoint{-0.006213in}{0.003608in}}{\pgfqpoint{-0.006944in}{0.001842in}}{\pgfqpoint{-0.006944in}{0.000000in}}%
\pgfpathcurveto{\pgfqpoint{-0.006944in}{-0.001842in}}{\pgfqpoint{-0.006213in}{-0.003608in}}{\pgfqpoint{-0.004910in}{-0.004910in}}%
\pgfpathcurveto{\pgfqpoint{-0.003608in}{-0.006213in}}{\pgfqpoint{-0.001842in}{-0.006944in}}{\pgfqpoint{0.000000in}{-0.006944in}}%
\pgfpathclose%
\pgfusepath{stroke,fill}%
}%
\begin{pgfscope}%
\pgfsys@transformshift{0.984722in}{3.025356in}%
\pgfsys@useobject{currentmarker}{}%
\end{pgfscope}%
\end{pgfscope}%
\begin{pgfscope}%
\definecolor{textcolor}{rgb}{0.000000,0.000000,0.000000}%
\pgfsetstrokecolor{textcolor}%
\pgfsetfillcolor{textcolor}%
\pgftext[x=1.159722in,y=2.991328in,left,base]{\color{textcolor}\sffamily\fontsize{7.000000}{8.400000}\selectfont adagrad 0.001 0.5 7 0.0 0.0 (82.02\% - 02:34:15) (top-1)}%
\end{pgfscope}%
\begin{pgfscope}%
\pgfsetbuttcap%
\pgfsetroundjoin%
\pgfsetlinewidth{0.250937pt}%
\definecolor{currentstroke}{rgb}{1.000000,0.647059,0.000000}%
\pgfsetstrokecolor{currentstroke}%
\pgfsetdash{{0.925000pt}{0.400000pt}}{0.000000pt}%
\pgfpathmoveto{\pgfqpoint{0.887500in}{2.882656in}}%
\pgfpathlineto{\pgfqpoint{1.081944in}{2.882656in}}%
\pgfusepath{stroke}%
\end{pgfscope}%
\begin{pgfscope}%
\pgfsetbuttcap%
\pgfsetroundjoin%
\definecolor{currentfill}{rgb}{1.000000,0.647059,0.000000}%
\pgfsetfillcolor{currentfill}%
\pgfsetlinewidth{1.003750pt}%
\definecolor{currentstroke}{rgb}{1.000000,0.647059,0.000000}%
\pgfsetstrokecolor{currentstroke}%
\pgfsetdash{}{0pt}%
\pgfsys@defobject{currentmarker}{\pgfqpoint{-0.003472in}{-0.003472in}}{\pgfqpoint{0.003472in}{0.003472in}}{%
\pgfpathmoveto{\pgfqpoint{0.000000in}{-0.003472in}}%
\pgfpathcurveto{\pgfqpoint{0.000921in}{-0.003472in}}{\pgfqpoint{0.001804in}{-0.003106in}}{\pgfqpoint{0.002455in}{-0.002455in}}%
\pgfpathcurveto{\pgfqpoint{0.003106in}{-0.001804in}}{\pgfqpoint{0.003472in}{-0.000921in}}{\pgfqpoint{0.003472in}{0.000000in}}%
\pgfpathcurveto{\pgfqpoint{0.003472in}{0.000921in}}{\pgfqpoint{0.003106in}{0.001804in}}{\pgfqpoint{0.002455in}{0.002455in}}%
\pgfpathcurveto{\pgfqpoint{0.001804in}{0.003106in}}{\pgfqpoint{0.000921in}{0.003472in}}{\pgfqpoint{0.000000in}{0.003472in}}%
\pgfpathcurveto{\pgfqpoint{-0.000921in}{0.003472in}}{\pgfqpoint{-0.001804in}{0.003106in}}{\pgfqpoint{-0.002455in}{0.002455in}}%
\pgfpathcurveto{\pgfqpoint{-0.003106in}{0.001804in}}{\pgfqpoint{-0.003472in}{0.000921in}}{\pgfqpoint{-0.003472in}{0.000000in}}%
\pgfpathcurveto{\pgfqpoint{-0.003472in}{-0.000921in}}{\pgfqpoint{-0.003106in}{-0.001804in}}{\pgfqpoint{-0.002455in}{-0.002455in}}%
\pgfpathcurveto{\pgfqpoint{-0.001804in}{-0.003106in}}{\pgfqpoint{-0.000921in}{-0.003472in}}{\pgfqpoint{0.000000in}{-0.003472in}}%
\pgfpathclose%
\pgfusepath{stroke,fill}%
}%
\begin{pgfscope}%
\pgfsys@transformshift{0.984722in}{2.882656in}%
\pgfsys@useobject{currentmarker}{}%
\end{pgfscope}%
\end{pgfscope}%
\begin{pgfscope}%
\definecolor{textcolor}{rgb}{0.000000,0.000000,0.000000}%
\pgfsetstrokecolor{textcolor}%
\pgfsetfillcolor{textcolor}%
\pgftext[x=1.159722in,y=2.848628in,left,base]{\color{textcolor}\sffamily\fontsize{7.000000}{8.400000}\selectfont adagrad 0.001 0.5 7 0.0 0.0 (96.30\% - 02:34:15) (top-5)}%
\end{pgfscope}%
\begin{pgfscope}%
\pgfsetrectcap%
\pgfsetroundjoin%
\pgfsetlinewidth{0.501875pt}%
\definecolor{currentstroke}{rgb}{0.000000,0.501961,0.000000}%
\pgfsetstrokecolor{currentstroke}%
\pgfsetdash{}{0pt}%
\pgfpathmoveto{\pgfqpoint{0.887500in}{2.739956in}}%
\pgfpathlineto{\pgfqpoint{1.081944in}{2.739956in}}%
\pgfusepath{stroke}%
\end{pgfscope}%
\begin{pgfscope}%
\pgfsetbuttcap%
\pgfsetroundjoin%
\definecolor{currentfill}{rgb}{0.000000,0.501961,0.000000}%
\pgfsetfillcolor{currentfill}%
\pgfsetlinewidth{1.003750pt}%
\definecolor{currentstroke}{rgb}{0.000000,0.501961,0.000000}%
\pgfsetstrokecolor{currentstroke}%
\pgfsetdash{}{0pt}%
\pgfsys@defobject{currentmarker}{\pgfqpoint{-0.006944in}{-0.006944in}}{\pgfqpoint{0.006944in}{0.006944in}}{%
\pgfpathmoveto{\pgfqpoint{0.000000in}{-0.006944in}}%
\pgfpathcurveto{\pgfqpoint{0.001842in}{-0.006944in}}{\pgfqpoint{0.003608in}{-0.006213in}}{\pgfqpoint{0.004910in}{-0.004910in}}%
\pgfpathcurveto{\pgfqpoint{0.006213in}{-0.003608in}}{\pgfqpoint{0.006944in}{-0.001842in}}{\pgfqpoint{0.006944in}{0.000000in}}%
\pgfpathcurveto{\pgfqpoint{0.006944in}{0.001842in}}{\pgfqpoint{0.006213in}{0.003608in}}{\pgfqpoint{0.004910in}{0.004910in}}%
\pgfpathcurveto{\pgfqpoint{0.003608in}{0.006213in}}{\pgfqpoint{0.001842in}{0.006944in}}{\pgfqpoint{0.000000in}{0.006944in}}%
\pgfpathcurveto{\pgfqpoint{-0.001842in}{0.006944in}}{\pgfqpoint{-0.003608in}{0.006213in}}{\pgfqpoint{-0.004910in}{0.004910in}}%
\pgfpathcurveto{\pgfqpoint{-0.006213in}{0.003608in}}{\pgfqpoint{-0.006944in}{0.001842in}}{\pgfqpoint{-0.006944in}{0.000000in}}%
\pgfpathcurveto{\pgfqpoint{-0.006944in}{-0.001842in}}{\pgfqpoint{-0.006213in}{-0.003608in}}{\pgfqpoint{-0.004910in}{-0.004910in}}%
\pgfpathcurveto{\pgfqpoint{-0.003608in}{-0.006213in}}{\pgfqpoint{-0.001842in}{-0.006944in}}{\pgfqpoint{0.000000in}{-0.006944in}}%
\pgfpathclose%
\pgfusepath{stroke,fill}%
}%
\begin{pgfscope}%
\pgfsys@transformshift{0.984722in}{2.739956in}%
\pgfsys@useobject{currentmarker}{}%
\end{pgfscope}%
\end{pgfscope}%
\begin{pgfscope}%
\definecolor{textcolor}{rgb}{0.000000,0.000000,0.000000}%
\pgfsetstrokecolor{textcolor}%
\pgfsetfillcolor{textcolor}%
\pgftext[x=1.159722in,y=2.705928in,left,base]{\color{textcolor}\sffamily\fontsize{7.000000}{8.400000}\selectfont adam 0.001 0.5 7 0.0 0.0 (39.12\% - 02:42:20) (top-1)}%
\end{pgfscope}%
\begin{pgfscope}%
\pgfsetbuttcap%
\pgfsetroundjoin%
\pgfsetlinewidth{0.250937pt}%
\definecolor{currentstroke}{rgb}{0.000000,0.501961,0.000000}%
\pgfsetstrokecolor{currentstroke}%
\pgfsetdash{{0.925000pt}{0.400000pt}}{0.000000pt}%
\pgfpathmoveto{\pgfqpoint{0.887500in}{2.597256in}}%
\pgfpathlineto{\pgfqpoint{1.081944in}{2.597256in}}%
\pgfusepath{stroke}%
\end{pgfscope}%
\begin{pgfscope}%
\pgfsetbuttcap%
\pgfsetroundjoin%
\definecolor{currentfill}{rgb}{0.000000,0.501961,0.000000}%
\pgfsetfillcolor{currentfill}%
\pgfsetlinewidth{1.003750pt}%
\definecolor{currentstroke}{rgb}{0.000000,0.501961,0.000000}%
\pgfsetstrokecolor{currentstroke}%
\pgfsetdash{}{0pt}%
\pgfsys@defobject{currentmarker}{\pgfqpoint{-0.003472in}{-0.003472in}}{\pgfqpoint{0.003472in}{0.003472in}}{%
\pgfpathmoveto{\pgfqpoint{0.000000in}{-0.003472in}}%
\pgfpathcurveto{\pgfqpoint{0.000921in}{-0.003472in}}{\pgfqpoint{0.001804in}{-0.003106in}}{\pgfqpoint{0.002455in}{-0.002455in}}%
\pgfpathcurveto{\pgfqpoint{0.003106in}{-0.001804in}}{\pgfqpoint{0.003472in}{-0.000921in}}{\pgfqpoint{0.003472in}{0.000000in}}%
\pgfpathcurveto{\pgfqpoint{0.003472in}{0.000921in}}{\pgfqpoint{0.003106in}{0.001804in}}{\pgfqpoint{0.002455in}{0.002455in}}%
\pgfpathcurveto{\pgfqpoint{0.001804in}{0.003106in}}{\pgfqpoint{0.000921in}{0.003472in}}{\pgfqpoint{0.000000in}{0.003472in}}%
\pgfpathcurveto{\pgfqpoint{-0.000921in}{0.003472in}}{\pgfqpoint{-0.001804in}{0.003106in}}{\pgfqpoint{-0.002455in}{0.002455in}}%
\pgfpathcurveto{\pgfqpoint{-0.003106in}{0.001804in}}{\pgfqpoint{-0.003472in}{0.000921in}}{\pgfqpoint{-0.003472in}{0.000000in}}%
\pgfpathcurveto{\pgfqpoint{-0.003472in}{-0.000921in}}{\pgfqpoint{-0.003106in}{-0.001804in}}{\pgfqpoint{-0.002455in}{-0.002455in}}%
\pgfpathcurveto{\pgfqpoint{-0.001804in}{-0.003106in}}{\pgfqpoint{-0.000921in}{-0.003472in}}{\pgfqpoint{0.000000in}{-0.003472in}}%
\pgfpathclose%
\pgfusepath{stroke,fill}%
}%
\begin{pgfscope}%
\pgfsys@transformshift{0.984722in}{2.597256in}%
\pgfsys@useobject{currentmarker}{}%
\end{pgfscope}%
\end{pgfscope}%
\begin{pgfscope}%
\definecolor{textcolor}{rgb}{0.000000,0.000000,0.000000}%
\pgfsetstrokecolor{textcolor}%
\pgfsetfillcolor{textcolor}%
\pgftext[x=1.159722in,y=2.563228in,left,base]{\color{textcolor}\sffamily\fontsize{7.000000}{8.400000}\selectfont adam 0.001 0.5 7 0.0 0.0 (75.10\% - 02:42:20) (top-5)}%
\end{pgfscope}%
\begin{pgfscope}%
\pgfsetrectcap%
\pgfsetroundjoin%
\pgfsetlinewidth{0.501875pt}%
\definecolor{currentstroke}{rgb}{1.000000,0.000000,0.000000}%
\pgfsetstrokecolor{currentstroke}%
\pgfsetdash{}{0pt}%
\pgfpathmoveto{\pgfqpoint{0.887500in}{2.454556in}}%
\pgfpathlineto{\pgfqpoint{1.081944in}{2.454556in}}%
\pgfusepath{stroke}%
\end{pgfscope}%
\begin{pgfscope}%
\pgfsetbuttcap%
\pgfsetroundjoin%
\definecolor{currentfill}{rgb}{1.000000,0.000000,0.000000}%
\pgfsetfillcolor{currentfill}%
\pgfsetlinewidth{1.003750pt}%
\definecolor{currentstroke}{rgb}{1.000000,0.000000,0.000000}%
\pgfsetstrokecolor{currentstroke}%
\pgfsetdash{}{0pt}%
\pgfsys@defobject{currentmarker}{\pgfqpoint{-0.006944in}{-0.006944in}}{\pgfqpoint{0.006944in}{0.006944in}}{%
\pgfpathmoveto{\pgfqpoint{0.000000in}{-0.006944in}}%
\pgfpathcurveto{\pgfqpoint{0.001842in}{-0.006944in}}{\pgfqpoint{0.003608in}{-0.006213in}}{\pgfqpoint{0.004910in}{-0.004910in}}%
\pgfpathcurveto{\pgfqpoint{0.006213in}{-0.003608in}}{\pgfqpoint{0.006944in}{-0.001842in}}{\pgfqpoint{0.006944in}{0.000000in}}%
\pgfpathcurveto{\pgfqpoint{0.006944in}{0.001842in}}{\pgfqpoint{0.006213in}{0.003608in}}{\pgfqpoint{0.004910in}{0.004910in}}%
\pgfpathcurveto{\pgfqpoint{0.003608in}{0.006213in}}{\pgfqpoint{0.001842in}{0.006944in}}{\pgfqpoint{0.000000in}{0.006944in}}%
\pgfpathcurveto{\pgfqpoint{-0.001842in}{0.006944in}}{\pgfqpoint{-0.003608in}{0.006213in}}{\pgfqpoint{-0.004910in}{0.004910in}}%
\pgfpathcurveto{\pgfqpoint{-0.006213in}{0.003608in}}{\pgfqpoint{-0.006944in}{0.001842in}}{\pgfqpoint{-0.006944in}{0.000000in}}%
\pgfpathcurveto{\pgfqpoint{-0.006944in}{-0.001842in}}{\pgfqpoint{-0.006213in}{-0.003608in}}{\pgfqpoint{-0.004910in}{-0.004910in}}%
\pgfpathcurveto{\pgfqpoint{-0.003608in}{-0.006213in}}{\pgfqpoint{-0.001842in}{-0.006944in}}{\pgfqpoint{0.000000in}{-0.006944in}}%
\pgfpathclose%
\pgfusepath{stroke,fill}%
}%
\begin{pgfscope}%
\pgfsys@transformshift{0.984722in}{2.454556in}%
\pgfsys@useobject{currentmarker}{}%
\end{pgfscope}%
\end{pgfscope}%
\begin{pgfscope}%
\definecolor{textcolor}{rgb}{0.000000,0.000000,0.000000}%
\pgfsetstrokecolor{textcolor}%
\pgfsetfillcolor{textcolor}%
\pgftext[x=1.159722in,y=2.420528in,left,base]{\color{textcolor}\sffamily\fontsize{7.000000}{8.400000}\selectfont rmsprop 0.001 0.5 7 0.0 0.0 (19.54\% - 02:36:59) (top-1)}%
\end{pgfscope}%
\begin{pgfscope}%
\pgfsetbuttcap%
\pgfsetroundjoin%
\pgfsetlinewidth{0.250937pt}%
\definecolor{currentstroke}{rgb}{1.000000,0.000000,0.000000}%
\pgfsetstrokecolor{currentstroke}%
\pgfsetdash{{0.925000pt}{0.400000pt}}{0.000000pt}%
\pgfpathmoveto{\pgfqpoint{0.887500in}{2.311856in}}%
\pgfpathlineto{\pgfqpoint{1.081944in}{2.311856in}}%
\pgfusepath{stroke}%
\end{pgfscope}%
\begin{pgfscope}%
\pgfsetbuttcap%
\pgfsetroundjoin%
\definecolor{currentfill}{rgb}{1.000000,0.000000,0.000000}%
\pgfsetfillcolor{currentfill}%
\pgfsetlinewidth{1.003750pt}%
\definecolor{currentstroke}{rgb}{1.000000,0.000000,0.000000}%
\pgfsetstrokecolor{currentstroke}%
\pgfsetdash{}{0pt}%
\pgfsys@defobject{currentmarker}{\pgfqpoint{-0.003472in}{-0.003472in}}{\pgfqpoint{0.003472in}{0.003472in}}{%
\pgfpathmoveto{\pgfqpoint{0.000000in}{-0.003472in}}%
\pgfpathcurveto{\pgfqpoint{0.000921in}{-0.003472in}}{\pgfqpoint{0.001804in}{-0.003106in}}{\pgfqpoint{0.002455in}{-0.002455in}}%
\pgfpathcurveto{\pgfqpoint{0.003106in}{-0.001804in}}{\pgfqpoint{0.003472in}{-0.000921in}}{\pgfqpoint{0.003472in}{0.000000in}}%
\pgfpathcurveto{\pgfqpoint{0.003472in}{0.000921in}}{\pgfqpoint{0.003106in}{0.001804in}}{\pgfqpoint{0.002455in}{0.002455in}}%
\pgfpathcurveto{\pgfqpoint{0.001804in}{0.003106in}}{\pgfqpoint{0.000921in}{0.003472in}}{\pgfqpoint{0.000000in}{0.003472in}}%
\pgfpathcurveto{\pgfqpoint{-0.000921in}{0.003472in}}{\pgfqpoint{-0.001804in}{0.003106in}}{\pgfqpoint{-0.002455in}{0.002455in}}%
\pgfpathcurveto{\pgfqpoint{-0.003106in}{0.001804in}}{\pgfqpoint{-0.003472in}{0.000921in}}{\pgfqpoint{-0.003472in}{0.000000in}}%
\pgfpathcurveto{\pgfqpoint{-0.003472in}{-0.000921in}}{\pgfqpoint{-0.003106in}{-0.001804in}}{\pgfqpoint{-0.002455in}{-0.002455in}}%
\pgfpathcurveto{\pgfqpoint{-0.001804in}{-0.003106in}}{\pgfqpoint{-0.000921in}{-0.003472in}}{\pgfqpoint{0.000000in}{-0.003472in}}%
\pgfpathclose%
\pgfusepath{stroke,fill}%
}%
\begin{pgfscope}%
\pgfsys@transformshift{0.984722in}{2.311856in}%
\pgfsys@useobject{currentmarker}{}%
\end{pgfscope}%
\end{pgfscope}%
\begin{pgfscope}%
\definecolor{textcolor}{rgb}{0.000000,0.000000,0.000000}%
\pgfsetstrokecolor{textcolor}%
\pgfsetfillcolor{textcolor}%
\pgftext[x=1.159722in,y=2.277828in,left,base]{\color{textcolor}\sffamily\fontsize{7.000000}{8.400000}\selectfont rmsprop 0.001 0.5 7 0.0 0.0 (55.90\% - 02:36:59) (top-5)}%
\end{pgfscope}%
\begin{pgfscope}%
\pgfsetrectcap%
\pgfsetroundjoin%
\pgfsetlinewidth{0.501875pt}%
\definecolor{currentstroke}{rgb}{0.501961,0.000000,0.501961}%
\pgfsetstrokecolor{currentstroke}%
\pgfsetdash{}{0pt}%
\pgfpathmoveto{\pgfqpoint{0.887500in}{2.169156in}}%
\pgfpathlineto{\pgfqpoint{1.081944in}{2.169156in}}%
\pgfusepath{stroke}%
\end{pgfscope}%
\begin{pgfscope}%
\pgfsetbuttcap%
\pgfsetroundjoin%
\definecolor{currentfill}{rgb}{0.501961,0.000000,0.501961}%
\pgfsetfillcolor{currentfill}%
\pgfsetlinewidth{1.003750pt}%
\definecolor{currentstroke}{rgb}{0.501961,0.000000,0.501961}%
\pgfsetstrokecolor{currentstroke}%
\pgfsetdash{}{0pt}%
\pgfsys@defobject{currentmarker}{\pgfqpoint{-0.006944in}{-0.006944in}}{\pgfqpoint{0.006944in}{0.006944in}}{%
\pgfpathmoveto{\pgfqpoint{0.000000in}{-0.006944in}}%
\pgfpathcurveto{\pgfqpoint{0.001842in}{-0.006944in}}{\pgfqpoint{0.003608in}{-0.006213in}}{\pgfqpoint{0.004910in}{-0.004910in}}%
\pgfpathcurveto{\pgfqpoint{0.006213in}{-0.003608in}}{\pgfqpoint{0.006944in}{-0.001842in}}{\pgfqpoint{0.006944in}{0.000000in}}%
\pgfpathcurveto{\pgfqpoint{0.006944in}{0.001842in}}{\pgfqpoint{0.006213in}{0.003608in}}{\pgfqpoint{0.004910in}{0.004910in}}%
\pgfpathcurveto{\pgfqpoint{0.003608in}{0.006213in}}{\pgfqpoint{0.001842in}{0.006944in}}{\pgfqpoint{0.000000in}{0.006944in}}%
\pgfpathcurveto{\pgfqpoint{-0.001842in}{0.006944in}}{\pgfqpoint{-0.003608in}{0.006213in}}{\pgfqpoint{-0.004910in}{0.004910in}}%
\pgfpathcurveto{\pgfqpoint{-0.006213in}{0.003608in}}{\pgfqpoint{-0.006944in}{0.001842in}}{\pgfqpoint{-0.006944in}{0.000000in}}%
\pgfpathcurveto{\pgfqpoint{-0.006944in}{-0.001842in}}{\pgfqpoint{-0.006213in}{-0.003608in}}{\pgfqpoint{-0.004910in}{-0.004910in}}%
\pgfpathcurveto{\pgfqpoint{-0.003608in}{-0.006213in}}{\pgfqpoint{-0.001842in}{-0.006944in}}{\pgfqpoint{0.000000in}{-0.006944in}}%
\pgfpathclose%
\pgfusepath{stroke,fill}%
}%
\begin{pgfscope}%
\pgfsys@transformshift{0.984722in}{2.169156in}%
\pgfsys@useobject{currentmarker}{}%
\end{pgfscope}%
\end{pgfscope}%
\begin{pgfscope}%
\definecolor{textcolor}{rgb}{0.000000,0.000000,0.000000}%
\pgfsetstrokecolor{textcolor}%
\pgfsetfillcolor{textcolor}%
\pgftext[x=1.159722in,y=2.135128in,left,base]{\color{textcolor}\sffamily\fontsize{7.000000}{8.400000}\selectfont rmsprop 0.01 0.5 7 0.0 0.0 (3.30\% - 02:38:14) (top-1)}%
\end{pgfscope}%
\begin{pgfscope}%
\pgfsetbuttcap%
\pgfsetroundjoin%
\pgfsetlinewidth{0.250937pt}%
\definecolor{currentstroke}{rgb}{0.501961,0.000000,0.501961}%
\pgfsetstrokecolor{currentstroke}%
\pgfsetdash{{0.925000pt}{0.400000pt}}{0.000000pt}%
\pgfpathmoveto{\pgfqpoint{0.887500in}{2.026456in}}%
\pgfpathlineto{\pgfqpoint{1.081944in}{2.026456in}}%
\pgfusepath{stroke}%
\end{pgfscope}%
\begin{pgfscope}%
\pgfsetbuttcap%
\pgfsetroundjoin%
\definecolor{currentfill}{rgb}{0.501961,0.000000,0.501961}%
\pgfsetfillcolor{currentfill}%
\pgfsetlinewidth{1.003750pt}%
\definecolor{currentstroke}{rgb}{0.501961,0.000000,0.501961}%
\pgfsetstrokecolor{currentstroke}%
\pgfsetdash{}{0pt}%
\pgfsys@defobject{currentmarker}{\pgfqpoint{-0.003472in}{-0.003472in}}{\pgfqpoint{0.003472in}{0.003472in}}{%
\pgfpathmoveto{\pgfqpoint{0.000000in}{-0.003472in}}%
\pgfpathcurveto{\pgfqpoint{0.000921in}{-0.003472in}}{\pgfqpoint{0.001804in}{-0.003106in}}{\pgfqpoint{0.002455in}{-0.002455in}}%
\pgfpathcurveto{\pgfqpoint{0.003106in}{-0.001804in}}{\pgfqpoint{0.003472in}{-0.000921in}}{\pgfqpoint{0.003472in}{0.000000in}}%
\pgfpathcurveto{\pgfqpoint{0.003472in}{0.000921in}}{\pgfqpoint{0.003106in}{0.001804in}}{\pgfqpoint{0.002455in}{0.002455in}}%
\pgfpathcurveto{\pgfqpoint{0.001804in}{0.003106in}}{\pgfqpoint{0.000921in}{0.003472in}}{\pgfqpoint{0.000000in}{0.003472in}}%
\pgfpathcurveto{\pgfqpoint{-0.000921in}{0.003472in}}{\pgfqpoint{-0.001804in}{0.003106in}}{\pgfqpoint{-0.002455in}{0.002455in}}%
\pgfpathcurveto{\pgfqpoint{-0.003106in}{0.001804in}}{\pgfqpoint{-0.003472in}{0.000921in}}{\pgfqpoint{-0.003472in}{0.000000in}}%
\pgfpathcurveto{\pgfqpoint{-0.003472in}{-0.000921in}}{\pgfqpoint{-0.003106in}{-0.001804in}}{\pgfqpoint{-0.002455in}{-0.002455in}}%
\pgfpathcurveto{\pgfqpoint{-0.001804in}{-0.003106in}}{\pgfqpoint{-0.000921in}{-0.003472in}}{\pgfqpoint{0.000000in}{-0.003472in}}%
\pgfpathclose%
\pgfusepath{stroke,fill}%
}%
\begin{pgfscope}%
\pgfsys@transformshift{0.984722in}{2.026456in}%
\pgfsys@useobject{currentmarker}{}%
\end{pgfscope}%
\end{pgfscope}%
\begin{pgfscope}%
\definecolor{textcolor}{rgb}{0.000000,0.000000,0.000000}%
\pgfsetstrokecolor{textcolor}%
\pgfsetfillcolor{textcolor}%
\pgftext[x=1.159722in,y=1.992428in,left,base]{\color{textcolor}\sffamily\fontsize{7.000000}{8.400000}\selectfont rmsprop 0.01 0.5 7 0.0 0.0 (15.20\% - 02:38:14) (top-5)}%
\end{pgfscope}%
\begin{pgfscope}%
\pgfsetrectcap%
\pgfsetroundjoin%
\pgfsetlinewidth{0.501875pt}%
\definecolor{currentstroke}{rgb}{0.647059,0.164706,0.164706}%
\pgfsetstrokecolor{currentstroke}%
\pgfsetdash{}{0pt}%
\pgfpathmoveto{\pgfqpoint{0.887500in}{1.883756in}}%
\pgfpathlineto{\pgfqpoint{1.081944in}{1.883756in}}%
\pgfusepath{stroke}%
\end{pgfscope}%
\begin{pgfscope}%
\pgfsetbuttcap%
\pgfsetroundjoin%
\definecolor{currentfill}{rgb}{0.647059,0.164706,0.164706}%
\pgfsetfillcolor{currentfill}%
\pgfsetlinewidth{1.003750pt}%
\definecolor{currentstroke}{rgb}{0.647059,0.164706,0.164706}%
\pgfsetstrokecolor{currentstroke}%
\pgfsetdash{}{0pt}%
\pgfsys@defobject{currentmarker}{\pgfqpoint{-0.006944in}{-0.006944in}}{\pgfqpoint{0.006944in}{0.006944in}}{%
\pgfpathmoveto{\pgfqpoint{0.000000in}{-0.006944in}}%
\pgfpathcurveto{\pgfqpoint{0.001842in}{-0.006944in}}{\pgfqpoint{0.003608in}{-0.006213in}}{\pgfqpoint{0.004910in}{-0.004910in}}%
\pgfpathcurveto{\pgfqpoint{0.006213in}{-0.003608in}}{\pgfqpoint{0.006944in}{-0.001842in}}{\pgfqpoint{0.006944in}{0.000000in}}%
\pgfpathcurveto{\pgfqpoint{0.006944in}{0.001842in}}{\pgfqpoint{0.006213in}{0.003608in}}{\pgfqpoint{0.004910in}{0.004910in}}%
\pgfpathcurveto{\pgfqpoint{0.003608in}{0.006213in}}{\pgfqpoint{0.001842in}{0.006944in}}{\pgfqpoint{0.000000in}{0.006944in}}%
\pgfpathcurveto{\pgfqpoint{-0.001842in}{0.006944in}}{\pgfqpoint{-0.003608in}{0.006213in}}{\pgfqpoint{-0.004910in}{0.004910in}}%
\pgfpathcurveto{\pgfqpoint{-0.006213in}{0.003608in}}{\pgfqpoint{-0.006944in}{0.001842in}}{\pgfqpoint{-0.006944in}{0.000000in}}%
\pgfpathcurveto{\pgfqpoint{-0.006944in}{-0.001842in}}{\pgfqpoint{-0.006213in}{-0.003608in}}{\pgfqpoint{-0.004910in}{-0.004910in}}%
\pgfpathcurveto{\pgfqpoint{-0.003608in}{-0.006213in}}{\pgfqpoint{-0.001842in}{-0.006944in}}{\pgfqpoint{0.000000in}{-0.006944in}}%
\pgfpathclose%
\pgfusepath{stroke,fill}%
}%
\begin{pgfscope}%
\pgfsys@transformshift{0.984722in}{1.883756in}%
\pgfsys@useobject{currentmarker}{}%
\end{pgfscope}%
\end{pgfscope}%
\begin{pgfscope}%
\definecolor{textcolor}{rgb}{0.000000,0.000000,0.000000}%
\pgfsetstrokecolor{textcolor}%
\pgfsetfillcolor{textcolor}%
\pgftext[x=1.159722in,y=1.849728in,left,base]{\color{textcolor}\sffamily\fontsize{7.000000}{8.400000}\selectfont sgd 0.001 0.5 7 0.9 0.0 with nesterov (83.59\% - 02:30:20) (top-1)}%
\end{pgfscope}%
\begin{pgfscope}%
\pgfsetbuttcap%
\pgfsetroundjoin%
\pgfsetlinewidth{0.250937pt}%
\definecolor{currentstroke}{rgb}{0.647059,0.164706,0.164706}%
\pgfsetstrokecolor{currentstroke}%
\pgfsetdash{{0.925000pt}{0.400000pt}}{0.000000pt}%
\pgfpathmoveto{\pgfqpoint{0.887500in}{1.741056in}}%
\pgfpathlineto{\pgfqpoint{1.081944in}{1.741056in}}%
\pgfusepath{stroke}%
\end{pgfscope}%
\begin{pgfscope}%
\pgfsetbuttcap%
\pgfsetroundjoin%
\definecolor{currentfill}{rgb}{0.647059,0.164706,0.164706}%
\pgfsetfillcolor{currentfill}%
\pgfsetlinewidth{1.003750pt}%
\definecolor{currentstroke}{rgb}{0.647059,0.164706,0.164706}%
\pgfsetstrokecolor{currentstroke}%
\pgfsetdash{}{0pt}%
\pgfsys@defobject{currentmarker}{\pgfqpoint{-0.003472in}{-0.003472in}}{\pgfqpoint{0.003472in}{0.003472in}}{%
\pgfpathmoveto{\pgfqpoint{0.000000in}{-0.003472in}}%
\pgfpathcurveto{\pgfqpoint{0.000921in}{-0.003472in}}{\pgfqpoint{0.001804in}{-0.003106in}}{\pgfqpoint{0.002455in}{-0.002455in}}%
\pgfpathcurveto{\pgfqpoint{0.003106in}{-0.001804in}}{\pgfqpoint{0.003472in}{-0.000921in}}{\pgfqpoint{0.003472in}{0.000000in}}%
\pgfpathcurveto{\pgfqpoint{0.003472in}{0.000921in}}{\pgfqpoint{0.003106in}{0.001804in}}{\pgfqpoint{0.002455in}{0.002455in}}%
\pgfpathcurveto{\pgfqpoint{0.001804in}{0.003106in}}{\pgfqpoint{0.000921in}{0.003472in}}{\pgfqpoint{0.000000in}{0.003472in}}%
\pgfpathcurveto{\pgfqpoint{-0.000921in}{0.003472in}}{\pgfqpoint{-0.001804in}{0.003106in}}{\pgfqpoint{-0.002455in}{0.002455in}}%
\pgfpathcurveto{\pgfqpoint{-0.003106in}{0.001804in}}{\pgfqpoint{-0.003472in}{0.000921in}}{\pgfqpoint{-0.003472in}{0.000000in}}%
\pgfpathcurveto{\pgfqpoint{-0.003472in}{-0.000921in}}{\pgfqpoint{-0.003106in}{-0.001804in}}{\pgfqpoint{-0.002455in}{-0.002455in}}%
\pgfpathcurveto{\pgfqpoint{-0.001804in}{-0.003106in}}{\pgfqpoint{-0.000921in}{-0.003472in}}{\pgfqpoint{0.000000in}{-0.003472in}}%
\pgfpathclose%
\pgfusepath{stroke,fill}%
}%
\begin{pgfscope}%
\pgfsys@transformshift{0.984722in}{1.741056in}%
\pgfsys@useobject{currentmarker}{}%
\end{pgfscope}%
\end{pgfscope}%
\begin{pgfscope}%
\definecolor{textcolor}{rgb}{0.000000,0.000000,0.000000}%
\pgfsetstrokecolor{textcolor}%
\pgfsetfillcolor{textcolor}%
\pgftext[x=1.159722in,y=1.707028in,left,base]{\color{textcolor}\sffamily\fontsize{7.000000}{8.400000}\selectfont sgd 0.001 0.5 7 0.9 0.0 with nesterov (97.00\% - 02:30:20) (top-5)}%
\end{pgfscope}%
\begin{pgfscope}%
\pgfsetrectcap%
\pgfsetroundjoin%
\pgfsetlinewidth{0.501875pt}%
\definecolor{currentstroke}{rgb}{1.000000,0.752941,0.796078}%
\pgfsetstrokecolor{currentstroke}%
\pgfsetdash{}{0pt}%
\pgfpathmoveto{\pgfqpoint{0.887500in}{1.598356in}}%
\pgfpathlineto{\pgfqpoint{1.081944in}{1.598356in}}%
\pgfusepath{stroke}%
\end{pgfscope}%
\begin{pgfscope}%
\pgfsetbuttcap%
\pgfsetroundjoin%
\definecolor{currentfill}{rgb}{1.000000,0.752941,0.796078}%
\pgfsetfillcolor{currentfill}%
\pgfsetlinewidth{1.003750pt}%
\definecolor{currentstroke}{rgb}{1.000000,0.752941,0.796078}%
\pgfsetstrokecolor{currentstroke}%
\pgfsetdash{}{0pt}%
\pgfsys@defobject{currentmarker}{\pgfqpoint{-0.006944in}{-0.006944in}}{\pgfqpoint{0.006944in}{0.006944in}}{%
\pgfpathmoveto{\pgfqpoint{0.000000in}{-0.006944in}}%
\pgfpathcurveto{\pgfqpoint{0.001842in}{-0.006944in}}{\pgfqpoint{0.003608in}{-0.006213in}}{\pgfqpoint{0.004910in}{-0.004910in}}%
\pgfpathcurveto{\pgfqpoint{0.006213in}{-0.003608in}}{\pgfqpoint{0.006944in}{-0.001842in}}{\pgfqpoint{0.006944in}{0.000000in}}%
\pgfpathcurveto{\pgfqpoint{0.006944in}{0.001842in}}{\pgfqpoint{0.006213in}{0.003608in}}{\pgfqpoint{0.004910in}{0.004910in}}%
\pgfpathcurveto{\pgfqpoint{0.003608in}{0.006213in}}{\pgfqpoint{0.001842in}{0.006944in}}{\pgfqpoint{0.000000in}{0.006944in}}%
\pgfpathcurveto{\pgfqpoint{-0.001842in}{0.006944in}}{\pgfqpoint{-0.003608in}{0.006213in}}{\pgfqpoint{-0.004910in}{0.004910in}}%
\pgfpathcurveto{\pgfqpoint{-0.006213in}{0.003608in}}{\pgfqpoint{-0.006944in}{0.001842in}}{\pgfqpoint{-0.006944in}{0.000000in}}%
\pgfpathcurveto{\pgfqpoint{-0.006944in}{-0.001842in}}{\pgfqpoint{-0.006213in}{-0.003608in}}{\pgfqpoint{-0.004910in}{-0.004910in}}%
\pgfpathcurveto{\pgfqpoint{-0.003608in}{-0.006213in}}{\pgfqpoint{-0.001842in}{-0.006944in}}{\pgfqpoint{0.000000in}{-0.006944in}}%
\pgfpathclose%
\pgfusepath{stroke,fill}%
}%
\begin{pgfscope}%
\pgfsys@transformshift{0.984722in}{1.598356in}%
\pgfsys@useobject{currentmarker}{}%
\end{pgfscope}%
\end{pgfscope}%
\begin{pgfscope}%
\definecolor{textcolor}{rgb}{0.000000,0.000000,0.000000}%
\pgfsetstrokecolor{textcolor}%
\pgfsetfillcolor{textcolor}%
\pgftext[x=1.159722in,y=1.564328in,left,base]{\color{textcolor}\sffamily\fontsize{7.000000}{8.400000}\selectfont sgd 0.001 0.5 7 0.9 0.0 without nesterov (83.35\% - 02:27:04) (top-1)}%
\end{pgfscope}%
\begin{pgfscope}%
\pgfsetbuttcap%
\pgfsetroundjoin%
\pgfsetlinewidth{0.250937pt}%
\definecolor{currentstroke}{rgb}{1.000000,0.752941,0.796078}%
\pgfsetstrokecolor{currentstroke}%
\pgfsetdash{{0.925000pt}{0.400000pt}}{0.000000pt}%
\pgfpathmoveto{\pgfqpoint{0.887500in}{1.455656in}}%
\pgfpathlineto{\pgfqpoint{1.081944in}{1.455656in}}%
\pgfusepath{stroke}%
\end{pgfscope}%
\begin{pgfscope}%
\pgfsetbuttcap%
\pgfsetroundjoin%
\definecolor{currentfill}{rgb}{1.000000,0.752941,0.796078}%
\pgfsetfillcolor{currentfill}%
\pgfsetlinewidth{1.003750pt}%
\definecolor{currentstroke}{rgb}{1.000000,0.752941,0.796078}%
\pgfsetstrokecolor{currentstroke}%
\pgfsetdash{}{0pt}%
\pgfsys@defobject{currentmarker}{\pgfqpoint{-0.003472in}{-0.003472in}}{\pgfqpoint{0.003472in}{0.003472in}}{%
\pgfpathmoveto{\pgfqpoint{0.000000in}{-0.003472in}}%
\pgfpathcurveto{\pgfqpoint{0.000921in}{-0.003472in}}{\pgfqpoint{0.001804in}{-0.003106in}}{\pgfqpoint{0.002455in}{-0.002455in}}%
\pgfpathcurveto{\pgfqpoint{0.003106in}{-0.001804in}}{\pgfqpoint{0.003472in}{-0.000921in}}{\pgfqpoint{0.003472in}{0.000000in}}%
\pgfpathcurveto{\pgfqpoint{0.003472in}{0.000921in}}{\pgfqpoint{0.003106in}{0.001804in}}{\pgfqpoint{0.002455in}{0.002455in}}%
\pgfpathcurveto{\pgfqpoint{0.001804in}{0.003106in}}{\pgfqpoint{0.000921in}{0.003472in}}{\pgfqpoint{0.000000in}{0.003472in}}%
\pgfpathcurveto{\pgfqpoint{-0.000921in}{0.003472in}}{\pgfqpoint{-0.001804in}{0.003106in}}{\pgfqpoint{-0.002455in}{0.002455in}}%
\pgfpathcurveto{\pgfqpoint{-0.003106in}{0.001804in}}{\pgfqpoint{-0.003472in}{0.000921in}}{\pgfqpoint{-0.003472in}{0.000000in}}%
\pgfpathcurveto{\pgfqpoint{-0.003472in}{-0.000921in}}{\pgfqpoint{-0.003106in}{-0.001804in}}{\pgfqpoint{-0.002455in}{-0.002455in}}%
\pgfpathcurveto{\pgfqpoint{-0.001804in}{-0.003106in}}{\pgfqpoint{-0.000921in}{-0.003472in}}{\pgfqpoint{0.000000in}{-0.003472in}}%
\pgfpathclose%
\pgfusepath{stroke,fill}%
}%
\begin{pgfscope}%
\pgfsys@transformshift{0.984722in}{1.455656in}%
\pgfsys@useobject{currentmarker}{}%
\end{pgfscope}%
\end{pgfscope}%
\begin{pgfscope}%
\definecolor{textcolor}{rgb}{0.000000,0.000000,0.000000}%
\pgfsetstrokecolor{textcolor}%
\pgfsetfillcolor{textcolor}%
\pgftext[x=1.159722in,y=1.421628in,left,base]{\color{textcolor}\sffamily\fontsize{7.000000}{8.400000}\selectfont sgd 0.001 0.5 7 0.9 0.0 without nesterov (96.90\% - 02:27:04) (top-5)}%
\end{pgfscope}%
\end{pgfpicture}%
\makeatother%
\endgroup%

					\end{center}
					\caption{Overview of best optimizer}
					\label{fig:evaluation_momentum}
				\end{figure}

				\noindent ...

				\paragraph{Influence of the momentum and the Nesterov momentum}

				\noindent ...

				\begin{figure}[H]
					\begin{center}
						\inputpgf{images/evaluation}{momentum.pgf}
					\end{center}
					\caption{Overview momentum vs nesterov momentum}
					\label{fig:evaluation_momentum}
				\end{figure}

				\noindent ...
		
			\subsubsection{Influence of the number of trained layers on the accuracy}
	
				\noindent ...

				\begin{figure}[H]
					\begin{center}
						\inputpgf{images/evaluation}{number_trainable_layers.pgf}
					\end{center}
					\caption{Overview of influence of the number of trained layers}
					\label{fig:evaluation_number_trainable_layers}
				\end{figure}

				\noindent ...
		
			\subsubsection{Influence of a dynamic learning rate on accuracy (scheduling)}

				\noindent ...

				\begin{figure}[H]
					\begin{center}
						\inputpgf{images/evaluation}{scheduling_learning_rate.pgf}
					\end{center}
					\caption{Overview of a dynamic learning rate on accuracy}
					\label{fig:evaluation_scheduling_learning_rate}
				\end{figure}

				\noindent ...
		
			\subsubsection{Different batch sizes}
	
				\noindent ...
		
			\subsubsection{Different image sizes}
	
				\noindent ...
		
			\subsubsection{Different number of learned epochs}
	
				\noindent ...


	
	% -------------------- %
	% Optimization process %
	% -------------------- %
	\section{Optimization process}
		This chapter contains ideas, approaches and evaluations of more complex ideas, which do not fit into the range of simple parameter changes.
	
		\subsection{Preamble}
			\noindent ...

		\subsection{Data augmentation}
			\noindent ...

			\begin{figure}[H]
				\begin{center}
					\inputpgf{images/pgf}{augment.pgf}
				\end{center}
				\caption{Data Augmentation}
				\label{fig:data_augmentation}
			\end{figure}

		\subsection{Enrichment of the data set from other data sources}
			\noindent ...
		
		\subsection{Analyses with multidimensional scaling}
			\noindent ...
		
		\subsection{Hierarchical classification}
			\de{Durch die Verwendung eines einzigen Modelles für alle Klassen, sind die bisherigen Klassifikatoren darauf trainiert, den Verlust am Klassenausgabevektor zu minimieren. Jede bisher verwendete Klasse hat den gleichen Rang sowohl beim Training, als auch bei der Klassifizierung. Die Vorhersage von "Pizza" kostet genauso viel wie die Vorhersage von "Martini".}
			\en{By using a single model for all classes, previous classifiers have been trained to minimize the loss of the class output vector. Each class used so far has the same rank in both training and classification. The prediction of "Pizza" costs the same as the prediction of "Martini".}

			\de{Die menschliche Fähigkeit Objekte einordnen zu können, funktioniert nicht nur auf einer Ebene. Kategorien werden sich natürlich überlappen und eine hierarchische Struktur aufweisen. So wird ein Mensch ein Bild beispielsweise unter "Pizza", "Thunfischpizza" oder sogar "Fastfood" einordnen, was so gesehen korrekt ist. Je nach Einordnung findet hier lediglich ein "Informationsverlust" statt. Jedoch wird der Mensch eine "Pizza" meist nicht fälschlicherweise als "Martini" verwechseln, welcher eher der Kategorie "Getränk" oder "Cocktail" einzuordnen ist\autocite{rosch2004basic}.}
			\en{The human ability to classify objects does not only work on one level. Categories will naturally overlap and have a hierarchical structure. For example, a human will classify a picture under "pizza", "tuna pizza" or even "fast food", which is correct from this point of view. Depending on the classification, there will only be a "loss of information". However, a person will not mistake a "pizza" as a "Martini", which is more likely to be classified as a "drink" or "cocktail"\autocite{rosch2004basic}.}
		
		\subsection{Binary classifiers}
			\noindent ...
		
		\subsection{Evaluation}
			\noindent ...
		
		\subsection{Use of the model across programming languages}
			\noindent ...



	% -------------------- %
	% Summary and outlook %
	% -------------------- %
	\pagebreak
	\section{Summary and outlook}
		\hl{What's the outcome? What else is possible? How can this work be continued? In here!}


	% -------------------- %
	% List of figures %
	% -------------------- %
	\pagebreak
	\renewcommand{\listfigurename}{List of figures}
	\addcontentsline{toc}{section}{List of figures}
	\listoffigures
	\listoftables
	


	% -------------------- %
	% List of literature %
	% -------------------- %
	\pagebreak	
	\section*{List of literature}
		\addcontentsline{toc}{section}{List of literature}
		\printbibliography[heading=none]
	


	% -------------------- %
	% List of literature %
	% -------------------- %
	\pagebreak	
	\section*{List of links}
		\addcontentsline{toc}{section}{List of links}
		\begin{itemize}
			\item Deep learning unbalanced training data?
			\begin{itemize}
				\item \url{https://towardsdatascience.com/deep-learning-unbalanced-training-data-solve-it-like-this-6c528e9efea6}
			\end{itemize}
			\item Data Augmentation
			\begin{itemize}
				\item \url{https://machinelearningmastery.com/how-to-configure-image-data-augmentation-when-training-deep-learning-neural-networks/}
			\end{itemize}
			\item Stop Feeding Garbage To Your Model! — The 6 biggest mistakes with datasets and how to avoid them.
			\begin{itemize}
				\item \url{https://hackernoon.com/stop-feeding-garbage-to-your-model-the-6-biggest-mistakes-with-datasets-and-how-to-avoid-them-3cb7532ad3b7}
			\end{itemize}
		\end{itemize}
	


	% -------------------- %
	% Declaration %
	% -------------------- %
	\pagebreak
	\section*{Declaration}
		\thispagestyle{empty}
		
		\noindent I hereby declare that the work presented in this thesis is solely my work and that to the best of my
		knowledge this work is original, except where indicated by references to other authors. No part of this
		work has been submitted for any other degree or diploma. 
		
		\begin{displaymath}
		% use packages: array
		\begin{array}{ll}
		Signature:~~~~~~~~~~~~~~~~~~~~~~~~~~~~~~~~~~~~~~~~~~
		& Place, Date:~~~~~~~~~~~~~~~~~~~~~~~~~~~~~~~~~~~~~~~~~~
		\end{array}
		\end{displaymath}

\end{document}